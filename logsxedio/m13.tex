\documentclass[A4,10pt,greek]{book}
\usepackage{xltxtra}
\usepackage{xgreek}
\usepackage[a4paper, inner=1.5cm, outer=3cm, top=2cm, bottom=3cm, bindingoffset=1cm]{geometry}
\setmainfont[Mapping=tex-text]{GFS Didot}
\usepackage{amsfonts}
\usepackage{booktabs}
\usepackage{ltablex}

\title{Ελληνικό Λογιστικό Σχέδιο}
\author{Λάζαρος Θεόδωρος}

\begin{document}
\maketitle

\tableofcontents

\frontmatter
\chapter{Πρόλογος}
Πλήρης ανάλυση του Ελληνικού Λογιστικού Σχεδίου

\mainmatter
\chapter{ΠΑΓΙΟ ΕΝΕΡΓΗΤΙΚΟ}
\section{Λογαριασμοί}
% Please add the following required packages to your document preamble:
% \usepackage{booktabs}
\noindent
\begin{tabularx}{\linewidth}{lX}
10 & Εδαφικές εκτάσεις\\
10.00 & Γήπεδα - Οικόπεδα\\
10.01 & Ορυχεία\\
10.02 & Μεταλλεία\\
10.03 & Λατομεία\\
10.04 & Αγροί\\
10.05 & Φυτείες\\
10.06 & Δάση\\
10.10 & Γήπεδα - Οικόπεδα εκτός εκμεταλλεύσεως\\
10.11 & Ορυχεία εκτός εκμεταλλεύσεως\\
10.12 & Μεταλλεία εκτός εκμεταλλεύσεως\\
10.13 & Λατομεία εκτός εκμεταλλεύσεως\\
10.14 & Αγροί εκτός εκμεταλλεύσεως\\
10.15 & Φυτείες εκτός εκμεταλλεύσεως\\
10.16 & Δάση εκτός εκμεταλλεύσεως\\
10.99 & Αποσβεσμένες εδαφικές εκτάσεις\\
10.99.01 & Αποσβεσμένα Ορυχεία\\
10.99.02 & Αποσβεσμένα Μεταλλεία\\
10.99.03 & Αποσβεσμένα Λατομεία\\
10.99.05 & Αποσβεσμένες Φυτείες\\
10.99.06 & Αποσβεσμένα Δάση\\
10.99.11 & Αποσβεσμένα Ορυχεία εκτός εκμεταλλεύσεως\\
10.99.12 & Αποσβεσμένα Μεταλλεία εκτός εκμεταλλεύσεως\\
10.99.13 & Αποσβεσμένα Λατομεία εκτός εκμεταλλεύσεως\\
10.99.15 & Αποσβεσμένες Φυτείες εκτός εκμεταλλεύσεως\\
10.99.16 & Αποσβεσμένα Δάση εκτός εκμεταλλεύσεως\\
11 & Κτίρια - Εγκαταστάσεις κτιρίων - Τεχνικά έργα\\
11.00 & Κτίρια - Εγκαταστάσεις Κτιρίων\\
11.01 & Τεχνικά έργα εξυπηρετήσεως μεταφορών\\
11.02 & Λοιπά τεχνικά έργα\\
11.03 & Υποκείμενες σε απόσβεση διαμορφώσεις γηπέδων\\
11.07 & Κτίρια - Εγκαταστάσεις κτιρίων σε ακίνητα τρίτων\\
11.08 & Τεχνικά έργα εξυπηρετήσεως μεταφορών σε ακίνητα τρίτων\\
11.09 & Λοιπά τεχνικά έργα σε ακίνητα τρίτων\\
11.10 & Υποκείμενες σε απόσβεση διαμορφώσεις γηπέδων τρίτων\\
11.14 & Κτίρια - Εγκαταστάσεις κτιρίων εκτός εκμεταλλεύσεως\\
11.15 & Τεχνικά έργα εξυπηρετήσεως μεταφορών εκτός εκμεταλλεύσεως\\
11.16 & Λοιπά τεχνικά έργα εκτός εκμεταλλεύσεως\\
11.17 & Υποκείμενες σε απόσβεση διαμορφώσεις γηπέδων εκτός εκμεταλλεύσεως\\
11.21 & Κτίρια - Εγκαταστάσεις κτιρίων σε ακίνητα τρίτων εκτός εκμεταλλεύσεως\\
11.22 & Τεχνικά έργα εξυπηρετήσεως μεταφορών σε ακίνητα τρίτων εκτός εκμεταλλεύσεως\\
11.23 & Λοιπά τεχνικά έργα σε ακίνητα τρίτων εκτός εκμεταλλεύσεως\\
11.24 & Υποκείμενες σε απόσβεση διαμορφώσεις γηπέδων τρίτων εκτός εκμεταλλεύσεως\\
11.99 & Αποσβεσμένα κτίρια - Εγκαταστάσεις κτιρίων - Τεχνικά έργα\\
11.99.00 & Αποσβεσμένα κτίρια - εγκαταστάσεις κτιρίων\\
11.99.01 & Αποσβεσμένα τεχνικά έργα εξυπηρετήσεως μεταφορών\\
11.99.02 & Αποσβεσμένα λοιπά τεχνικά έργα\\
11.99.03 & Αποσβεσμένες διαμορφώσεις γηπέδων\\
11.99.07 & Αποσβεσμένα κτίρια - εγκαταστάσεις κτιρίων σε ακίνητα τρίτων\\
11.99.08 & Αποσβεσμένα τεχνικά έργα εξυπηρετήσεως μεταφορών σε ακίνητα τρίτων\\
11.99.09 & Αποσβεσμένα λοιπά τεχνικά έργα σε ακίνητα τρίτων\\
11.99.10 & Αποσβεσμένες διαμορφώσεις γηπέδων τρίτων\\
11.99.14 & Αποσβεσμένα κτίρια - εγκαταστάσεις κτιρίων εκτός εκμεταλλεύσεως\\
11.99.15 & Αποσβεσμένα τεχνικά έργα εξυπηρετήσεως μεταφορών εκτός εκμεταλλεύσεως\\
11.99.16 & Αποσβεσμένα λοιπά τεχνικά έργα εκτός εκμεταλλεύσεως\\
11.99.17 & Αποσβεσμένες διαμορφώσεις γηπέδων εκτός εκμεταλλεύσεως\\
11.99.21 & Αποσβεσμένα κτίρια - εγκαταστάσεις κτιρίων σε ακίνητα τρίτων εκτός εκμεταλλεύσεως\\
11.99.22 & Αποσβεσμένα τεχνικά έργα εξυπηρετήσεως μεταφορών σε ακίνητα τρίτων εκτός εκμεταλλεύσεως\\
11.99.23 & Αποσβεσμένα λοιπά τεχνικά έργα σε ακίνητα τρίτων εκτός εκμεταλλεύσεως\\
11.99.24 & Αποσβεσμένες διαμορφώσεις γηπέδων τρίτων εκτός εκμεταλλεύσεως\\
12 & Μηχανήματα - Τεχνικές εγκαταστάσεις - Λοιπός μηχανολογικός εξοπλισμός\\
12.00 & Μηχανήματα\\
12.01 & Τεχνικές εγκαταστάσεις\\
12.02 & Φορητά μηχανήματα «χειρός»\\
12.03 & Εργαλεία\\
12.04 & Καλούπια - Ιδιοσυσκευές\\
12.05 & Μηχανολογικά όργανα\\
12.06 & Λοιπός μηχανολογικός εξοπλισμός\\
12.07 & Μηχανήματα σε ακίνητα τρίτων\\
12.08 & Τεχνικές εγκαταστάσεις σε ακίνητα τρίτων\\
12.09 & Λοιπός μηχανολογικός εξοπλισμός σε ακίνητα τρίτων\\
12.10 & Μηχανήματα εκτός εκμεταλλεύσεως\\
12.11 & Τεχνικές εγκαταστάσεις εκτός εκμεταλλεύσεως\\
12.12 & Φορητά μηχανήματα «χειρός» εκτός εκμεταλλεύσεως\\
12.13 & Εργαλεία εκτός εκμεταλλεύσεως\\
12.14 & Καλούπια - Ιδιοσυσκευές εκτός εκμεταλλεύσεως\\
12.15 & Μηχανολογικά όργανα εκτός εκμεταλλεύσεως\\
12.16 & Λοιπός μηχανολογικός εξοπλισμός εκτός εκμεταλλεύσεως\\
12.17 & Μηχανήματα σε ακίνητα τρίτων εκτός εκμεταλλεύσεως\\
12.18 & Τεχνικές εγκαταστάσεις σε ακίνητα τρίτων εκτός εκμεταλλεύσεως\\
12.19 & Λοιπός μηχανολογικός εξοπλισμός σε ακίνητα τρίτων εκτός εκμεταλλεύσεως\\
12.90 & Μηχανήματα και λοιπός εξοπλισμός στον ΟΔΔΥ για εκποίηση (Γνωμ. 253/2243/1995)\\
12.99 & Αποσβεσμένα μηχανήματα - τεχνικές εγκαταστάσεις - λοιπός μηχανολογικός εξοπλισμός\\
12.99.00 & Αποσβεσμένα μηχανήματα\\
12.99.01 & Αποσβεσμένες τεχνικές εγκαταστάσεις\\
12.99.02 & Αποσβεσμένα φορητά μηχανήματα «χειρός»\\
12.99.03 & Αποσβεσμένα εργαλεία\\
12.99.04 & Αποσβεσμένα καλούπια - ιδιοσυσκευές\\
12.99.05 & Αποσβεσμένα μηχανολογικά όργανα\\
12.99.06 & Αποσβεσμένος λοιπός μηχανολογικός εξοπλισμός\\
12.99.07 & Αποσβεσμένα μηχανήματα σε ακίνητα τρίτων\\
12.99.08 & Αποσβεσμένες τεχνικές εγκαταστάσεις σε ακίνητα τρίτων\\
12.99.09 & Αποσβεσμένος λοιπός μηχανολογ. εξοπλισμός σε ακίνητα τρίτων\\
12.99.10 & Αποσβεσμένα μηχανήματα εκτός εκμεταλλεύσεως\\
12.99.11 & Αποσβεσμένες τεχνικές εγκαταστάσεις εκτός εκμεταλλεύσεως\\
12.99.12 & Αποσβεσμένα φορητά μηχανήματα «χειρός» εκτός εκμεταλλεύσεως\\
12.99.13 & Αποσβεσμένα εργαλεία εκτός εκμεταλλεύσεως\\
12.99.14 & Αποσβεσμένα καλούπια - ιδιοσυσκευές εκτός εκμεταλλεύσεως\\
12.99.15 & Αποσβεσμένα μηχανολογικά όργανα εκτός εκμεταλλεύσεως\\
12.99.16 & Αποσβεσμένος λοιπός μηχανολογικός εξοπλισμός εκτός εκμεταλλεύσεως\\
12.99.17 & Αποσβεσμένα μηχανήματα σε ακίνητα τρίτων εκτός εκμεταλλεύσεως\\
12.99.18 & Αποσβεσμένες τεχνικές εγκαταστάσεις σε ακίνητα τρίτων εκτός εκμεταλλεύσεως\\
12.99.19 & Αποσβεσμένος λοιπός μηχανολογικός εξοπλισμός σε ακίνητα τρίτων εκτός εκμεταλλεύσεως\\
13 & Μεταφορικά μέσα\\
13.00 & Αυτοκίνητα λεωφορεία\\
13.01 & Λοιπά επιβατικά αυτοκίνητα\\
13.02 & Αυτοκίνητα φορτηγά - Ρυμούλκες - Ειδικής Χρήσεως\\
13.03 & Σιδηροδρομικά οχήματα\\
13.04 & Πλωτά μέσα\\
13.05 & Εναέρια μέσα\\
13.06 & Μέσα εσωτερικών μεταφορών\\
13.09 & Λοιπά μέσα μεταφοράς\\
13.10 & Αυτοκίνητα λεωφορεία εκτός εκμεταλλεύσεως\\
13.11 & Λοιπά επιβατικά αυτοκίνητα εκτός εκμεταλλεύσεως\\
13.12 & Αυτοκίνητα φορτηγά - Ρυμούλκες - Ειδικής χρήσεως εκτός εκμεταλλεύσεως\\
13.13 & Σιδηροδρομικά οχήματα εκτός εκμεταλλεύσεως\\
13.14 & Πλωτά μέσα εκτός εκμεταλλεύσεως\\
13.15 & Εναέρια μέσα εκτός εκμεταλλεύσεως\\
13.16 & Μέσα εσωτερικών μεταφορών εκτός εκμεταλλεύσεως\\
13.19 & Λοιπά μέσα μεταφοράς εκτός εκμεταλλεύσεως\\
13.90 & Μεταφορικά μέσα στον ΟΔΔΥ για εκποίηση (Γνωμ. 253/2243/1995)\\
13.99 & Αποσβεσμένα μέσα μεταφοράς\\
13.99.00 & Αποσβεσμένα αυτοκίνητα λεωφορεία\\
13.99.01 & Αποσβεσμένα λοιπά επιβατικά αυτοκίνητα\\
13.99.02 & Αποσβεσμένα φορτηγά - Ρυμούλκες - Ειδικής χρήσεως\\
13.99.03 & Αποσβεσμένα σιδηροδρομικά οχήματα\\
13.99.04 & Αποσβεσμένα πλωτά μέσα\\
13.99.05 & Αποσβεσμένα εναέρια μέσα\\
13.99.06 & Αποσβεσμένα μέσα εσωτερικών μεταφορών\\
13.99.09 & Αποσβεσμένα λοιπά μέσα μεταφοράς\\
13.99.10 & Αποσβεσμένα αυτοκίνητα λεωφορεία εκτός εκμεταλλεύσεως\\
13.99.11 & Αποσβεσμένα λοιπά επιβατικά αυτοκίνητα εκτός εκμεταλλεύσεως12 Αποσβεσμένα αυτοκίνητα φορτηγά - Ρυμούλκες - Ειδικής χρήσεως εκτός εκμεταλλεύσεως\\
13.99.13 & Αποσβεσμένα σιδηροδρομικά οχήματα εκτός εκμεταλλεύσεως\\
13.99.14 & Αποσβεσμένα πλωτά μέσα εκτός εκμεταλλεύσεως\\
13.99.15 & Αποσβεσμένα εναέρια μέσα εκτός εκμεταλλεύσεως\\
13.99.16 & Αποσβεσμένα μέσα εσωτερικών μεταφορών εκτός εκμεταλλεύσεως\\
13.99.19 & Αποσβεσμένα λοιπά μέσα μεταφοράς εκτός εκμεταλλεύσεως\\
14 & Έπιπλα και λοιπός εξοπλισμός\\
14.00 & Έπιπλα\\
14.01 & Σκεύη\\
14.02 & Μηχανές γραφείων\\
14.03 & Ηλεκτρονικοί υπολογιστές και ηλεκτρονικά συγκροτήματα\\
14.04 & Μέσα αποθηκεύσεως και μεταφοράς\\
14.05 & Επιστημονικά όργανα\\
14.06 & Ζώα για πάγια εκμετάλλευση (γεωργικών και κτηνοτροφικών επιχ/σεων)\\
14.08 & Εξοπλισμός τηλεπικοινωνιών\\
14.09 & Λοιπός εξοπλισμός\\
14.10 & Έπιπλα εκτός εκμεταλλεύσεως\\
14.11 & Σκεύη εκτός εκμεταλλεύσεως\\
14.12 & Μηχανές γραφείων εκτός εκμεταλλεύσεως\\
14.13 & Ηλεκτρονικοί υπολογιστές και ηλεκτρονικά συγκροτήματα εκτός εκμεταλλεύσεως\\
14.14 & Μέσα αποθηκεύσεως και μεταφοράς εκτός εκμεταλλεύσεως\\
14.15 & Επιστημονικά όργανα εκτός εκμεταλλεύσεως\\
14.16 & Ζώα για πάγια εκμετάλλευση εκτός εκμεταλλεύσεως\\
14.18 & Εξοπλισμός τηλεπικοινωνιών εκτός εκμεταλλεύσεως\\
14.19 & Λοιπός εξοπλισμός εκτός εκμεταλλεύσεως\\
14.90 & Έπιπλα και λοιπός εξοπλισμός στον ΟΔΔΥ για εκποίηση (Γνωμ. 253/2243/1995)\\
14.99 & Αποσβεσμένα έπιπλα και αποσβεσμένος λοιπός εξοπλισμός\\
14.99.00 & Αποσβεσμένα έπιπλα\\
14.99.01 & Αποσβεσμένα σκεύη\\
14.99.02 & Αποσβεσμένες μηχανές γραφείων\\
14.99.03 & Αποσβεσμένοι ηλεκτρονικοί υπολογιστές και αποσβεσμένα ηλεκτρονικά συγκροτήματα\\
14.99.04 & Αποσβεσμένα μέσα αποθηκεύσεως και μεταφοράς\\
14.99.05 & Αποσβεσμένα επιστημονικά όργανα\\
14.99.06 & Αποσβεσμένα ζώα για πάγια εκμετάλλευση\\
14.99.08 & Αποσβεσμένος εξοπλισμός τηλεπικοινωνιών\\
14.99.09 & Αποσβεσμένος λοιπός εξοπλισμός\\
14.99.10 & Αποσβεσμένα έπιπλα εκτός εκμεταλλεύσεως\\
14.99.11 & Αποσβεσμένα σκεύη εκτός εκμεταλλεύσεως\\
14.99.12 & Αποσβεσμένες μηχανές γραφείων εκτός εκμεταλλεύσεως\\
14.99.13 & Αποσβεσμένοι ηλεκτρονικοί υπολογιστές και αποσβεσμένα ηλεκτρονικά συγκροτήματα εκτός εκμεταλλεύσεως\\
14.99.14 & Αποσβεσμένα μέσα αποθηκεύσεως και μεταφοράς εκτός εκμεταλλεύσεως\\
14.99.15 & Αποσβεσμένα επιστημονικά όργανα εκτός εκμεταλλεύσεως\\
14.99.16 & Αποσβεσμένα ζώα για πάγια εκμετάλλευση εκτός εκμεταλλεύσεως\\
14.99.18 & Αποσβεσμένος εξοπλισμός τηλεπικοινωνιών εκτός εκμεταλλεύσεως\\
14.99.19 & Αποσβεσμένος λοιπός εξοπλισμός εκτός εκμεταλλεύσεως\\
15 & Ακινητοποιήσεις υπό εκτέλεση και προκαταβολές κτήσεως πάγιων στοιχείων\\
15.01 & Κτίρια - Εγκαταστάσεις κτιρίων - Τεχνικά έργα υπό εκτέλεση\\
15.02 & Μηχανήματα - Τεχνικές εγκαταστάσεις - Λοιπός μηχανολογικός εξοπλισμός υπό εκτέλεση\\
15.03 & Μεταφορικά μέσα υπό εκτέλεση\\
15.04 & Έπιπλα και λοιπός εξοπλισμός υπό εκτέλεση\\
15.09 & Προκαταβολές κτήσεως πάγιων στοιχείων\\
16 & Ασώματες ακινητοποιήσεις και έξοδα πολυετούς αποσβέσεως\\
16.00 & Υπεραξία επιχειρήσεως (Goodwill)\\
16.01 & Δικαιώματα βιομηχανικής ιδιοκτησίας\\
16.01.00 & Διπλώματα ευρεσιτεχνίας\\
16.01.01 & Άδειες παραγωγής και εκμεταλλεύσεως (Licences)\\
16.01.02 & Σήματα\\
16.01.03 & Μέθοδοι (Know How)\\
16.01.04 & Πρότυπα\\
16.01.05 & Σχέδια\\
16.02 & Δικαιώματα (όπως π.χ. παραχωρήσεις) εκμεταλλεύσεως ορυχείων - μεταλλείων - λατομείων\\
16.03 & Λοιπές παραχωρήσεις\\
16.04 & Δικαιώματα χρήσεως ενσώματων πάγιων στοιχείων\\
16.05 & Λοιπά δικαιώματα\\
16.10 & Έξοδα ιδρύσεως και πρώτης εγκαταστάσεως\\
16.11 & Έξοδα ερευνών ορυχείων - μεταλλείων - λατομείων\\
16.12 & Έξοδα λοιπών ερευνών\\
16.13 & Έξοδα αυξήσεως κεφαλαίου και εκδόσεως ομολογιακών δανείων\\
16.14 & Έξοδα κτήσεως ακινητοποιήσεων\\
16.15 & Συναλλαγματικές διαφορές από πιστώσεις και δάνεια για κτήσεις πάγιων στοιχείων. Υποχρεωτική ανάπτυξη κατά πίστωση ή δάνειο\\
16.16 & Διαφορές εκδόσεως και εξοφλήσεως ομολογιών\\
16.17 & Έξοδα αναδιοργανώσεως\\
16.17.00 & Λογισμικά προγράμματα Η/Υ (Γνωμ. 142/1948/1993)\\
16.18 & Τόκοι δανείων κατασκευαστικής περιόδου\\
16.19 & Λοιπά έξοδα πολυετούς αποσβέσεως\\
16.90 & Έξοδα μετεγκαταστάσεως της επιχειρήσεως (Γνωμ. 260/2258/1995)\\
16.98 & Προκαταβολές κτήσεως ασώματων ακινητοποιήσεων\\
16.99 & Αποσβεσμένες ασώματες ακινητοποιήσεις και αποσβεσμένα έξοδα πολυετούς αποσβέσεως\\
16.99.00 & Αποσβεσμένη υπεραξία επιχειρήσεως\\
16.99.01 & Αποσβεσμένα δικαιώματα βιομηχανικής ιδιοκτησίας\\
16.99.02 & Αποσβεσμένα δικαιώματα (όπως π.χ. παραχωρήσεις) εκμεταλλεύσεως ορυχείων - μεταλλείων - λατομείων\\
16.99.03 & Αποσβεσμένες λοιπές παραχωρήσεις\\
16.99.04 & Αποσβεσμένα δικαιώματα χρήσεως ενσώματων πάγιων στοιχείων\\
16.99.05 & Αποσβεσμένα λοιπά δικαιώματα\\
16.99.10 & Αποσβεσμένα έξοδα ιδρύσεως και πρώτης εγκαταστάσεως\\
16.99.11 & Αποσβεσμένα έξοδα ερευνών ορυχείων - μεταλλείων - λατομείων\\
16.99.12 & Αποσβεσμένα έξοδα λοιπών ερευνών\\
16.99.13 & Αποσβεσμένα έξοδα αυξήσεως κεφαλαίου και εκδόσεως ομολογιακών δανείων\\
16.99.14 & Αποσβεσμένα έξοδα κτήσεως ακινητοποιήσεων\\
16.99.16 & Αποσβεσμένες διαφορές εκδόσεως και εξοφλήσεως ομολογιών\\
16.99.17 & Αποσβεσμένα έξοδα αναδιοργανώσεως\\
16.99.18 & Αποσβεσμένοι τόκοι δανείων κατασκευαστικής περιόδου\\
16.99.19 & Αποσβεσμένα λοιπά έξοδα πολυετούς αποσβέσεως\\
16.99.90 & Αποσβεσμένα έξοδα μετεγκατεστάσεως της επιχειρήσεως (Γνωμ.260/2258/1995)\\
18 & Συμμετοχές και λοιπές μακροπρόθεσμες απαιτήσεις\\
18.00 & Συμμετοχές σε συνδεμένες επιχειρήσεις\\
18.00.00 & Μετοχές εισαγμένες στο Χρηματιστήριο εταιριών εσωτερικού\\
18.00.01 & Μετοχές μη εισαγμένες στο Χρηματιστήριο εταιριών εσωτερικού\\
18.00.02 & Ανεξόφλητες μετοχές εισαγμένες στο Χρηματιστήριο εταιριών εσωτερικού\\
18.00.03 & Ανεξόφλητες μετοχές μη εισαγμένες στο Χρηματιστήριο εταιριών εσωτερικού\\
18.00.04 & Μετοχές εισαγμένες στο Χρηματιστήριο εταιριών εξωτερικού\\
18.00.05 & Μετοχές μη εισαγμένες στο Χρηματιστήριο εταιριών εξωτερικού\\
18.00.06 & Ανεξόφλητες μετοχές εισαγμένες στο Χρηματιστήριο εταιριών εξωτερικού\\
18.00.07 & Ανεξόφλητες μετοχές μη εισαγμένες στο Χρηματιστήριο εταιριών εξωτερικού\\
18.00.08 & Συμμετοχές σε λοιπές (πλην Α.Ε.) επιχειρήσεις εσωτερικού\\
18.00.09 & Συμμετοχές σε λοιπές (πλην Α.Ε.) επιχειρήσεις εξωτερικού\\
18.00.10 & Προεγγραφές σε υπό έκδοση μετοχές εταιριών εσωτερικού\\
18.00.11 & Προεγγραφές σε υπό έκδοση μετοχές εταιριών εξωτερικού\\
18.00.12 & Μετοχές σε τρίτους για εγγύηση\\
18.00.19 & Προβλέψεις για υποτιμήσεις συμμετοχών σε λοιπές (πλην Α.Ε.) επιχειρήσεις (αντίθετος λογ/σμός των 18.00.08-09.Ο Λογαριασμός καταργήθηκε με το άρθρο 3 παρ. 1 Π.Δ. 367/1994)\\
18.00.99 & Προβλέψεις για υποτιμήσεις συμμετοχών σε συνδεδεμένες  επιχειρήσεις (Π.Δ. 367/1994, άρθρο 3 παρ. 1)\\
18.01 & Συμμετοχές σε λοιπές επιχειρήσεις\\
18.01.00 & Μετοχές εισαγμένες στο Χρηματιστήριο εταιριών εσωτερικού\\
18.01.01 & Μετοχές μη εισαγμένες στο Χρηματιστήριο εταιριών  εσωτερικού\\
18.01.02 & Ανεξόφλητες μετοχές εισαγμένες στο Χρηματιστήριο εταιριών εσωτερικού\\
18.01.03 & Ανεξόφλητες μετοχές μη εισαγμένες στο Χρηματιστήριο εταιριών εσωτερικού\\
18.01.04 & Μετοχές εισαγμένες στο Χρηματιστήριο εταιριών εξωτερικού\\
18.01.05 & Μετοχές μη εισαγμένες στο Χρηματιστήριο εταιριών εξωτερικού\\
18.01.06 & Ανεξόφλητες μετοχές εισαγμένες στο Χρηματιστήριο εταιριών εξωτερικού\\
18.01.07 & Ανεξόφλητες μετοχές μη εισαγμένες στο Χρηματιστήριο εταιριών εξωτερικού\\
18.01.08 & Συμμετοχές σε λοιπές (πλην Α.Ε.) επιχειρήσεις εσωτερικού\\
18.01.09 & Συμμετοχές σε λοιπές (πλην Α.Ε.) επιχειρήσεις εξωτερικού\\
18.01.10 & Προεγγραφές σε υπό έκδοση μετοχές εταιριών εσωτερικού\\
18.01.11 & Προεγγραφές σε υπό έκδοση μετοχές εταιριών εξωτερικού\\
18.01.12 & Μετοχές σε τρίτους για εγγύηση\\
18.01.19 & Προβλέψεις για υποτιμήσεις συμμετοχών σε λοιπές (πλην Α.Ε.) επιχειρήσεις (αντίθετος λογ. των 18.01.08-09) (Καταργήθηκε με το άρθρο 3 παρ.1 ΠΔ 367/94)\\
18.01.99 & Προβλέψεις για υποτιμήσεις συμμετοχών σε λοιπές επιχειρήσεις (ΠΔ 367/94 άρθρο 3 παρ.1)\\
18.02 & Μακροπρόθεσμες απαιτήσεις κατά συνδεμένων επιχειρήσεων\\
18.03 & Μακροπρόθεσμες απαιτήσεις κατά συνδεμένων επιχειρήσεων σε Ξ.Ν.\\
18.04 & Μακροπρόθεσμες απαιτήσεις κατά λοιπών συμμετοχικού ενδιαφέροντος επιχειρήσεων\\
18.05 & Μακροπρόθεσμες απαιτήσεις κατά λοιπών συμμετοχικού ενδιαφέροντος επιχειρήσεων σε Ξ.Ν.\\
18.06 & Μακροπρόθεσμες απαιτήσεις κατά εταίρων\\
18.07 & Γραμμάτια Εισπρακτέα μακροπρόθεσμα\\
18.08 & Γραμμάτια Εισπρακτέα μακροπρόθεσμα σε Ξ.Ν.\\
18.09 & Μη δουλευμένοι τόκοι γραμματίων εισπρακτέων μακροπρόθεσμων\\
18.10 & Μη δουλευμένοι τόκοι γραμματίων εισπρακτέων μακροπρόθεσμων σε Ξ.Ν.\\
18.11 & Δοσμένες εγγυήσεις\\
18.12 & Οφειλόμενο κεφάλαιο\\
18.13 & Λοιπές μακροπρόθεσμες απαιτήσεις\\
18.14 & Λοιπές μακροπρόθεσμες απαιτήσεις σε Ξ.Ν.\\
18.15 & Τίτλοι με χαρακτήρα ακινητοποιήσεων\\
18.16 & Τίτλοι με χαρακτήρα ακινητοποιήσεων σε Ξ.Ν.\\
19 & Πάγιο ενεργητικό υποκαταστημάτων ή άλλων κέντρων (Όμιλος λ/σμών προαιρετικής χρήσεως)\\
190 & ΕΔΑΦΙΚΕΣ ΕΚΤΑΣΕΙΣ (Ανάπτυξη αντίστοιχη του λ/σμού 10)\\
191 & ΚΤΙΡΙΑ - ΕΓΚΑΤΑΣΤΑΣΕΙΣ ΚΤΙΡΙΩΝ - ΤΕΧΝΙΚΑ ΕΡΓΑ (Ανάπτυξη αντίστοιχη του λ/σμού 11)\\
192 & ΜΗΧΑΝΗΜΑΤΑ - ΤΕΧΝΙΚΕΣ ΕΓΚΑΤΑΣΤΑΣΕΙΣ - ΛΟΙΠΟΣ ΜΗΧΑΝΟΛΟΓΙΚΟΣ ΕΞΟΠΛΙΣΜΟΣ (Ανάπτυξη αντίστοιχη του λ/σμού 12)\\
193 & ΜΕΤΑΦΟΡΙΚΑ ΜΕΣΑ (Ανάπτυξη αντίστοιχη του λ/σμού 13)\\
194 & ΕΠΙΠΛΑ ΚΑΙ ΛΟΙΠΟΣ ΕΞΟΠΛΙΣΜΟΣ (Ανάπτυξη αντίστοιχη του λ/σμού 14)\\
195 & ΑΚΙΝΗΤΟΠΟΙΗΣΕΙΣ ΥΠΟ ΕΚΤΕΛΕΣΗ ΚΑΙ ΠΡΟΚΑΤΑΒΟΛΕΣ ΚΤΗΣΕΩΣ ΠΑΓΙΩΝ ΣΤΟΙΧΕΙΩΝ (Ανάπτυξη αντίστοιχη του λ/σμού 15)\\
196 & ΑΣΩΜΑΤΕΣ ΑΚΙΝΗΤΟΠΟΙΗΣΕΙΣ ΚΑΙ ΕΞΟΔΑ ΠΟΛΥΕΤΟΥΣ ΑΠΟΣΒΕΣΕΩΣ (Ανάπτυξη αντίστοιχη του λ/σμού 16)\\
198 & ΣΥΜΜΕΤΟΧΕΣ ΚΑΙ ΛΟΙΠΕΣ ΜΑΚΡΟΠΡΟΘΕΣΜΕΣ ΑΠΑΙΤΗΣΕΙΣ (Ανάπτυξη αντίστοιχη του λ/σμού 18)\\

\end{tabularx}

\section{Περιουσιακά στοιχεία που περιλαμβάνονται}
\begin{enumerate}

\item Στην πρώτη ομάδα περιλαμβάνεται το σύνολο των αγαθών, αξιών και δικαιωμάτων, που προορίζονται να παραμείνουν μακροχρόνια, με την ίδια περίπου μορφή, στην οικονομική μονάδα, καθώς και τα έξοδα πολυετούς αποσβέσεως και οι μακροπρόθεσμες απαιτήσεις.

\item Στο πάγιο ενεργητικό περιλαμβάνονται οι εξής μερικότερες κατηγορίες περιουσιακών στοιχείων:
\begin{enumerate}
\item Ενσώματα πάγια στοιχεία (λογαριασμοί 10-15): Είναι τα υλικά αγαθά που αποκτάει η οικονομική μονάδα με σκοπό να τα χρησιμοποιεί ως μέσα δράσεώς της κατά τη διάρκεια της ωφέλιμης ζωής τους, η οποία είναι οπωσδήποτε μεγαλύτερη από ένα έτος.

\item Ασώματες ακινητοποιήσεις ή άυλα πάγια στοιχεία (λογαριασμοί 16.00-16.09): Είναι τα ασώματα οικονομικά αγαθά που αποκτούνται από την οικονομική μονάδα με σκοπό να χρησιμοποιούνται παραγωγικά για χρονικό διάστημα οπωσδήποτε μεγαλύτερο από ένα έτος.

\item Έξοδα πολυετούς αποσβέσεως (λογαριασμοί 16.10-16.19): Είναι τα έξοδα που αποσβένονται τμηματικά και πραγματοποιούνται για την ίδρυση και οργάνωση της οικονομικής μονάδας, για την απόκτηση διαρκών μέσων εκμεταλλεύσεως και για την επέκταση και αναδιοργάνωσή της.

\item Συμμετοχές και μακροπρόθεσμες απαιτήσεις (λογαριασμός 18): Είναι οι συμμετοχές σε άλλες οικονομικές μονάδες, οποιασδήποτε νομικής μορφής - Α.Ε., Ε.Π.Ε., Ε.Ε., Ο.Ε. και άλλες -, οι οποίες εξασφαλίζουν την άσκηση επιρροής πάνω σ' αυτές και αποκτούνται με σκοπό διαρκούς κατοχής τους, και οι κατά τρίτων απαιτήσεις της οικονομικής μονάδας, για τις οποίες η προθεσμία εξοφλήσεως λήγει μετά από το τέλος της επόμενης χρήσεως.
\end{enumerate}
\end{enumerate}

\section{Επέκταση, προσθήκη, βελτίωση, συντήρηση και επισκευή ενσώματων πάγιων περιουσιακών στοιχείων}
\begin{enumerate}
\item Επέκταση ή προσθήκη κτιρίου, κτιριακής εγκαταστάσεως και τεχνικού έργου είναι οποιαδήποτε μόνιμη αύξηση του όγκου, του μεγέθους ή της ωφελιμότητάς του, που γίνεται με τη χρησιμοποίηση κατά κανόνα δομικών υλικών.

\item Επέκταση ή προσθήκη μηχανήματος, τεχνικής εγκαταστάσεως και μηχανολογικού εξοπλισμού είναι κάθε προσθήκη ή εργασία που γίνεται σ' αυτά και αυξάνει το μέγεθος και κατά κανόνα την παραγωγική τους δυναμικότητα.

\item Βελτίωση ενσώματου πάγιου περιουσιακού στοιχείου είναι κάθε μεταβολή που γίνεται σ' αυτό μετά από τεχνολογική επέμβαση και που έχει ως αποτέλεσμα, είτε την αύξηση του χρόνου της ωφέλιμης ζωής του, είτε την αύξηση της παραγωγικότητάς του, είτε τη μείωση του κόστους λειτουργίας του ή τη βελτίωση των συνθηκών χρησιμοποιήσεώς του.

\item Συντήρηση ενσώματου πάγιου περιουσιακού στοιχείου είναι η τεχνολογική επέμβαση που γίνεται σ' αυτό με σκοπό να διατηρείται στην αρχική του παραγωγική ικανότητα για όσο το δυνατό μεγαλύτερο χρονικό διάστημα.

\item Επισκευή ενσώματου πάγιου περιουσιακού στοιχείου είναι η αντικατάσταση ή επιδιόρθωση μερών αυτού, που έχουν καταστραφεί ή υποστεί βλάβη, με σκοπό την επαναφορά της παραγωγικής του ικανότητας ή των συνθηκών λειτουργίας του στο επίπεδο που ήταν πριν από την καταστροφή ή τη βλάβη.

\item Το κόστος των επεκτάσεων, προσθηκών και βελτιώσεων προσαυξάνει την αξία κτήσεως των πάγιων περιουσιακών στοιχείων και καταχωρείται στους σχετικούς λογαριασμούς των στοιχείων αυτών.

\item Τα έξοδα συντηρήσεως και επισκευής των πάγιων περιουσιακών στοιχείων είναι κόστος τρέχουσας μορφής και καταχωρούνται στους οικείους λογαριασμούς εξόδων κατ' είδος της ομάδας 6.
\end{enumerate}

\section{Οι αποσβέσεις των πάγιων περιουσιακών στοιχείων}
\subsection{Εννοιολογικοί προσδιορισμοί}
\begin{enumerate}
\item Απόσβεση είναι η χρονική κατανομή της αποσβεστέας αξίας του πάγιου περιουσιακού στοιχείου, που υπολογίζεται με βάση την ωφέλιμη διάρκεια ζωής του και, συνακόλουθα, η λογιστική απεικόνιση και ο καταλογισμός της σε καθεμία χρήση. Οι αποσβέσεις κάθε χρήσεως βαρύνουν το λειτουργικό κόστος, ή απευθείας τα αποτελέσματα χρήσεως όταν πρόκειται για αποσβέσεις που δεν ενσωματώνονται στο λειτουργικό κόστος. Το ποσό της ετήσιας αποσβέσεως αντιπροσωπεύει τη μείωση της αξίας του πάγιου στοιχείου, που επέρχεται λόγω της χρήσεώς του, της παρόδου του χρόνου και της οικονομικής του απαξιώσεως.

\item Αποσβέσιμο πάγιο περιουσιακό στοιχείο είναι το ενσώματο ή άυλο πάγιο στοιχείο που αποκτάται από την οικονομική μονάδα για διαρκή παραγωγική χρήση και έχει ωφέλιμη διάρκεια ζωής περιορισμένη, πάντως μεγαλύτερη από ένα έτος.

\item Ωφέλιμη διάρκεια ζωής είναι, είτε η χρονική περίοδος κατά την οποία υπολογίζεται ότι το αποσβέσιμο πάγιο στοιχείο θα χρησιμοποιείται παραγωγικά από την οικονομική μονάδα, είτε η ολική ποσότητα παραγωγής ή το ολικό έργο το οποίο αναμένεται να επιτύχει η οικονομική μονάδα από το πάγιο αυτό στοιχείο (π.χ.  ωφέλιμη διάρκεια ζωής μηχανήματος μετρημένη σε παραγωγικές ώρες).

\item Αποσβεστέα αξία ενός αποσβέσιμου πάγιου περιουσιακού στοιχείου είναι το ιστορικό κόστος του ή άλλο ποσό που αντικατέστησε νομότυπα το ιστορικό κόστος (π.χ. αξία αναπροσαρμογής που επιβλήθηκε από το νόμο ή αξία που έχει προκύψει από εκτίμηση λόγω συγχωνεύσεως), μειωμένο κατά την υπολειμματική αξία του, εφόσον αυτή είναι αξιόλογη. Αν η υπολειμματική αξία δεν είναι αξιόλογη, δε λαμβάνεται υπόψη για τον προσδιορισμό της αποσβεστέας αξίας. Η κρίση για την αξιολόγηση αυτή αφήνεται στην οικονομική μονάδα.

\item Υπολειμματική αξία ενός αποσβέσιμου πάγιου περιουσιακού στοιχείου είναι η καθαρή ρευστοποιήσιμη αξία του, που υπολογίζεται να πραγματοποιηθεί κατά το τέλος της ωφέλιμης διάρκειας της ζωής του.
\end{enumerate}

\subsection{Γενικές αρχές λογισμού των αποσβέσεων}
\begin{enumerate}
\item Η αποσβεστέα αξία των πάγιων περιουσιακών στοιχείων κατανέμεται σε κάθε λογιστική χρήση, κατά τη διάρκεια της ωφέλιμης ζωής τους, με ομοιόμορφο τρόπο.  Για τον υπολογισμό των αποσβέσεων εφαρμόζεται η μέθοδος της σταθερής αποσβέσεως.

\item Οι αποσβέσεις υπολογίζονται με βάση τους προβλεπόμενους από την κείμενη νομοθεσία συντελεστές ετήσιας τακτικής αποσβέσεως για κάθε κατηγορία αποσβέσιμων πάγιων στοιχείων. Οι συντελεστές αυτοί, κατά τεκμήριο, καλύπτουν τη φυσική φθορά (από τη χρήση και από την πάροδο του χρόνου) καθώς και την οικονομική απαξίωση των οικείων στοιχείων.

\item Δεν επιτρέπεται ο λογισμός αποσβέσεων με συντελεστές μεγαλύτερους από εκείνους που προβλέπονται από την κείμενη νομοθεσία. Επίσης, δεν επιτρέπεται ο λογισμός αποσβέσεων με συντελεστές μικρότερους από τους ελάχιστους συντελεστές, που η κείμενη νομοθεσία προβλέπει ως υποχρεωτικούς.

\item Η διενέργεια αποσβέσεων για κάθε έτος με τους θεσπισμένους ελάχιστους συντελεστές είναι υποχρεωτική, ανεξάρτητα από την ύπαρξη ή μη κερδών. Η διενέργεια αποσβέσεων διακόπτεται από τη στιγμή που το σύνολο των διενεργημένων αποσβέσεων για κάθε αποσβέσιμο στοιχείο γίνει ίσο με την αποσβεστέα αξία αυτού του στοιχείου (μείον μιας μονάδας), ανεξάρτητα από το αν εξακολουθεί η παραγωγική χρησιμοποίησή του.

\item Ο υπολογισμός των αποσβέσεων γίνεται από τη στιγμή που το πάγιο στοιχείο αρχίζει να χρησιμοποιείται ή να λειτουργεί. Αν ο χρόνος αυτός δε συμπίπτει με την έναρξη της λογιστικής χρήσεως, η απόσβεση υπολογίζεται σε τόσα δωδέκατα της ετήσιας αποσβέσεως, όσοι είναι οι μήνες μέχρι το τέλος της χρήσεως, στους οποίους περιλαμβάνεται και ο μήνας μέσα στον οποίο το πάγιο στοιχείο αρχίζει να χρησιμοποιείται ή να λειτουργεί.

\item Οι αποσβέσεις των πάγιων στοιχείων τα οποία παραμένουν σε αδράνεια για χρονικό διάστημα που διαρκεί συνέχεια πέρα από έξι μήνες υπολογίζονται, για το διάστημα αυτό, με μειωμένους συντελεστές. Το ποσοστό μειώσεως καθορίζεται, κατά κλάδους οικονομικών μονάδων ή κατηγορίες στοιχείων, συγχρόνως με τον καθορισμό των ετήσιων συντελεστών τακτικών αποσβέσεων.
\end{enumerate}

\subsection{Άλλα γενικά θέματα σχετικά με τις αποσβέσεις}
\begin{enumerate}

\item Οι αποσβέσεις που διενεργούνται για κάθε λογιστική χρήση καταλογίζονται σ' αυτή, με χρέωση των λογαριασμών 66 «αποσβέσεις πάγιων στοιχείων ενσωματωμένες στο λειτουργικό κόστος» και 85 «αποσβέσεις πάγιων στοιχείων μη ενσωματωμένες στο λειτουργικό κόστος» και με πίστωση των από το Σχέδιο Λογαριασμών προβλεπόμενων αντίθετων λογαριασμών 10.99, 11.99, 12.99, 13.99, 14.99 και 16.99.

\item Από τις αποσβέσεις που διενεργούνται σε κάθε χρήση, οι τακτικές, που θεωρείται ότι αφορούν το λειτουργικό κόστος (δηλαδή τη λειτουργία παραγωγής, τη διοικητική λειτουργία, τη λειτουργία ερευνών και αναπτύξεως και τη λειτουργία διαθέσεως), όπως ειδικότερα ορίζεται στην παρ. 5.213 του πέμπτου μέρους, καταχωρούνται στη χρέωση του λογαριασμού 66 και, τελικά, μεταφέρονται στο λογαριασμό 80.00 της Γενικής Εκμεταλλεύσεως.

Στην περίπτωση που η οικονομική μονάδα, προκειμένου να προσδιορίσει το κόστος και τα αναλυτικά αποτελέσματα, δεν κάνει χρήση των αρχών του πέμπτου μέρους, οι τακτικές αποσβέσεις του λογαριασμού 66 κατανέμονται εξωλογιστικά στις επιμέρους λειτουργίες της οικονομικής μονάδας (παραγωγής, διοικήσεως, ερευνών - αναπτύξεως και διαθέσεως).

\item Οι προβλεπόμενες από τη φορολογική νομοθεσία, με τη μορφή αναπτυξιακών κινήτρων, πρόσθετες (επιταχυνόμενες) αποσβέσεις καταχωρούνται στη χρέωση του λογαριασμού 85 και, τελικά, μεταφέρονται στα αποτελέσματα χρήσεως (λογαριασμός 86.03).
\end{enumerate}

\section{Μητρώο πάγιων περιουσιακών στοιχείων}
Για τη διαχειριστική παρακολούθηση κάθε πάγιου στοιχείου και για τη λογιστική παρακολούθηση της αξίας κτήσεως και των αποσβέσεών του και γενικότερα της τύχης του, τηρείται υποχρεωτικά μητρώο πάγιων στοιχείων, το οποίο αποτελεί την τελευταία ανάλυση των λογαριασμών των πάγιων περιουσιακών στοιχείων (λογαριασμοί τρίτου ή τέταρτου κλπ. βαθμού).

Από το μητρώο πάγιων στοιχείων, οι λεπτομέρειες και ο τρόπος τηρήσεως του οποίου αφήνονται στην κρίση της οικονομικής μονάδας, πρέπει να προκύπτουν τουλάχιστο τα παρακάτω στοιχεία:

\begin{enumerate}

\item Τα στοιχεία που εξατομικεύουν το είδος του παγίου (ονοματολογία και διακριτικά στοιχεία).

\item Τα στοιχεία της λογιστικής του εντάξεως (τίτλοι και κωδικοί αριθμοί του πρωτοβάθμιου και του λογαριασμού της τελευταίας βαθμίδας).

\item Η αιτιολογία και τα σχετικά στοιχεία κτήσεως, η αρχική αξία κτήσεως και οι μεταβολές αυτής (προσθήκες, βελτιώσεις, μειώσεις).

\item Ο τόπος εγκαταστάσεως ή ο τρίτος στις εγκαταστάσεις του οποίου τυχόν βρίσκεται.

\item Η ημερομηνία κατά την οποία άρχισε η χρησιμοποίηση ή λειτουργία του, καθώς και η ημερομηνία που τυχόν τέθηκε σε αδράνεια.

\item Η τυχόν κτήση του με ευεργετική φορολογική διάταξη.

\item Η τυχόν ύπαρξη βαρών πάνω σ' αυτό (π.χ. είδος βάρους, αιτία, ποσό).

\item Ο κωδικός αριθμός της τελευταίας βαθμίδας του λογαριασμού αποσβέσεων.

\item Οι λογισμένες αποσβέσεις (συντελεστής και ποσά) και τα στοιχεία της λογιστικής τους εγγραφής (α/α παραστατικού, ημερομηνία), καθώς και οι αντιλογισμένες αποσβέσεις, π.χ. σε περίπτωση πωλήσεως ή καταστροφής.

\item Τα στοιχεία και η αιτία του τερματισμού της παραγωγικής του ζωής (π.χ.  εκποίηση, διάλυση ή καταστροφή).

\end{enumerate}

Με σκοπό να αντιμετωπιστούν δυσχέρειες που ενδεχόμενα θα ανακύψουν κατά την υποχρεωτική τήρηση του μητρώου πάγιων στοιχείων σύμφωνα με τα παραπάνω, παρέχεται η δυνατότητα της τηρήσεως αυτού κατά ομάδες ομοειδών πάγιων στοιχείων (π.χ. πάγια στοιχεία του λογαριασμού 14 που κτήθηκαν κατά τη διάρκεια του αυτού μήνα παρακολουθούνται σε μία ατομική μερίδα) με την προϋπόθεση ότι το συγκεκριμένο πάγιο στοιχείο, όταν κρίνεται αναγκαίο (π.χ. κατά την απογραφή ή την πώληση), θα είναι δυνατό να εξατομικεύεται.

\section{10.ΕΔΑΦΙΚΕΣ ΕΚΤΑΣΕΙΣ}

Εδαφικές εκτάσεις είναι τα οικόπεδα, γήπεδα, αγροτεμάχια, δάση, ορυχεία, μεταλλεία, λατομεία, οι φυτείες και γενικά οποιαδήποτε έκταση γης της οποίας η κυριότητα ανήκει στην οικονομική μονάδα.

Οι εδαφικές εκτάσεις διακρίνονται σ' εκείνες που έχουν απεριόριστη διάρκεια ωφέλιμης ζωής, όπως π.χ. είναι τα οικόπεδα, γήπεδα ή τα αγροτεμάχια, και σ' αυτές που η διάρκεια της ωφέλιμης ζωής τους είναι περιορισμένη και για το λόγο αυτό η αξία τους είναι αποσβεστέα. Στην τελευταία αυτή κατηγορία ανήκουν π.χ.  τα ορυχεία, μεταλλεία και λατομεία.

Στο λογαριασμό 10.00 «γήπεδα - οικόπεδα» παρακολουθούνται οι εκτάσεις γης πάνω στις οποίες έχουν κατασκευαστεί και οργανωθεί τα εργοστάσια ή εργοτάξια της οικονομικής μονάδας ή έχουν ανεγερθεί λοιπά κτίρια και εγκαταστάσεις αυτής (π.χ. για γραφεία, καταστήματα ή κατοικίες), καθώς και εκείνες που προορίζονται για την εξυπηρέτηση παρόμοιων σκοπών.

Στους λογαριασμούς 10.01 «ορυχεία», 10.02 «μεταλλεία» και 10.03 «λατομεία» παρακολουθούνται οι ιδιόκτητες εκτάσεις γης, από τις οποίες, με κατάλληλα τεχνικά μέσα, αντλείται ο υπόγειος ή επιφανειακός φυσικός πλούτος τους (π.χ.  ορυκτά, μεταλλεύματα ή λατομικά προϊόντα).

Οι λογαριασμοί 10.01 και 10.02 δημιουργούνται ως εξής:

\begin{enumerate}

\item Με μεταφορά από το λογαριασμό 10.00 «γήπεδα - οικόπεδα» ή το 10.10 «γήπεδα - οικόπεδα εκτός εκμεταλλεύσεως» της αξίας κτήσεως των ιδιόκτητων γηπέδων τα οποία χαρακτηρίζονται από αρμόδια Αρχή ως ορυχεία ή μεταλλεία έπειτα από χορήγηση σχετικής άδεια εκμεταλλεύσεώς τους. Η μεταφορά γίνεται τη στιγμή που αρχίζει η εκμετάλλευσή τους, ενώ πριν από αυτή η παρακολούθηση των χαρακτηρισμένων ως ορυχείων ή μεταλλείων ιδιόκτητων γηπέδων γίνεται στους λογαριασμούς 10.11 ή 10.12.

\item Με καταχώριση της αξίας κτήσεως των αγορασμένων ή με άλλο νόμιμο τρόπο αποκτημένων (π.χ. με εισφορά σε είδος) ορυχείων ή μεταλλείων, δηλαδή των γηπέδων τα οποία είναι ήδη χαρακτηρισμένα, με σχετική άδεια αρμόδιας Αρχής, ως ορυχεία ή μεταλλεία και τα οποία, μαζί με την άδεια αυτή, αποκτούνται κατά πλήρη κυριότητα.

\item Με μεταφορά από το λογαριασμό 10.11 «ορυχεία εκτός εκμεταλλεύσεως» ή το 10.12. «μεταλλεία εκτός εκμεταλλεύσεως» των ορυχείων ή μεταλλείων που ήταν εκτός εκμεταλλεύσεως, τη στιγμή που αρχίζει η εκμετάλλευσή τους.

\end{enumerate}

Όσα καθορίζονται παραπάνω για τα ορυχεία και μεταλλεία ισχύουν ανάλογα και για τις πετρελαιοπηγές, καθώς και για άλλες παρόμοιες περιπτώσεις (π.χ. πηγές φυσικών αερίων ή ιαματικών νερών).

Στο λογαριασμό 16.02 «δικαιώματα (π.χ. παραχωρήσεις) εκμεταλλεύσεως ορυχείων - μεταλλείων - λατομείων» παρακολουθείται η αξία κτήσεως του δικαιώματος εκμεταλλεύσεως του ορυχείου ή μεταλλείου, δηλαδή του εμπράγματου δικαιώματος της «μεταλλειοκτησίας», το οποίο αποσβένεται σύμφωνα με όσα ορίζονται από τη σχετική νομοθεσία. Σε υπολογαριασμό του 16.02 παρακολουθούνται και τα σχετικά έξοδα τα οποία πραγματοποιούνται για τη χορήγηση, από αρμόδια Αρχή, του δικαιώματος εκμεταλλεύσεως ορυχείου ή μεταλλείου που βρίσκεται σε ιδιόκτητο έδαφος, εφόσον τα έξοδα αυτά είναι αξιόλογα. Η κρίση για την αξιολόγηση αυτή αφήνεται στην οικονομική μονάδα.

Τα έξοδα που πραγματοποιούνται για έρευνες ανευρέσεως ή αξιοποιήσεως ορυχείου ή μεταλλείου παρακολουθούνται στο λογαριασμό 16.11 «έξοδα ερευνών ορυχείων - μεταλλείων - λατομείων». Τα έξοδα αυτά αποσβένονται σύμφωνα με όσα ορίζονται από τη σχετική νομοθεσία.

Στο λογαριασμό 10.03 «λατομεία» παρακολουθούνται οι ιδιόκτητες εκτάσεις γης, από τις οποίες, με κατάλληλα τεχνικά μέσα, γίνεται εξόρυξη λατομικών προϊόντων.

Λατομικά προϊόντα είναι τα ορυκτά εκείνα τα οποία δε χαρακτηρίζονται ως μεταλλεύματα από τη νομοθεσία περί μεταλλείων, όπως είναι ιδίως τα διάφορα πετρώματα, τα μάρμαρα, οι κοινοί λίθοι, τα κονιάματα και τα χώματα.

Όσα αναφέρονται στην παραπάνω περίπτωση 4 σχετικά με τα ορυχεία και μεταλλεία εφαρμόζονται ανάλογα και για τα λατομεία.

Στους λογαριασμούς 10.04 «αγροί», 10.05 «φυτείες» και 10.06 «δάση» παρακολουθούνται οι καλλιεργήσιμες καθώς και οι με οποιοδήποτε άλλο φυσικό τρόπο εκμεταλλεύσιμες εκτάσεις γης.

Στους λογαριασμούς 10.10 έως και 10.16 παρακολουθούνται οι εδαφικές εκτάσεις οι οποίες δε χρησιμοποιούνται παραγωγικά για τις ανάγκες της βασικής επαγγελματικής δραστηριότητας της οικονομικής μονάδας, ούτε και για τις παρεπόμενες ασχολίες αυτής.

Οι εδαφικές εκτάσεις παρακολουθούνται στους οικείους υπολογαριασμούς του 10 με καταχώριση σ' αυτούς της αξίας κτήσεώς τους (αγοράς, εκτιμήσεως όταν πρόκειται για συγχώνευση ή εισφορά σε είδος) ή της αξίας η οποία προκύπτει έπειτα από νόμιμη αναπροσαρμογή της αξίας κτήσεως.

Τα έξοδα κτήσεως των εδαφικών εκτάσεων (π.χ. φόροι μεταβιβάσεως, συμβολαιογραφικά και μεσιτικά) καταχωρούνται και παρακολουθούνται στο λογαριασμό 16.14 «έξοδα κτήσεως ακινητοποιήσεων».

Τα έξοδα διαμορφώσεως των γηπέδων και άλλων εδαφικών εκτάσεων, τα οποία προσδίνουν αξία σ' αυτές επειδή τα σχετικά έργα (π.χ. εκβραχισμοί ή ισοπεδώσεις) δε φθείρονται, φέρονται σε αύξηση της αξίας κτήσεώς τους. Αν τα έργα αυτά φθείρονται και συνεπώς αποσβένονται, με την προϋπόθεση ότι τα σχετικά έξοδα δεν έχουν περιληφθεί στο κόστος κτιρίων ή τεχνικών έργων σαν κόστος υποδομής της κατασκευής τους, καταχωρούνται και παρακολουθούνται στο λογαριασμό 11.03 «υποκείμενες σε απόσβεση διαμορφώσεις γηπέδων».

Τα γήπεδα-οικόπεδα και άλλες εδαφικές εκτάσεις δε φθείρονται από τη χρήση τους ή την πάροδο του χρόνου και για το λόγο αυτό δεν αποσβένονται. Όταν όμως για τις εδαφικές αυτές εκτάσεις υπάρχει κίνδυνος οικονομικής απαξιώσεως και υποτιμήσεως, για τις ειδικές αυτές περιπτώσεις, σχηματίζεται ειδική πρόβλεψη, η οποία καταχωρείται στο λογαριασμό 44.10 «προβλέψεις απαξιώσεων και υποτιμήσεων πάγιων στοιχείων», με χρέωση του λογαριασμού 83.10.

Σε περίπτωση εκποιήσεως μη οικοδομημένης εδαφικής εκτάσεως (π.χ. γηπέδου - οικοπέδου) ισχύουν τα παρακάτω:

\begin{enumerate}

\item Στην πίστωση του οικείου λογαριασμού της εδαφικής εκτάσεως καταχωρούνται το τίμημα πωλήσεως του πωλητήριου συμβολαίου και η τυχόν σχηματισμένη πρόβλεψη για υποτίμηση της πωλούμενης εκτάσεως (από το λογαριασμό 44.10). Στη χρέωση του ίδιου λογαριασμού φέρονται τα τυχόν έξοδα που δημιουργούνται για την πραγματοποίηση της πωλήσεως.

\item Στη χρέωση του οικείου λογαριασμού της εδαφικής εκτάσεως μεταφέρονται, επίσης, το αναπόσβεστο υπόλοιπο των εξόδων κτήσεως και των τυχόν εξόδων διαμορφώσεώς της. Οι μεταφορές αυτές γίνονται από τους λογαριασμούς 16.14 και 11.03, αντίστοιχα, στους οποίους προηγουμένως μεταφέρονται, από τους λογαριασμούς 16.99.14 και 11.99.03, οι διενεργημένες αποσβέσεις.

\item Το αποτέλεσμα που προκύπτει μετά από τις παραπάνω καταχωρήσεις και μεταφορές μεταφέρεται στο λογαριασμό 81.02.00 «ζημίες από εκποίηση ακινήτων», όταν είναι ζημία, ή στο λογαριασμό 81.03.00 «κέρδη από εκποίηση ακινήτων», όταν είναι κέρδος.

\end{enumerate}

Στην περίπτωση εκποιήσεως οικοδομημένης εδαφικής εκτάσεως ισχύουν όσα αναφέρονται στην περίπτωση 9 της επόμενης παραγράφου 2.2.105.

\section{11.ΚΤΙΡΙΑ - ΕΓΚΑΤΑΣΤΑΣΕΙΣ ΚΤΙΡΙΩΝ - ΤΕΧΝΙΚΑ ΕΡΓΑ}

Κτίρια είναι οι οικοδομικές κατασκευές που γίνονται με τη χρησιμοποίηση δομικών υλικών και προορίζονται για κατοικίες, βιομηχανοστάσια, αποθήκες ή οποιαδήποτε άλλη εκμετάλλευση ή δραστηριότητα της οικονομικής μονάδας.

Εγκαταστάσεις κτιρίων είναι πρόσθετες εγκαταστάσεις, όπως ηλεκτρικές, υδραυλικές, μηχανολογικές, κλιματιστικές, τηλεπικοινωνιακές, αποχετεύσεως, πνευματικής ή μη μεταφοράς, ενδοσυνεννοήσεως και άλλες, οι οποίες είναι συνδεδεμένες με το κτίριο κατά τέτοιο τρόπο, ώστε ο αποχωρισμός τους να μην είναι δυνατό να γίνει εύκολα και χωρίς βλάβη της ουσίας τους ή του κτιρίου. Οι εγκαταστάσεις αυτές παρακολουθούνται στους ίδιους υπολογαριασμούς του 11.00 στους οποίους παρακολουθούνται τα κτίρια στα οποία είναι ενσωματωμένες ή συνδεδεμένες.

Τεχνικά έργα είναι μόνιμες, κατά κανόνα, τεχνικές κατασκευές με τις οποίες τροποποιείται το φυσικό περιβάλλον με σκοπό την εξυπηρέτηση των δραστηριοτήτων της οικονομικής μονάδας (π.χ. δρόμοι, πλατείες, λιμάνια, φράγματα, λίμνες, διώρυγες, περιφράξεις, σήραγγες, γέφυρες, αεροδρόμια ή στάδια).

Στο λογαριασμό 11.01 «τεχνικά έργα εξυπηρετήσεως μεταφορών» παρακολουθούνται όσα από τα έργα αυτά εξυπηρετούν τις μεταφορές της οικονομικής μονάδας. Τα τεχνικά έργα που εξυπηρετούν άλλους σκοπούς παρακολουθούνται στο λογαριασμό 11.02 «λοιπά τεχνικά έργα».

Στο λογαριασμό 11.03 «υποκείμενες σε απόσβεση διαμορφώσεις γηπέδων» παρακολουθούνται οι δαπάνες διαμορφώσεως γηπέδων και άλλων εδαφικών εκτάσεων, όταν συντρέχουν οι εξής δύο βασικές προϋποθέσεις: α) οι δαπάνες αυτές δεν πρέπει να έχουν περιληφθεί στο κόστος κτιρίων ή τεχνικών έργων σαν κόστος υποδομής της κατασκευής τους και β) οι διαμορφώσεις να φθείρονται και για το λόγο αυτό να αποσβένονται.

Στους λογαριασμούς 11.07 «κτίρια - εγκαταστάσεις κτιρίων σε ακίνητα τρίτων», 11.08 «τεχνικά έργα εξυπηρετήσεως μεταφορών σε ακίνητα τρίτων», 11.09 «λοιπά τεχνικά έργα σε ακίνητα τρίτων» και 11.10 «υποκείμενες σε απόσβεση διαμορφώσεις γηπέδων τρίτων» παρακολουθούνται τα κτίρια και τεχνικά έργα που κατασκευάζονται, καθώς και τα έξοδα που γίνονται σε ακίνητα κυριότητας τρίτων, όταν η οικονομική μονάδα έχει δικαίωμα χρήσεως για ορισμένο χρόνο που καθορίζεται συμβατικά, μετά την πάροδο του οποίου τα εν λόγω έργα (π.χ. κτίρια ή διαμορφώσεις) περιέρχονται στον κύριο του ακινήτου χωρίς αντάλλαγμα. Στους λογαριασμούς αυτούς καταχωρούνται: α) το κόστος ανεγέρσεως κτιρίων και τεχνικών έργων, β) το κόστος διαμορφώσεων, βελτιώσεων και προσθηκών πάνω σε κτίρια και τεχνικά έργα και γ) τα έξοδα διαμορφώσεως εδαφικών εκτάσεων.

Τα κτίρια και τα τεχνικά έργα που κατασκευάζονται από την οικονομική μονάδα σε ακίνητα τρίτων, καθώς και τα έξοδα που πραγματοποιούνται γι' αυτά, αποσβένονται ανάλογα με το χρόνο της συμβατικής χρησιμοποιήσεώς τους, με τον όρο ότι ο συντελεστής αποσβέσεως που προσδιορίζεται με βάση το χρόνο χρησιμοποιήσεως δε θα είναι μικρότερος από το συντελεστή που εφαρμόζεται σε ομοειδή ιδιόκτητα πάγια στοιχεία.

Στους λογαριασμούς 11.14 έως και 11.17 καθώς και 11.21 έως και 11.24 παρακολουθούνται τα κτίρια - εγκαταστάσεις κτιρίων και τα τεχνικά έργα τα οποία δε χρησιμοποιούνται παραγωγικά για τις ανάγκες της βασικής επαγγελματικής δραστηριότητας της οικονομικής μονάδας, ούτε και για τις παρεπόμενες ασχολίες της.

Τα κτίρια - εγκαταστάσεις κτιρίων και τα τεχνικά έργα παρακολουθούνται στους οικείους υπολογαριασμούς του 11 έπειτα από καταχώριση σ' αυτούς:

\begin{enumerate}

\item της αξίας κτήσεώς τους (αγοράς, εκτιμήσεως όταν πρόκειται για συγχώνευση ή εισφορά σε είδος) ή της αξίας που προκύπτει μετά από νόμιμη αναπροσαρμογή της αξίας κτήσεως και 

\item του κόστους κατασκευής τους, όταν πρόκειται για ιδιοκατασκευές, το οποίο προκύπτει από το λογαριασμό 15.01 «κτίρια - εγκαταστάσεις κτιρίων - τεχνικά έργα υπό εκτέλεση».

\end{enumerate}

Τα έξοδα κτήσεως των κτιρίων - τεχνικών έργων (π.χ. φόροι μεταβιβάσεως, συμβολαιογραφικά και μεσιτικά) καταχωρούνται και παρακολουθούνται στο λογαριασμό 16.14 «έξοδα κτήσεως ακινητοποιήσεων».

Η αξία κτήσεως των κτιρίων και τεχνικών έργων προσαυξάνεται με την αξία των επεκτάσεων ή προσθηκών και βελτιώσεων που γίνονται κάθε φορά.

Σχετικά με τις αποσβέσεις των κτιρίων - εγκαταστάσεων κτιρίων - τεχνικών έργων, ισχύουν όσα αναφέρονται στην παρ. 2.2.102.

Σχετικά με την οικονομική απαξίωση και υποτίμηση των κτιρίων - εγκαταστάσεων κτιρίων - τεχνικών έργων, ισχύουν αναλόγως όσα αναφέρονται στις περιπτ. 11 και 12 της παρ. 2.2.104.

Σε περίπτωση εκποιήσεως ακινήτου ισχύουν τα ακόλουθα:

\begin{enumerate}

\item Στην πίστωση του οικείου λογαριασμού του κτιρίου ή του τεχνικού έργου καταχωρείται το τίμημα πωλήσεως του πωλητηρίου συμβολαίου και στη χρέωσή του καταχωρούνται τα έξοδα που τυχόν δημιουργούνται για την επίτευξη της πωλήσεως.

\item Στη χρέωση του ίδιου λογαριασμού μεταφέρονται η αξία κτήσεως του αντίστοιχου γηπέδου ή άλλης εδαφικής εκτάσεως και το αναπόσβεστο υπόλοιπο των εξόδων διαμορφώσεως του γηπέδου (δηλαδή το υπόλοιπο του λογαριασμού 11.03 που προκύπτει μετά τη μεταφορά στο λογαριασμό αυτό των αποσβέσεων του λογαριασμού 11.99.03). Στην πίστωση του ίδιου λογαριασμού μεταφέρεται η τυχόν σχηματισμένη πρόβλεψη για υποτίμηση του πωλούμενου γηπέδου (από το λογαριασμό 44.10).

\item Στην πίστωση του ίδιου λογαριασμού μεταφέρονται οι αποσβέσεις που διενεργήθηκαν μέχρι την πώληση και στη χρέωσή του μεταφέρεται το αναπόσβεστο υπόλοιπο των εξόδων κτήσεως του ακινήτου (δηλαδή το υπόλοιπο του λογαριασμού 16.14 που προκύπτει μετά τη μεταφορά στο λογαριασμό αυτό των αποσβέσεων του λογαριασμού 16.99.14).

\item Το αποτέλεσμα που προκύπτει μετά από τις παραπάνω καταχωρήσεις και μεταφορές μεταφέρεται στο λογαριασμό 81.02.00 «ζημίες από εκποίηση ακινήτων» ή 81.02.01 «ζημίες από εκποίηση τεχνικών έργων», όταν είναι ζημία, ή στο λογαριασμό 81.03.00 «κέρδη από εκποίηση ακινήτων» ή 81.03.01 «κέρδη από εκποίηση τεχνικών έργων», όταν είναι κέρδος.

\end{enumerate}

Σε περίπτωση κατεδαφίσεως κτιρίου, το οποίο δεν έχει αποσβεστεί ολοκληρωτικά, η αναπόσβεστη αξία του μεταφέρεται στη χρέωση του λογαριασμού 16.19 «λοιπά έξοδα πολυετούς αποσβέσεως».

Τα έξοδα κατεδαφίσεως παλαιού κτιρίου καταχωρούνται στο λογαριασμό 11.03 «υποκείμενες σε απόσβεση διαμορφώσεις γηπέδων», εκτός αν επακολουθεί ανέγερση νέου κτιρίου, οπότε τα έξοδα αυτά προσαυξάνουν το κόστος ανεγέρσεώς του. Οι αποζημιώσεις που τυχόν καταβάλλονται σε μισθωτές του υπό κατεδάφιση παλαιού κτιρίου προσαυξάνουν το κόστος ανεγέρσεως του νέου.

\section{12.ΜΗΧΑΝΗΜΑΤΑ - ΤΕΧΝΙΚΕΣ ΕΓΚΑΤΑΣΤΑΣΕΙΣ - ΛΟΙΠΟΣ ΜΗΧΑΝΟΛΟΓΙΚΟΣ ΕΞΟΠΛΙΣΜΟΣ}

Στο λογαριασμό 12.00 παρακολουθούνται τα μηχανήματα της οικονομικής μονάδας, δηλαδή οι μηχανολογικές κατασκευές, μόνιμα εγκαταστημένες ή κινητές, οι οποίες χρησιμεύουν για να αποσπούν από τη φύση, να επεξεργάζονται ή να μετασχηματίζουν υλικά αγαθά ή για να παράγουν υπηρεσίες που αποτελούν το αντικείμενο δραστηριότητάς της.

Στο λογαριασμό 12.01 παρακολουθούνται οι τεχνικές εγκαταστάσεις της οικονομικής μονάδας, δηλαδή τεχνικές κατασκευές και γενικά τεχνολογικές διευθετήσεις που γίνονται για τη μόνιμη εγκατάσταση μηχανημάτων και τη σύνδεσή τους στο παραγωγικό κύκλωμά της. Στον ίδιο λογαριασμό παρακολουθούνται και οι κάθε είδους εγκαταστάσεις της οικονομικής μονάδας, οι οποίες, χωρίς να συσχετίζονται με τα μηχανήματα, έχουν σχέση με το παραγωγικό και γενικά με το λειτουργικό κύκλωμά της (π.χ. εγκαταστάσεις θερμάνσεως, υδραυλικές και τηλεφωνικές εγκαταστάσεις ή αποθηκευτικές δεξαμενές), με την προϋπόθεση ότι δεν είναι συνδεδεμένες με τις κτιριακές εγκαταστάσεις ή είναι συνδεδεμένες με αυτές, αλλά κατά τρόπο που ο αποχωρισμός τους είναι δυνατό να συντελεστεί εύκολα και χωρίς βλάβη της ουσίας τους ή των κτιριακών εγκαταστάσεων.

Στο λογαριασμό 12.02 παρακολουθούνται τα φορητά μηχανήματα «χειρός», δηλαδή τα φορητά μικρομηχανήματα που έχουν παραγωγική ζωή μεγαλύτερη από ένα έτος και μικρότερη από την παραγωγική ζωή των μηχανημάτων του λογαριασμού 12.00.

Στο λογαριασμό 12.03 παρακολουθούνται τα εργαλεία, δηλαδή τα μηχανολογικά και άλλης φύσεως αντικείμενα που χρησιμοποιούνται με το χέρι και έχουν παραγωγική ζωή μεγαλύτερη από ένα έτος. Τα μικροεργαλεία που αποσβένονται εφάπαξ στη χρήση που θα χρησιμοποιηθούν παρακολουθούνται στο λογαριασμό 25.00 «μικρά εργαλεία».

Στο λογαριασμό 12.04 παρακολουθούνται τα καλούπια και οι ιδιοσυσκευές της οικονομικής μονάδας, δηλαδή οι μηχανολογικές και άλλης φύσεως κατασκευές, οι οποίες προσαρμόζονται στα καθ' αυτό μηχανήματα για την παραγωγή εξειδικευμένων αντικειμένων, αποχωρίζονται από αυτά μετά από την εκτέλεση του συγκεκριμένου έργου και παραμένουν σε αδράνεια μέχρι να επαναχρησιμοποιηθούν (π.χ. καλούπια, μήτρες ή κεφαλές).

Στο λογαριασμό 12.05 παρακολουθούνται τα διάφορα μηχανολογικά όργανα, π.χ.  μετρήσεων, πειραματισμών ή ελέγχων.

Στο λογαριασμό 12.06 παρακολουθείται ο λοιπός μηχανολογικός εξοπλισμός της οικονομικής μονάδας ο οποίος δεν είναι δυνατό να ενταχθεί σε μια από τις κατηγορίες εξοπλισμού των λογαριασμών 12.00-12.05.

Στους λογαριασμούς 12.07, 12.08 και 12.09 παρακολουθούνται τα μηχανήματα, οι εγκαταστάσεις και ο λοιπός μηχανολογικός εξοπλισμός της οικονομικής μονάδας, που έχουν εγκατασταθεί σε ακίνητα τρίτων και που, μετά παρέλευση ορισμένου χρόνου, συμβατικά καθορισμένου, η κυριότητά τους περιέρχεται στους κυρίους των ακινήτων χωρίς αντάλλαγμα. Σχετικά με τον υπολογισμό των αποσβέσεων των παγίων αυτών ισχύουν όσα αναφέρονται στην περίπτ. 5 της παρ. 2.2.105 για το λογαριασμό 11.07.

Στους λογαριασμούς 12.10 έως και 12.19 παρακολουθούνται τα μηχανήματα, οι τεχνικές εγκαταστάσεις και ο λοιπός μηχανολογικός εξοπλισμός, όταν δε χρησιμοποιούνται παραγωγικά για τις ανάγκες της βασικής επαγγελματικής δραστηριότητας της οικονομικής μονάδας, ούτε και για τις παρεπόμενες ασχολίες της. Στους ίδιους λογαριασμούς παρακολουθούνται και τα μηχανήματα και άλλα πάγια τα οποία θεωρούνται ως οριστικά εκτός εκμεταλλεύσεως, είτε έχουν ολοκληρωτικά αποσβεστεί, οπότε η παρακολούθησή τους γίνεται με μία λογιστική μονάδα, είτε δεν έχουν ολοκληρωτικά αποσβεστεί, οπότε η παρακολούθησή τους γίνεται με την αξία κτήσεώς τους.

Πριν από τη μεταφορά της αξίας, π.χ. των μηχανημάτων, στους οικείους λογαριασμούς εκτός εκμεταλλεύσεως, προηγείται η μεταφορά των αποσβέσεων, είτε στην πίστωση των οικείων υπολογαριασμών του 12, όταν έχουν ολοκληρωτικά αποσβεστεί, είτε στους οικείους υπολογαριασμούς αποσβέσεων, π.χ. των μηχανημάτων εκτός εκμεταλλεύσεως του 12.99, όταν δεν έχουν ολοκληρωτικά αποσβεστεί.

Τα μηχανήματα, οι τεχνικές εγκαταστάσεις και ο λοιπός μηχανολογικός εξοπλισμός παρακολουθούνται στους οικείους υπολογαριασμούς του 12 έπειτα από καταχώριση σ' αυτούς: 

\begin{enumerate}

\item της αξίας κτήσεώς τους (αγοράς, εκτιμήσεως όταν πρόκειται για συγχώνευση ή εισφορά σε είδος), η οποία προσαυξάνεται με τα ειδικά έξοδα αγοράς, όπως τα έξοδα εγκαταστάσεως και συναρμολογήσεως (μέχρι να τεθούν σε κατάσταση λειτουργίας) ή της αξίας η οποία προκύπτει μετά από νόμιμη αναπροσαρμογή της αξίας κτήσεως και 

\item του κόστους κατασκευής τους, όταν πρόκειται για ιδιοκατασκευές, το οποίο προκύπτει από το λογαριασμό 15.02 «μηχανήματα - τεχνικές εγκαταστάσεις - λοιπός μηχανολογικός εξοπλισμός υπό εκτέλεση» και το οποίο προσαυξάνεται με τα έξοδα εγκαταστάσεως και συναρμολογήσεώς τους.

\end{enumerate}

Η παραπάνω αξία κτήσεως προσαυξάνεται με την αξία των επεκτάσεων ή προσθηκών και βελτιώσεων που γίνονται κάθε φορά.

Σχετικά με τις αποσβέσεις των μηχανημάτων - τεχνικών εγκαταστάσεων - λοιπού μηχανολογικού εξοπλισμού, ισχύουν όσα αναφέρονται στην παρ. 2.2.102. Σχετικά με την οικονομική απαξίωση και υποτίμηση των μηχανημάτων - τεχνικών εγκαταστάσεων - λοιπού μηχανολογικού εξοπλισμού, ισχύουν αναλόγως όσα αναφέρονται στις περιπτ. 11 και 12 της παρ. 2.2.104.

Σε περίπτωση πωλήσεως μηχανήματος και γενικά περιουσιακού στοιχείου του λογαριασμού 12 ισχύουν τα εξής:

\begin{enumerate}

\item Στην πίστωση του οικείου λογαριασμού, π.χ. του μηχανήματος, καταχωρείται το τίμημα πωλήσεως και στη χρέωσή του καταχωρούνται τα έξοδα που τυχόν δημιουργούνται για την επίτευξη της πωλήσεως.

\item Στην πίστωση του ίδιου λογαριασμού μεταφέρονται οι αποσβέσεις που διενεργήθηκαν μέχρι την πώληση.

\item Το αποτέλεσμα που προκύπτει μετά από τις παραπάνω καταχωρήσεις και μεταφορές μεταφέρεται στο λογαριασμό 81.02.02 «ζημίες από εκποίηση μηχανημάτων - τεχνικών εγκαταστάσεων - λοιπού μηχανολογικού εξοπλισμού», όταν είναι ζημία, ή στο λογαριασμό 81.03.02 «κέρδη από εκποίηση μηχανημάτων - τεχνικών εγκαταστάσεων - λοιπού μηχανολογικού εξοπλισμού», όταν είναι κέρδος.

\end{enumerate}

Σε περίπτωση ολοκληρωτικής αχρηστεύσεως ή καταστροφής, π.χ. μηχανημάτων τα οποία δεν έχουν ολοκληρωτικά αποσβεστεί, η αναπόσβεστη αξία τους μεταφέρεται στη χρέωση του λογαριασμού 81.02.99 «λοιπές έκτακτες ζημίες».

\section{13.ΜΕΤΑΦΟΡΙΚΑ ΜΕΣΑ}

Στους οικείους υπολογαριασμούς του 13 παρακολουθούνται τα κάθε είδους οχήματα με τα οποία η οικονομική μονάδα διενεργεί μεταφορές και μετακινήσεις του προσωπικού και των υλικών αγαθών της (π.χ. εμπορευμάτων, έτοιμων προϊόντων ή υλικών), είτε μέσα στους χώρους εκμεταλλεύσεως, είτε έξω από αυτούς.

Σχετικά με τη λειτουργία και άλλες λεπτομέρειες που αφορούν τους υπολογαριασμούς του 13 ισχύουν ανάλογα όσα ορίζονται στην προηγούμενη παράγραφο 2.2.106 για το λογαριασμό 12.

\section{14.ΕΠΙΠΛΑ ΚΑΙ ΛΟΙΠΟΣ ΕΞΟΠΛΙΣΜΟΣ}

Στους οικείους υπολογαριασμούς του 14 παρακολουθούνται τα έπιπλα και ο λοιπός εξοπλισμός των διαφόρων κτιριακών χώρων της οικονομικής μονάδας (π.χ.  γραφείων, εργοστασίων, εργαστηρίων, καταστημάτων ή αποθηκών).

Στον υπολογαριασμό 14.00 «έπιπλα» παρακολουθούνται τα κινητά αντικείμενα ή εκείνα που είναι εγκαταστημένα αλλά είναι δυνατό να αποχωριστούν εύκολα, και τα οποία προορίζονται για τη συμπλήρωση ή τον καλλωπισμό των κτιριακών χώρων και χρησιμοποιούνται, κατά κανόνα, από το προσωπικό της οικονομικής μονάδας στην οποία ανήκουν.

Στον υπολογαριασμό 14.01 «σκεύη» παρακολουθούνται τα διάφορα είδη εστιάσεως, τα οποία χρησιμοποιούνται για την εξυπηρέτηση αναγκών εστιατορίων, κυλικείων, ξενοδοχείων κλπ. (π.χ. ψύκτες νερού, ψυγεία, ηλεκτρικοί φούρνοι ή σκεύη κουζίνας).

Στον υπολογαριασμό 14.02 «μηχανές γραφείων» παρακολουθούνται οι κάθε είδους μηχανικές μηχανές γραφείων (π.χ. λογιστικές, αριθμομηχανές ή γραφομηχανές) της οικονομικής μονάδας.

Στον υπολογαριασμό 14.03 «ηλεκτρονικοί υπολογιστές και ηλεκτρονικά συστήματα» παρακολουθούνται τα κάθε είδους ηλεκτρονικά μηχανήματα που εξυπηρετούν τις ανάγκες της οικονομικής μονάδας, όπως π.χ. οι διερευνητές, οι ηλεκτρονικές λογιστικές μηχανές, οι ηλεκτρονικές οθόνες, οι διατρητικές μηχανές.

Στον υπολογαριασμό 14.04 «μέσα αποθηκεύσεως και μεταφοράς» παρακολουθούνται τα περιουσιακά στοιχεία τα οποία χρησιμοποιούνται ως μέσα αποθηκεύσεως και μεταφοράς, έχουν παραγωγική ζωή μεγαλύτερη από ένα έτος και αποσβένονται τμηματικά (π.χ. δεξαμενές, δοχεία, σιλό, κοντέινερ ή παλέτες).

Στον υπολογαριασμό 14.05 «επιστημονικά όργανα» παρακολουθούνται τα φορητά μέσα με τα οποία εξασφαλίζονται οι αναγκαίες αναλύσεις, μετρήσεις και δοκιμές πάνω σε υλικά, δυνάμεις και διάφορες μορφές ενέργειας (π.χ. αντιδραστήρες, αποστακτήρες, ζυγοί ακριβείας, μετρητές αντοχής υλικού σε κρούσεις, εφελκυσμό ή θραύσεις, συσκευές τεχνητής δημιουργίας διαφόρων συνθηκών περιβάλλοντος ή συσκευές δημιουργίας κενού).

Στον υπολογαριασμό 14.06 «ζώα για πάγια εκμετάλλευση» παρακολουθούνται τα ζώα τα οποία προορίζονται για πάγια εκμετάλλευση, ιδίως από τις γεωργικές και κτηνοτροφικές οικονομικές μονάδες (π.χ. άλογα, βόδια, που χρησιμοποιούνται π.χ.  για το όργωμα αγρών ή για μεταφορές, αγελάδες που παρέχουν π.χ. τα νεογέννητα μοσχάρια ή το γάλα).

Στον υπολογαριασμό 14.08 «εξοπλισμός τηλεπικοινωνιών» παρακολουθούνται τα κάθε είδους φορητά ή εγκαταστημένα μέσα τηλεπικοινωνιών (π.χ. τηλεφωνικά κέντρα, τηλεφωνικές συσκευές ή συσκευές τέλεξ).

Σχετικά με τη λειτουργία και άλλες λεπτομέρειες που αφορούν τους υπολογαριασμούς του 14 ισχύουν ανάλογα όσα ορίζονται στην παρ. 2.2.106 για το λογαριασμό 12.

\section{15.ΑΚΙΝΗΤΟΠΟΙΗΣΕΙΣ ΥΠΟ ΕΚΤΕΛΕΣΗ ΚΑΙ ΠΡΟΚΑΤΑΒΟΛΕΣ ΚΤΗΣΕΩΣ ΠΑΓΙΩΝ ΣΤΟΙΧΕΙΩΝ}

Στο λογαριασμό 15 παρακολουθούνται τα ποσά τα οποία διαθέτονται για την κατασκευή νέων ενσώματων πάγιων στοιχείων, καθώς και τα ποσά τα οποία προκαταβάλλονται για την αγορά όμοιων στοιχείων. Το κατά την ημέρα κλεισίματος του ισολογισμού υπόλοιπο του λογαριασμού 15 απεικονίζει το μη ολοκληρωμένο κόστος των πάγιων στοιχείων, τα οποία μέχρι την ημέρα εκείνη, δεν είχαν παραληφθεί ή δεν είχε συντελεστεί η αποπεράτωσή τους.

Το κόστος κατασκευής των πάγιων στοιχείων τα οποία κατασκευάζονται από την οικονομική μονάδα με δικά της μέσα παρακολουθείται και προσδιορίζεται με τους λογαριασμούς παραγωγής της ομάδας 9, όπως ειδικότερα ορίζεται στην παρ. 5.214 του πέμπτου μέρους, ή υπολογίζεται εξωλογιστικά, αν η οικονομική μονάδα δεν εφαρμόζει σύστημα αναλυτικής λογιστικής εκμεταλλεύσεως. Το κόστους που προσδιορίζεται με τον τρόπο αυτό καταχωρείται στη χρέωση των οικείων υπολογαριασμών του 15, με πίστωση των αντίστοιχων υπολογαριασμών του 78.00. Το κόστος αυτό παραμένει στους οικείους υπολογαριασμούς του 15 και κατά την επόμενη χρήση, κατά την οποία προσαυξάνεται και με το κόστος που πραγματοποιείται μέσα στη νέα αυτή χρήση, και ούτω καθεξής, μέχρι την ολοκλήρωση της κατασκευής, οπότε το συνολικό κόστος κατασκευής του πάγιου στοιχείου μεταφέρεται από τους οικείους υπολογαριασμούς του 15 στους οικείους λογαριασμούς των πάγιων στοιχείων (11-14 και 16).

Αν η κατασκευή του πάγιου στοιχείου ολοκληρώνεται μέσα στη χρήση που άρχισε η κατασκευή αυτή, το κόστος που προσδιορίζεται από τους λογαριασμούς παραγωγής της ομάδας 9 ή εξωλογιστικά, είναι δυνατό να καταχωρείται απευθείας στους οικείους λογαριασμούς των πάγιων στοιχείων (11-14 και 16), με πίστωση των αντίστοιχων υπολογαριασμών του 78.00.

Το κόστος κατασκευής των πάγιων στοιχείων τα οποία κατασκευάζονται από τρίτους με υλικά που παρέχονται από την οικονομική μονάδα προσδιορίζεται από τους οικείους υπολογαριασμούς του 15, στη χρέωση των οποίων καταχωρείται η αξία των υλικών που αγοράζονται και η αξία των τιμολογίων των τρίτων κατασκευαστών.

Στην περίπτωση κατά την οποία τα υλικά που αγοράζονται εισάγονται πρωτύτερα στις αποθήκες με καταχώρισή τους στους οικείους λογαριασμούς της ομάδας 2 και από τις αποθήκες αυτές παραδίνονται στους τρίτους, ή γενικά στην περίπτωση που τα υλικά χορηγούνται στους τρίτους από τις αποθήκες, η αξία των υλικών αυτών καταχωρείται στη χρέωση των οικείων υπολογαριασμών του 15, με πίστωση των αντίστοιχων υπολογαριασμών του 78.00.

Όταν η κατασκευή των πάγιων στοιχείων γίνεται από τρίτους, στους οποίους, εκτός από τη χορήγηση των υλικών, παρέχεται και συμπαράσταση των υπηρεσιών της οικονομικής μονάδας, η οποία συνεπάγεται πρόσθετο κόστος επιπλέον των υλικών, για τον προσδιορισμό του ολοκληρωμένου κόστους κατασκευής και για τη λειτουργία των σχετικών λογαριασμών ακολουθείται η διαδικασία της παραπάνω περιπτ. 2 (κατασκευή πάγιων στοιχείων από την οικονομική μονάδα).

Οι προκαταβολές που δίνονται σε κατασκευαστές πάγιων στοιχείων, καθώς και σε προμηθευτές υλικών κατασκευής ή προμηθευτές αυτούσιων όμοιων στοιχείων, καταχωρούνται στη χρέωση του λογαριασμού 15.09 «προκαταβολές κτήσεως πάγιων στοιχείων». Μετά τη λήψη του σχετικού τιμολογίου ή, προκειμένου για εισαγωγή από το εξωτερικό, μετά τον προσδιορισμό του κόστους αγοράς, πιστώνεται ο λογαριασμός 15.09 με χρέωση άλλων υπολογαριασμών του 15 ή των οικείων λογαριασμών των πάγιων στοιχείων (11-14 και 16).

Παρέχεται η ευχέρεια στις οικονομικές μονάδες να παρακολουθούν τις προκαταβολές για κτήση πάγιων στοιχείων, ως εξής:

\begin{enumerate}

\item Οι προκαταβολές που δίνονται για εισαγωγή υλικών ή αυτούσιων πάγιων στοιχείων από το εξωτερικό, να παρακολουθούνται στο λογαριασμό 32.00 «παραγγελίες πάγιων στοιχείων».

\item Οι προκαταβολές που δίνονται σε προμηθευτές ή κατασκευαστές πάγιων στοιχείων του εσωτερικού, να παρακολουθούνται στο λογαριασμό 50.08 «προμηθευτές εσωτερικού πάγιων στοιχείων».

\item Στο τέλος κάθε χρήσεως το υπόλοιπο του λογαριασμού 15 «ακινητοποιήσεις υπό εκτέλεση και προκαταβολές κτήσεως πάγιων στοιχείων» και τα χρεωστικά υπόλοιπα των υπολογαριασμών των 32.00 «παραγγελίες πάγιων στοιχείων» και 50.08 «προμηθευτές εσωτερικού λογ/σμός πάγιων στοιχείων» εμφανίζονται στον ισολογισμό σε ένα ενιαίο κονδύλι.

\end{enumerate}

\section{16.ΑΣΩΜΑΤΕΣ ΑΚΙΝΗΤΟΠΟΙΗΣΕΙΣ ΚΑΙ ΕΞΟΔΑ ΠΟΛΥΕΤΟΥΣ ΑΠΟΣΒΕΣΕΩΣ}

\begin{enumerate}
\item Ασώματες ακινητοποιήσεις (άυλα πάγια στοιχεία)

Άυλα πάγια στοιχεία (ασώματες ακινητοποιήσεις) είναι τα ασώματα εκείνα οικονομικά αγαθά τα οποία είναι δεκτικά χρηματικής αποτιμήσεως και είναι δυνατό να αποτελέσουν αντικείμενο συναλλαγής, είτε αυτά μόνα, είτε μαζί με την οικονομική μονάδα. Τα άυλα πάγια στοιχεία αποκτούνται με σκοπό να χρησιμοποιούνται παραγωγικά για χρονικό διάστημα μεγαλύτερο από ένα έτος, διακρίνονται δε στις εξής δύο κατηγορίες:

\begin{enumerate}
\item δικαιώματα, όπως π.χ. διπλώματα ευρεσιτεχνίας, εμποροβιομηχανικά σήματα ή πνευματική ιδιοκτησία.

\item πραγματικές καταστάσεις, ιδιότητες και σχέσεις, όπως π.χ. η πελατεία, η φήμη, η πίστη, η καλή οργάνωση της οικονομικής μονάδας ή η ειδίκευση στην παραγωγή ορισμένων αγαθών. Στη δεύτερη αυτή κατηγορία ανήκουν τα στοιχεία τα οποία συνθέτουν, κατά κύριο λόγο, την έννοια του γνωστού όρου της «υπεραξίας» ή «φήμης και πελατείας» (Goodwill, Fonds de Commerce) της οικονομικής μονάδας και τα οποία προσδίνουν στη μονάδα που λειτουργεί παραγωγικά συνολική αξία μεγαλύτερη από εκείνη που προκύπτει από την αποτίμηση των επιμέρους περιουσιακών της στοιχείων.
\end{enumerate}

Τα άυλα πάγια στοιχεία που αποκτούνται από τρίτους καταχωρούνται στους οικείους υπολογαριασμούς του 16 με την αξία κτήσεώς τους, ενώ εκείνα που δημιουργούνται από την οικονομική μονάδα απεικονίζονται λογιστικά, μόνο όταν για τη δημιουργία τους πραγματοποιούνται έξοδα και εφόσον τα έξοδα αυτά αποσβένονται τμηματικά και όχι εφάπαξ μέσα σε ένα χρόνο.

Στο λογαριασμό 16.00 «υπεραξία επιχειρήσεως (Goodwill)» παρακολουθείται η υπεραξία που δημιουργείται κατά την εξαγορά ή συγχώνευση ολόκληρης οικονομικής μονάδας και που είναι ίση με τη διαφορά μεταξύ του ολικού τιμήματος αγοράς και της πραγματικής αξίας των επιμέρους περιουσιακών της στοιχείων. Στην περίπτωση συγχωνεύσεως η πραγματική αξία της οικονομικής μονάδας προσδιορίζεται κατά τη διαδικασία που ορίζει το άρθρο 9 του Ν. 2190/1920.

Η υπεραξία της οικονομικής μονάδας στηρίζεται στην εκτίμηση για την ικανότητά της να πραγματοποιεί υψηλά κέρδη λόγω κυρίως της καλής φήμης, της εκτεταμένης πελατείας, της μεγάλης πίστεως στην αγορά, της καλής οργανώσεως, της ιδιαίτερής της εξειδικεύσεως στην παραγωγή ορισμένων αγαθών, της καλής προοπτικής αναπτύξεως του κλάδου στον οποίο ανήκει, των εξαιρετικών πλεονεκτημάτων της θέσεως όπου είναι εγκαταστημένη, της υψηλής στάθμης των στελεχών που απασχολεί (επιστημονική κατάρτιση, εμπειρία) και του κύρους, δυναμισμού και αποτελεσματικότητας του διοικητικού και διευθυντικού της μηχανισμού.

Η υπεραξία της οικονομικής μονάδας αποσβένεται, είτε εφάπαξ, είτε τμηματικά και ισόποσα σε περισσότερες από μία χρήσεις, οι οποίες δεν είναι δυνατό να υπερβαίνουν τα πέντε έτη.

Στο λογαριασμό 16.01 «δικαιώματα βιομηχανικής ιδιοκτησίας» παρακολουθούνται τα άυλα περιουσιακά στοιχεία τα οποία αποκτούνται με αντάλλαγμα, είτε λόγω αγοράς, είτε λόγω παραγωγής από την ίδια την οικονομική μονάδα. Με την κατοχή και αξιοποίηση των στοιχείων αυτών η μονάδα αποκτάει πλεονεκτήματα μονοπωλιακής ή εξειδικευμένης δράσεως στην αγορά, για το χρονικό διάστημα που διαρκεί π.χ. η προστασία του σχετικού δικαιώματος ή η γνώση του τρόπου παραγωγής ενός προϊόντος ή της μεθόδου κατεργασίας υλικών.

Για την καταχώριση εξόδων αγοράς ή παραγωγής στο λογαριασμό 16.01 αποτελεί προϋπόθεση η προσδοκία ότι τα δικαιώματα για τα οποία γίνονται έξοδα θα αποδώσουν αποτελέσματα στην οικονομική μονάδα. Απαγορεύεται η αποθεματοποίηση εξόδων στους υπολογαριασμούς του 16, όταν είναι βέβαιο ότι δεν προσδοκάται οποιοδήποτε αποτέλεσμα από την πραγματοποίηση των εξόδων αυτών.

Η αξία κτήσεως των άυλων περιουσιακών στοιχείων του λογαριασμού 16.01 αποσβένεται με ισόποση ετήσια απόσβεση μέσα στο χρόνο της παραγωγικής χρησιμότητας κάθε άυλου στοιχείου και, σε περίπτωση που το άυλο δικαίωμα έχει από το νόμο προστασία περιορισμένης διάρκειας, μέσα στο χρόνο της περιορισμένης αυτής διάρκειας.

Στους λογαριασμούς 16.02 «δικαιώματα (π.χ. παραχωρήσεις) εκμεταλλεύσεως ορυχείων - μεταλλείων - λατομείων» και 16.03 «λοιπές παραχωρήσεις» παρακολουθείται η αξία, π.χ. κτήσεως, των δικαιωμάτων αυτών, σύμφωνα με όσα ορίζονται στην περίπτ. 5 της παρ. 2.2.104.

Στο λογαριασμό 16.04 «δικαιώματα χρήσεως ενσώματων πάγιων στοιχείων» παρακολουθείται η αξία της εισφοράς κατά χρήση, στην οικονομική μονάδα, ενσώματων πάγιων στοιχείων, για ορισμένο χρόνο, η οποία καθορίζεται με νόμιμη διαδικασία εκτιμήσεως.

Η παραπάνω αξία εισφοράς κατά χρήση αποσβένεται με ισόποσες δόσεις μέσα στο χρόνο που καθορίζεται συμβατικά για τη χρησιμοποίηση κάθε άυλου πάγιου στοιχείου.

Στο λογαριασμό 16.05 «λοιπά δικαιώματα» παρακολουθούνται τα άυλα εκείνα περιουσιακά στοιχεία τα οποία δεν εντάσσονται σε μία από τις προηγούμενες κατηγορίες των λογαριασμών 16.00-16.04, όπως είναι π.χ. τα μισθωτικά δικαιώματα.

Στην περίπτωση μισθωτικών δικαιωμάτων (δηλαδή μεταβιβάσεως από μισθωτή ακινήτου στην οικονομική μονάδα των μισθωτικών του δικαιωμάτων σε ορισμένο ακίνητο) που απορρέουν από σχετική σύμβαση μισθώσεως και το νόμο που ισχύει κάθε φορά, η αξία που καταβάλλεται στο μισθωτή αυτό ως αποζημίωση για τη μεταβίβαση των δικαιωμάτων του καταχωρείται στη χρέωση οικείου υπολογαριασμού του 16.05 και αποσβένεται σε ισόποσες δόσεις μέσα στο χρόνο ισχύος του μισθωτικού δικαιώματος.

Σε περίπτωση ολοκληρώσεως της αποσβέσεως της αξίας κτήσεως άυλων περιουσιακών στοιχείων, μεταφέρονται από τους αντίστοιχους υπολογαριασμούς του 16.99 στους οικείους λογαριασμούς του 16 οι αποσβέσεις και έτσι οι λογαριασμοί αυτοί εξισώνονται.

Σε περίπτωση οριστικής παύσεως της χρησιμοποιήσεως άυλου περιουσιακού στοιχείου, πριν ολοκληρωθεί η απόσβεση της αξίας κτήσεώς του, το αναπόσβεστο υπόλοιπο αυτής μεταφέρεται στη χρέωση του λογαριασμού 81.02.99 «λοιπές έκτακτες ζημίες».

Σε περίπτωση πωλήσεως άυλου πάγιου περιουσιακού στοιχείου ισχύουν ανάλογα όσα ορίζονται στην περίπτ. 12 της παρ. 2.2.106 για το λογαριασμό 12.

\item Έξοδα πολυετούς αποσβέσεως

Έξοδα πολυετούς αποσβέσεως είναι εκείνα που γίνονται για την ίδρυση και αρχική οργάνωση της οικονομικής μονάδας, την απόκτηση διαρκών μέσων εκμεταλλεύσεως, καθώς και για την επέκταση και αναδιοργάνωσή της. Τα έξοδα αυτά εξυπηρετούν την οικονομική μονάδα για μεγάλη χρονική περίοδο - οπωσδήποτε μεγαλύτερη από ένα έτος - και για το λόγο αυτό αποσβένονται τμηματικά.

Στο λογαριασμό 16.10 «έξοδα ιδρύσεως και πρώτης εγκαταστάσεως» παρακολουθούνται τα έξοδα καταρτίσεως και δημοσιεύσεως του καταστατικού των οικονομικών μονάδων εταιρικής μορφής, τα έξοδα δημόσιας προβολής της ιδρύσεως, της καλύψεως του εταιρικού κεφαλαίου, της εκπονήσεως τεχνικών, εμπορικών και οργανωτικών μελετών, καθώς και τα έξοδα διοικήσεως που πραγματοποιούνται μέχρι την έναρξη της εκμεταλλεύσεως. Στον ίδιο λογαριασμό παρακολουθούνται και τα μεταγενέστερα (δηλαδή μετά την έναρξη της παραγωγικής δράσεως) έξοδα που δημιουργούνται για την επέκταση της δραστηριότητας της οικονομικής μονάδας.

Ειδικά, προκειμένου για έξοδα τεχνικών μελετών, αν το έργο για το οποίο πραγματοποιούνται κατασκευαστεί, τα έξοδα μελέτης του δεν καταχωρούνται στο λογαριασμό 16.10, αλλά ενσωματώνονται στο κόστος του έργου και αποσβένονται όπως αυτό.

Στο λογαριασμό 16.11 «έξοδα ερευνών ορυχείων - μεταλλείων - λατομείων» παρακολουθείται η αξία, π.χ., κτήσεως, των εξόδων αυτών, σύμφωνα με όσα ορίζονται στην περίπτ. 6 της παρ. 2.2.104.

Στο λογαριασμό 16.12 «έξοδα λοιπών ερευνών» παρακολουθούνται τα έξοδα που γίνονται για έρευνες σε άλλους κλάδους και τομείς δραστηριότητας της οικονομικής μονάδας, εκτός από τα ορυχεία - μεταλλεία - λατομεία.

Στο λογαριασμό 16.13 «έξοδα αυξήσεως κεφαλαίου και εκδόσεως ομολογιακών δανείων» παρακολουθούνται έξοδα, όπως π.χ. εκείνα που γίνονται για εκτυπώσεις ή ανακοινώσεις, όταν αυξάνεται το κεφάλαιο της οικονομικής μονάδας ή εκδίδεται από αυτή ομολογιακό δάνειο.

Στο λογαριασμό 16.14 «έξοδα κτήσεως ακινητοποιήσεων» παρακολουθούνται όλα τα έξοδα που γίνονται για την απόκτηση των ενσώματων ή ασώματων ακινητοποιήσεων, όπως π.χ. είναι ο φόρος μεταβιβάσεως, τα συμβολαιογραφικά έξοδα, τα μεσιτικά και οι αμοιβές μελετητών ή δικηγόρων, τα οποία, σύμφωνα με όσα ορίζονται στους οικείους λογαριασμούς των ακινητοποιήσεων, δεν προσαυξάνουν το κόστος κτήσεώς τους.

Στο λογαριασμό 16.15 «συναλλαγματικές διαφορές από πιστώσεις και δάνεια για κτήσεις πάγιων στοιχείων» παρακολουθούνται σε υπολογαριασμούς κατά πίστωση ή δάνειο, οι συναλλαγματικές διαφορές που προκύπτουν κατά την πληρωμή ή την αποτίμηση σε δραχμές των πιστώσεων ή δανείων σε ξένο νόμισμα, που συνάπτονται ειδικά και μόνο για την αγορά, κατασκευή ή εγκατάσταση πάγιων στοιχείων της οικονομικής μονάδας, σύμφωνα με όσα ορίζονται στην παρακάτω περίπτωση 23.

Στο λογαριασμό 16.16 «διαφορές εκδόσεως και εξοφλήσεως ομολογιών» παρακολουθούνται οι διαφορές από τη διάθεση ομολογιών σε τιμή μικρότερη από την ονομαστική τους, καθώς και οι διαφορές από την εξόφληση ομολογιών σε τιμή μεγαλύτερη από την ονομαστική τους.

Στο λογαριασμό 16.17 «έξοδα αναδιοργανώσεως» παρακολουθούνται τα έξοδα μελετών οικονομικής, εμπορικής, τεχνικής και διοικητικής αναδιοργανώσεως ριζικού χαρακτήρα, με τα οποία επιδιώκεται η κάλυψη νέων αναγκών που προκύπτουν από το μεγάλωμα της οικονομικής μονάδας σαν αποτέλεσμα σημαντικών επεκτάσεων του παραγωγικού της δυναμικού ή αλλαγών στην οργανωτική της δομή ή ριζικών μεταβολών στην εμπορική της δραστηριότητα.

Στο λογαριασμό 16.18 «τόκοι δανείων κατασκευαστικής περιόδου» παρακολουθούνται οι τόκοι, μόνο της κατασκευαστικής περιόδου, πιστώσεων ή δανείων τα οποία χρησιμοποιούνται αποκλειστικά για κτήσεις πάγιων περιουσιακών στοιχείων.

Στο λογαριασμό 16.19 «λοιπά έξοδα πολυετούς αποσβέσεως» παρακολουθούνται τα έξοδα εκείνα που δεν εντάσσονται σε μία από τις προηγούμενες κατηγορίες των λογαριασμών 16.10-16.18, όπως π.χ. η αναπόσβεστη αξία κτιρίου σε περίπτωση κατεδαφίσεώς του (βλ. περίπτ. 10 παρ. 2.2.105).

Στο λογαριασμό 16.98 «προκαταβολές κτήσεως ασώματων ακινητοποιήσεων» παρακολουθούνται οι προκαταβολές που δίνονται για το λόγο αυτό, σύμφωνα με όσα αναφέρονται στην περίπτ. 5 της παρ. 2.2.109 για το λογαριασμό 15.09.

Τα έξοδα των λογαριασμών 16.10, 16.12, 16.13, 16.14, 16.17, 16.18 και 16.19, αποσβένονται, είτε εφάπαξ κατά το έτος πραγματοποιήσεώς τους, είτε τμηματικά και ισόποσα μέσα σε μια πενταετία.

Οι χρεωστικές συναλλαγματικές διαφορές του λογαριασμού 16.15, κατά πίστωση ή δάνειο, έπειτα από συμψηφισμό τυχόν πιστωτικών που μεταφέρονται από τον οικείο υπολογαριασμό του 44.15 «προβλέψεις για συναλλαγματικές διαφορές από πιστώσεις και δάνεια για κτήσεις πάγιων στοιχείων», αποσβένονται τμηματικά ανάλογα με την υπόλοιπη κανονική χρονική διάρκεια της πιστώσεως ή του δανείου ως εξής:

\begin{itemize}

\item Στο τέλος της κλειόμενης χρήσεως μεταφέρεται από τον οικείο κατά πίστωση ή δάνειο υπολογαριασμό του 16.15 στο λογαριασμό 81.00.04 «συναλλαγματικές διαφορές» ποσό ίσο με το πηλίκον της διαιρέσεως του χρεωστικού υπολοίπου του οικείου υπολογαριασμού του 16.15 με τον αριθμό των ετών από τη λήξη της χρήσεως αυτής μέχρι τη λήξη της αντίστοιχης πιστώσεως ή του αντίστοιχου δανείου. Για τον προσδιορισμό του πηλίκου της παραγράφου αυτής χρονική περίοδος μικρότερη από δώδεκα μήνες λογίζεται ως περίοδος ενός έτους.

\item Σε περίπτωση που από την ημερομηνία χορηγήσεως της πιστώσεως ή του δανείου μέχρι την ημερομηνία ενάρξεως της παραγωγικής λειτουργίας των χρηματοδοτούμενων πάγιων στοιχείων μεσολαβεί κατασκευαστική περίοδος, η τμηματική μεταφορά του χρεωστικού υπολοίπου του οικείου υπολογαριασμού του 16.15 αρχίζει από τη χρήση μέσα στην οποία λήγει η κατασκευαστική περίοδος ή διακόπτεται, για οποιοδήποτε λόγο, η κατασκευή του έργου.

\item Σε περίπτωση ληξιπρόθεσμων πιστώσεων ή δανείων, κατά το όλο ή μέρος αυτών, τα χρεωστικά υπόλοιπα των οικείων υπολογαριασμών του 16.15, που αντιστοιχούν στο ληξιπρόθεσμο μέρος, μεταφέρονται στο λογαριασμό 81.00.04 στο τέλος της χρήσεως μέσα στην οποία οι αντίστοιχες πιστώσεις ή τα αντίστοιχα δάνεια έγιναν ληξιπρόθεσμα.

\end{itemize}

Οι πιστωτικές συναλλαγματικές διαφορές του λογαριασμού 16.15, κατά πίστωση ή δάνειο, στο τέλος της χρήσεως, μειώνουν τις χρεωστικές ή, στο μέτρο που δεν υπάρχουν χρεωστικές, μεταφέρονται σε αντίστοιχο κατά πίστωση ή δάνειο υπολογαριασμό του 44.15. Από τον τελευταίο αυτό λογαριασμό, κατά το κλείσιμο του ισολογισμού, μεταφέρεται στο λογαριασμό 81.01.04 «συναλλαγματικές διαφορές» το μέρος εκείνο που αντιστοιχεί στο ποσό των πιστώσεων ή δανείων που πληρώθηκε μέσα στην κλειόμενη χρήση.

Οι διαφορές εκδόσεως και εξοφλήσεως ομολογιών του λογαριασμού 16.16 αποσβένονται με τμηματικές ισόποσες δόσεις μέχρι τη λήξη της προθεσμίας εξοφλήσεως του ομολογιακού δανείου.

Σε περίπτωση ολοκληρώσεως της αποσβέσεως της αξίας κτήσεως εξόδων πολυετούς αποσβέσεως, μεταφέρονται από τους αντίστοιχους υπολογαριασμούς του 16.99 στους οικείους λογαριασμούς του 16 οι αποσβέσεις και έτσι οι λογαριασμοί αυτοί εξισώνονται. 

\end{enumerate}

\section{18.ΣΥΜΜΕΤΟΧΕΣ ΚΑΙ ΛΟΙΠΕΣ ΜΑΚΡΟΠΡΟΘΕΣΜΕΣ ΑΠΑΙΤΗΣΕΙΣ}

Στους λογαριασμούς 18.00 «συμμετοχές σε συνδεμένες επιχειρήσεις» και 18.01 «συμμετοχές σε λοιπές επιχειρήσεις» παρακολουθούνται οι μετοχές ανώνυμων εταιρειών, τα εταιρικά μερίδια Ε.Π.Ε. και οι εταιρικές μερίδες των άλλης νομικής μορφής εταιρειών, που η διαρκής κατοχή τους κρίνεται ιδιαίτερα χρήσιμη για τη δραστηριότητα της οικονομικής μονάδας, κυρίως γιατί της εξασφαλίζει άσκηση επιρροής στις αντίστοιχες εταιρείες. Οι συμμετοχές χαρακτηρίζονται σαν μορφή πάγιας επενδύσεως όταν κατά την απόκτησή τους υπάρχει σκοπός για διαρκή κατοχή τους και επί πλέον το ποσοστό συμμετοχής υπερβαίνει το 10\% του κεφαλαίου κάθε εταιρείας. Στην αντίθετη περίπτωση χαρακτηρίζονται σαν χρεόγραφα και παρακολουθούνται στο λογαριασμό 34. Σχετικά με τη διάκριση των λογαριασμών 18.00 και 18.01 ισχύουν τα εξής:


\begin{enumerate}
\item Στο λογαριασμό 18.00 καταχωρούνται οι συμμετοχές της οικονομικής μονάδας σε συνδεμένες επιχειρήσεις της παρακάτω περιπτώσεως 10 της παραγράφου αυτής.

\item Στο λογαριασμό 18.01 καταχωρούνται οι συμμετοχές της οικονομικής μονάδας σε μη συνδεμένες επιχειρήσεις.

\end{enumerate}

Οι συμμετοχές καταχωρούνται στους οικείους υπολογαριασμούς του 18.00 ή 18.01 με την αξία κτήσεώς τους. Αξία κτήσεως είναι το ποσό που καταβάλλεται, είτε απευθείας στην εταιρεία κατά τη συγκρότηση του κεφαλαίου της, είτε για την αγορά της συμμετοχής, καθώς και η ονομαστική αξία των τίτλων που δίνονται στην οικονομική μονάδα χωρίς αντάλλαγμα λόγω νόμιμης αναπροσαρμογής των περιουσιακών στοιχείων του ισολογισμού της εκδότριας εταιρείας ή κεφαλαιοποιήσεως αποθεματικών της.

Στην περίπτωση λήψεως τίτλων χωρίς αντάλλαγμα χρεώνεται ο οικείος υπολογαριασμός του 18.00 ή 18.01, με πίστωση του λογαριασμού 41.06 «διαφορές από αναπροσαρμογή αξίας συμμετοχών και χρεογράφων». Τα ειδικά έξοδα αγοράς τίτλων συμμετοχής καταχωρούνται στο λογαριασμό 64.10.00 «προμήθειες και λοιπά έξοδα αγοράς συμμετοχών και χρεογράφων».

Όταν αναλαμβάνεται η κάλυψη μέρους του μετοχικού κεφαλαίου ανώνυμης εταιρείας, με τον όρο η καταβολή του να γίνει σε δόσεις, οι μετοχές που αποκτούνται με τον τρόπο αυτό καταχωρούνται στους λογαριασμούς 18.00.02, 18.00.03, 18.00.06 και 18.00.07 ή στους λογαριασμούς 18.01.02, 18.01.03, 18.01.06 και 18.01.07, κατά περίπτωση, με τη συνολική τους αξία, με πίστωση του λογαριασμού 53.06 «οφειλόμενες δόσεις συμμετοχών» με την αξία των οφειλόμενων δόσεων. Μετά την εξόφληση των δόσεων γίνεται η μεταφορά της συνολικής αξίας των μετοχών που εξοφλούνται από τους λογαριασμούς 18.00.02, 18.00.03, 18.00.06 και 18.00.07 ή από τους λογαριασμούς 18.01.02, 18.01.03, 18.01.06 και 18.01.07, κατά περίπτωση, στους λογαριασμούς 18.00.00, 18.00.01 18.00.04 και 18.00.05 ή στους λογαριασμούς 18.01.00, 18.01.01, 18.01.04 και 18.01.05.

Στο λογαριασμό 18.00.19 ή 18.01.19 παρακολουθούνται οι προβλέψεις για υποτιμήσεις συμμετοχών σε λοιπές (πλην Α.Ε.) επιχειρήσεις, με χρέωση του λογαριασμού 68.01.

Σε περίπτωση πωλήσεως συμμετοχών, το τίμημα πωλήσεως καταχωρείται στην πίστωση του οικείου υπολογαριασμού του 18.00 ή 18.01, στην οποία μεταφέρεται και η τυχόν σχηματισμένη πρόβλεψη (από το λογαριασμό 18.00.19 ή 18.01.19), όταν πρόκειται για συμμετοχές σε λοιπές (πλην Α.Ε.) επιχειρήσεις, το αποτέλεσμα δε που προκύπτει καταχωρείται στο λογαριασμό 64.12.00 ή 64.12.01, κατά περίπτωση, αν πρόκειται για ζημία, και στο λογαριασμό 76.04.00 ή 76.04.01, κατά περίπτωση, αν πρόκειται για κέρδος.

Για την αποτίμηση των συμμετοχών και χρεογράφων ισχύουν τα εξής:

\begin{enumerate}

\item Με την επιφύλαξη της παρακάτω υποπεριπτώσεως στ. αυτής της περιπτώσεως, οι συμμετοχές σε ανώνυμες εταιρείες, καθώς και οι μετοχές ανώνυμων εταιρειών και οι κάθε φύσεως τίτλοι χρεογράφων με χαρακτήρα ακινητοποιήσεων, αποτιμούνται μαζί με τα χρεόγραφα, που παρακολουθούνται στο λογαριασμό 34, στη συνολικά χαμηλότερη τιμή μεταξύ της τιμής κτήσεως και της τρέχουσας τιμής τους. Στην περίπτωση αποτιμήσεως στην τρέχουσα τιμή, αυτή θεωρείται για τη νέα χρήση σαν τιμή κτήσεως.

Ως τρέχουσα τιμή ορίζεται:

\begin{enumerate}

\item Για τους εισαγμένους στο χρηματιστήριο τίτλους (μετοχές, ομολογίες κλπ), ο μέσος όρος της επίσημης τιμής τους κατά τον τελευταίο μήνα της χρήσεως.

\item Για τους μη εισαγμένους στο χρηματιστήριο τίτλους:

\begin{itemize}
\item Αν πρόκειται για μετοχές ανώνυμων εταιρειών, η εσωτερική λογιστική αξία που προκύπτει από το δημοσιευμένο τελευταίο ισολογισμό της εταιρείας, για τις μετοχές της οποίας πρόκειται να γίνει η αποτίμηση.

\item Αν πρόκειται για τους λοιπούς, εκτός από τις μετοχές, τίτλους, η τιμή κτήσεώς τους.

\item Αν πρόκειται για μερίδια αμοιβαίων κεφαλαίων, ο μέσος όρος της καθαρής τιμής τους κατά τον τελευταίο μήνα της χρήσεως.

\end{itemize}

\end{enumerate}
\item Όταν η συνολική τρέχουσα τιμή είναι χαμηλότερη από την αντίστοιχη τιμή κτήσεως, η διαφορά μεταξύ τους καταχωρείται σε χρέωση του λογαριασμού 64.11 «διαφορές αποτιμήσεως συμμετοχών και χρεογράφων», με χρεοπίστωση των οικείων λογαριασμών συμμετοχών και χρεογράφων.

\item Με την επιφύλαξη της παρακάτω υποπεριπτώσεως στ. αυτής της περιπτώσεως, οι συμμετοχές σε άλλες επιχειρήσεις, που δεν έχουν τη μορφή ανώνυμης εταιρείας, και οι τυχόν τίτλοι με χαρακτήρα ακινητοποιήσεων των επιχειρήσεων αυτών, αποτιμούνται στην κατ' είδος χαμηλότερη τιμή, μεταξύ της τιμής κτήσεως και της τρέχουσας τιμής. Ως τρέχουσα τιμή ορίζεται η εσωτερική λογιστική αξία που προκύπτει από τον τελευταίο ισολογισμό της επιχειρήσεως, για τις συμμετοχές της οποίας πρόκειται να γίνει η αποτίμηση.

\item Οι διαφορές που προκύπτουν από την αποτίμηση της παραπάνω περιπτώσεως (γ) καταχωρούνται σε χρέωση του λογαριασμού 68.01 «προβλέψεις για υποτιμήσεις συμμετοχών σε λοιπές (πλην Α.Ε.) επιχειρήσεις», με αντίστοιχη πίστωση του λογαριασμού 18.00.19 ή του 18.01.19 «προβλέψεις για υποτιμήσεις συμμετοχών σε λοιπές (πλην Α.Ε.) επιχειρήσεις».

\item Στην περίπτωση που ο τελευταίος δημοσιευμένος ισολογισμός, με βάση τον οποίο έγινε η αποτίμηση των μη εισαγμένων στο χρηματιστήριο μετοχών ή συμμετοχών, δεν έχει ελεγχθεί από αναγνωρισμένο κατά νόμο λογιστή, στο κάτω μέρος του ισολογισμού της οικονομικής μονάδας που έχει τις συμμετοχές και τα χρεόγραφα αναγράφεται σημείωση που αναφέρει ότι, στις συμμετοχές και στα χρεόγραφα περιλαμβάνονται και μετοχές ανώνυμων εταιρειών ή συμμετοχές σε λοιπές (πλην Α.Ε.) επιχειρήσεις, αξίας κτήσεως (ή αποτιμήσεως) δρχ. ....................  και .................., αντίστοιχα, μη εισαγμένες στο χρηματιστήριο, και ότι ο ισολογισμός, με βάση τον οποίο έγινε ο προσδιορισμός της εσωτερικής λογιστικής αξίας αυτών των μετοχών και συμμετοχών, δεν έχει ελεγχθεί από αναγνωρισμένο κατά νόμο λογιστή.

\item Τα κάθε φύσεως χρεόγραφα και τίτλοι, που έχουν χαρακτήρα προθεσμιακής καταθέσεως και δεν είναι εισαγμένα στο χρηματιστήριο, όπως είναι τα έντοκα γραμμάτια δημοσίου, αποτιμούνται στην κατ' είδος παρούσα αξία τους κατά την ημέρα κλεισίματος του ισολογισμού. Η αξία αυτή προσδιορίζεται με βάση το ετήσιο επιτόκιο του κάθε χρεογράφου ή τίτλου.

\end{enumerate}

Στους λογαριασμούς 18.02 έως και 18.16 παρακολουθούνται οι μακροπρόθεσμες απαιτήσεις της οικονομικής μονάδας (δηλαδή οι απαιτήσεις για τις οποίες η προθεσμία εξοφλήσεως λήγει μετά από το τέλος της επόμενης χρήσεως).

Οι λοιπές απαιτήσεις (δηλαδή εκείνες των οποίων η προθεσμία εξοφλήσεως λήγει μέχρι το τέλος της επόμενης του ισολογισμού χρήσεως), παρακολουθούνται στους οικείους λογαριασμούς των βραχυπρόθεσμων απαιτήσεων της ομάδας 3 του Σχεδίου Λογαριασμών.

Κατά την κατάρτιση του ισολογισμού: 

\begin{enumerate}

\item κάθε μακροπρόθεσμη απαίτηση, η οποία έχει καταστεί βραχυπρόθεσμη, μεταφέρεται στον οικείο λογαριασμό των βραχυπρόθεσμων απαιτήσεων, 

\item τα ποσά των μακροπρόθεσμων απαιτήσεων που είναι εισπρακτέα μέσα στη νέα χρήση μεταφέρονται από τους λογαριασμούς 18.02-18.14 στους λογαριασμούς 33.19 και 33.20 «μακροπρόθεσμες απαιτήσεις εισπρακτέες στην επόμενη χρήση» (σε δρχ. και σε Ξ.Ν.) και επαναφέρονται στους πρώτους κατά την έναρξη της νέας χρήσεως, εφόσον η επαναφορά είναι επιθυμητή (για την ενότητα της παρακολουθήσεως). Παρέχεται η δυνατότητα να μη διενεργούνται σχετικές λογιστικές εγγραφές (μεταφοράς και επαναφοράς), αλλά μόνο διαχωρισμός για την εμφάνιση των σχετικών κονδυλίων στις βραχυπρόθεσμες απαιτήσεις του ισολογισμού.

\end{enumerate}

Κάθε απαίτηση που εμφανίζεται στους λογαριασμούς 18.02-18.14, μόλις γίνει επισφαλής, μεταφέρεται στους οικείους λογαριασμούς της ομάδας 3 του Σχεδίου Λογαριασμών, με τους οποίους παρακολουθούνται οι επισφαλείς απαιτήσεις.

Για την εφαρμογή του Γενικού Λογιστικού Σχεδίου και των Κλαδικών Λογιστικών Σχεδίων, συνδεμένες επιχειρήσεις είναι οι ακόλουθες:

\begin{enumerate}

\item Οι επιχειρήσεις εκείνες μεταξύ των οποίων υπάρχει σχέση μητρικής προς θυγατρική. Στην περίπτωση που υπάρχει η σχέση αυτή, η θυγατρική επιχείρηση είναι συνδεμένη με τη μητρική και η μητρική είναι συνδεμένη με τη θυγατρική.

Σχέση μητρικής επιχειρήσεως προς θυγατρική υπάρχει όταν μία επιχείρηση (μητρική):

\begin{itemize}

\item έχει την πλειοψηφία του κεφαλαίου ή των δικαιωμάτων ψήφου μιας άλλης επιχειρήσεως (θυγατρικής), έστω και αν η πλειοψηφία αυτή σχηματίζεται ύστερα από συνυπολογισμό των τίτλων και δικαιωμάτων που κατέχονται από τρίτους για λογαριασμό της μητρικής επιχειρήσεως ή έχει αποκτηθεί ύστερα από συμφωνία με άλλο μέτοχο ή εταίρο της θυγατρικής αυτής επιχειρήσεως, ή

\item συμμετέχει στο κεφάλαιο μιας άλλης επιχειρήσεως και έχει το δικαίωμα, είτε άμεσα, είτε μέσω τρίτων, να διορίζει ή να παύει την πλειοψηφία των μελών των οργάνων διοικήσεως της επιχειρήσεως αυτής (θυγατρικής), ή

\item ασκεί δεσπόζουσα επιρροή σε μια άλλη επιχείρηση (θυγατρική).

\end{itemize}
Δεσπόζουσα επιρροή υπάρχει όταν η μητρική επιχείρηση διαθέτει, άμεσα ή έμμεσα, τουλάχιστο το 20\% του κεφαλαίου ή των δικαιωμάτων ψήφου της θυγατρικής και ασκεί κυριαρχική επιρροή στη διοίκηση ή τη λειτουργία της τελευταίας.

Για τον υπολογισμό της παραπάνω πλειοψηφίας του κεφαλαίου ή των δικαιωμάτων ψήφου, στο ποσοστό της συμμετοχής της μητρικής επιχειρήσεως σε μια άλλη επιχείρηση προσθέτονται και τα ποσοστά του κεφαλαίου ή των δικαιωμάτων ψήφου της άλλης αυτής επιχειρήσεως, που κατέχονται από άλλη ή άλλες θυγατρικές επιχειρήσεις.

\item Οι συνδεμένες επιχειρήσεις της προηγούμενης περιπτώσεως 10-α και κάθε μία από τις θυγατρικές ή τις θυγατρικές των θυγατρικών των συνδεμένων αυτών επιχειρήσεων.

\item οι θυγατρικές επιχειρήσεις των προηγούμενων περιπτώσεων 10-α και 10-β, άσχετα αν μεταξύ τους δεν υπάρχει απευθείας δεσμός συμμετοχής.

\end{enumerate}

Στους λογαριασμούς 18.02 και 18.03 παρακολουθούνται οι μακροπρόθεσμες απαιτήσεις κατά συνδεμένων επιχειρήσεων των παραπάνω περιπτώσεων 10-α, 10-β και 10-γ.

Στους λογαριασμούς 18.04 και 18.05 παρακολουθούνται οι μακροπρόθεσμες απαιτήσεις κατά των λοιπών επιχειρήσεων στις οποίες η οικονομική μονάδα έχει συμμετοχικό ενδιαφέρον λόγω του ότι διαθέτει συμμετοχές της φύσεως του λογαριασμού 18.01.

Στο λογαριασμό 18.06 «μακροπρόθεσμες απαιτήσεις κατά εταίρων» παρακολουθούνται οι απαιτήσεις της κατηγορίας αυτής που η οικονομική μονάδα έχει κατά των εταίρων της.

Στους λογαριασμούς 18.07 γραμμάτια εισπρακτέα μακροπρόθεσμα και 18.08 γραμμάτια εισπρακτέα μακροπρόθεσμα σε Ξ.Ν. είναι δυνατό να παρακολουθούνται τα γραμμάτια εισπρακτέα που η λήξη τους υπερβαίνει τους δώδεκα (12) μήνες από το τέλος της χρήσεως του ισολογισμού. Σχετικά με την επαναφορά στους αρμόδιους λογαριασμούς της ομάδας 3 των βραχυπρόθεσμης λήξεως γραμματίων εισπρακτέων, ισχύουν όσα αναφέρονται στην παραπάνω περίπτ. 8.

Τα γραμμάτια εισπρακτέα που η λήξη τους υπερβαίνει τους δώδεκα (12) μήνες από το τέλος της χρήσεως του ισολογισμού, είναι δυνατό να παρακολουθούνται στους οικείους δευτεροβάθμιους λογαριασμούς του 31 «γραμμάτια εισπρακτέα», εφόσον συντρέχουν οι ακόλουθες προϋποθέσεις:

\begin{enumerate}

\item Στο τέλος κάθε χρήσεως, το άληκτο υπόλοιπο του λογαριασμού 31 διαχωρίζεται, εξωλογιστικά, σε δύο μέρη - λήξεως μέχρι δώδεκα (12) μήνες και λήξεως πέρα από δώδεκα (12) μήνες.

\item Το μέχρι δώδεκα (12) μήνες άληκτο υπόλοιπο εμφανίζεται στον ισολογισμό στις βραχυπρόθεσμες απαιτήσεις με τον τίτλο «γραμμάτια εισπρακτέα».

\item Το πέρα από δώδεκα (12) μήνες άληκτο υπόλοιπο εμφανίζεται στον ισολογισμό στις μακροπρόθεσμες απαιτήσεις με τον τίτλο «γραμμάτια εισπρακτέα μακροπρόθεσμης λήξεως».

\item Οι τραπεζικές χρηματοδοτήσεις που είναι εγγυημένες με γραμμάτια εισπρακτέα και αντιστοιχούν στο πέρα από δώδεκα (12) μήνες άληκτο υπόλοιπο του λογαριασμού 31, διαχωρίζονται κατ' αναλογία των δύο μερών - λήξεως μέχρι 12 μήνες και λήξεως πέρα από 12 μήνες - ή κατά άλλο προσφορότερο τρόπο. Το δεύτερο αυτό τμήμα των τραπεζικών χρηματοδοτήσεων στον ισολογισμό εμφανίζεται στις μακροπρόθεσμες υποχρεώσεις με τον τίτλο «Τράπεζες λογαριασμοί μακροπρόθεσμων χρηματοδοτήσεων με εγγύηση γραμματίων εισπρακτέων».

\end{enumerate}

Στους λογαριασμούς 18.09 «μη δουλευμένοι τόκοι γραμματίων εισπρακτέων μακροπρόθεσμων σε Δρχ.» και 18.10 «μη δουλευμένοι τόκοι γραμματίων εισπρακτέων μακροπρόθεσμων σε Ξ.Ν.» είναι δυνατό να παρακολουθούνται οι μη δουλευμένοι τόκοι των γραμματίων εισπρακτέων των κατηγοριών αυτών. Σχετικά με τον υπολογισμό των μη δουλευμένων αυτών τόκων ισχύουν όσα αναφέρονται στην περίπτ.  6 της παρ. 2.2.302.

Όταν εφαρμόζεται η παραπάνω περίπτωση 13, ο διαχωρισμός των μη δουλευμένων τόκων γίνεται όπως ορίζεται στην περίπτωση αυτή σχετικά με το διαχωρισμό των γραμματίων εισπρακτέων.

Στο λογαριασμό 18.11 «δοσμένες εγγυήσεις» παρακολουθούνται τα ποσά που καταβάλλονται ως εγγύηση, όταν η επιστροφή τους δεν προβλέπεται να πραγματοποιηθεί μέχρι το τέλος της επόμενης χρήσεως (π.χ. εγγυήσεις στη ΔΕΗ, στον ΟΤΕ ή σε εκμισθωτές ακινήτων).

Στο λογαριασμό 18.12 «οφειλόμενο κεφάλαιο» παρακολουθούνται οι, μετά το τέλος της επόμενης του ισολογισμού χρήσεως, καταβλητέες δόσεις του οφειλόμενου κεφαλαίου της οικονομικής μονάδας το οποίο έχει κληθεί να καταβληθεί, καθώς και το οφειλόμενο κεφάλαιο που δεν έχει κληθεί να καταβληθεί, σύμφωνα με όσα ορίζονται στην περίπτ. 8β της παρ. 2.2.401.

Στους λογαριασμούς 18.13 «λοιπές μακροπρόθεσμες απαιτήσεις σε Δρχ.» και 18.14 «λοιπές μακροπρόθεσμες απαιτήσεις σε Ξ.Ν.» παρακολουθούνται οι μακροπρόθεσμες απαιτήσεις της οικονομικής μονάδας που δεν εντάσσονται σε μία από τις κατηγορίες των λογαριασμών 18.00 έως και 18.12 και 18.15 έως και 18.16.

Στους λογαριασμούς 18.15 «τίτλοι με χαρακτήρα ακινητοποιήσεων σε δρχ.» και 18.16 «τίτλοι με χαρακτήρα ακινητοποιήσεων σε Ξ.Ν.» παρακολουθούνται οι μακροπρόθεσμες τοποθετήσεις κεφαλαίων, για τις οποίες εκδίδονται τίτλοι διάφοροι από εκείνους που εντάσσονται στις συμμετοχές του λογαριασμού 18.00 ή του 18.01.

\section{19.ΠΑΓΙΟ ΕΝΕΡΓΗΤΙΚΟ ΥΠΟΚΑΤΑΣΤΗΜΑΤΩΝ ή ΑΛΛΩΝ ΚΕΝΤΡΩΝ}

Στις περιπτώσεις κατά τις οποίες τα υποκαταστήματα ή άλλα κέντρα (π.χ.  εργοστάσια ή καταστήματα) των οικονομικών μονάδων δεν έχουν λογιστική αυτοτέλεια, παρέχεται η δυνατότητα αναπτύξεως των λογαριασμών τους στους ομίλους λογαριασμών 19, 29, 39, 49, 59, 69, 79, και 99, καθώς και 09 υπό ορισμένες προϋποθέσεις, αντί της αναπτύξεως των λογαριασμών αυτών στους πρωτοβάθμιους λογαριασμούς των ομάδων 1, 2, 3, 4, 5, 6, 7, 9 και 10 (0), αντίστοιχα.

Η ανάπτυξη των λογαριασμών του ομίλου 19 (καθώς και των ομίλων 29, 39, 49, 59, 69, 79, 99 και 09), αναφορικά με τους πρωτοβάθμιους και τους υποχρεωτικούς δευτεροβάθμιους λογαριασμούς, ακολουθεί υποχρεωτικά το Σχέδιο Λογαριασμών, ώστε να είναι ευχερής η συγκέντρωση των πληροφοριών σύμφωνα με όσα προβλέπονται από το Σχέδιο αυτό.

Τόσο για τον όμιλο λογαριασμών 19, όσο και για τους λοιπούς ομίλους (29, 39, 49, 59, 69, 79, 99 και 09), η ανάπτυξη των πρωτοβάθμιων λογαριασμών των υποκαταστημάτων ή των λοιπών κέντρων υποχρεωτικά είναι αντίστοιχη με την ανάπτυξη των πρωτοβάθμιων λογαριασμών της αντίστοιχης ομάδας.

Ο τρόπος αναπτύξεως κάθε πρωτοβάθμιου λογαριασμού από τους ανωτέρω (190-198) αφήνεται στην κρίση της οικονομικής μονάδας, με τον περιορισμό όμως ότι στους δευτεροβάθμιους λογαριασμούς, στους οποίους θα αναπτύσσονται οι πρωτοβάθμιοι, θα περιλαμβάνονται τουλάχιστον οι υποχρεωτικοί δευτεροβάθμιοι και τριτοβάθμιοι λογαριασμοί του Σχεδίου Λογαριασμών.

Σχετικά με τον τρόπο λειτουργίας κλπ. των πρωτοβάθμιων λογαριασμών 190-198 ισχύουν, αντίστοιχα, όσα ορίζονται παραπάνω στις παρ. 2.2.100 έως και 2.2.112 για τους πρωτοβάθμιους λογαριασμούς 10-18.

Σε περίπτωση που η οικονομική μονάδα κάνει χρήση του ομίλου λογαριασμών 19, τα κονδύλια των λογαριασμών του ομίλου αυτού, στον ισολογισμό τέλους χρήσεως, συναθροίζονται και εμφανίζονται μαζί με τα αντίστοιχα κονδύλια των λογαριασμών 10-18.

\chapter{ΑΠΟΘΕΜΑΤΑ}

\section{Λογαριασμοί}

\begin{tabularx}{\linewidth}{lX}

20 & Εμπορεύματα\\
21 & Προϊόντα έτοιμα και ημιτελή\\
22 & Υποπροϊόντα και υπολείμματα\\
23 & Παραγωγή σε εξέλιξη (προϊόντα, υποπροϊόντα και υπολείμματα στo στάδιο της κατεργασίας)\\
24 & Πρώτες και βοηθητικές ύλες - Υλικά συσκευασίας\\
25 & Αναλώσιμα υλικά\\
26 & Ανταλλακτικά πάγιων στοιχείων\\
28 & Είδη συσκευασίας\\
29 & Αποθέματα υποκαταστημάτων ή άλλων κέντρων\\

\end{tabularx}

\section{Περιεχόμενο και εννοιολογικοί προσδιορισμοί}

Στην ομάδα 2 παρακολουθούνται τα αποθέματα της οικονομικής μονάδας που προέρχονται, είτε από απογραφή, είτε από αγορά, είτε από ιδιοπαραγωγή και, σε εξαιρετικές περιπτώσεις, είτε από ανταλλαγή, είτε από εισφορά σε είδος, είτε από δωρεά.

Αποθέματα είναι τα υλικά αγαθά που ανήκουν στην οικονομική μονάδα, τα οποία: 
\begin{enumerate}
\item προορίζονται να πωληθούν κατά τη συνήθη πορεία των εργασιών της, 
\item βρίσκονται στη διαδικασία της παραγωγής και προορίζονται να πωληθούν όταν πάρουν τη μορφή των έτοιμων προϊόντων, 
\item προορίζονται να αναλωθούν για την παραγωγή έτοιμων αγαθών ή την παροχή υπηρεσιών, 
\item προορίζονται να αναλωθούν για την καλή λειτουργία, τη συντήρηση ή επισκευή, καθώς και την ιδιοπαραγωγή πάγιων στοιχείων, 
\item προορίζονται να χρησιμοποιηθούν για τη συσκευασία π.χ. των παραγόμενων έτοιμων προϊόντων ή των εμπορευμάτων που προορίζονται για πώληση.

\end{enumerate}

Στην ομάδα 2 περιλαμβάνονται οι εξής μερικότερες κατηγορίες αποθεμάτων:

\begin{enumerate}


\item Εμπορεύματα (λογαριασμός 20): Είναι τα υλικά αγαθά (αντικείμενα, ύλες, υλικά) που αποκτούνται από την οικονομική μονάδα με σκοπό να μεταπωλούνται στην κατάσταση που αγοράζονται.

\item Έτοιμα προϊόντα (λογαριασμός 21): Είναι τα υλικά αγαθά που παράγονται, κατασκευάζονται ή συναρμολογούνται από την οικονομική μονάδα με σκοπό την πώλησή τους.

\item Ημιτελή προϊόντα (λογαριασμός 21): Είναι τα υλικά αγαθά που, μετά από κατεργασία σε ορισμένο στάδιο (ή στάδια), είναι έτοιμα για παραπέρα βιομηχανοποίηση (ή κατεργασία) ή για πώληση στην ημιτελή τους κατάσταση.

\item Υποπροϊόντα (λογαριασμός 22): Είναι τα υλικά αγαθά (προϊόντα) που παράγονται μαζί με τα κύρια προϊόντα, σε διάφορα στάδια της παραγωγικής διαδικασίας, από τις ίδιες πρώτες και βοηθητικές ύλες. Τα υποπροϊόντα επαναχρησιμοποιούνται από την οικονομική μονάδα σαν πρώτη ύλη ή πωλούνται αυτούσια.

\item Υπολείμματα (λογαριασμός 22): Είναι υλικά κατάλοιπα της παραγωγικής διαδικασίας, κατά κανόνα άχρηστα. Τα υπολείμματα, όταν, σαν άχρηστα, απορρίπτονται, αντιπροσωπεύουν μέρος της βιομηχανικής απώλειας (π.χ. φύρας).

Στην κατηγορία των υπολειμμάτων (λογαριασμός 22) εντάσσονται και τα ακατάλληλα για βιομηχανοποίηση ή κανονική αξιοποίηση διάφορα υλικά ή έτοιμα ή ημιτελή προϊόντα.

\item Παραγωγή σε εξέλιξη (λογαριασμός 23): Είναι πρώτες ύλες, βοηθητικά υλικά, ημιτελή προϊόντα και άλλα στοιχεία (π.χ. εργασία, γενικά βιομηχανικά έξοδα), τα οποία κατά τη διάρκεια της χρήσεως ή στο τέλος αυτής, κατά την απογραφή, βρίσκονται στο κύκλωμα της παραγωγικής διαδικασίας για κατεργασία.

\item Πρώτες και βοηθητικές ύλες (λογαριασμός 24): Είναι τα υλικά αγαθά που η οικονομική μονάδα αποκτάει με σκοπό τη βιομηχανική επεξεργασία ή συναρμολόγησή τους για την παραγωγή ή κατασκευή προϊόντων.

\item Υλικά συσκευασίας (λογαριασμός 24): Είναι τα υλικά αγαθά που η οικονομική μονάδα αποκτάει με σκοπό τη χρησιμοποίησή τους για τη συσκευασία των προϊόντων της, ώστε τα τελευταία να φτάνουν στην κατάσταση εκείνη στην οποία είναι δυνατό ή σκόπιμο να προσφέρονται στην πελατεία.

\item Αναλώσιμα υλικά (λογαριασμός 25): Είναι τα υλικά αγαθά που η οικονομική μονάδα αποκτάει με προορισμό την ανάλωσή τους για συντήρηση του πάγιου εξοπλισμού της και γενικά για την εξασφάλιση των αναγκαίων συνθηκών λειτουργίας των κύριων και βοηθητικών υπηρεσιών της.

\item Ανταλλακτικά πάγιων στοιχείων (λογαριασμός 26): Είναι τα υλικά αγαθά που η οικονομική μονάδα αποκτάει με σκοπό την ανάλωσή τους για συντήρηση και επισκευή του πάγιου εξοπλισμού της.

\item Είδη συσκευασίας (λογαριασμός 28): Είναι τα υλικά μέσα που χρησιμοποιούνται από την οικονομική μονάδα για τη συσκευασία εμπορευμάτων ή προϊόντων της και παραδίνονται στους πελάτες μαζί με το περιεχόμενό τους. Τα είδη συσκευασίας είναι επιστρεπτέα ή μη επιστρεπτέα, ανάλογα με τη συμφωνία που γίνεται κατά την πώληση σχετικά με την επιστροφή τους ή μη.

\end{enumerate}

Πρώτες και βοηθητικές ύλες, υλικά συσκευασίας, αναλώσιμα υλικά, ανταλλακτικά πάγιων στοιχείων και είδη συσκευασίας που αγοράζονται ή παράγονται από την οικονομική μονάδα με σκοπό να μεταπωλούνται, θεωρούνται σαν εμπορεύματα ή έτοιμα προϊόντα και παρακολουθούνται, αντίστοιχα, στους λογαριασμούς 20 ή 21.

\section{Τρόπος αναπτύξεως των λογαριασμών αποθεμάτων}

Οι πρωτοβάθμιοι λογαριασμοί αποθεμάτων (20-28), οι οποίοι είναι υποχρεωτικοί, αναπτύσσονται σε δευτεροβάθμιους, τριτοβάθμιους ή αναλυτικότερους λογαριασμούς, σύμφωνα με τις πληροφοριακές ανάγκες κάθε οικονομικής μονάδας.

Οι αγορές που πραγματοποιούνται κατά τη διάρκεια της χρήσεως, καθώς και τα αρχικά και τελικά αποθέματα, για κάθε κατηγορία αγαθών των λογαριασμών 20-28, παρακολουθούνται υποχρεωτικά σε χωριστούς δευτεροβάθμιους, τριτοβάθμιους ή αναλυτικότερους λογαριασμούς.

Ακολουθεί, ενδεικτικά, η ανάπτυξη του λογαριασμού 20.

20    ΕΜΠΟΡΕΥΜΑΤΑ

20.00   Είδος Α'

20.00.00    Αποθέματα

20.00.01    Αγορές χρήσεως

20.00.02    Εκπτώσεις αγορών (είδους Α)

20.01    Είδος Β'

20.01.00 Αποθέματα

20.01.01 Αγορές χρήσεως

20.01.02 Εκπτώσεις αγορών (είδους Β)

20.02    Είδος Γ'

  κ.ο.κ.

20.98    Εκπτώσεις αγορών (για περισσότερα από ένα είδη)

20.99    Προϋπολογισμένες αγορές (Λ/58.14)

Το τελευταίο όριο αναλύσεως για καθένα από τους πρωτοβάθμιους λογαριασμούς 20-28 είναι η μερίδα αποθήκης, εκτός αν οι μερίδες αποθήκης (διαρκής απογραφή) εξυπηρετούνται στην ομάδα 9 της αναλυτικής λογιστικής (οικείοι υπολογαριασμοί του λογαριασμού 94 «αποθέματα»). Το αναλυτικό περιεχόμενο της μερίδας αποθήκης περιγράφεται στην παρ. 5.215 του πέμπτου μέρους.

\section{Λειτουργία των λογαριασμών αποθεμάτων}

Οι λογαριασμοί 20-28 λειτουργούν σύμφωνα με τα παρακάτω:

\begin{itemize}


\item Κατά την έναρξη της χρήσεως χρεώνονται (οι ειδικοί υπολογαριασμοί αποθεμάτων) με την αξία των αποθεμάτων της προηγούμενης απογραφής.

\item Κατά τη διάρκεια της χρήσεως χρεώνονται (οι ειδικοί υπολογαριασμοί αγορών) με την αξία κτήσεως των αγοραζόμενων αγαθών, όπως η αξία αυτή προσδιορίζεται παρακάτω στην παρ. 2.2.203, όπου περιγράφεται αναλυτικότερα και ο τρόπος λειτουργίας των λογαριασμών αποθεμάτων κατά την αγορά των υλικών αγαθών, και πιστώνονται με τις τυχόν επιστροφές αγορών και τις εκτός τιμολογίου εκπτώσεις.

\item Κατά το τέλος της χρήσεως πιστώνονται, με χρέωση του λογαριασμού 80.00 «λογαριασμός γενικής εκμεταλλεύσεως», με την αξία των αρχικών αποθεμάτων και την αξία των καθαρών, μετά την αφαίρεση των επιστροφών και των εκτός τιμολογίου εκπτώσεων, αγορών της χρήσεως, και χρεώνονται, με πίστωση του αυτού λογαριασμού 80.00, με την αξία των τελικών αποθεμάτων, όπως η αξία αυτή προκύπτει κατά την αποτίμησή τους που διενεργείται σύμφωνα με τους κανόνες που καθορίζονται παρακάτω στην παρ. 2.2.205.

\end{itemize}

Ειδικά για τα επιστρεπτέα από τους πελάτες είδη συσκευασίας ορίζονται τα εξής:

\begin{enumerate}


\item Όταν η αξία τους περιλαμβάνεται στο τιμολόγιο πωλήσεως (ή στο δελτίο λιανικής πωλήσεως), χρεώνεται με αυτήν ο λογαριασμός 30.00 «πελάτες εσωτερικού» (ή 30.01 ή 30.02 ή 30.03 ή ο 38.00), με πίστωση του λογαριασμού 30.07 «πελάτες αντίθετος λογαριασμός αξίας ειδών συσκευασίας». Κατά την επιστροφή στην οικονομική μονάδα των ειδών συσκευασίας ενεργείται αντίστροφη εγγραφή. Σε περίπτωση που τα τιμολογημένα είδη συσκευασίας δεν επιστρέφονται μέσα στην καθορισμένη προθεσμία, χρεώνεται ο λογαριασμός 30.07, με την αξία με την οποία προηγούμενα είχε πιστωθεί, και πιστώνεται ο οικείος υπολογαριασμός «πωλήσεις ειδών συσκευασίας» του 72.

\item Όταν η αξία τους δεν τιμολογείται, αλλά μόνο η ποσότητά τους αναγράφεται στο τιμολόγιο πωλήσεως του περιεχομένου τους ή σε άλλο ιδιαίτερο στοιχείο (π.χ. δελτίο παραδόσεως ειδών συσκευασίας), η λογιστική τους παρακολούθηση γίνεται σύμφωνα με τις ιδιαίτερες συνθήκες κάθε οικονομικής μονάδας, με την προϋπόθεση ότι από τους λογαριασμούς που τηρούνται (λογιστικά ή εξωλογιστικά) προκύπτουν πάντοτε τα μη τιμολογημένα είδη συσκευασίας που βρίσκονται στα χέρια κάθε πελάτη.

\item Τα ποσά που η οικονομική μονάδα εισπράττει από τους πελάτες της για εγγύηση της επιστροφής των ειδών συσκευασίας, τα οποία παραδίνονται σ' αυτούς χωρίς να τιμολογούνται, καταχωρούνται σε πίστωση του λογαριασμού 30.04 «πελάτες εγγυήσεις ειδών συσκευασίας». Αν τα μη τιμολογημένα είδη συσκευασίας δεν επιστρέφονται από τους πελάτες μέσα στην καθορισμένη προθεσμία, για το ποσό της αποζημιώσεως, που η οικονομική μονάδα εξασφαλίζει για αποκατάσταση της ζημίας από την μη επιστροφή των ειδών αυτών, εκδίδεται τιμολόγιο πωλήσεως (ή δελτίο λιανικής πωλήσεως) και χρεώνεται ο προσωπικός λογαριασμός του πελάτη, με πίστωση του οικείου υπολογαριασμού «πωλήσεις ειδών συσκευασίας» του 72, το δε ποσό της σχετικής εγγυήσεως του πελάτη μεταφέρεται από την πίστωση του λογαριασμού 30.04 στην πίστωση του προσωπικού του λογαριασμού.

\item Τα ποσά που η οικονομική μονάδα ενδεχόμενα, εισπράττει από τους πελάτες της για τη χρησιμοποίηση απ' αυτούς των επιστρεπτέων ειδών συσκευασίας, καταχωρούνται σε πίστωση του λογαριασμού 74.98.01 «έσοδα από μερική χρησιμοποίηση ειδών συσκευασίας».

\end{enumerate}

Για τους λογαριασμούς «εκπτώσεις αγορών» και «προϋπολογισμένες αγορές» ισχύουν τα ακόλουθα:

\begin{enumerate}

\item Εκπτώσεις αγορών: Οι προαιρετικοί υπολογαριασμοί 20.98, 24.98, 25.98, 26.98 και 28.98 πιστώνονται με τις εκτός τιμολογίου χορηγούμενες εκπτώσεις επί αγορών, όταν η διάκρισή τους κατ' είδος αγορών είναι αδύνατη ή παρουσιάζει δυσκολίες.

\item Προϋπολογισμένες αγορές: Οι προαιρετικοί υπολογαριασμοί 20.99, 24.99, 25.99, 26.99 και 28.99, κατά το τέλος της περιόδου λογισμού (π.χ. στο τέλος του μήνα), χρεώνονται με τις προϋπολογισμένες αγορές (αξία αγορασμένων αγαθών που παραλαμβάνονται χωρίς τιμολόγιο ή άλλο παραστατικό αξίας κατά τη διάρκεια της χρήσεως), με πίστωση των αντίστοιχων και οικείων υπολογαριασμών του 58 «λογαριασμοί περιοδικής κατανομής».

\end{enumerate}

Στο τέλος της επόμενης περιόδου λογισμού (π.χ. στο τέλος του επόμενου μήνα ή της επόμενης τριμηνίας) ακυρώνονται οι εγγραφές των προϋπολογισμένων αγορών, οι οποίες έγιναν στο τέλος της προηγούμενης περιόδου λογισμού, και διενεργούνται νέες εγγραφές προϋπολογισμένων αγορών, σύμφωνα με τα στοιχεία της νέας (επόμενης) περιόδου λογισμού, κατά τον ίδιο τρόπο που ορίζεται παραπάνω.

Σε όλες τις περιπτώσεις που παρουσιάζεται ανάγκη προϋπολογισμού της αξίας αγορασμένων αγαθών, αντί της εφαρμογής της παραπάνω διαδικασίας, παρέχεται η δυνατότητα πιστώσεως του λογαριασμού 56.02 «αγορές υπό τακτοποίηση», σύμφωνα με όσα ορίζονται στην περίπτ. 7 της παρακάτω παρ. 2.2.203.

Οι σχετικοί με τη συλλειτουργία των παραπάνω λογαριασμών προϋπολογισμένων αγορών με τους οικείους υπολογαριασμούς του 58 κανόνες, περιλαμβάνονται στην παρ. 2.2.509.

\section{Αγορές}

Οι αγορές αποθεμάτων που πραγματοποιούνται κατά τη διάρκεια της χρήσεως καταχωρούνται στη χρέωση των λογαριασμών της ομάδας 2 (των ειδικών υπολογαριασμών αγορών) με την τιμή κτήσεώς τους, δηλαδή με την τιμολογιακή τους αξία προσαυξημένη με τα ειδικά έξοδα αγοράς.

Τιμολογιακή αξία είναι η αξία που αναγράφεται στα τιμολόγια αγοράς, μειωμένη κατά τις ενδεχόμενες εκπτώσεις που χορηγούνται από τους προμηθευτές και απαλλαγμένη από τα ποσά των φόρων και τελών τα οποία δεν βαρύνουν, τελικά, την οικονομική μονάδα.

Τα ποσά Φ.Κ.Ε. (ή φόρου προστιθέμενης αξίας, σε περίπτωση αντικαταστάσεως του φόρου κύκλου εργασιών με το φόρο προστιθέμενης αξίας) που καταβάλλονται κατά τις αγορές των προς βιομηχανοποίηση υλών, τα οποία, σύμφωνα με τη φορολογική νομοθεσία, συμψηφίζονται με οφειλές της οικονομικής μονάδας από Φ.Κ.Ε. που αντιστοιχεί στις πωλήσεις της, καταχωρούνται σε χρέωση του λογαριασμού 54.00 «φόρος κύκλου εργασιών». Οι δασμοί, φόροι και τέλη, που καταβάλλονται προσωρινά κατά την εισαγωγή από το εξωτερικό αγαθών, τα οποία προορίζονται για βιομηχανοποίηση και επανεξαγωγή, είναι δυνατό να καταχωρούνται σε χρέωση του λογαριασμού 33.14.01 «δασμοί και λοιποί φόροι εισαγωγής προς επιστροφή».

Τα ειδικά έξοδα αγορών, δηλαδή εκείνα που πραγματοποιούνται κατά τρόπο άμεσο για κάθε συγκεκριμένη αγορά μέχρι την παραλαβή και αποθήκευση των αγαθών (π.χ. οι δασμοί εισαγωγής ή τα έξοδα μεταφοράς και παραλαβής των αγαθών), καταχωρούνται απευθείας σε χρέωση των οικείων λογαριασμών της ομάδας 2. Τα έξοδα αυτά, αν για οποιοδήποτε λόγο καταχωρηθούν σε λογαριασμούς της ομάδας 6, μεταφέρονται με αντιλογισμό στους συγκεκριμένους λογαριασμούς της ομάδας 2, τους οποίους αφορούν. Στους λογαριασμούς της ομάδας 2 μεταφέρονται επίσης τα ειδικά έξοδα αγορών, τα οποία προηγούμενα έχουν συγκεντρωθεί σε υπολογαριασμούς του 32.01 «παραγγελίες κυκλοφορούντων στοιχείων».

Ο τρόπος υπολογισμού του κόστους των αγαθών που εισάγονται από το εξωτερικό καθορίζεται στην παρ. 2.3.1 «υπολογισμός της αξίας των αγορών από το εξωτερικό και των πωλήσεων στο εξωτερικό».

Σε περιπτώσεις που, για διάφορους λόγους, οι εκπτώσεις αγορών δεν είναι δυνατό να μειώνουν την τιμολογιακή αξία αγοράς, καταχωρούνται στους ιδιαίτερους, κατά κατηγορία αποθεμάτων, υπολογαριασμούς, είτε κατ' είδος αποθεμάτων, είτε για περισσότερα είδη όταν δεν είναι δυνατός ή όταν είναι δυσχερής ο διαχωρισμός τους στα επιμέρους είδη. Στο τέλος της χρήσεως, τα υπόλοιπα των υπολογαριασμών εκπτώσεων αγορών μεταφέρονται στους αντίστοιχους υπολογαριασμούς «αγορές χρήσεως». Όταν οι εκπτώσεις αγορών αναφέρονται σε περισσότερα από ένα είδη αποθεμάτων και ο διαχωρισμός τους κατ' είδος είναι αδύνατος ή δυσχερής, η κατανομή τους στα είδη αυτά γίνεται ανάλογα με την πριν από τις εκπτώσεις αξία κτήσεώς τους.

Η χρέωση των λογαριασμών αποθεμάτων με την αξία των αγαθών που αγοράζονται διενεργείται κατά την παραλαβή τους με βάση τα τιμολόγια και λοιπά δικαιολογητικά αγοράς. Σε περίπτωση που τα τιμολόγια ή τα λοιπά δικαιολογητικά αγοράς δεν περιέρχονται στην οικονομική μονάδα κατά την παραλαβή των αγαθών, χρεώνονται οι λογαριασμοί αποθεμάτων με τη συμφωνημένη αξία των αγαθών που παραλαμβάνονται, με πίστωση του λογαριασμού 56.02 «αγορές υπό τακτοποίηση», ο οποίος χρεώνεται αμέσως μετά τη λήψη του οικείου στοιχείου π.χ. τιμολογίου, με πίστωση του λογαριασμού του προμηθευτή ή των λογαριασμών των χρηματικών διαθεσίμων ή των οικείων υπολογαριασμών του 32. Ενδεχόμενη διαφορά μεταξύ της αξίας του τιμολογίου και εκείνης που λαμβάνεται υπόψη κατά τη χρέωση των λογαριασμών των αποθεμάτων, καταχωρείται στους οικείους λογαριασμούς αυτών, εκτός αν η τακτοποιητική εγγραφή γίνεται έπειται από το κλείσιμο του ισολογισμού, οπότε η διαφορά αυτή καταχωρείται στον οικείο υπολογαριασμό του 82 «έξοδα και έσοδα προηγούμενων χρήσεων».

Σε περιπτώσεις που, κατά τη διάρκεια της χρήσεως, περιέρχονται στην οικονομική μονάδα τιμολόγια πριν από την παραλαβή των αντίστοιχων αγαθών, δε διενεργούνται εγγραφές. Σε περιπτώσεις που, κατά το τέλος της χρήσεως, λαμβάνονται τιμολόγια για αγορές αγαθών που δεν έχουν παραληφθεί, αλλά έχουν φορτωθεί για λογαριασμό και με ευθύνη της οικονομικής μονάδας, πιστώνεται με την αξία τους ο προσωπικός λογαριασμός του προμηθευτή, με χρέωση του υπολογαριασμού 36.02 «αγορές υπό παραλαβή». Ο λογαριασμός 36.02 τακτοποιείται στην επόμενη χρήση, κατά την παραλαβή των αγαθών, με χρέωση των οικείων λογαριασμών της ομάδας 2. Αν τα τιμολόγια αφορούν αγορές αγαθών από το εξωτερικό που, κατά το τέλος της χρήσεως, δεν έχουν παραληφθεί, η οικονομική μονάδα μπορεί να μη διενεργεί σχετικές εγγραφές.

\section{Απογραφή}

Οι οικονομικές μονάδες είναι υποχρεωμένες να πραγματοποιούν πραγματικές (φυσικές) απογραφές των αποθεμάτων τους τουλάχιστο μία φορά μέσα σε κάθε χρήση και μάλιστα στο τέλος αυτής. Κατά την απογραφή πρέπει να αναγνωρίζονται, να καταμετρούνται και να καταγράφονται όλα τα αποθέματα κατ' είδος, ποιότητα και ποσότητα και να γίνεται η κατάταξη αυτών σε κατηγορίες που να αντιστοιχούν στους επιμέρους λογαριασμούς των αποθεμάτων. Είδη που βρίσκονται σε τρίτους για πώληση, για ενέχυρο ή για άλλους λόγους, καταχωρούνται ιδιαίτερα στην απογραφή.

Οι οικονομικές μονάδες που τηρούν τους λογαριασμούς αποθεμάτων κατά τη μέθοδο της διαρκούς απογραφής, σύμφωνα με όσα καθορίζονται στην παρ. 5.215 του πέμπτου μέρους, έχουν τη δυνατότητα, αντί να διενεργούν πραγματική απογραφή για όλα τα είδη κατά τη λήξη της χρήσεως, να εφαρμόζουν τη μέθοδο της περιοδικής απογραφής. Σύμφωνα με τη μέθοδο αυτή η απογραφή, για κάθε κατηγορία ειδών, γίνεται μέσα στη χρήση, αλλά σε καθορισμένους χρόνους που κρίνονται κατάλληλοι από την οικονομική μονάδα, με την προϋπόθεση ότι όλα τα είδη θα απογράφονται τουλάχιστο μία φορά μέσα στη χρήση.

Η αποτίμηση των ειδών που απογράφονται με τη μέθοδο της περιοδικής απογραφής γίνεται στο τέλος της χρήσεως, με βάση τις ποσότητες που προκύπτουν από τα λογιστικά δεδομένα της τελευταίας εργάσιμης ημέρας.

\subsection{Αποτίμηση αποθεμάτων απογραφής}

\begin{enumerate}

\item Τιμές που πρέπει να εφαρμόζονται για την αποτίμηση των αποθεμάτων της απογραφής

Τα αποθέματα που προέρχονται από αγορές αποτιμούνται στην κατ' είδος χαμηλότερη τιμή μεταξύ τιμής κτήσεως και τρέχουσας τιμής αγοράς.

Τα αποθέματα (εκτός από τα υπολείμματα και υποπροϊόντα) που προέχονται από την παραγωγή της οικονομικής μονάδας και προορίζονται, είτε για πώληση ως έτοιμα προϊόντα, είτε για παραπέρα επεξεργασία προς παραγωγή έτοιμων προϊόντων, αποτιμούνται στην κατ' είδος χαμηλότερη τιμή μεταξύ τιμών ιστορικού κόστους παραγωγής και καθαρής ρευστοποιήσιμης αξίας.

Τα υπολείμματα αποτιμούνται στην πιθανή τιμή πωλήσεώς τους, μειωμένη με τα άμεσα έξοδα που υπολογίζεται ότι θα πραγματοποιηθούν για την πώλησή τους.

Τα υποπροϊόντα, εφόσον προορίζονται για πώληση, αποτιμούνται στην πιθανή τιμή πωλήσεώς τους, μειωμένη με τα άμεσα έξοδα πωλήσεως, όπως και στην περίπτωση των υπολειμμάτων. Όταν όμως προορίζονται να χρησιμοποιηθούν από την ίδια την οικονομική μονάδα, αποτιμούνται στην τιμή χρησιμοποιήσεώς τους, δηλαδή στην τιμή που θα αγοράζονταν, είτε τα συγκεκριμένα υποπροϊόντα, είτε άλλα ισοδύναμης αξίας, με σκοπό να χρησιμοποιηθούν από αυτή.

Οι οικονομικές μονάδες που εφαρμόζουν σύστημα πρότυπης κοστολογήσεως έχουν τη δυνατότητα να αποτιμούν τα αποθέματά τους στις τιμές του πρότυπου κόστους, με την προϋπόθεση ότι οι αποκλίσεις, που ενδεχόμενα θα προκύψουν ανάμεσα στο ιστορικό και στο πρότυπο κόστος, θα κατανέμονται στα απώλητα (μένοντα) και στα πωλημένα αποθέματά τους. Το ποσό των αποκλίσεων που αναλογεί στα απώλητα αποθέματα της απογραφής εμφανίζεται ιδιαίτερα.

\item Εννοιολογικός προσδιορισμός των τιμών και μεθόδων που εφαρμόζονται για την αποτίμηση των αποθεμάτων.

Τιμή κτήσεως: Είναι η τιμολογιακή αξία αγοράς των αποθεμάτων, αυξημένη με τα ειδικά έξοδα αγοράς και μειωμένη με τις εκπτώσεις (άμεσο κόστος αγοράς).

Τιμολογιακή αξία είναι η αξία αγοράς που αναγράφεται στα τιμολόγια, μειωμένη κατά τις εκπτώσεις που κάθε φορά χορηγούνται από τους προμηθευτές και απαλλαγμένη από τους φόρους και τα τέλη που δε βαρύνουν τελικά, την οικονομική μονάδα.

Ειδικά έξοδα αγοράς είναι τα άμεσα έξοδα αγοράς που γίνονται μέχρι την παραλαβή και αποθήκευση του αγαθού και ιδιαίτερα οι δασμοί και λοιποί φόροι-τέλη εισαγωγής, καθώς και τα έξοδα μεταφοράς και παραλαβής των σχετικών ειδών.

Μέθοδοι υπολογισμού της τιμής κτήσεως: Η τιμή κτήσεως υπολογίζεται με οποιαδήποτε από τις παρακάτω μεθόδους, καθώς και με οποιαδήποτε άλλη παραδεγμένη μέθοδο.

\begin{enumerate}


\item Η μέθοδος του μέσου σταθμικού κόστους: Κατά τη μέθοδο αυτή η μέση σταθμική τιμή κτήσεως υπολογίζεται με τον εξής τύπο:

Αξία αποθέματος
ενάρξεως της περιόδου
	

+
	

αξία αγορών της περιόδου
στην τιμή κτήσεως


Ποσότητα αποθέματος
ενάρξεως της περιόδου
	

+
	

ποσότητα που αγοράζεται
στην περίοδο

\item Η μέθοδος του κυκλοφοριακού μέσου όρου ή των διαδοχικών υπολοίπων: Κατά τη μέθοδο αυτή μετά από κάθε εισαγωγή καθορίζεται η μέση τιμή του υπολοίπου με τον εξής τύπο:

Αξία προηγούμενου υπολοίπου
	

+
	

αξία νέας αγοράς στην τιμή κτήσεως


Ποσότητα προηγούμενου υπολοίπου
	
+
	
ποσότητα νέας αγοράς

\item Η μέθοδος πρώτη εισαγωγή - πρώτη εξαγωγή (F.I.F.O.): Κατά τη μέθοδο αυτή θεωρείται ότι η πρώτη εισαγωγή (αγορά) εξάγεται πρώτη (First In - First Out) και ότι τα αποθέματα της απογραφής προέρχονται από τις τελευταίες αγορές της χρήσεως και αποτιμούνται στις τιμές που αντίστοιχα αγοράστηκαν. Η αρχή του σχετικού υπολογισμού γίνεται από την τελευταία αγορά.

\item Η μέθοδος τελευταία εισαγωγή - πρώτη εξαγωγή (L.I.F.O): Κατά τη μέθοδο αυτή θεωρείται ότι η πρώτη εξαγωγή προέρχεται από την τελευταία εισαγωγή (Last In - First Out) και ότι τα αποθέματα τέλους χρήσεως προέρχονται από τις παλαιότερες εισαγωγές. Η αρχή του σχετικού υπολογισμού γίνεται από την πρώτη αγορά της χρήσεως.

\item Η μέθοδος του βασικού αποθέματος: Κατά τη μέθοδο αυτή τα αποθέματα τέλους χρήσεως διακρίνονται σε δύο μέρη. Το ένα αντιστοιχεί στο βασικό απόθεμα που αντιπροσωπεύει την ελάχιστη ποσότητα (στοκ ασφαλείας) η οποία κρίνεται αναγκαία για την ομαλή διεξαγωγή της συνήθους δραστηριότητας της οικονομικής μονάδας. Το άλλο προορίζεται για εξυπηρέτηση μελλοντικών αναγκών πωλήσεων, όταν πρόκειται για εμπορεύματα ή έτοιμα προϊόντα, ή αναγκών βιομηχανοποιήσεων, όταν πρόκειται για υλικά που αναλώνονται στην παραγωγική διαδικασία. Το βασικό απόθεμα αποτιμάται στην αξία της αρχικης κτήσεώς του. Το υπόλοιπο μέρος (υπεραπόθεμα) αποτιμάται με μία από τις παραπάνω (α - δ) μεθόδους υπολογισμού της τιμής κτήσεως.


Σε περίπτωση που η ποσότητα των κατ' είδος αποθεμάτων δε διακυμαίνεται σημαντικά από χρήση σε χρήση, είναι δυνατό να χαρακτηρίζεται ολόκληρη η ποσότητα αυτή σαν βασικό απόθεμα και ανάλογα να γίνεται η αποτίμησή της.

\item Η μέθοδος του εξατομικευμένου κόστους: Κατά τη μέθοδο αυτή τα αποθέματα παρακολουθούνται όχι μόνο κατ' είδος, αλλά και κατά συγκεκριμένες παρτίδες αγοράς ή παραγωγής, οι οποίες έτσι αποκτούν αυτοτέλεια κόστους (π.χ. παρτίδα μαλλιών, ακατέργαστων δερμάτων, πλαστικών πρώτων υλών, μερών μηχανημάτων). Κατά την αποτίμηση των αποθεμάτων της απογραφής, αυτά αναλύονται σε ποσότητες κατά παρτίδα από την οποία προέρχονται και αποτιμούνται στο κόστος της συγκεκριμένης παρτίδας, ανεξάρτητα από το χρόνο παραγωγής ή αγοράς τους.

\item Η μέθοδος του πρότυπου κόστους: Κατά τη μέθοδο αυτή τα αποθέματα αποτιμούνται στην τιμή του πρότυπου κόστους. Η μέθοδος του πρότυπου κόστους εφαρμόζεται με την προϋπόθεση της παραπάνω περιπτ. 5.

\end{enumerate}

Η οικονομική μονάδα έχει τη δυνατότητα να εφαρμόζει οποιαδήποτε από τις παραδεγμένες μεθόδους προσδιορισμού της τιμής κτήσεως, με την προϋπόθεση ότι τη μέθοδο που θα επιλέξει θα την εφαρμόζει κατά τρόπο πάγιο. Σε περιπτώσεις αλλαγής των συνθηκών ή υπάρξεως σοβαρών λόγων επιτρέπεται η αλλαγή της μεθόδου προσδιορισμού της τιμής κτήσεως, με την προϋπόθεση ότι στις δημοσιευόμενες οικονομικές καταστάσεις θα δηλώνονται οι λόγοι που οδήγησαν στην αλλαγή, καθώς και η επίδραση που είχε η αλλαγή αυτή στη διαμόρφωση των αποτελεσμάτων.

Τρέχουσα τιμή αγοράς: Είναι η τιμή αντικαταστάσεως του συγκεκριμένου αποθέματος, δηλαδή η τιμή στην οποία η οικονομική μονάδα έχει τη δυνατότητα να προμηθευτεί το αγαθό, κατά την ημέρα συντάξεως της απογραφής, από τη συνήθη αγορά, με συνήθεις όρους και κάτω από κανονικές συνθήκες, χωρίς να λαμβάνονται υπόψη περιπτωσιακά και προσωρινά γεγονότα που προκαλούν αδικαιολόγητες προσωρινές διακυμάνσεις τιμών. Η τρέχουσα τιμή αγοράς διαμορφώνεται με το συνυπολογισμό όλων των στοιχείων του κόστους κτήσεως.

Σε περίπτωση αδυναμίας προσδιορισμού της τρέχουσας τιμής, εφαρμόζεται η καθαρή ρευστοποιήσιμη αξία της παρακάτω περιπτ. 10.

Ιστορικό κόστος παραγωγής: Είναι το άμεσο κόστος αγοράς (η τιμή κτήσεως) των πρώτων υλών και των διάφορων υλικών που χρησιμοποιήθηκαν στην παραγωγή των αγαθών, προσαυξημένο με τα γενικά (έμμεσα) έξοδα αγορών, καθώς και με τα άμεσα και έμμεσα έξοδα παραγωγής (κόστος κατεργασίας) που δαπανήθηκαν για να φτάσουν τα παραγμένα αγαθά στη θέση και κατάσταση που βρίσκονται κατά την απογραφή.

Το ιστορικό κόστος παραγωγής υπολογίζεται με μία από τις μεθόδους που υπολογίζεται και η τιμή κτήσεως, κατά τον τρόπο που ορίζεται παραπάνω στην περίπτ. 7. 

Ο προσδιορισμός του ιστορικού κόστους παραγωγής γίνεται κατά τον τρόπο που περιγράφεται στις παρ. 5.213 και 5.214 του πέμπτου μέρους.

 Καθαρή ρευστοποιήσιμη αξία: Είναι η τιμή πωλήσεως του αποθέματος, στην οποία υπολογίζεται ότι αυτό θα πωληθεί κάτω από συνθήκες ομαλής πορείας των εργασιών της οικονομικής μονάδας, μειωμένη με το κόστος ολοκληρώσεως της επεξεργασίας (όταν πρόκειται για ημιτελή αποθέματα ή αποθέματα που βρίσκονται στο στάδιο της κατεργασίας) και με τα έξοδα που υπολογίζεται ότι θα πραγματοποιηθούν για την επίτευξη της πωλήσεως.

\item Η αποτίμηση των συμπαράγωγων προϊόντων

Συμπαράγωγα είναι τα προϊόντα που παράγονται από την επεξεργασία της αυτής πρώτης ύλης κατά τη διάρκεια της αυτής παραγωγικής διαδικασίας.

Τα συμπαράγωγα προϊόντα έχουν ενιαίο κόστος παραγωγής, το οποίο, μετά τη μείωσή του κατά την αξία αποτιμήσεως των τυχόν υποπροϊόντων και υπολειμμάτων, κατανέμεται μεταξύ αυτών με κριτήριο την αξία τους σε καθαρές τιμές πωλήσεως.

Παράδειγμα: Από τη βιομηχανοποίηση της πρώτης ύλης Υ1 συμπαράγονται τα προϊόντα Π1 και Π2. Κατά τη διάρκεια μιας κοστολογικής περιόδου (π.χ. στο μήνα Μάρτη) τα σχετικά απολογιστικά δεδομένα έχουν ως εξής:

Βιομηχανοποίηση πρώτης ύλης Υ1 μον. 110.000 Χ 10
	

Δρχ.
	

1.100.000

Κόστος κατεργασίας
	

»
	

 500.000

Συνολικό κόστος
	

»
	

1.600.000

Μείον αξία αποτιμήσεως υποπροϊόντων και υπλειμμάτων
	

»
	

100.000

Κόστος συμπαράγωγων προϊόντων
	

Δρχ.
	

1.500.000

 

Παραγωγή προϊόντος Π1 μον. 40.000 τιμής πωλήσεως δρχ. 25 κατά μονάδα

Παραγωγή προϊόντος Π2 μον. 50.000 τιμής πωλήσεως δρχ. 20 κατά μονάδα

Η κατανομή του ενωμένου κόστους γίνεται ως εξής:

Προϊόν Π1 μον. 40.000 Χ 25 = 1.000.000 Χ (1.500.000 : 2.000.000) = 750.000

Προϊόν Π2 μον. 50.000 Χ 20 = 1.000.000 Χ (1.500.000 : 2.000.000) = 750.000

Κόστος μονάδας προϊόντος Π1 750.000 : 40.000 = δρχ. 18,75

Κόστος μονάδας προϊόντος Π2 750.000 : 50.000 = δρχ. 15,00

\item Η αποτίμηση των ελαττωματικών προϊόντων

13. Ελαττωματικά είναι τα προϊόντα τα οποία, εξαιτίας ελαττωματικής παραγωγής ή κατασκευής, διαφέρουν από τα λοιπά κανονικά προϊόντα και πωλούνται με το χαρακτηρισμό του ελαττωματικού σε τιμή κατώτερη της κανονικής.

Τα ελαττωματικά προϊόντα, ανάλογα με τις περιπτώσεις που παρουσιάζονται, αποτιμούνται ως εξής:

\begin{enumerate}


\item Σε περίπτωση που το ελαττωματικό προϊόν πρόκειται να διατεθεί με το ελάττωμά του σε μικρότερη τιμή, η αποτίμησή του γίνεται στην πιθανή τιμή πωλήσεώς του. Το κόστος που προκύπτει με τον τρόπο αυτό μειώνει το συνολικό κόστος παραγωγής, η διαφορά δε αποτελεί το κόστος της παραγωγής του κανονικού ή των κανονικών προϊόντων.

\item Σε περίπτωση που το ελαττωματικό προϊόν πρόκειται να διατεθεί με το ελάττωμά του με μικρή έκπτωση, αποτιμάται (κοστολογείται) όπως και το κανονικό προϊόν. Στην περίπτωση δηλαδή αυτή το συνολικό κόστος παραγωγής διαιρείται με τη συνολική σε μονάδες παραγωγή, κανονικών και ελαττωματικών προϊόντων, και από τη διαίρεση αυτή προκύπτει ενιαίο κατά μονάδα κόστος παραγωγής.

\item Σε περίπτωση που το ελαττωματικό προϊόν δεν είναι δυνατό ή δε συμφέρει να διατεθεί στην αγορά με το ελάττωμά του, και για το λόγο αυτό επανεισάγεται στην παραγωγική διαδικασία με σκοπό την εξάλειψη του ελαττώματος, τα έξοδα της πρόσθετης κατεργασίας βαρύνουν το σύνολο της παραγωγής και όχι μόνο εκείνη που προέρχεται από την επεξεργασία των ελαττωματικών προϊόντων. Στην περίπτωση συνεπώς αυτή η αποτίμηση του ελαττωματικού προϊόντος γίνεται στο ιστορικό κόστος παραγωγής του. 

\item Σε περίπτωση που το ελαττωματικό προϊόν, για διάφορους λόγους, επαναχρησιμοποιείται στην παραγωγική διαδικασία σαν πρώτη ύλη, η αποτίμησή του γίνεται στην τιμή της πρώτης ύλης που υποκαθιστά.

\end{enumerate}

\item Η αποτίμηση των αποθεμάτων που δεν παρακολουθούνται με το σύστημα της διαρκούς απογραφής


Οι οικονομικές μονάδες που τηρούν τους λογαριασμούς των αποθεμάτων τους με το σύστημα της διαρκούς απογραφής, σύμφωνα με όσα καθορίζονται στην παρ. 5.215 του πέμπτου μέρους, για την αποτίμηση των αποθεμάτων απογραφής παίρνουν ως βάση τα δεδομένα των λογαριασμών αυτών.

Οι λοιπές οικονομικές μονάδες έχουν υποχρέωση να διαμορφώνουν τα βιβλία τους με κατάλληλο τρόπο, ώστε να προκύπτει από αυτά το κόστος αγοράς. Οι οικονομικές αυτές μονάδες θα υπολογίζουν το κόστος παραγωγής με τη βοήθεια εξωλογιστικών στοιχείων.

\end{enumerate}

\chapter{ΑΠΑΙΤΗΣΕΙΣ ΚΑΙ ΔΙΑΘΕΣΙΜΑ}

\section{Λογαριασμοί}

\begin{tabularx}{\linewidth}{lX}

30 & Πελάτες\\
31 & Γραμμάτια εισπρακτέα\\
32 & Παραγγελίες στο εξωτερικό\\
33 & Χρεώστες διάφοροι\\
34 & Χρεόγραφα\\
35 & Λογαριασμοί διαχειρίσεως προκαταβολών και πιστώσεων\\
36 & Μεταβατικοί λογαριασμοί ενεργητικού\\
38 & Χρηματικά διαθέσιμα\\
39 & Απαιτήσεις και διαθέσιμα υποκαταστημάτων ή άλλων κέντρων\\

\end{tabularx}

\section {2.2.3 ΑΠΑΙΤΗΣΕΙΣ ΚΑΙ ΔΙΑΘΕΣΙΜΑ}

\subsection{2.2.300 Περιεχόμενο και εννοιολογικοί προσδιορισμοί}

Στην ομάδα 3 παρακολουθούνται οι βραχυπρόθεσμες απαιτήσεις, τα αξιόγραφα και τα διαθέσιμα περιουσιακά στοιχεία της οικονομικής μονάδας.

Βραχυπρόθεσμες απαιτήσεις θεωρούνται εκείνες που, κατά το κλείσιμο του ισολογισμού, είναι εισπρακτέες μέσα στη χρήση που ακολουθεί. Σύμφωνα με την έννοια αυτή, κάθε απαίτηση της οικονομικής μονάδας, της οποίας η προθεσμία εξοφλήσεως λήγει μέσα στην επόμενη χρήση, καταχωρείται στον οικείο λογαριασμό της ομάδας 3.

Για τις μακροπρόθεσμες απαιτήσεις οι οποίες κατά την κατάρτιση του ισολογισμού μετατρέπονται σε βραχυπρόθεσμες, καθώς και για τον τρόπο παρακολουθήσεώς τους, ισχύουν όσα ορίζονται στην περίπτ. 8 της παραγρ.  2.2.112.

30.00   Πελάτες εσωτερικού
30.01   Πελάτες εξωτερικού
30.02   Ελληνικό Δημόσιο
30.03   Ν.Π.Δ.Δ. και Δημόσιες Επιχειρήσεις
30.04   Πελάτες - Εγγυήσεις ειδών συσκευασίας
30.05   Προκαταβολές πελατών
30.06   Πελάτες - Παρακρατημένες εγγυήσεις
30.07   Πελάτες αντίθετος λογ. αξίας ειδών συσκευασίας
30.80   Πελάτες εσωτερικού εκχωρηθέντες με σύμβαση Factoring (Γνωμ. 216/2176/1994)
30.81   Πελάτες εξωτερικού εκχωρηθέντες με σύμβαση Factoring (Γνωμ. 216/2176/1994)
30.82   Πελάτης Ελληνικό Δημόσιο εκχωρηθείς με σύμβαση Factoring (Γνωμ. 216/2176/1994)
30.83   Πελάτες ΝΠΔΔ και Δημόσιες επιχειρήσεις εκχωρηθέντες με σύμβαση Factoring (Γνωμ. 216/2176/1994)
30.90   Έξοδα για λογ/σμό πελατών - λογ. διάμεσος (Γνωμ. 197/2121/1994 και 204/2136/1994)
30.97   Πελάτες επισφαλείς
30.98   Ελληνικό Δημόσιο (με την ιδιότητα του πελάτη) λογ. Επίδικων απαιτήσεων
30.99   Λοιποί πελάτες λογαριασμός επίδικων απαιτήσεων

\subsection{2.2.301 Λογαριασμός 30 «Πελάτες»}

Στους υπολογαριασμούς του 30 παρακολουθούνται οι απαιτήσεις και κατ' εξαίρεση, για λόγους ενιαίας παρακολουθήσεως, ορισμένες υποχρεώσεις της οικονομικής μονάδας έναντι πελατών της, που απορρέουν από τις πωλήσεις της.

Στο λογαριασμό 30.00 «πελάτες εσωτερικού» παρακολουθούνται οι απαιτήσεις από πωλήσεις που γίνονται στο εσωτερικό της χώρας, εκτός από εκείνες που προέρχονται από πωλήσεις, είτε προς το Ελληνικό Δημόσιο, είτε προς Νομικά Πρόσωπα Δημοσίου Δικαίου και Δημόσιες Επιχειρήσεις, οι οποίες παρακολουθούνται αντίστοιχα στους λογαριασμούς 30.02 «Ελληνικό Δημόσιο» και 30.03 «Ν.Π.Δ.Δ. και Δημόσιες Επιχειρήσεις».

Στο λογαριασμό 30.01 «πελάτες εξωτερικού» παρακολουθούνται οι απαιτήσεις από πωλήσεις στο εξωτερικό. Ο λογαριασμός αυτός χρεώνεται με την αξία του τιμολογίου, η οποία, για τη χρέωση αυτή, μετατρέπεται σε δραχμές με βάση την επίσημη τιμή συναλλάγματος (τιμή αγοράς της Τράπεζας Ελλάδος) της ημέρας εκδόσεως του τιμολογίου, σύμφωνα με όσα καθορίζονται στην περίπτ. 10 της παραγρ. 2.2.704. Οι αναλυτικοί λογαριασμοί του 30.01 τηρούνται κατά τρόπο που να προκύπτει απ' αυτούς η απαίτηση σε ξένο νόμισμα. Ο τρόπος αντιμετωπίσεως των συναλλαγματικών διαφορών που προκύπτουν κατά την αποτίμηση, στο τέλος της χρήσεως, των απαιτήσεων έναντι πελατών εξωτερικού καθορίζεται στην παρ. 2.3.2.

Οι δοσοληψίες με τους αντιπροσώπους - υπαντιπροσώπους της οικονομικής μονάδας παρακολουθούνται στους λογαριασμούς 30.00 και 30.01, όταν αφορούν πωλήσεις και, γενικά, όταν είναι όμοιες με τις δοσοληψίες με πελάτες, και στους λογαριασμούς 33.95 «λοιποί χρεώστες διάφοροι σε Δρχ.» και 33.96 «λοιποί χρεώστες διάφοροι σε Ξ.Ν.», όταν αναφέρονται σε άλλες συναλλαγές ή σχέσεις.

Στο λογαριασμό 30.04 «πελάτες-εγγυήσεις ειδών συσκευασίας» παρακολουθούνται τα ποσά που καταβάλλονται στην οικονομική μονάδα από τους πελάτες της για εγγύηση της επιστροφής των ειδών συσκευασίας, τα οποία παραδίνονται σ' αυτούς χωρίς να τιμολογούνται. Σχετικά με τον τρόπο λειτουργίας του λογαριασμού 30.04 ισχύουν όσα ορίζονται στην περίπτ. 2 της παρ. 2.2.202.

Στο λογαριασμό 30.05 «προκαταβολές πελατών» παρακολουθούνται οι προκαταβολές που λαμβάνονται από πελάτες για παραγγελίες που γίνονται από τους τελευταίους προς την οικονομική μονάδα, όταν αυτή δεν επιθυμεί την παρακολούθησή τους στους οικείους λογαριασμούς 30.00-30.03. Ο λογαριασμός 30.05 πιστώνεται με το ποσό της προκαταβολής και χρεώνεται με το όλο ή μέρος αυτού, ανάλογα με τη μερική ή ολική εκτέλεση της παραγγελίας, με πίστωση του οικείου λογαριασμού του πελάτη (30.00-30.03).

Σε περίπτωση που ο πελάτης, επειδή δεν τηρεί κάποιον όρο της παραγγελίας, χάνει την προκαταβολή ή μέρος της, που περιέρχεται στην οικονομική μονάδα, το ποσό αυτό μεταφέρεται από το λογαριασμό 30.05 (ή 30.00-30.03) στο λογαριασμό 74.98.00 «αποζημιώσεις από πελάτες». Αν όμως πρόκειται για ποινική ρήτρα, τότε μεταφέρεται στο λογαριασμό 81.01.02 «καταπτώσεις εγγυήσεων - ποινικών ρητρών».

Στο λογαριασμό 30.06 «πελάτες - παρακρατημένες εγγυήσεις» παρακολουθούνται τα ποσά που, με βάση κάποιο συμβατικό όρο, παρακρατούν για εγγύηση οι πελάτες της οικονομικής μονάδας, όταν αυτή δεν επιθυμεί την παρακολούθησή τους στους οικείους λογαριασμούς 30.00-30.03. Ο λογαριασμός 30.06 χρεώνεται με τα ποσά που παρακρατούνται για εγγύηση και παραμένει χρεωμένος μέχρι την εκπλήρωση του όρου, για ασφάλεια του οποίου γίνεται η παρακράτηση.

Στο λογαριασμό 30.07 «πελάτες αντίθετος λογαριασμός αξίας ειδών συσκευασίας» παρακολουθείται η αξία των τιμολογημένων ειδών συσκευασίας, για τα οποία οι πελάτες διατηρούν το δικαίωμα της επιστροφής (πολλές φορές έχουν υποχρέωση να τα επιστρέψουν), σύμφωνα με όσα ορίζονται στην περίπτ. 2 της παρ. 2.2.202.

Παρέχεται στις οικονομικές μονάδες η δυνατότητα, αντί να αναπτύσσουν σε τρίτο και πέρα βαθμούς τους υποχρεωτικούς και προαιρετικούς υπολογαριασμούς του 30, να χρησιμοποιούν για την ανάπτυξη αυτών τους κενούς δευτεροβάθμιους 30.08 - 30.96, με την προϋπόθεση ότι θα προκύπτουν κατά οποιοδήποτε τρόπο (π.χ. από τα ισοζύγια) οι πληροφορίες που θα προέκυπταν αν είχαν αναπτυχθεί οι δευτεροβάθμιοι 30.00-30.04.

Στο λογαριασμό 30.97 «πελάτες επισφαλείς» παρακολουθούνται οι απαιτήσεις κατά πελατών που η είσπραξή τους γίνεται επισφαλής (αμφίβολης ρευστοποιήσεως), οι οποίες μεταφέρονται στο λογαριασμό αυτό από τους οικείους υπολογαριασμούς του 30.

Στους λογαριασμούς 30.98 «Ελληνικό Δημόσιο λογαριασμός επίδικων απαιτήσεων» και 30.99 «λοιποί πελάτες λογαριασμός επίδικων απαιτήσεων» παρακολουθούνται όσες απαιτήσεις της οικονομικής μονάδας κατά πελατών της μετατρέπονται σε επίδικες.  Οι προβλέψεις πιθανών ζημιών, που γίνονται για επισφαλείς και επίδικες απαιτήσεις κατά πελατών, καταχωρούνται σε χρέωση του λογαριασμού 83.11 «προβλέψεις για επισφαλείς απαιτήσεις», με αντίστοιχη πίστωση του λογαριασμού 44.11 «προβλέψεις για επισφαλείς απαιτήσεις».

31.00   Γραμμάτια στο χαρτοφυλάκιο
31.01   Γραμμάτια στις Τράπεζες για είσπραξη
31.02   Γραμμάτια στις Τράπεζες σε εγγύηση
31.03   Γραμμάτια σε καθυστέρηση
31.04   Γραμμάτια μεταβιβασμένα σε τρίτους (αντίθετος λογ.)
31.05   Γραμμάτια προεξοφλημένα (αντίθετος λογ.)
31.06   Μη δουλευμένοι τόκοι γραμματίων εισπρακτέων (αντίθετος λογ.)
31.07   Γραμμάτια σε Ξ.Ν. στο χαρτοφυλάκιο
31.08   Γραμμάτια σε Ξ.Ν. στις Τράπεζες για είσπραξη
31.09   Γραμμάτια σε Ξ.Ν. στις Τράπεζες σε εγγύηση
31.10   Γραμμάτια σε Ξ.Ν. σε καθυστέρηση
31.11   Γραμμάτια σε Ξ.Ν. μεταβιβασμένα σε τρίτους (αντίθετος λογ.)
31.12   Γραμμάτια σε Ξ.Ν. προεξοφλημένα (αντίθετος λογ.)
31.13   Μη δουλευμένοι τόκοι γραμματίων εισπρακτέων σε Ξ.Ν. (αντίθετος λογ.)
31.90   Υποσχετικές επιστολές εισπρακτέες σε δρχ. (Γνωμ. 79/1623/1991)
31.91   Υποσχετικές επιστολές εισπρακτέες σε Ξ.Ν. (Γνωμ. 79/1623/1991)
31.92   Μη δουλευμένοι τόκοι υποσχετικών επιστολών εισπρακτέων σε δρχ.
31.93   Μη δουλευμένοι τόκοι υποσχετικών επιστολών εισπρακτέων σε Ξ.Ν.
31.94   Γραμμάτια στις τράπεζες για είσπραξη με σύμβαση Factoring (Γνωμ. 216/2176/1994)
31.95   Τίτλοι trade credit (Γνωμ. 256/2252/1995)
31.99   Διάμεσος λογ. ελέγχου διακινήσεως γραμματίων εισπρακτέων

\subsection{2.2.302 Λογαριασμός 31 «Γραμμάτια εισπρακτέα»}

Στους υπολογαριασμούς του 31 παρακολουθούνται οι απαιτήσεις κατά τρίτων που είναι ενσωματωμένες σε τίτλους συναλλαγματικών ή γραμματίων «εις διαταγήν». Οι τίτλοι αυτοί, στις επόμενες παραγράφους, αναφέρονται με την ονομασία «γραμμάτια εισπρακτέα».

Τα γραμμάτια εισπρακτέα σε Δρχ. καταχωρούνται στο λογαριασμό 31.00 «γραμμάτια στο χαρτοφυλάκιο». Όσα από τα γραμμάτια αυτά μεταβιβάζονται στις Τράπεζες για είσπραξη ή σε εγγύηση, μεταφέρονται από το λογαριασμό 31.00 στη χρέωση των λογαριασμών 31.01 «γραμμάτια στις Τράπεζες για είσπραξη» ή 31.02 «γραμμάτια στις Τράπεζες σε εγγύηση», αντίστοιχα.

Στο λογαριασμό 31.03 «γραμμάτια σε καθυστέρηση» παρακολουθούνται οι απαιτήσεις κατά οφειλετών (π.χ. αποδεκτών) γραμματίων εισπρακτέων, τα οποία δεν εξοφλούνται κατά την ημερομηνία λήξεώς τους και παραμένουν απλήρωτα στα χέρια της οικονομικής μονάδας. Η ανάπτυξη του λογαριασμού γίνεται σύμφωνα με τις ανάγκες κάθε μονάδας, αλλά πάντοτε κατά τέτοιο τρόπο ώστε να προκύπτει η απαίτησή της από κάθε οφειλέτη.

Σε περίπτωση που οι απαιτήσεις του λογαριασμού 31.03 (ή 31.10) μετατρέπονται σε επισφαλείς ή επίδικες, μεταφέρονται στους λογαριασμούς 30.97 ή 33.97, όταν πρόκειται για επισφαλείς, ή στους λογαριασμούς 30.99 ή 33.99, όταν πρόκειται για επίδικες, ανάλογα με τη φύση τους.

Στους λογαριασμούς 31.04 «γραμμάτια μεταβιβασμένα σε τρίτους» και 31.05 «γραμμάτια προεξοφλημένα» είναι δυνατό να παρακολουθούνται, ανάλογα με την περίπτωση, τα γραμμάτια εισπρακτέα τα οποία μεταβιβάζονται σε τρίτους, π.χ.  προμηθευτές, με χρέωση του προσωπικού λογαριασμού του τρίτου, ή προεξοφλούνται, με χρέωση του οικείου υπολογαριασμού χρηματικών διαθεσίμων του 38 για το προϊόν της προεξοφλήσεως και του λογαριασμού 65.02 «προεξοφλητικοί τόκοι και έξοδα Τραπεζών» για τους τόκους και τα έξοδα προεξοφλήσεως.

Τα μεταβιβασμένα σε τρίτους, καθώς και τα προεξοφλημένα γραμμάτια εισπρακτέα, στην περίπτωση που παρακολουθούνται στους λογαριασμούς 31.04 και 31.05, έπειτα από την πληρωμή τους ή, αν δεν υπάρχουν στοιχεία, αφού περάσει εύλογος χρόνος από τη λήξη τους, μεταφέρονται από τους λογαριασμούς αυτούς στην πίστωση του λογαριασμού 31.00.

Σε περίπτωση που η οικονομική μονάδα δεν παρακολουθεί τα μεταβιβασμένα σε τρίτους και τα προεξοφλημένα γραμμάτια εισπρακτέα με τους λογαριασμούς 31.04 και 31.05, αλλά με τη μεταβίβαση ή προεξόφλησή τους πιστώνει απευθείας το λογαριασμό 31.00, είναι υποχρεωμένη να παρακολουθεί τα γραμμάτια αυτά στους λογαριασμούς τάξεως 02.20 «προεξοφλημένα γραμμάτια εισπρακτέα» - 06.20 «προεξοφλήσεις γραμματίων εισπρακτέων» και 02.21 «μεταβιβασμένα σε τρίτους γραμμάτια εισπρακτέα» - 06.21 «μεταβιβάσεις σε τρίτους γραμματίων εισπρακτέων», σύμφωνα με όσα ορίζονται στην περίπτωση 2 της παρ. 3.2.103.

Τα γραμμάτια εισπρακτέα σε Ξ.Ν. καταχωρούνται στο λογαριασμό 31.07 «γραμμάτια σε Ξ.Ν. στο χαρτοφυλάκιο». Για όσα από τα γραμμάτια αυτά μεταβιβάζονται ή προεξοφλούνται ή δεν πληρώνονται κατά την ημερομηνία λήξεώς τους, ισχύουν ανάλογα όσα ορίζονται παραπάνω για τα γραμμάτια εισπρακτέα σε δραχμές, με τη διαφορά ότι, αντί των λογαριασμών 31.01, 31.02, 31.03, 31.04 και 31.05, κινούνται οι λογαριασμοί 31.08, 31.09, 31.10, 31.11 και 31.12, αντίστοιχα.

Η απεικόνιση σε δραχμές των γραμματίων εισπρακτέων σε ξένο νόμισμα γίνεται με βάση την επίσημη τιμή του ξένου συναλλάγματος (τιμή αγοράς της Τράπεζας της Ελλάδος) της ημέρας που αυτά περιέρχονται στην οικονομική μονάδα. Οι συναλλαγματικές διαφορές που προκύπτουν μεταξύ της δραχμικής αξίας με την οποία ένα γραμμάτιο απεικονίζεται στα βιβλία και της αξίας που τελικά εισπράττεται, καταχωρούνται σε χρέωση ή πίστωση των λογαριασμών 81.00.04 ή 81.01.04, ανάλογα με την περίπτωση. Οι συναλλαγματικές διαφορές που προκύπτουν κατά την αποτίμηση, στο τέλος της χρήσεως, των γραμματίων σε ξένο νόμισμα, αντιμετωπίζονται σύμφωνα με όσα ορίζονται στην παρ. 2.3.2.

Στους λογαριασμούς 31.06 «μη δουλευμένοι τόκοι γραμματίων εισπρακτέων σε Δρχ.» και 31.13 «μη δουλευμένοι τόκοι γραμματίων εισπρακτέων σε Ξ.Ν.» καταχωρούνται οι τόκοι που περιλαμβάνονται στα άληκτα γραμμάτια στο τέλος της χρήσεως. Ο χειρισμός αυτός δεν είναι υποχρεωτικός για τις οικονομικές μονάδες, αν όμως γίνει σε κάποια χρήση, υποχρεωτικά εφαρμόζεται πάγια και στις επόμενες χρήσεις.

Οι τρόποι αντιμετωπίσεως των επιμέρους περιπτώσεων των τόκων που περιλαμβάνονται στα άληκτα γραμμάτια εισπρακτέα είναι οι εξής:

α. Στην περίπτωση που οι τόκοι των γραμματίων διαχωρίζονται από τα έσοδα από πωλήσεις, ισχύουν τα εξής:

- Οι τόκοι των γραμματίων που εκδίδονται και λήγουν μέσα στη χρήση καταχωρούνται απευθείας στο λογαριασμό 76.02 «δουλευμένοι τόκοι γραμματίων εισπρακτέων».

- Από τους τόκους των γραμματίων που εκδίδονται μέσα στη χρήση και λήγουν μετά το τέλος της, εκείνοι που αναλογούν στη χρονική περίοδο μέχρι τη λήξη της χρήσεως αυτής καταχωρούνται απευθείας στο λογαριασμό 76.02 και εκείνοι που αναλογούν στη χρονική περίοδο μετά τη λήξη της χρήσεως αυτής καταχωρούνται στους αντίθετους λογαριασμούς 31.06 ή 31.13, κατά περίπτωση.

- Στο τέλος κάθε χρήσεως, οι δουλευμένοι τόκοι των γραμματίων που έληξαν μέσα στη χρήση αυτή (γραμμάτια που εκδόθηκαν σε προηγούμενες χρήσεις), καθώς και οι τόκοι των λοιπών γραμματίων (γραμμάτια που εκδόθηκαν σε προηγούμενες χρήσεις και λήγουν μετά το τέλος της χρήσεως), που αναλογούν στη χρονική περίοδο μέχρι τη λήξη της χρήσεως αυτής, μεταφέρονται από τους λογαριασμούς 31.06 ή 31.13, κατά περίπτωση, στο λογαριασμό 76.02.

β. Στην περίπτωση που οι τόκοι των γραμματίων περιλαμβάνονται στο τίμημα ή στα λοιπά έσοδα πωλήσεων, εμφανίζονται δε μαζί με τα έσοδα αυτά στους οικείους λογαριασμούς της ομάδας 7, ο διαχωρισμός και η εμφάνισή τους στους αντίθετους λογαριασμούς 31.06 ή 31.13 γίνεται ως εξής:

- Τα γραμμάτια εισπρακτέα που είναι στο τέλος της χρήσεως άληκτα εκτοκίζονται με βάση το τραπεζικό επιτόκιο προεξοφλήσεως που ισχύει κατά τη χρονολογία αυτή, προσαυξημένο κατά την τραπεζική προμήθεια. Ο εκτοκισμός αυτός γίνεται για χρονική περίοδο από την ημερομηνία κλεισίματος της χρήσεως μέχρι την ημερομηνία λήξεως κάθε γραμματίου.

- Με τους μη δουλευμένους τόκους των άληκτων γραμματίων χρεώνονται οι οικείοι αντίθετοι λογαριασμοί εσόδων της ομάδας 7 (70.97, 71.97, 72.97 και 73.97) και πιστώνονται οι αντίθετοι λογαριασμοί 31.06 ή 31.13, κατά περίπτωση.

- Σε περίπτωση που ο εκτοκισμός των άληκτων γραμματίων κατά κατηγορίες εσόδων (λογαριασμών 70, 71, 72 και 73) είναι, είτε αδύνατος, είτε δυσχερής, η κατανομή των μη δουλευμένων τόκων στις επιμέρους αυτές κατηγορίες γίνεται με βάση κριτήρια που επιλέγονται από την οικονομική μονάδα (π.χ. ανάλογα με τα ακαθάριστα έσοδα που διακανονίζονται με γραμμάτια).

- Όταν, κατά το τέλος κάθε χρήσεως, οι μη δουλευμένοι τόκοι που προκύπτουν από τον παραπάνω εκτοκισμό των άληκτων γραμματίων είναι μικρότερης αξίας από τους μη δουλευμένους τόκους που εμφανίζονται στους αντίθετους λογαριασμούς 31.06 και 31.13, η διαφορά μεταφέρεται στο λογαριασμό 76.02, με χρέωση των αντίθετων αυτών λογαριασμών.

- Παρέχεται η δυνατότητα στην οικονομική μονάδα να διαχωρίζει και να μεταφέρει στο λογαριασμό 76.02 και τους δουλευμένους τόκους που περιλαμβάνονται στο τίμημα πωλήσεως, δηλαδή στα έσοδα των λογαριασμών 70, 71, 72 και 73.

7. Παρέχεται η δυνατότητα χρησιμοποιήσεως του λογαριασμού 31.99 «διάμεσος λογαριασμός ελέγχου διακινήσεως γραμματίων εισπρακτέων» για τον έλεγχο της διακινήσεως των γραμματίων μεταξύ υποκαταστημάτων ή μεταξύ του κεντρικού και των υποκαταστημάτων των οικονομικών μονάδων. Το εκάστοτε υπόλοιπο του λογαριασμού 31.99 απεικονίζει τα υπό διακίνηση γραμμάτια εισπρακτέα.

8. Τα γραμμάτια που λαμβάνονται σε εγγύηση για την καλή εκτέλεση όρου συμβάσεως με τρίτους ή για οποιοδήποτε άλλο λόγο, καταχωρούνται στους λογαριασμούς τάξεως 02.02 «γραμμάτια εισπρακτέα εξασφαλίσεως εκτελέσεως όρων συμβάσεων κλπ.» - 06.02 «αποδέκτες γραμματίων εισπρακτέων εγγυήσεων», σύμφωνα με όσα ορίζονται στην περίπτ. 1 της παρ. 3.2.103.


32.00   Παραγγελίες πάγιων στοιχείων
32.01   Παραγγελίες κυκλοφορούντων στοιχείων
32.02   Προεμβάσματα μέσω Τραπεζών
32.03   Ανέκκλητες πιστώσεις μέσω Τραπεζών
32.04   Δεσμευμένα περιθώρια και δασμοί εισαγωγής
32.90   Κόστος παραγγελιών εξωτερικού λογισμένο (λογ. αντίθετος) (Γνωμ. 184/2098/1993)


\subsection{2.2.303 Λογαριασμός 32 «Παραγγελίες στο εξωτερικό»}

Στους υπολογαριασμούς του 32 για κάθε παραγγελία συγκεντρώνεται και παρακολουθείται προσωρινά η αξία κτήσεως των αγαθών που εισάγονται από το εξωτερικό. Μετά από την παραλαβή των αγαθών και την ολοκλήρωση της συγκεντρώσεως της αξίας κτήσεώς τους, η τελευταία μεταφέρεται στους οικείους λογαριασμούς των ομάδων 1 και 2, κατά περίπτωση.

Στο λογαριασμό 32.00 «παραγγελίες πάγιων στοιχείων» παρακολουθούνται, κατά παραγγελία, τα πάγια περιουσιακά στοιχεία που εισάγονται από το εξωτερικό. Τα λοιπά αγαθά που εισάγονται από το εξωτερικό παρακολουθούνται στο λογαριασμό 32.01 «παραγγελίες κυκλοφορούντων στοιχείων».

Στους διάμεσους λογαριασμούς 32.02 «προεμβάσματα μέσω Τραπεζών» και 32.03 «ανέκκλητες πιστώσεις μέσω Τραπεζών» είναι δυνατό να παρακολουθούνται, προσωρινά, προεμβάσματα που γίνονται ή ανέκκλητες πιστώσεις που ανοίγονται για πολλές παραγγελίες μαζί, όταν ο άμεσος διαχωρισμός τους κατά παραγγελία, είτε είναι αδύνατος, είτε είναι δυσχερής. Μόλις εκλείψουν οι λόγοι των προσωρινών καταχωρήσεων, τα σχετικά ποσά μεταφέρονται στους λογαριασμούς 32.00 και 32.01, κατά περίπτωση.

Στο λογαριασμό 32.04 «δεσμευμένα περιθώρια και δασμοί εισαγωγής» καταχωρούνται τα ποσά τα οποία καταβάλλονται για να παραμείνουν δεσμευμένα για ορισμένο χρονικό διάστημα, σύμφωνα με τους κανόνες περί εισαγωγών που ισχύουν κάθε φορά.

33.00   Προκαταβολές προσωπικού
33.01   Χρηματικές διευκολύνσεις προσωπικού
33.02   Δάνεια προσωπικού
33.03   Μέτοχοι (ή εταίροι) λογ/σμός καλύψεως κεφαλαίου
33.04   Οφειλόμενο κεφάλαιο
33.05   Δόσεις μετοχικού κεφαλαίου σε καθυστέρηση
33.06   Προμερίσματα
33.07   Δοσοληπτικοί λογ/σμοί εταίρων
33.08   Δοσοληπτικοί λογ/σμοί διαχειριστών
33.09   Δοσοληπτικοί λογαριασμοί ιδρυτών Α.Ε. και μελών Δ.Σ.
33.10   Δοσοληπτικοί λογ/σμοί γενικών διευθυντών ή διευθυντών Α.Ε.
33.11   Βραχυπρόθεσμες απαιτήσεις κατά συνδεμένων επιχειρήσεων σε Δρχ.
33.12   Βραχυπρόθεσμες απαιτήσεις κατά συνδεμένων επιχειρήσεων σε Ξ.Ν.
33.13   Ελληνικό Δημόσιο - προκαταβλημένοι και παρακρατημένοι φόροι
33.13.00   Προκαταβολή φόρου εισοδήματος
01   Παρακρατημένος φόρος εισοδήματος από μερίσματα μετοχών εισαγμένων στο Χρηματιστήριο Ο Λογ/σμός μετά την εφαρμογή των διατάξεων του Ν. 2065/92 (άρθρα 109 \& 114 Ν. 2238/94) παραμένει ανενεργός.
02   Παρακρατημένος φόρος εισοδήματος από μερίσματα μετοχών μη εισαγμένων στο Χρηματιστήριο
03   Παρακρατημένος φόρος εισοδήματος από μερίσματα μετοχών αλλοδαπής
04   Παρακρατημένος φόρος εισοδήματος από συμμετοχές σε ΕΠΕ αλλοδαπής
05   Παρακρατημένος φόρος εισοδήματος από μερίδια αμοιβαίων κεφαλαίων
06   Παρακρατημένος φόρος εισοδήματος από τόκους
07   Παρακρατημένος φόρος εισοδήματος από συμμετοχές σε ΕΠΕ, Ο.Ε. Ε.Ε. και κοινοπραξίες εκτελέσεως τεχνικών έργων ημεδαπής
33.13.10   Παρακρατημένος φόρος εισοδήματος από πωλήσεις προς Ελληνικό Δημόσιο ή ΝΠΔΔ (Γνωμ. 236/2221/1995)
33.13.90   Συμψηφιστέος στην επόμενη χρήση Φ.Π.Α. (Γνωμ.243/2162/1995)
33.13.99   Λοιποί παρακρατημένοι φόροι εισοδήματος
33.14   Ελληνικό Δημόσιο - λοιπές απαιτήσεις
33.14.00   Απαιτήσεις από ειδικές επιχορηγήσεις
      01   Δασμοί και λοιποί φόροι εισαγωγής προς επιστροφή (όταν δενεπιβαρύνεται η αξία κτήσεως των πρώτων υλών, βλ. και λογ/σμό 74.01).
33.15   Λογαριασμοί ενεργοποιήσεως εγγυήσεων προμηθευτών σε Δρχ. (Guarantees)
33.16   Λογαριασμοί ενεργοποιήσεως εγγυήσεων προμηθευτών σε Ξ.Ν. (Guarantees)
33.17   Λογαριασμοί δεσμευμένων (Bloques) καταθέσεων σε Δρχ.
33.18   Λογαριασμοί δεσμευμένων (Bloques) καταθέσεων σε Ξ.Ν.
33.19   Μακροπρόθεσμες απαιτήσεις εισπρακτέες στην επόμενη χρήση σε Δρχ.
33.20   Μακροπρόθεσμες απαιτήσεις εισπρακτέες στην επόμενη χρήση σε Ξ.Ν.
33.21   Βραχυπρόθεσμες απαιτήσεις κατά λοιπών συμμετοχικού ενδιαφέροντος επιχειρήσεων σε Δρχ.
33.22   Βραχυπρόθεσμες απαιτήσεις κατά λοιπών συμμετοχικού ενδιαφέροντος επιχειρήσεων σε Ξ.Ν.
33.90   Επιταγές εισπρακτέες (μεταχρονολογημένες) (Γνωμ. 26/971/1988)
33.91   Επιταγές σε καθυστέρηση (σφραγισμένες) (Γνωμ. 26/971/1988)
33.95   Λοιποί χρεώστες διάφοροι σε Δρχ.
33.96   Λοιποί χρεώστες διάφοροι σε Ξ.Ν.
33.97   Χρεώστες επισφαλείς
33.98    Επίδικες απαιτήσεις κατά Ελληνικού Δημοσίου
33.99    Λοιποί χρεώστες επίδικοι

2.2.304 Λογαριασμός 33 «Χρεώστες διάφοροι»

1. Στους υπολογαριασμούς του 33 παρακολουθούνται οι απαιτήσεις που δεν υπάγονται σε οποιαδήποτε κατηγορία απαιτήσεων από εκείνες που παρακολουθούνται στους λοιπούς πρωτοβάθμιους λογαριασμούς της ομάδας 3.

2. Στο λογαριασμό 33.00 «προκαταβολές προσωπικού» καταχωρούνται οι προκαταβολές που δίνονται στο προσωπικό έναντι των αποδοχών της μισθολογικής περιόδου (π.χ.  μήνα ή εβδομάδας), η οποία αποτελεί τη βάση υπολογισμού τους (εκκαθαρίσεως). Ο λογαριασμός 33.00 πιστώνεται με τα ποσά που παρακρατούνται κατά την εκκαθάριση των αποδοχών της οικείας περιόδου, τα οποία είναι ίσα με τις δοσμένες προκαταβολές, οπότε εξισώνεται.

3. Στο λογαριασμό 33.01 «χρηματικές διευκολύνσεις προσωπικού» παρακολουθούνται οι χρηματικές διευκολύνσεις, προσωρινού χαρακτήρα, που γίνονται στο προσωπικό.

4. Στο λογαριασμό 33.02 «δάνεια προσωπικού» παρακολουθούνται τα ποσά που καταβάλλονται στο προσωπικό με μορφή δανείου.

5. Για το περιεχόμενο και τον τρόπο λειτουργίας των λογαριασμών 33.03 «μέτοχοι λογαριασμός καλύψεως κεφαλαίου», 33.04 «οφειλόμενο κεφάλαιο» και 33.05 «δόσεις μετοχικού κεφαλαίου σε καθυστέρηση» ισχύουν όσα καθορίζονται στην παρ. 2.2.401 σχετικά με το λογαριασμό 40 «κεφάλαιο» με τον οποίο οι λογαριασμοί αυτοί συλλειτουργούν.

6. Ο λογαριασμός 33.06 «προμερίσματα» χρεώνεται με πίστωση του λογαριασμού 53.02 «προμερίσματα πληρωτέα» με το συνολικό ποσό που αποφασίζεται νόμιμα να καταβληθεί ως προμέρισμα. Κατά το κλείσιμο του ισολογισμού το υπόλοιπο του λογαριασμού 33.06 μεταφέρεται στο λογαριασμό 53.01 «μερίσματα πληρωτέα».

7. Στο λογαριασμό 33.07 «δοσοληπτικός λογαριασμός εταίρων» παρακολουθούνται όλες οι χρηματικές δοσοληψίες της οικονομικής μονάδας με τα πρόσωπα που συμμετέχουν σε εταιρίες κεφαλαίου, προσωπικές και συμμετοχικές, ή, όταν πρόκειται για ατομικές επιχειρήσεις, με τον επιχειρηματία, οπότε ο λογαριασμός αυτός μετονομάζεται σε «ατομικός λογαριασμός επιχειρηματία».

8. Στους λογαριασμούς 33.08 «δοσοληπτικοί λογαριασμοί διαχειριστών», 33.09 «δοσοληπτικοί λογαριασμοί ιδρυτών Α.Ε. και μελών διοικητικού συμβουλίου» και 33.10 «δοσοληπτικοί λογαριασμοί γενικών διευθυντών ή διευθυντών Α.Ε.» παρακολουθούνται οι χρηματικές δοσοληψίες της οικονομικής μονάδας με τα όργανα διοικήσεώς της κατά τρόπο που να είναι δυνατή η εμφάνιση των υπολοίπων των λογαριασμών αυτών στον ισολογισμό για την πληροφόρηση των πιστωτών και του κοινού.

9. Στους λογαριασμούς 33.11 «βραχυπρόθεσμες απαιτήσεις κατά συνδεμένων επιχειρήσεων σε Δρχ.» και 33.12 «βραχυπρόθεσμες απαιτήσεις κατά συνδεμένων επιχειρήσεων σε Ξ.Ν.», παρακολουθούνται οι βραχυπρόθεσμες απαιτήσεις της οικονομικής μονάδας, οι οποίες δεν προέρχονται από συναλλαγές που αφορούν το αντικείμενο αυτής, κατά των συνδεμένων επιχειρήσεων της περιπτ. 10 της παρ.  2.2.112. Οι βραχυπρόθεσμες απαιτήσεις κατά λοιπών συμμετοχικού ενδιαφέροντος επιχειρήσεων οι οποίες, επίσης, δεν προέρχονται από συναλλαγές που αφορούν το αντικείμενο της οικονομικής μονάδας, παρακολουθούνται στους λογαριασμούς 33.21 «βραχυπρόθεσμες απαιτήσεις κατά λοιπών συμμετοχικού ενδιαφέροντος επιχειρήσεων σε Δρχ.» και 33.22 «βραχυπρόθεσμες απαιτήσεις κατά λοιπών συμμετοχικού ενδιαφέροντος επιχειρήσεων σε Ξ.Ν.», κατά περίπτωση.

10. Στο λογαριασμό 33.13 «Ελληνικό Δημόσιο - προκαταβλημένοι και παρακρατημένοι φόροι» παρακολουθούνται οι απαιτήσεις της οικονομικής μονάδας κατά του Ελληνικού Δημοσίου, οι οποίες προέρχονται από φόρους που προκαταβάλλονται ή παρακρατούνται κατά την είσπραξη μερισμάτων ή άλλων εισοδημάτων.

Ειδικότερα για τους υπολογαριασμούς του 33.13 ισχύουν τα ακόλουθα:

α. Ο λογαριασμός 33.13.00 «προκαταβολή φόρου εισοδήματος» χρεώνεται στο τέλος της χρήσεως με το ποσό της προκαταβολής φόρου εισοδήματος για την επόμενη χρήση, που προκύπτει από τη δήλωση φόρου εισοδήματος της κλειόμενης χρήσεως, με αντίστοιχη πίστωση του λογαριασμού 54.08 «λογαριασμός εκκαθαρίσεως φόρων-τελών ετήσιας δηλώσεως φόρου εισοδήματος». Στο τέλος της επόμενης χρήσεως, το υπόλοιπο του λογαριασμού 33.13.00 μεταφέρεται στη χρέωση του λογαριασμού 54.08, σύμφωνα με όσα καθορίζονται στην περίπτ. 8 της παρ. 2.2.505.

β. Στους λογαριασμούς 33.13.01-99 καταχωρούνται τα ποσά που παρακρατούνται, κατά τη διάρκεια της χρήσεως, για φόρο εισοδήματος από τα εισοδήματα που εισπράττονται από την οικονομική μονάδα από κινητές αξίες ή από τις εισπράξεις για άλλες αιτίες (π.χ. φόρος εργολάβων). Στο τέλος της χρήσεως, όσα από τα ποσά αυτά, σύμφωνα με τη φορολογική νομοθεσία που ισχύει κάθε φορά, είναι δυνατό να συμψηφίζονται με το φόρο εισοδήματος της κλειόμενης χρήσεως, μεταφέρονται στη χρέωση του λογαριασμού 54.08, ενώ τα υπόλοιπα που δε συμψηφίζονται, μεταφέρονται στη χρέωση του λογαριασμού 63.00 «φόρος εισοδήματος μη συμψηφιζόμενος».

11. Στο λογαριασμό 33.14 «Ελληνικό Δημόσιο λοιπές απαιτήσεις» παρακολουθούνται οι λοιπές απαιτήσεις της οικονομικής μονάδας κατά του Ελληνικού Δημοσίου, οι οποίες δεν προέρχονται από συναλλαγές που αφορούν το αντικείμενό της.

Ειδικότερα για τους υπολογαριασμούς του 33.14 ισχύουν τα ακόλουθα:

α. Στο λογαριασμό 33.14.00 «απαιτήσεις από ειδικές επιχορηγήσεις» καταχωρούνται οι απαιτήσεις που προέρχονται από δικαιώματα της οικονομικής μονάδας για την είσπραξη ποσών π.χ. λόγω επιχορηγήσεων ή συμμετοχής του Ελληνικού Δημοσίου σε έξοδα ή επενδυτικές δαπάνες της.

β. Στο λογαριασμό 33.14.01 «δασμοί και λοιποί φόροι εισαγωγής προς επιστροφή» είναι δυνατό να καταχωρούνται τα ποσά δασμών, φόρων και τελών που καταβάλλονται προσωρινά κατά την εισαγωγή από το εξωτερικό διαφόρων αγαθών, τα οποία προορίζονται για βιομηχανοποίηση και εξαγωγή ή επανεξαγωγή ή πώληση σε δικαιούχα ατέλειας πρόσωπα εσωτερικού. Τα ποσά τα οποία, μετά την επανεξαγωγή, επιστρέφονται στην οικονομική μονάδα καταχωρούνται στην πίστωση του λογαριασμού 33.14.01 και οι τυχόν διαφορές μεταφέρονται στο λογαριασμό 63.98.99 «λοιποί φόροι - τέλη».

12. Στους λογαριασμούς 33.15 «λογαριασμοί ενεργοποιήσεως εγγυήσεων προμηθευτών σε Δρχ. (Guarantee)» και 33.16 «λογαριασμοί ενεργοποιήσεως εγγυήσεων προμηθευτών σε Ξ.Ν. (Guarantee)» παρακολουθούνται τα ποσά που καταβάλλονται από την οικονομική μονάδα για αποκατάσταση ζημιών πελατών της σε αγαθά που πωλήθηκαν από αυτή με σύγχρονη χορήγηση εγγυήσεως του οικείου προμηθευτή της.

Τα ποσά που εισπράττει η οικονομική μονάδα από τους προμηθευτές της, σε αναγνώριση της εγγυήσεως που αυτοί χορηγούν, καταχωρούνται στην πίστωση των οικείων υπολογαριασμών των 33.15 και 33.16, κατά περίπτωση. Τα ποσά που δεν καταβάλλονται στην οικονομική μονάδα, λόγω μη αναγνωρίσεώς τους από τους προμηθευτές, μεταφέρονται στη χρέωση, είτε του λογαριασμού του οικείου πελάτη, είτε του λογαριασμού 64.02.08 «έξοδα λόγω εγγυήσεως πωλήσεων», ανάλογα με την περίπτωση που συντρέχει.

13. Στους λογαριασμούς 33.17 «λογαριασμοί δεσμευμένων (Bloques) καταθέσεων σε Δρχ.» και 33.18 «λογαριασμοί δεσμευμένων (Bloques) καταθέσεων σε Ξ.Ν.» παρακολουθούνται οι καταθέσεις που γίνονται, κατά κύριο λόγο στις Τράπεζες, με τη μορφή δεσμεύσεως για διάφορους λόγους, όπως π.χ. για την έκδοση εγγυητικών επιστολών ή την παροχή εγγυήσεως για χορήγηση πιστώσεων σε τρίτους.

14. Για το περιεχόμενο και τον τρόπο λειτουργίας των λογαριασμών 33.19 «μακροπρόθεσμες απαιτήσεις εισπρακτέες στην επόμενη χρήση σε δρχ.» και 33.20 «μακροπρόθεσμες απαιτήσεις εισπρακτέες στην επόμενη χρήση σε Ξ.Ν.» ισχύουν όσα καθορίζονται στην περίπτ. 8 της παρ. 2.2.112.

15. Στους λογαριασμούς 33.95 «λοιποί χρεώστες διάφοροι σε Δρχ.» και 33.96 «λοιποί χρεώστες διάφοροι σε Ξ.Ν.» παρακολουθούνται οι βραχυπρόθεσμες απαιτήσεις της οικονομικής μονάδας που δεν είναι δυνατό να ενταχθούν σε οποιοδήποτε άλλο λογαριασμό της ομάδας 3.

16. Στο λογαριασμό 33.97 «χρεώστες επισφαλείς» παρακολουθούνται οι διάφοροι χρεώστες της οικονομικής μονάδας, οι οποίοι χαρακτηρίζονται ως επισφαλείς λόγω αμφίβολης ρευστοποιήσεως των κατ' αυτών απαιτήσεων.

17. Στους λογαριασμούς 33.98 «επίδικες απαιτήσεις κατά Ελληνικού Δημοσίου» και 33.99 «λοιποί χρεώστες επίδικοι» παρακολουθούνται, κατά περίπτωση, όσες απαιτήσεις της οικονομικής μονάδας κατά χρεωστών της μετατρέπονται σε επίδικες.

18. Ειδικότερα για τους λογαριασμούς 33.97, 33.98 και 33.99 ισχύουν και τα ακόλουθα:

α. Οι προβλέψεις για πιθανές ζημίες, που γίνονται γι' αυτές τις απαιτήσεις, καταχωρούνται σε χρέωση του λογαριασμού 83.11 «προβλέψεις για επισφαλείς απαιτήσεις», με αντίστοιχη πίστωση του λογαριασμού 44.11 «προβλέψεις για επισφαλείς απαιτήσεις».

β. Ο λογαριασμός 33.98 χρεώνεται με τα ποσά των φόρων που βεβαιώνονται σε βάρος της οικονομικής μονάδας, για τα οποία έχει προηγηθεί η άσκηση προσφυγών στα διοικητικά δικαστήρια, με πίστωση του λογαριασμού 54.99 «φόροι-τέλη προηγούμενων χρήσεων».

γ. Για τις επίδικες απαιτήσεις κατά του Ελληνικού Δημοσίου, κατά το κλείσιμο του ισολογισμού, σχηματίζεται ισόποση πρόβλεψη, με χρέωση του λογαριασμού 83.12 «προβλέψεις για εξαιρετικούς κινδύνους και έκτακτα έξοδα» ή 83.13 «προβλέψεις για έξοδα προηγούμενων χρήσεων» και με πίστωση, αντίστοιχα, του λογαριασμού 44.12 ή 44.13. Σε περίπτωση που, σύμφωνα με τα στοιχεία της οικονομικής μονάδας, είναι πολύ πιθανό ότι το ποσό με το οποίο αυτή θα επιβαρυνθεί τελικά για συγκεκριμένη φορολογική υπόθεση θα διαφέρει από εκείνο που έχει βεβαιωθεί, η πρόβλεψη πρέπει να είναι ανάλογη.

δ. Όταν οριστικοποιηθεί η εκκρεμοδικία, ο λογαριασμός 33.98 πιστώνεται με το ποσό που είχε χρεωθεί όταν βεβαιώθηκε ο φόρος και χρεώνονται, ο λογαριασμός 38.00 «ταμείο» με το ποσό που τυχόν επιστρέφεται στην οικονομική μονάδα και ο λογαριασμός 82.00.05 «οριστικοποιημένοι επίδικοι φόροι Δημοσίου» με το υπόλοιπο ποσό, εκτός αν πρόκειται για φόρο εισοδήματος, οπότε αντί του 82.00.05 χρεώνεται ο 42.04 «διαφορές φορολογικού ελέγχου προηγούμενων χρήσεων» ή ο 42.00 «υπόλοιπο κερδών εις νέο» ή ο 42.01 «υπόλοιπο ζημιών χρήσεως εις νέο», κατά περίπτωση.

ε. Μετά από τις παραπάνω (δ) λογιστικές τακτοποιήσεις επακολουθεί η τακτοποίηση της σχετικής προβλέψεως, σύμφωνα με όσα καθορίζονται στην περίπτ. 5-στ της παρ.  2.2.405.

 34   ΧΡΕΟΓΡΑΦΑ

        34.00   Μετοχές εισαγμένες στο Χρηματιστήριο εταιριών εσωτερικού

        34.01   Μετοχές μη εισαγμένες στο Χρηματιστήριο εταιριών εσωτερικού

        34.02   Ανεξόφλητες μετοχές εισαγμένες στο Χρηματιστ. Εταιριών
                    εσωτερικού

        34.03   Ανεξόφλητες μετοχές μη εισαγμένες στο Χρηματιστήριο εταιριών
                     εσωτερικού

        34.04   Μερισματαποδείξεις εισπρακτέες μετοχών εταιριών εσωτερικού

        34.05   Ομολογίες ελληνικών δανείων (Γνωμ. 94/1680/1992)

                    34.05.00   Ομολογιακό δάνειο 1889 παγίου 2% (Γνωμ. 94/1680/1992)

                                      34.05.00.00   Αξία κτήσεως τίτλων

                                      34.05.00.01   Δουλευμένοι τόκοι αγορασμένων τίτλων

                    34.05.01   Ομολογιακό δάνειο 1890 Λαρίσης 2,50%
                                      (Γνωμ. 94/1680/1992)

                                      34.05.01.00    Αξία κτήσεως τίτλων

                                      34.05.01.01    Δουλευμένοι τόκοι αγορασμένων τίτλων

                    34.05.02   Ομολογιακό δάνειο .......

                                      34.05.02.00   Αξία κτήσεως τίτλων

                                      34.05.02.01   Δουλευμένοι τόκοι αγορασμένων τίτλων

                    34.05.03

                    ..............

                    34.05.99

        34.06   Ανεξόφλητες ομολογίες ελληνικών δανείων

        34.07   Μερίδια αμοιβαίων κεφαλαίων εσωτερικού

        34.08    Έντοκα γραμμάτια Ελληνικού Δημοσίου (Γνωμ. 94/1680/1992)
                    (Η ανάλυσή του είναι ίδια με την ανάλυση του λογαριασμού 34.05)

        34.09   Λοιπά χρεόγραφα εσωτερικού

        34.10   Μετοχές εισαγμένες στο Χρηματιστήριο εταιριών εξωτερικού

        34.11   Μετοχές μη εισαγμένες στο Χρηματιστήριο εταιριών εξωτερικού

        34.12   Ανεξόφλητες μετοχές εισαγμένες στο Χρηματιστ. Εταιριών
                    εξωτερικού

        34.13   Ανεξόφλητες μετοχές μη εισαγμένες στο Χρηματιστήριο εταιριών
                    εξωτερικού

        34.14   Μερισματαποδείξεις εισπρακτέες μετοχών εταιριών εξωτερικού

        34.15   Ομολογίες αλλοδαπών δανείων

        34.16   Ανεξόφλητες ομολογίες αλλοδαπών δανείων

        34.17   Μερίδια αμοιβαίων κεφαλαίων εξωτερικού

        34.18   ..............................................

        34.19   Λοιπά χρεόγραφα εξωτερικού

        34.20   Προεγγραφές σε υπό έκδοση μετοχές εταιριών εσωτερικού

        34.21   Προεγγραφές σε υπό έκδοση μετοχές εταιριών εξωτερικού

        34.22   Προεγγραφές σε ομολογιακά δάνεια εσωτερικού

        34.23   Προεγγραφές σε ομολογιακά δάνεια εξωτερικού

        34.24   Χρεόγραφα σε τρίτους για εγγύηση

        34.25   Ίδιες μετοχές

        .........

        34.91   Ομόλογα Ελληνικού Δημοσίου (Γνωμ. 94/1680/1992)
                    (Η ανάλυσή του είναι όμοια με την ανάλυση του λογαριασμού 34.05)

        34.92    Τραπεζικά ομόλογα (Γνωμ. 94/1680/1992)
                      (Η ανάλυσή του είναι όμοια με την ανάλυση του λογαριασμού 34.05)

        34.99   Προβλέψεις για υποτιμήσεις χρεογράφων (Π.Δ. 367/94, άρθ. 3 παρ. 1)
                     (τριτοβάθμιοι κατ' είδος χρεογράφου)

2.2.305 Λογαριασμός 34 «Χρεόγραφα»

1. Στους υπολογαριασμούς του 34 παρακολουθούνται τα χρεόγραφα - μετοχές ανώνυμων εταιρειών, ομολογίες, έντοκα γραμμάτια του Ελληνικού Δημοσίου, μερίδια αμοιβαίων κεφαλαίων, ομόλογα Τραπεζών - τα οποία αποκτούνται από την οικονομική μονάδα με σκοπό την τοποθέτηση κεφαλαίων της και την πραγματοποίηση από αυτά άμεσης προσόδου, σύμφωνα και με όσα καθορίζονται στην περίπτ. 1 της παρ.  2.2.112.

Σε ειδικούς υπολογαριασμούς του 34 παρακολουθούνται και οι μερισματαποδείξεις των μετοχών, όταν αυτές, όπως ορίζεται παρακάτω στην περίπτ. 4, αποκόπτονται από τις αντίστοιχες μετοχές τους.

2. Τα χρεόγραφα καταχωρούνται στους οικείους υπολογαριασμούς του 34 με την αξία κτήσεώς τους, για την οποία, καθώς και για τα ειδικά έξοδα αγορών τους, ισχύουν ανάλογα όσα καθορίζονται στην παρ. 2.2.112 για τους λογαριασμούς 18.00 και 18.01.

3. Στους λογαριασμούς 34.02 «ανεξόφλητες μετοχές εισαγμένες στο χρηματιστήριο εταιρειών εσωτερικού», 34.03 «ανεξόφλητες μετοχές μη εισαγμένες στο χρηματιστήριο εταιρειών εσωτερικού», 34.12 «ανεξόφλητες μετοχές εισαγμένες στο χρηματιστήριο εταιρειών εξωτερικού» και 34.13 «ανεξόφλητες μετοχές μη εισαγμένες στο χρηματιστήριο εταιρειών εξωτερικού», καταχωρείται αντίστοιχα η συνολική αξία των μετοχών που η οικονομική μονάδα αποκτάει - χωρίς πρόθεση διαρκούς κατοχής - από την κάλυψη μέρους του μετοχικού κεφαλαίου ανώνυμης εταιρείας, με τον όρο της τμηματικής καταβολής αυτού. Το συνολικό ποσό που οφείλεται για τις μετοχές αυτές καταχωρείται αντίστοιχα στην πίστωση του λογαριασμού 53.07 «οφειλόμενες δόσεις ομολογιών και λοιπών χρεογράφων». Μετά από ολοκληρωτική εξόφληση μετοχών, η συνολική αξία των εξοφλημένων μεταφέρεται από τους λογαριασμούς 34.02-03 και 34.12-13 στους λογαριασμούς 34.00-01 και 34.10-11, αντίστοιχα.

4. Στους λογαριασμούς 34.04 «μερισματαποδείξεις εισπρακτέες μετοχών εταιρειών εσωτερικού» και 34.14 «μερισματαποδείξεις εισπρακτέες μετοχών εταιρειών εξωτερικού» παρακολουθούνται οι μερισματαποδείξεις που αποκόπτονται μετά την έγκριση του ισολογισμού της οικείας ανώνυμης εταιρείας από τη Γενική Συνέλευση των μετόχων της και τη γνωστοποίηση της ημερομηνίας πληρωμής των μερισμάτων. Με τα ποσά των μερισμάτων που δικαιούνται η οικονομική μονάδα, με βάση τις μερισματαποδείξεις που αποκόπτονται, χρεώνονται οι λογαριασμοί αυτοί (34.04 ή 34.14) και πιστώνονται οι λογαριασμοί 76.00 «έσοδα συμμετοχών», όταν πρόκειται για μερισματαποδείξεις από συμμετοχές, και 76.01 «έσοδα χρεογράφων», όταν πρόκειται για μερισματαποδείξεις από χρεόγραφα.

Στις περιπτώσεις που στις μετοχές δεν είναι ενσωματωμένες σχετικές μερισματαποδείξεις (π.χ. όταν πρόκειται για ονομαστικές μετοχές), η απαίτηση της οικονομικής μονάδας για είσπραξη μερισμάτων καταχωρείται στους λογαριασμούς 33.95 «λοιποί χρεώστες διάφοροι σε Δρχ.» και 33.96 «λοιποί χρεώστες σε Ξ.Ν.», κατά περίπτωση.

5. Στους λογαριασμούς 34.05 «ομολογίες ελληνικών δανείων», 34.06 «ανεξόφλητες ομολογίες ελληνικών δανείων», 34.15 «ομολογίες αλλοδαπών δανείων» και 34.16 «ανεξόφλητες ομολογίες αλλοδαπών δανείων» παρακολουθούνται οι ομολογίες οι οποίες αποκτούνται από την οικονομική μονάδα. Σχετικά με τον τρόπο λειτουργίας των λογαριασμών αυτών ισχύουν όσα ορίζονται στις παραπάνω περιπτώσεις, ειδικά δε για τις ανεξόφλητες ομολογίες εφαρμόζεται, αναλογικά, η περίπτωση 3 που αναφέρεται στις ανεξόφλητες μετοχές.

Οι λαχειοφόρες ομολογίες παρακολουθούνται με τον αύξοντα αριθμό τους κατά τρόπο που να γίνεται εξατομίκευση κάθε ομολογίας, για να διασφαλίζονται τα συμφέροντα της οικονομικής μονάδας κατά τις κληρώσεις των λαχνών.

6. Στους λογαριασμούς 34.00 «μετοχές εισαγμένες στο χρηματιστήριο εταιρειών εσωτερικού», 34.01 «μετοχές μη εισαγμένες στο χρηματιστήριο εταιρειών εσωτερικού., 34.10 «μετοχές εισαγμένες στο χρηματιστήριο εταιρειών εξωτερικού» και 34.11 «μετοχές μη εισαγμένες στο χρηματιστήριο εταιρειών εξωτερικού» παρακολουθούνται οι εξοφλημένες μετοχές οι οποίες αποκτούνται από την οικονομική μονάδα.

7. Στους λογαριασμούς 34.07 «μερίδια αμοιβαίων κεφαλαίων εσωτερικού», 34.08 «έντοκα γραμμάτια Ελληνικού Δημοσίου» και 34.17 «μερίδια αμοιβαίων κεφαλαίων εξωτερικού» παρακολουθούνται, αντίστοιχα, τα μερίδια αμοιβαίων κεφαλαίων και τα έντοκα γραμμάτια του Ελληνικού Δημοσίου που αποκτούνται από την οικονομική μονάδα.

8. Στους λογαριασμούς 34.09 «λοιπά χρεόγραφα εσωτερικού» και 34.19 «λοιπά χρεόγραφα εξωτερικού» παρακολουθούνται τα χρεόγραφα που δεν εντάσσονται σε μια από τις συγκεκριμένες κατηγορίες που προβλέπονται από τους υπολογαριασμούς του 34.

9. Στους λογαριασμούς 34.20 «προεγγραφές σε υπό έκδοση μετοχές εταιρειών εσωτερικού», 34.21 «προεγγραφές σε υπό έκδοση μετοχές εταιρειών εξωτερικού», 34.22 «προεγγραφές σε ομολογιακά δάνεια εσωτερικού» και 34.23 «προεγγραφές σε ομολογιακά δάνεια εξωτερικού» καταχωρούνται τα ποσά που καταβάλλονται στις περιπτώσεις προεγγραφών σε εκδόσεις μετοχών και ομολογιακών δανείων. Μετά από την ολοκλήρωση της διαδικασίας για την οριστική εγγραφή στις υπό έκδοση μετοχές ή ομολογίες, η αξία κτήσεώς τους μεταφέρεται στους οικείους υπολογαριασμούς του 34.

10. Στο λογαριασμό 34.24 «χρεόγραφα σε τρίτους για εγγύηση» παρακολουθούνται τα χρεόγραφα που ενεχυριάζονται για διάφορους λόγους, όπως π.χ. για εγγύηση χρηματοδοτήσεων που η οικονομική μονάδα λαβαίνει ή για εγγύηση εκδόσεως εγγυητικών επιστολών. Ο λογαριασμός αυτός χρεώνεται, με αντίστοιχη πίστωση του οικείου υπολογαριασμού του 34, όταν γίνεται π.χ. η ενεχυρίαση, και πιστώνεται με αντίστοιχη χρέωση του οικείου υπολογαριασμού του 34, όταν γίνεται η αποδέσμευση ή επιστροφή των χρεογράφων.

11. Σε περίπτωση πωλήσεως χρεογράφων, το τίμημα καταχωρείται στην πίστωση του οικείου υπολογαριασμού του 34, το αποτέλεσμα δε που προκύπτει καταχωρείται στο λογαριασμό 64.12.02 «διαφορές (ζημίες) από πώληση χρεογράφων», όταν πρόκειται για ζημία, και στο λογαριασμό 76.04.02 «διαφορές (κέρδη) από πώληση χρεογράφων», όταν πρόκειται για κέρδος. Αποτέλεσμα είναι η διαφορά μεταξύ της αξίας κτήσεως (ή απογραφής) και της τιμής πωλήσεως.

12. Για την αποτίμηση των χρεογράφων ισχύουν όσα καθορίζονται στην περίπτ. 6 της παρ. 2.2.112.

13. Στο λογαριασμό 34.25 «ίδιες μετοχές» παρακολουθούνται οι μετοχές εκδόσεως της εταιρείας, στις περιπτώσεις εκείνες που επιτρέπεται από τη νομοθεσία η απόκτησή τους.

Ο λογαριασμός 34.25 εμφανίζεται στον ισολογισμό στην κατηγορία ΔΙΙΙ «χρεόγραφα» του ενεργητικού του υποδείγματος της παρ. 4.1.103, όταν σχηματίζεται στο τέλος της χρήσεως αποθεματικό ισόποσο με την αξία κτήσεως των ίδιων μετοχών.

Αν δεν υπάρχουν κέρδη, για σχηματισμό του παραπάνω αποθεματικού, ο λογαριασμός 34.25 εμφανίζεται στο παθητικό του ισολογισμού αφαιρετικά από το άθροισμα των ιδίων κεφαλαίων.

 35   ΛΟΓΑΡΙΑΣΜΟΙ ΔΙΑΧΕΙΡΙΣΕΩΣ ΠΡΟΚΑΤΑΒΟΛΩΝ ΚΑΙ ΠΙΣΤΩΣΕΩΝ

        35.00   Εκτελωνιστές - Λογ/σμοί προς απόδοση

        35.01   Προσωπικό - Λογ/σμοί προς απόδοση

        35.02   Λοιποί συνεργάτες τρίτοι - Λογ/σμοί προς απόδοση

        35.03   Πάγιες προκαταβολές

        35.04   Πιστώσεις υπέρ τρίτων

        .........

        35.99

2.2.306 Λογαριασμός 35 «Λογαριασμοί διαχειρίσεως προκαταβολών και πιστώσεων»

1. Στους υπολογαριασμούς του 35 παρακολουθούνται οι ομοιογενούς φύσεως και έντονου διαχειριστικού χαρακτήρα απαιτήσεις της οικονομικής μονάδας από τους υπαλλήλους και τους λοιπούς συνεργάτες της, που προέρχονται από καταβολές ποσών που γίνονται σ' αυτούς προσωρινά για την εκτέλεση, για λογαριασμό της, συγκεκριμένου έργου ή εργασίας (π.χ. εκτελωνισμός ή ταξίδι για λήψη παραγγελιών).

Οι υπολογαριασμοί του 35 χρεώνονται με τα ποσά που καταβάλλονται στους προσωρινούς διαχειριστές της οικονομικής μονάδας και πιστώνονται με αντίστοιχη χρέωση των οικείων κατά περίπτωση λογαριασμών, π.χ. αποθεμάτων ή εξόδων, όταν εκτελείται το έργο ή η εργασία και γίνεται η σχετική απόδοση. Η απόδοση αυτή γίνεται αμέσως μετά την εκτέλεση του έργου ή της εργασίας, ή αυτοτελούς τμήματος αυτών.

2. Οι πιστώσεις που ανοίγονται στις Τράπεζες για λογαριασμό συνεργατών (π.χ.  αντιπροσώπων ή εντολοδόχων) της οικονομικής μονάδας είναι δυνατό να παρακολουθούνται στο λογαριασμό 35.04 «πιστώσεις υπέρ τρίτων».

3. Οι πάγιες προκαταβολές (δηλαδή τα ποσά που καταβάλλονται σε διαχειριστές και μετά από κάθε απόδοση συμπληρώνονται και πάλι ώστε οι διαχειριστές αυτοί να κρατούν πάγια το αυτό συνολικό ποσό) παρακολουθούνται στο λογαριασμό 35.03 «πάγιες προκαταβολές». Περιπτώσεις πάγιων προκαταβολών παρουσιάζονται π.χ. στα εργοτάξια, στα κινητά συνεργεία ή στη συγκέντρωση γεωργικών προϊόντων, καθώς επίσης και στις περιπτώσεις αντιμετωπίσεως των μικροεξόδων.

 36   ΜΕΤΑΒΑΤΙΚΟΙ ΛΟΓΑΡΙΑΣΜΟΙ ΕΝΕΡΓΗΤΙΚΟΥ

        36.00   Έξοδα επόμενων χρήσεων
                    (Ανάπτυξη αντίστοιχη των λογ/σμών εξόδων)

        36.01   Έσοδα χρήσεως εισπρακτέα
                    (Ανάπτυξη αντίστοιχη των λογ/σμών εσόδων και κατά οφειλέτη)

        36.02   Αγορές υπό παραλαβή

        36.03   Εκπτώσεις επί αγορών χρήσεως υπό διακανονισμό

 

2.2.307 Λογαριασμός 36 «Μεταβατικοί λογαριασμοί ενεργητικού»

1. Οι μεταβατικοί λογαριασμοί ενεργητικού και παθητικού δημιουργούνται, κατά κανόνα, στο τέλος κάθε χρήσεως με σκοπό τη χρονική τακτοποίηση των εξόδων και εσόδων, έτσι ώστε στα αποτελέσματά της να περιλαμβάνονται μόνο τα έσοδα και έξοδα που πράγματι αφορούν τη συγκεκριμένη αυτή χρήση. Με την τακτοποίηση αυτή πραγματοποιείται ταυτόχρονα η αναμόρφωση των λογαριασμών του ισολογισμού στο πραγματικό μέγεθός τους κατά την ημερομηνία λήξεως της χρήσεως.

Ειδικότερα, στους μεταβατικούς λογαριασμούς ενεργητικού καταχωρούνται τα έξοδα που πληρώνονται μεν μέσα στη χρήση, ανήκουν όμως στην επόμενη ή σε επόμενες χρήσεις. Στους ίδιους μεταβατικούς λογαριασμούς καταχωρούνται και τα έσοδα που ανήκουν στην κλειόμενη χρήση (δουλευμένα), αλλά που δεν εισπράττονται μέσα σ' αυτή, ούτε επιτρέπεται η καταχώρισή τους στη χρέωση προσωπικών λογαριασμών απαιτήσεων, επειδή δεν είναι ακόμη απαιτητά.

2. Στο λογαριασμό 36.00 «έξοδα επόμενων χρήσεων», σε περίπτωση που δεν καταχωρούνται απευθείας σ' αυτόν, μεταφέρονται από τους οικείους λογαριασμούς εξόδων, εκείνα από αυτά που δεν αφορούν την κλειόμενη, αλλά την επόμενη ή τις επόμενες χρήσεις.

Η ανάλυση του λογαριασμού 36.00 σε τριτοβάθμιους υπολογαριασμούς είναι αντίστοιχη με τις αναλύσεις των λογαριασμών εξόδων, στους οποίους μεταφέρονται τα κονδύλια που αφορούν τη νέα (επόμενη) χρήση, αμέσως με την έναρξή της.

3. Στο λογαριασμό 36.01 «έσοδα χρήσεως εισπρακτέα» καταχωρούνται, με αντίστοιχη πίστωση των οικείων λογαριασμών εσόδων της ομάδας 7, τα έσοδα που ανήκουν στην κλειόμενη χρήση αλλά δεν εισπράττονται μέσα σ' αυτή και τα οποία, σύμφωνα π.χ.  με τις σχετικές συμβάσεις, δεν είναι στο τέλος της χρήσεως απαιτητά και για το λόγο αυτό δεν κρίνεται ορθό ή σκόπιμο να φέρονται σε χρέωση των οικείων λογαριασμών απαιτήσεων.

4. Στο λογαριασμό 36.02 «αγορές υπό παραλαβή», στο τέλος της χρήσεως, παρακολουθούνται οι υπό παραλαβή αγορές για τις οποίες περιέρχονται τα τιμολόγια στην οικονομική μονάδα προς της λήξεως της χρήσεως, ενώ τα αγαθά δεν έχουν ακόμη παραληφθεί. Σχετικά με τον τρόπο λειτουργίας του λογαριασμού αυτού ισχύουν όσα καθορίζονται στην περίπτ. 8 της παρ. 2.2.203.

5. Στο λογαριασμό 36.03 «εκπτώσεις επί αγορών χρήσεως υπό διακανονισμό» καταχωρούνται, με αντίστοιχη πίστωση των οικείων λογαριασμών εκπτώσεων, οι εκπτώσεις αγορών που η οικονομική μονάδα δικαιούται στο τέλος της χρήσεως, εφόσον δεν έχει αναγγελθεί το ποσό αυτών και από το λόγο αυτό δεν είναι σκόπιμη η χρέωση του οικείου λογαριασμού του προμηθευτή.

 38   ΧΡΗΜΑΤΙΚΑ ΔΙΑΘΕΣΙΜΑ

        38.00   Ταμείο

        38.01   Διάμεσος λογ/σμός ελέγχου διακινήσεως μετρητών

        38.02   Ληγμένα τοκομερίδια για είσπραξη

        38.03   Καταθέσεις όψεως σε Δρχ.

        38.04   Καταθέσεις προθεσμίας σε Δρχ.

        38.05   Καταθέσεις όψεως σε Ξ.Ν.

        38.06   Καταθέσεις προθεσμίας σε Ξ.Ν.

        .........

        38.99

 

2.2.309 Λογαριασμός 38 «Χρηματικά διαθέσιμα»

1. Στους υπολογαριασμούς του 38 παρακολουθούνται τα διαθέσιμα περιουσιακά στοιχεία της οικονομικής μονάδας, στα οποία περιλαμβάνονται, εκτός από τα μετρητά και τις εισπρακτέες επιταγές επί λογαριασμών όψεως, τα ληξιπρόθεσμα τοκομερίδια, οι καταθέσεις όψεως και οι καταθέσεις προθεσμίας, εκτός αν υπάρχει ειδικός απαγορευτικός λόγος αναλήψεώς τους, οπότε πρόκειται για δεσμευμένες καταθέσεις, οι οποίες παρακολουθούνται στους λογαριασμούς 33.17 «λογαριασμοί δεσμευμένων καταθέσεων σε Δρχ.» και 33.18 «λογαριασμοί δεσμευμένων καταθέσεων σε Ξ.Ν.», σύμφωνα με όσα καθορίζονται στην περίπτ. 13 της παρ. 2.2.304.

2. Στο λογαριασμό 38.01 «διάμεσος λογαριασμός ελέγχου διακινήσεως μετρητών» οι οικονομικές μονάδες έχουν τη δυνατότητα να παρακολουθούν τα μετρητά που διακινούνται από το ταμείο ενός κέντρου (π.χ. έδρας) στο ταμείο άλλου κέντρου (π.χ. υποκαταστήματος) και αντίστροφα.

3. Στο λογαριασμό 38.02 «ληγμένα τοκομερίδια προς είσπραξη» καταχωρούνται τα τοκομερίδια μετά τη λήξη τους, με αντίστοιχη πίστωση του λογαριασμού 76.01.02 «έσοδα ομολογιών ελληνικών δανείων» ή 76.01.07 «έσοδα ομολογιών αλλοδαπών δανείων», κατά περίπτωση.

 39   ΑΠΑΙΤΗΣΕΙΣ ΚΑΙ ΔΙΑΘΕΣΙΜΑ ΥΠΟΚΑΤΑΣΤΗΜΑΤΩΝ ή ΑΛΛΩΝ
        ΚΕΝΤΡΩΝ (Όμιλος λογ/σμών προαιρετικής χρήσεως)

       390   ΠΕΛΑΤΕΣ
                Ανάπτυξη αντίστοιχη του λογ. 30

       391   ΓΡΑΜΜΑΤΙΑ ΕΙΣΠΡΑΚΤΕΑ
                Ανάπτυξη αντίστοιχη του λογ. 31

       392   ΠΑΡΑΓΓΕΛΙΕΣ ΣΤΟ ΕΞΩΤΕΡΙΚΟ
                Ανάπτυξη αντίστοιχη του λογ. 32

       393   ΧΡΕΩΣΤΕΣ ΔΙΑΦΟΡΟΙ
                Ανάπτυξη αντίστοιχη του λογ. 33

       394   ΧΡΕΟΓΡΑΦΑ
                Ανάπτυξη αντίστοιχη του λογ. 34

       395    ΛΟΓΑΡΙΑΣΜΟΙ ΔΙΑΧΕΙΡΙΣΕΩΣ ΠΡΟΚΑΤΑΒΟΛΩΝ
                ΚΑΙ ΠΙΣΤΩΣΕΩΝ
                Ανάπτυξη αντίστοιχη του λογ. 35

       396   ΜΕΤΑΒΑΤΙΚΟΙ ΛΟΓΑΡΙΑΣΜΟΙ ΕΝΕΡΓΗΤΙΚΟΥ
                Ανάπτυξη αντίστοιχη του λογ. 36

       397   .......................................

       398   ΧΡΗΜΑΤΙΚΑ ΔΙΑΘΕΣΙΜΑ
                Ανάπτυξη αντίστοιχη του λογ. 38

2.2.310 Όμιλος λογαριασμών 39 «Απαιτήσεις και διαθέσιμα υποκαταστημάτων ή άλλων κέντρων» (όμιλος λογαριασμών προαιρετικής χρήσεως)

1. Σχετικά με τον τρόπο αναπτύξεως κάθε πρωτοβάθμιου λογαριασμού (390-398) ισχύουν όσα καθορίζονται στην περίπτ. 1 της παρ. 2.2.113.

2. Σχετικά με τον τρόπο λειτουργίας των πρωτοβάθμιων λογαριασμών 390-398 ισχύουν, αντίστοιχα, όσα ορίζονται παραπάνω στις παρ. 2.2.300 έως και 2.2.309 για τους πρωτοβάθμιους λογαριασμούς 30-38.

3. Σε περίπτωση που η οικονομική μονάδα κάνει χρήση του ομίλου λογαριασμών 39, τα κονδύλια των λογαριασμών του ομίλου αυτού, στον ισολογισμό τέλους χρήσεως, συναθροίζονται και εμφανίζονται μαζί με τα αντίστοιχα κονδύλια των λογαριασμών 30-38.

\chapter{ΚΕΦΑΛΑΙΟ}

\section{Λογαριασμοί}

\begin{tabularx}{\linewidth}{lX}

\end{tabularx}

40 Κεφάλαιο

41 Αποθεματικά - Διαφορές αναπροσαρμογής - Επιχορηγήσεις επενδύσεων 

42 Αποτελέσματα εις νέο

43 Ποσά προορισμένα για αύξηση κεφαλαίου 

44 Προβλέψεις

45 Μακροπρόθεσμες υποχρεώσεις 

46 ................................................

47 ................................................

48 Λογαριασμοί συνδέσμου με τα υποκαταστήματα 

49 Προβλέψεις - Μακροπρόθεσμες υποχρεώσεις υποκαταστημάτων ή άλλων κέντρων

2.2.4 ΟΜΑΔΑ 4η: ΚΑΘΑΡΗ ΘΕΣΗ - ΠΡΟΒΛΕΨΕΙΣ - ΜΑΚΡΟΠΡΟΘΕΣΜΕΣ ΥΠΟΧΡΕΩΣΕΙΣ

 2.2.400 Περιεχόμενο και εννοιολογικοί προσδιορισμοί

1. Στην ομάδα 4 παρακολουθούνται η καθαρή θέση της οικονομικής μονάδας, οι προβλέψεις και οι μακροπρόθεσμες υποχρεώσεις της.

2. Καθαρή θέση ή καθαρή περιουσία είναι το ίδιο κεφάλαιο κάθε οικονομικής μονάδας, το οποίο για τις εταιρείες αποτελείται από το μετοχικό ή εταιρικό κεφάλαιο, από τα κάθε είδους και φύσεως αποθεματικά και από το εκάστοτε υπόλοιπο εις νέο (κερδών ή ζημιών).

3. Οι έννοιες των προβλέψεων και μακροπρόθεσμων υποχρεώσεων δίνονται παρακάτω στις παρ. 2.2.405 περίπτ. 1 και 2.2.406 περίπτ. 1, αντίστοιχα.

 40   ΚΕΦΑΛΑΙΟ

        40.00   Καταβλημένο μετοχικό κεφάλαιο κοινών μετοχών  

        40.01   Καταβλημένο μετοχικό κεφάλαιο προνομιούχων μετοχών  

        40.02   Οφειλόμενο μετοχικό κεφάλαιο κοινών μετοχών  

        40.03   Οφειλόμενο μετοχικό κεφάλαιο προνομιούχων μετοχών  

        40.04   Κοινό μετοχικό κεφάλαιο αποσβεσμένο 

        40.05   Προνομιούχο μετοχικό κεφάλαιο αποσβεσμένο  

        40.06   Εταιρικό κεφάλαιο 

        40.07   Κεφάλαιο ατομικών επιχειρήσεων  

        ........

        40.90   Αμοιβαίο κεφάλαιο (Γνωμ. 233/2213/1994) 

        40.91   Καταβλημένο συνεταιριστικό κεφάλαιο (ν. 2169/1993) 

        40.92   Οφειλόμενο συνεταιριστικό κεφάλαιο (ν. 2169/1993) 

        ........

        40.99

2.2.401 Λογαριασμός 40 «Κεφάλαιο»

1. Το κεφάλαιο, στις μεν ατομικές επιχειρήσεις αντιστοιχεί στην καθαρή περιουσία τους, ενώ στις εταιρείες αντιπροσωπεύει την ονομαστική αξία των μετοχών ή των εταιρικών μεριδίων ή μερίδων.

2. Το μετοχικό κεφάλαιο της ανώνυμης εταιρείας διαιρείται σε μετοχές που δίνονται στους μετόχους της. Το κεφάλαιο αυτό σχηματίζεται: (1) από τις εισφορές των μετόχων που καταβάλλονται κατά τη σύσταση της εταιρείας για να συγκροτηθεί το αρχικό κεφάλαιό της, καθώς και μεταγενέστερα για την αύξησή του και (2) από τη διάθεση αποθεματικών ή αδιανέμητων καθαρών κερδών, εφόσον η διάθεση αυτή αποφασίζεται και πραγματοποιείται για την αύξηση του μετοχικού κεφαλαίου. Για την καταβολή του μετοχικού κεφαλαίου - αρχικού και αυξήσεών του - εφαρμόζονται όσα ορίζονται από τη νομοθεσία που ισχύει κάθε φορά.

3. Το εταιρικό κεφάλαιο των λοιπών εταιρειών σχηματίζεται: (1) από τις εισφορές των εταίρων που καταβάλλονται κατά τη σύσταση της εταιρείας, καθώς και μεταγενέστερα για την αύξησή του και (2) από τη διάθεση αποθεματικών ή κερδών, εφόσον η διάθεση αυτή αποφασίζεται και πραγματοποιείται για την αύξηση του εταιρικού κεφαλαίου.

4. Καταβλημένο κεφάλαιο ανώνυμης εταιρείας είναι το μέρος της ονομαστικής αξίας των μετοχών της που έχει εισπραχτεί. Καταβλημένο κεφάλαιο εταιρείας άλλης μορφής είναι το μέρος που έχει εισπραχτεί από την ονομαστική αξία των εταιρικών μεριδίων ή μερίδων που συγκροτούν το κεφάλαιό της. Για τη μερική καταβολή του κεφαλαίου εφαρμόζονται όσα ορίζονται από τη νομοθεσία που ισχύει κάθε φορά.

5. Οφειλόμενο κεφάλαιο είναι, στην περίπτωση ανώνυμης εταιρείας, το μέρος του μετοχικού της κεφαλαίου που οι μέτοχοι οφείλουν να καταβάλουν για να εξοφληθεί η αξία των μετοχών τους, και στην περίπτωση εταιρείας άλλης μορφής, το μέρος του εταιρικού της κεφαλαίου που οι εταίροι οφείλουν να καταβάλουν για να εξοφληθεί η αξία των εταιρικών μεριδίων ή μερίδων τους.

6. Απόσβεση του μετοχικού κεφαλαίου ανώνυμης εταιρείας είναι η απόδοση στους μετόχους της της ονομαστικής αξίας των μετοχών τους, η οποία προβλέπεται από το καταστατικό και γίνεται, είτε από τα προς διάθεση κέρδη, είτε από αποθεματικά της εταιρείας.

7. Μείωση του μετοχικού κεφαλαίου ανώνυμης εταιρείας είναι η απόδοση, στους μετόχους της εταιρείας, της ονομαστικής αξίας των μετοχών τους, που γίνεται από το καταβλημένο μετοχικό κεφάλαιο έπειτα από νομότυπη απόφαση της Γενικής Συνελεύσεως.

8. Η κάλυψη του μετοχικού κεφαλαίου κατά τη σύσταση ανώνυμης εταιρείας, καθώς και σε μεταγενέστερες αυξήσεις, γίνεται με πίστωση των οικείων υπολογαριασμών του 40 και με αντίστοιχη χρέωση του λογαριασμού 33.03 «μέτοχοι λογαριασμός καλύψεως κεφαλαίου». Ειδικότερα για την κάλυψη και καταβολή του μετοχικού κεφαλαίου εφαρμόζονται τα ακόλουθα:

α. Η κάλυψη του μετοχικού κεφαλαίου απεικονίζεται με χρέωση του λογαριασμού 33.03 και με πίστωση του λογαριασμού 40.02 «οφειλόμενο μετοχικό κεφάλαιο κοινών μετοχών» ή 40.03 «οφειλόμενο μετοχικό κεφάλαιο προνομιούχων μετοχών», κατά περίπτωση, και του λογαριασμού 41.01 «οφειλόμενη διαφορά από έκδοση μετοχών υπέρ το άρτιο», εφόσον συντρέχει λόγος.

β. Επακολουθεί η πίστωση του λογαριασμού 33.03, με χρέωση του λογαριασμού 33.04 «οφειλόμενο κεφάλαιο», με το τμήμα το οποίο είναι καταβλητέο συγχρόνως με την κάλυψη, καθώς και με το τμήμα που έχει κληθεί να καταβληθεί και είναι καταβλητέο σε δόσεις οι οποίες λήγουν μέχρι το τέλος της επόμενης χρήσεως. Το τμήμα που δεν έχει κληθεί να καταβληθεί και οι δόσεις που λήγουν μετά το τέλος της επόμενης χρήσεως καταχωρούνται στη χρέωση του λογαριασμού 18.12 «οφειλόμενο κεφάλαιο». Από το λογαριασμό αυτό μεταφέρονται στο τέλος κάθε χρήσεως στη χρέωση του λογαριασμού 33.04, από το κεφάλαιο που έχει κληθεί να καταβληθεί, οι δόσεις που πρέπει να καταβληθούν μέσα στην επόμενη χρήση. Με τις εγγραφές αυτές ο λογαριασμός 33.03 εξισώνεται.

Διευκρινίζεται ότι στο λογαριασμό 33.04 απεικονίζεται το κεφάλαιο που είναι καταβλητέο συγχρόνως με την κάλυψη και από το κεφάλαιο που είναι καταβλητέο σε δόσεις μόνο οι δόσεις εκείνου που έχει κληθεί να καταβληθεί, εφόσον οι δόσεις αυτές λήγουν μέχρι το τέλος της επόμενης χρήσεως. Το κεφάλαιο που δεν έχει κληθεί να καταβληθεί και οι δόσεις εκείνου που έχει κληθεί να καταβληθεί, εφόσον λήγουν μετά το τέλος της επόμενης χρήσεως, απεικονίζονται στο λογαριασμό 18.12, σύμφωνα με όσα ορίζονται στην περίπτ. 16 της παρ. 2.2.112.

γ. Η καταβολή του μετοχικού κεφαλαίου παρακολουθείται στο λογαριασμό 33.04, ο οποίος πιστώνεται με τις εκάστοτε καταβολές, με χρέωση των οικείων λογαριασμών.  Σε περίπτωση που οποιαδήποτε ληξιπρόθεσμη δόση δεν καταβάλλεται εμπρόθεσμα, η δόση αυτή μεταφέρεται από το λογαριασμό 33.04 στον 33.05 «δόσεις μετοχικού κεφαλαίου σε καθυστέρηση», με τον οποίο παρακολουθείται η παραπέρα τύχη τους, σύμφωνα με όσα προβλέπονται από το καταστατικό και τη νομοθεσία που ισχύει κάθε φορά.

δ. Μόλις το μετοχικό κεφάλαιο καταβληθεί ολόκληρο ή - σε περίπτωση που η καταβολή του γίνεται με δόσεις - μόλις καταβληθεί οποιαδήποτε από τις δόσεις, με την πίστωση του λογαριασμού 33.04 γίνεται και ισόποση με την καταβολή μεταφορά από το λογαριασμό 40.02 ή 40.03 στο λογαριασμό 40.00 ή 40.01, κατά περίπτωση. Αν η καταβολή περιλαμβάνει και διαφορά υπέρ το άρτιο, γίνεται ισόποση με την καταβολή της διαφοράς αυτής μεταφορά από το λογαριασμό 41.01 «οφειλόμενη διαφορά από έκδοση μετοχών υπέρ το άρτιο» στο λογαριασμό 41.00 «καταβλημένη διαφορά από έκδοση μετοχών υπέρ το άρτιο». Διευκρινίζεται ότι η μεσολάβηση του λογαριασμού 41.01 είναι πρόσκαιρη και γίνεται για την απεικόνιση της καλύψεως της διαφοράς από έκδοση μετοχών υπέρ το άρτιο, δεδομένου ότι δεν επιτρέπεται καταβολή της διαφοράς αυτής σε δόσεις.

9. Σε περίπτωση αποσβέσεως του μετοχικού κεφαλαίου ανώνυμης εταιρείας εφαρμόζονται τα ακόλουθα:

α. Με το ποσό της αποσβέσεως του μετοχικού κεφαλαίου, που αποφασίζεται νομότυπα, χρεώνεται ο λογαριασμός 88.99 «κέρδη προς διάθεση», με πίστωση του λογαριασμού 53.16 «μέτοχοι αξία μετοχών τους προς απόδοση λόγω αποσβέσεως ή μειώσεως του κεφαλαίου». Ο λογαριασμός 53.16 εξισώνεται με την καταβολή στους μετόχους των σχετικών ποσών.

β. Σε περίπτωση που η απόσβεση ή μέρος αυτής γίνεται με χρησιμοποίηση αποθεματικών, τα σχετικά ποσά μεταφέρονται προηγούμενα από τους οικείους υπολογαριασμούς του 41 στο λογαριασμό 88.07 «λογαριασμός αποθεματικών προς διάθεση».

γ. Με το ποσό της αποσβέσεως χρεώνεται ο λογαριασμός 40.00 ή 40.01 κατά περίπτωση, με αντίστοιχη πίστωση του λογαριασμού 40.04 «κοινό μετοχικό κεφάλαιο αποσβεσμένο» ή 40.05 «προνομιούχο μετοχικό κεφάλαιο αποσβεσμένο».

δ. Οι μετοχές επικαρπίας, που εκδίδονται σε αντικατάσταση των μετοχών που ακυρώνονται, καταχωρούνται σε λογαριασμούς τάξεως (λογ. 04 και 08).

10. Σε περίπτωση μειώσεως του μετοχικού κεφαλαίου ανώνυμης εταιρείας, με το σχετικό ποσό της μειώσεως που αποφασίζεται νομότυπα χρεώνεται ο λογαριασμός 40.00 ή 40.01, κατά περίπτωση, με πίστωση του λογαριασμού 53.16, ο οποίος τελικά εξισώνεται με την καταβολή στους μετόχους των σχετικών ποσών.

11. Στο λογαριασμό 40.06 «εταιρικό κεφάλαιο» παρακολουθείται το κεφάλαιο των λοιπών, εκτός από τις ανώνυμες, εταιρειών.

12. Στο λογαριασμό 40.07 «κεφάλαιο ατομικών επιχειρήσεων» παρακολουθείται το κεφάλαιο που καταθέτει ο επιχειρηματίας στην ατομική του επιχείρηση. Ο λογαριασμός 40.07 παραμένει αμετάβλητος σε όλη τη διάρκεια της χρήσεως, εκτός αν γίνουν νέες καταθέσεις του επιχειρηματία, οι οποίες καταχωρούνται στην πίστωση του λογαριασμού αυτού. Οι αναλήψεις του επιχειρηματία που γίνονται κατά τη διάρκεια της χρήσεως καταχωρούνται στο λογαριασμό 33.07, ο οποίος μετονομάζεται σε «ατομικός λογαριασμός επιχειρηματία». Στον τελευταίο αυτό λογαριασμό μεταφέρεται από το λογαριασμό 88 το αποτέλεσμα κάθε χρήσεως. Το υπόλοιπο του λογαριασμού 33.07, που προκύπτει μετά τη μεταφορά σ' αυτόν του αποτελέσματος της χρήσεως, είναι δυνατό, κατά την κρίση του επιχειρηματία, να μεταφέρεται, είτε στο σύνολό του, είτε κατά μέρος, στο λογαριασμό 40.07.

 41   ΑΠΟΘΕΜΑΤΙΚΑ - ΔΙΑΦΟΡΕΣ ΑΝΑΠΡΟΣΑΡΜΟΓΗΣ - ΕΠΙΧΟΡΗΓΗΣΕΙΣ ΕΠΕΝΔΥΣΕΩΝ

        41.00   Καταβλημένη διαφορά από έκδοση μετοχών υπέρ το άρτιο 

        41.01   Οφειλόμενη διαφορά από έκδοση μετοχών υπέρ το άρτιο 

        41.02   Τακτικό αποθεματικό

        41.03   Αποθεματικά καταστατικού 

        41.04   Ειδικά αποθεματικά 

        41.05   Έκτακτα αποθεματικά 

        41.06   Διαφορές από αναπροσαρμογή αξίας συμμετοχών και χρεογράφων
                    [Ανάπτυξη κατά νόμο περί αναπροσαρμογής
                    (π.χ. α.ν. 148/67, ν. 542/77, ν.1249/1982 κ.λπ.)]

        41.07   Διαφορές από αναπροσαρμογή αξίας λοιπών περιουσιακών στοιχείων
                    (Ανάπτυξη κατά νόμο περί αναπροσαρμογής)

        41.08   Αφορολόγητα αποθεματικά ειδικών διατάξεων νόμων
                    (Ανάπτυξη κατά διάταξη νόμων περί κινήτρων κ.λπ.)

        41.09   Αποθεματικό για ίδιες μετοχές 

        41.10   Επιχορηγήσεις πάγιων επενδύσεων 

        ........

        41.90   Αποθεματικά από απαλλασσόμενα της φορολογίας έσοδα
                    (Απόφ. Υπ. Οικ. 1044770/10159/ΠΟΛ. 1117/1993)

        41.91   Αποθεματικά έσοδα φορολογηθέντα κατ' ειδικό τρόπο
                    (Απόφ. Υπ. Οικ. 1044770/10159/ΠΟΛ. 1117/1993)

        41.92   Αφορολόγητα κέρδη τεχνικών και οικοδομικών επιχειρήσεων
                    (Απόφ. Υπ. Οικ. 1044770/10159/ΠΟΛ.1117/1993)

        ........

        41.95   Διαφορά από εισφορά μηχανολογικού εξοπλισμού ως συμμετοχή μας
                    σε εταιρία του εξωτερικού (Γνωμ.182/2096/1992 \& 205/2135/1994) 

        ........

        41.99

2.2.402 Λογαριασμός 41 «Αποθεματικά - Διαφορές αναπροσαρμογής - Επιχορηγήσεις επενδύσεων»

1. Αποθεματικά είναι συσσωρευμένα καθαρά κέρδη τα οποία δεν έχουν διανεμηθεί, ούτε έχουν ενσωματωθεί στο μετοχικό ή εταιρικό κεφάλαιο. Στην κατηγορία των αποθεματικών κατατάσσεται και η διαφορά από έκδοση μετοχών ή εταιρικών μεριδίων υπέρ το άρτιο.

2. Διαφορά από έκδοση μετοχών ή εταιρικών μεριδίων υπέρ το άρτιο είναι το πλεόνασμα που προκύπτει από την έκδοση μετοχών ή εταιρικών μεριδίων Ε.Π.Ε. σε τιμή μεγαλύτερη από την ονομαστική τους.

3. Τακτικό αποθεματικό είναι εκείνο που σχηματίζεται σύμφωνα με τις διατάξεις περί ανωνύμων εταιρειών και Ε.Π.Ε. που ισχύουν κάθε φορά.

4. Αποθεματικά καταστατικού είναι εκείνα που σχηματίζονται σύμφωνα με ειδικές διατάξεις του καταστατικού της εταιρείας.

5. Ειδικά και έκτακτα ή προαιρετικά αποθεματικά είναι εκείνα τα οποία σχηματίζονται σύμφωνα με απόφαση της τακτικής γενικής συνελεύσεως των μετόχων.

6. Διαφορά από αναπροσαρμογή είναι η υπεραξία που προκύπτει από αναπροσαρμογή της αξίας περιουσιακών στοιχείων του ισολογισμού της οικονομικής μονάδας, η οποία γίνεται σύμφωνα με διατάξεις της νομοθεσίας που ισχύει κάθε φορά.

7. Αφορολόγητα αποθεματικά είναι εκείνα που σχηματίζονται από καθαρά κέρδη τα οποία, σύμφωνα με διατάξεις της φορολογικής νομοθεσίας που ισχύει κάθε φορά, δεν υπάγονται σε φορολογία εισοδήματος.

8. Οι υπολογαριασμοί του 41 λειτουργούν σύμφωνα με τους ακόλουθους κανόνες:

α. Για τους λογαριασμούς 41.00 «καταβλημένη διαφορά από έκδοση μετοχών υπέρ το άρτιο» και 41.01 «οφειλόμενη διαφορά από έκδοση μετοχών υπέρ το άρτιο» ισχύουν όσα καθορίζονται στην περίπτ. 8 της παρ. 2.2.401.

β. Οι λογαριασμοί 41.02 «τακτικό αποθεματικό», 41.03 «αποθεματικά καταστατικού», 41.04 «ειδικά αποθεματικά», 41.05 «έκτακτα αποθεματικά» και 41.08 «αφορολόγητα αποθεματικά ειδικών διατάξεων νόμων» πιστώνονται με τα αποθεματικά που, κάθε φορά, σχηματίζονται από τα καθαρά κέρδη χρήσεως, με χρέωση του λογαριασμού 88.99 «κέρδη προς διάθεση».

γ. Ο λογαριασμός 41.06 «διαφορές από αναπροσαρμογή αξίας συμμετοχών και χρεογράφων» πιστώνεται με την ονομαστική αξία μετοχών ή εταιρικών μεριδίων άλλων εταιρειών, στις οποίες η οικονομική μονάδα συμμετέχει, με χρέωση των οικείων υπολογαριασμών του 18 ή του 34. Οι μετοχές αυτές (ή τα εταιρικά μερίδια) λαμβάνονται χωρίς αντάλλαγμα έπειτα από νόμιμη αναπροσαρμογή ισολογισμών ή κεφαλαιοποίηση αποθεματικών.

δ. Ο λογαριασμός 41.07 «διαφορές από αναπροσαρμογή αξίας λοιπών περιουσιακών στοιχείων» πιστώνεται με τη διαφορά που προκύπτει κατά την αναπροσαρμογή της αξίας περιουσιακών στοιχείων του ισολογισμού της οικονομικής μονάδας, που γίνεται με βάση ειδικό εκάστοτε νόμο, με χρέωση των οικείων λογαριασμών στους οποίους παρακολουθούνται τα περιουσιακά αυτά στοιχεία, για τα οποία γίνεται αναπροσαρμογή της αξίας τους.

ε. Σε περίπτωση που θα χρησιμοποιηθεί αποθεματικό για την κάλυψη ζημίας ή για να διανεμηθεί στους μετόχους, το σχετικό ποσό μεταφέρεται από τον οικείο υπολογαριασμό του 41 στην πίστωση του λογαριασμού 88.07 «λογαριασμός αποθεματικών προς διάθεση».

στ. Ο λογαριασμός 41.09 «αποθεματικό για ίδιες μετοχές» πιστώνεται με το αποθεματικό που σχηματίζεται για την κάλυψη της αξίας κτήσεως μετοχών εκδόσεως της εταιρείας, σύμφωνα με τη νομοθεσία που ισχύει κάθε φορά. 

ζ. Ο λογαριασμός 41.10 «επιχορηγήσεις πάγιων επενδύσεων» πιστώνεται με τις χορηγούμενες επιχορηγήσεις για τη χρηματοδότηση πάγιων στοιχείων της οικονομικής μονάδας, με χρέωση του οικείου λογαριασμού του ενεργητικού λογαριασμός τρίτων ή λογαριασμός ταμιακών διαθεσίμων).

Στο τέλος της χρήσεως, από το λογαριασμό 41.10 μεταφέρεται στο λογαριασμό 81.01.05 «αναλογούσες στη χρήση επιχορηγήσεις πάγιων επενδύσεων» ποσό ίσο με τις τακτικές και τις πρόσθετες αποσβέσεις πάγιων στοιχείων των λογαριασμών 66 και 85, που αναλογούν στην αξία των αποσβέσιμων πάγιων στοιχείων που χρηματοδοτήθηκε από τις πιο πάνω επιχορηγήσεις.

Σε περίπτωση εκποιήσεως, καταστροφής ή αχρηστεύσεως οποιουδήποτε πάγιου στοιχείου που χρηματοδοτήθηκε από τις παραπάνω επιχορηγήσεις, από το λογαριασμό 41.10 μεταφέρεται στην πίστωση του οικείου λογαριασμού του πάγιου στοιχείου το υπόλοιπο της επιχορηγήσεως που αφορά το στοιχείο αυτό.

Σε περίπτωση που οι επιχορηγήσεις παραχωρούνται στις οικονομικές μονάδες με όρους ή δεσμεύσεις που θέτονται από τις αρχές ή τους οργανισμούς που τις παραχωρούν, οι λογιστικοί χειρισμοί προσαρμόζονται στο νομικό πλαίσιο, το οποίο διέπει κάθε επιχορήγηση. Έτσι, π.χ. σε περίπτωση που οι επιχορηγήσεις χαρακτηρίζονται ως αφορολόγητα αποθεματικά, με χρέωση του οικείου λογαριασμού του ενεργητικού (λογαριασμός τρίτων ή λογαριασμός ταμιακών διαθεσίμων), πιστώνεται ο οικείος υπολογαριασμός του 41.08 «αφορολόγητα αποθεματικά ειδικών διατάξεων νόμων», χωρίς στο τέλος κάθε χρήσεως, να γίνεται μεταφορά των επιχορηγήσεων στο λογαριασμό 81.01.05.

 42   ΑΠΟΤΕΛΕΣΜΑΤΑ ΕΙΣ ΝΕΟ 

        42.00   Υπόλοιπο κερδών εις νέο
                    (Ανάπτυξη κατά χρήση)

        42.01   Υπόλοιπο ζημιών χρήσεως εις νέο 

        42.02   Υπόλοιπο ζημιών προηγούμενων χρήσεων
                    (Ανάπτυξη κατά χρήση)

        42.03   .........................................

        42.04   Διαφορές φορολογικού ελέγχου προηγούμενων χρήσεων

        ........

        42.99

2.2.403 Λογαριασμός 42 «Αποτελέσματα εις νέο»

1. Στο λογαριασμό 42.00 «υπόλοιπο κερδών εις νέο» μεταφέρεται από το λογαριασμό 88.99 «κέρδη προς διάθεση» το τελικό υπόλοιπο που απομένει μετά τη διάθεση των κερδών.

2. Στο λογαριασμό 42.01 «υπόλοιπο ζημιών χρήσεως εις νέο» μεταφέρεται από το λογαριασμό 88.98 «ζημίες εις νέο» το ποσό των ζημιών που, τελικά, μένει ακάλυπτο.

3. Στο λογαριασμό 42.02 «υπόλοιπο ζημιών προηγούμενων χρήσεων» μεταφέρεται από το λογαριασμό 42.01 «υπόλοιπο ζημιών χρήσεως εις νέο» το ποσό εκείνο των ζημιών που δεν καλύπτεται κατά την επόμενη χρήση από κέρδη της ή από διάθεση αποθεματικών.

4. Ο λογαριασμός 42.04 «διαφορές φορολογικού ελέγχου προηγούμενων χρήσεων» λειτουργεί κατά τον ακόλουθο τρόπο:

α. Πιστώνεται, με χρέωση των οικείων λογαριασμών του πάγιου ενεργητικού ή των απαιτήσεων, για να επαναφερθούν στους λογαριασμούς αυτούς τα ποσά αποσβέσεων προηγούμενων χρήσεων που ο φορολογικός έλεγχος δεν αναγνωρίζει για έκπτωση, ιδίως διότι:

- ποσά που καταχωρούνται στα έξοδα της χρήσεως χαρακτηρίζονται από το φορολογικό έλεγχο ότι ανήκουν στο πάγιο ενεργητικό,

- διενεργούνται αποσβέσεις πάγιων στοιχείων μεγαλύτερες από εκείνες που αναγνωρίζει η φορολογική νομοθεσία,

- απαιτήσεις που αποσβέστηκαν σαν ανεπίδεκτες εισπράξεως δεν αναγνωρίζονται από το φορολογικό έλεγχο.

β. Πιστώνεται, με χρέωση των οικείων λογαριασμών ταμείου ή απαιτήσεων κατά Ελληνικού Δημοσίου, με τους φόρους εισοδήματος προηγούμενων χρήσεων οι οποίοι αναγνωρίζονται για επιστροφή οριστικά και αμετάκλητα.

γ. Πιστώνεται, με χρέωση του λογαριασμού 41.08 «αφορολόγητα αποθεματικά ειδικών διατάξεων νόμων», με τα ποσά αφορολόγητων αποθεματικών που δεν αναγνωρίζονται από το φορολογικό έλεγχο.

δ. Χρεώνεται, με πίστωση του λογαριασμού 33.98 «επίδικες απαιτήσεις κατά Ελληνικού Δημοσίου» έπειτα από την οριστικοποίηση της εκκρεμοδικίας, με τα ποσά φόρου εισοδήματος και σχετικών προσαυξήσεων που, τελικά, καταλογίζονται σε βάρος της οικονομικής μονάδας.

Χρεώνεται επίσης, με πίστωση του λογαριασμού 54.99 «φόροι - τέλη καθυστερούμενοι προηγούμενων χρήσεων», με τα ποσά φόρου εισοδήματος προηγούμενων χρήσεων και σχετικών προσαυξήσεων που βεβαιώνονται μέσα στη χρήση οριστικά και αμετάκλητα.

ε. Το υπόλοιπο που παρουσιάζει ο λογαριασμός 42.04 στο τέλος κάθε χρήσεως μεταφέρεται στο λογαριασμό 88.06 «διαφορές φορολογικού ελέγχου προηγούμενων χρήσεων».

 43   ΠΟΣΑ ΠΡΟΟΡΙΣΜΕΝΑ ΓΙΑ ΑΥΞΗΣΗ ΚΕΦΑΛΑΙΟΥ 

        43.00   Καταθέσεως μετόχων 

        43.01   Καταθέσεις εταίρων 

        43.02   Διαθέσιμα μερίσματα χρήσεως για αύξηση μετοχικού κεφαλαίου 

        ........

        43.90   Αποθεματικά διατιθέμενα για αύξηση κεφαλαίου
                    (Γνωμ. 241/2228/1995)

        ........

        43.99

2.2.404 Λογαριασμός 43 «Ποσά προορισμένα για αύξηση του κεφαλαίου»

1. Στο λογαριασμό 43.00 «καταθέσεις μετόχων» παρακολουθούνται οι καταθέσεις που γίνονται από τους μετόχους για να καλυφτεί, μερικά ή ολικά, η αύξηση του μετοχικού κεφαλαίου ανώνυμης εταιρείας. Προϋποθέσεις για να κινηθεί ο λογαριασμός 43.00 είναι: (1) να μην έχει ολοκληρωθεί η σχετική διαδικασία αυξήσεως του μετοχικού κεφαλαίου και (2) να έχει ληφθεί ανάλογη απόφαση του διοικητικού συμβουλίου της εταιρείας. Στην περίπτωση που, σύμφωνα με τις διατάξεις του νόμου και του καταστατικού, απαιτείται απόφαση της γενικής συνελεύσεως των μετόχων, η ημερομηνία συγκλήσεώς της καθορίζεται με την ίδια απόφαση του διοικητικού συμβουλίου και δεν απέχει περισσότερο από ένα έτος από την ημερομηνία της αποφάσεως αυτής. Ανάληψη των καταθέσεων της παραγράφου αυτής επιτρέπεται μόνο στην περίπτωση που η αύξηση του μετοχικού κεφαλαίου δεν πραγματοποιείται, είτε επειδή η πρόταση του διοικητικού συμβουλίου της εταιρείας δεν εγκρίνεται από τη γενική συνέλευση των μετόχων της, όταν απαιτείται η έγκριση αυτή, είτε επειδή η ίδια η απόφαση του διοικητικού συμβουλίου για την αύξηση του μετοχικού κεφαλαίου ανακαλείται από αυτό πριν ολοκληρωθεί η σχετική διαδικασία, όταν η αύξηση γίνεται με απόφαση του διοικητικού συμβουλίου.

Αμέσως έπειτα από την ολοκλήρωση της σχετικής διαδικασίας για την αύξηση του μετοχικού κεφαλαίου, το υπόλοιπο του λογαριασμού 43.00 μεταφέρεται στην πίστωση του λογαριασμού 33.04 «οφειλόμενο κεφάλαιο».

2. Στο λογαριασμό 43.01 «καταθέσεις εταίρων» παρακολουθούνται οι καταθέσεις που γίνονται από τους εταίρους των λοιπών, εκτός από τις ανώνυμες, εταιρειών για να καλυφτεί η προσεχής αύξηση του εταιρικού κεφαλαίου. Σε περίπτωση που, μέσα σε ένα εξάμηνο αφότου οι εταίροι καταθέσουν τα σχετικά ποσά, δεν πραγματοποιηθεί η αύξηση του εταιρικού κεφαλαίου, τα ποσά αυτά μεταφέρονται στην πίστωση του λογαριασμού 53.14 «βραχυπρόθεσμες υποχρεώσεις προς εταίρους και διοικούντες».

3. Στο λογαριασμό 43.02 «διαθέσιμα μερίσματα χρήσεως για αύξηση μετοχικού κεφαλαίου» μεταφέρεται, από το λογαριασμό 53.01 «μερίσματα πληρωτέα», το μέρος του πρώτου ή πρόσθετου μερίσματος που διανέμεται, το οποίο προτείνεται να διατεθεί για αύξηση του μετοχικού κεφαλαίου.

Αμέσως έπειτα από την ολοκλήρωση της σχετικής διαδικασίας για την αύξηση του μετοχικού κεφαλαίου, το υπόλοιπο του λογαριασμού 43.02 μεταφέρεται στην πίστωση του λογαριασμού 33.04.

 44   ΠΡΟΒΛΕΨΕΙΣ 

        44.00   Προβλέψεις για αποζημίωση προσωπικού λόγω εξόδου από την
                    υπηρεσία 

                    44.00.00   Σχηματισμένες προβλέψεις 

                                    Ανάπτυξη σύμφωνα με τις ανάγκες κάθε οικονομικής μονάδας,
                                    με διάκριση των προβλέψεων για αποζημίωση προσωπικού,
                                    σε προβλέψεις για το έμμισθο προσωπικό και σε προβλέψεις
                                    για το ημερομίσθιο προσωπικό.

                    44.00.01   Χρησιμοποιημένες προβλέψεις
                                    Ανάπτυξη αντίστοιχη του λογ/σμοί 44.00.00

        44.01

        ........

        44.09   Λοιπές προβλέψεις εκμεταλλεύσεως

                    44.09.00    Σχηματισμένες προβλέψεις
                                    Ανάπτυξη σύμφωνα με τις ανάγκες κάθε οικονομικής μονάδας

                    44.09.01    Χρησιμοποιημένες προβλέψεις
                                    Ανάπτυξη αντίστοιχη του λογ/σμού 44.09.00

        44.10   Προβλέψεις απαξιώσεων και υποτιμήσεων πάγιων στοιχείων 

        44.11   Προβλέψεις για επισφαλείς απαιτήσεις 

                    44.11.00 Προβλέψεις για απόσβεση επισφαλών πελατών
                                    (άρθρο 31 παρ. 1 - θ' Ν. 2238/1994)

        44.12   Προβλέψεις για εξαιρετικούς κινδύνους και έκτακτα έξοδα 

        44.13   Προβλέψεις για έξοδα προηγούμενων χρήσεων 

        44.14   Προβλέψεις για συναλλαγματικές διαφορές από αποτίμηση
                    απαιτήσεων και λοιπών υποχρεώσεων
                    (Ανάπτυξη κατά Ξ.Ν.)

        44.15   Προβλέψεις για συναλλαγματικές διαφορές από πιστώσεις και
                    δάνεια για κτήσεις πάγιων στοιχείων

                    (Ανάπτυξη κατά πίστωση ή δάνειο)

        ........

        44.98   Λοιπές έκτακτες προβλέψεις

        44.99 
        
 45   ΜΑΚΡΟΠΡΟΘΕΣΜΕΣ ΥΠΟΧΡΕΩΣΕΙΣ 

        45.00   Ομολογιακά δάνεια σε Δρχ. μη μετατρέψιμα σε μετοχές 

        45.01   Ομολογιακά δάνεια σε Δρχ. μετατρέψιμα σε μετοχές 

        45.02   Ομολογιακά δάνεια σε Δρχ. με ρήτρα Ξ.Ν. μη μετατρέψιμα
                     σε μετοχές

        45.03   Ομολογιακά δάνεια σε Δρχ. με ρήτρα Ξ.Ν. μετατρέψιμα σε μετοχές

        45.04   Ομολογιακά δάνεια σε Ξ.Ν. μη μετατρέψιμα σε μετοχές

        45.05   Ομολογιακά δάνεια σε Ξ.Ν. μετατρέψιμα σε μετοχές 

        45.06

        ........

        45.10   Τράπεζες - λογ/σμοί μακροπρόθεσμων υποχρεώσεων σε Δρχ. 

        45.11   Τράπεζες - λογαριασμοί μακροπρόθεσμων υποχρεώσεων σε Δρχ.
                     με ρήτρα Ξ.Ν. 

        45.12   Τράπεζες - λογ/σμοί μακροπρόθεσμων υποχρεώσεων σε Ξ.Ν. 

        45.13   Ταμιευτήρια - λογ/σμοί μακροπρόθεσμων υποχρεώσεων

        45.14   Μακροπρόθεσμες υποχρεώσεις προς συνδεμένες επιχειρήσεις σε Δρχ.

        45.15   Μακροπρόθεσμες υποχρεώσεις προς συνδεμένες επιχειρήσεις σε Ξ.Ν.

        45.16   Μακροπρόθεσμες υποχρεώσεις προς λοιπές συμμετοχικού
                     ενδιαφέροντος επιχειρήσεις σε Δρχ. 

        45.17   Μακροπρόθεσμες υποχρεώσεις προς λοιπές συμμετοχικού
                     ενδιαφέροντος επιχειρήσεις σε Ξ.Ν. 

        45.18   Μακροπρόθεσμες υποχρεώσεις προς εταίρους και διοικούντες 

        45.19   Γραμμάτια πληρωτέα σε Δρχ.

        45.20   Γραμμάτια πληρωτέα σε Ξ.Ν.

        45.21   Γραμμάτια πληρωτέα εκδόσεως Ν.Π.Δ.Δ. και Δημόσιων
                     Επιχειρήσεων

        45.22   Ελληνικό Δημόσιο (οφειλόμενοι φόροι) 

        45.23   Ασφαλιστικοί Οργανισμοί 

        45.24   Μη δουλευμένοι τόκοι γραμματίων πληρωτέων σε Δρχ.
                     (αντίθετος λ/σμός)

        45.25   Μη δουλευμένοι τόκοι γραμματίων πληρωτέων σε Ξ.Ν. (αντίθετος
                     λ/σμός)

        45.26   Μη δουλευμένοι τόκοι γραμματίων πληρωτέων εκδόσεως Ν.Π.Δ.Δ.
                     και Δημόσιων Επιχειρήσεων (αντίθετος λ/σμός)

        ........

        45.98   Λοιπές μακροπρόθεσμες υποχρεώσεις σε Δρχ.

        45.99   Λοιπές μακροπρόθεσμες υποχρεώσεις σε Ξ.Ν.

 

2.2.406 Λογαριασμός 45 «Μακροπρόθεσμες υποχρεώσεις»

1. Οι υποχρεώσεις διακρίνονται, ανάλογα με το χρόνο ληκτότητάς τους, σε μακροπρόθεσμες και βραχυπρόθεσμες. Μακροπρόθεσμες είναι οι υποχρεώσεις εκείνες για τις οποίες η προθεσμία εξοφλήσεώς τους λήγει μετά από το τέλος της επόμενης χρήσεως. Οι λοιπές υποχρεώσεις, δηλαδή εκείνες για τις οποίες η προθεσμία εξοφλήσεώς τους λήγει ως το τέλος της επόμενης χρήσεως, θεωρούνται βραχυπρόθεσμες και παρακολουθούνται στους οικείους λογαριασμούς της ομάδας 5.

2. Για να τακτοποιηθούν οι μακροπρόθεσμες υποχρεώσεις, κατά την κατάρτιση κάθε ισολογισμού, εφαρμόζονται οι ακόλουθοι κανόνες:

α. Κάθε μακροπρόθεσμη υποχρέωση που μετατρέπεται σε βραχυπρόθεσμη μεταφέρεται στον αρμόδιο λογαριασμό της ομάδας 5.

β. Οι ομολογίες που είναι πληρωτέες μέσα στη νέα χρήση μεταφέρονται από τους λογαριασμούς 45.00-45.05 στο λογαριασμό 53.04 «ομολογίες πληρωτέες».

γ. Τα ποσά των μακροπρόθεσμων υποχρεώσεων των λογαριασμών 45.10-45.99 που είναι πληρωτέα μέσα στη νέα χρήση μεταφέρονται από τους λογαριασμούς αυτούς στους λογαριασμούς 53.17 «μακροπρόθεσμες υποχρεώσεις πληρωτέες στην επόμενη χρήση σε Δρχ.» και 53.18 «μακροπρόθεσμες υποχρεώσεις πληρωτέες στην επόμενη χρήση σε Ξ.Ν.» και επαναφέρονται στους λογαριασμούς 45.10-45.99 κατά την έναρξη της νέας χρήσεως, εφόσον, για την ενότητα της παρακολουθήσεως ή για άλλο λόγο, η οικονομική μονάδα επιθυμεί αυτή τη μεταφορά. Παρέχεται η δυνατότητα, αντί από τη μεταφορά και επαναφορά, να εμφανίζονται τα σχετικά ποσά των λογαριασμών 53.17 και 53.18 στην κατηγορία των βραχυπρόθεσμων υποχρεώσεων του ισολογισμού, χωρίς να μεσολαβεί άνοιγμα των λογαριασμών τούτων στα λογιστικά βιβλία.

δ. Οι μακροπρόθεσμες υποχρεώσεις σε ξένο νόμισμα αποτιμούνται με βάση την επίσημη τιμή του ξένου νομίσματος (τιμή πωλήσεως της Τράπεζας της Ελλάδος) κατά την ημέρα κλεισίματος του ισολογισμού, σύμφωνα με όσα καθορίζονται στην παρ. 2.3.2 σχετικά με την αποτίμηση των απαιτήσεων και υποχρεώσεων σε ξένο νόμισμα.

3. Στους λογαριασμούς 45.00-45.09 παρακολουθούνται τα δάνεια που συνάπτονται με έκδοση ομολογιών. Η καταχώριση των δανείων αυτών γίνεται στην τιμή στην οποία η οικονομική μονάδα είναι υποχρεωμένη να εξοφλεί τις ομολογίες. Σε περίπτωση που προκύπτει διαφορά από την έκδοση των ομολογιών κάτω από το άρτιο ή από την εξόφλησή τους πάνω από το άρτιο, η διαφορά αυτή καταχωρείται στο λογαριασμό 16.16 «διαφορές εκδόσεως και εξοφλήσεως ομολογιών» και αποσβένεται σύμφωνα με όσα καθορίζονται στην περίπτ. 23 της παρ. 2.2.110.

4. Στους λογαριασμούς 45.10-45.13 παρακολουθούνται οι μακροπρόθεσμες υποχρεώσεις προς Τράπεζες και Ταμιευτήρια, από δάνεια ή άλλες χορηγήσεις που γίνονται από τους οργανισμούς αυτούς προς την οικονομική μονάδα.

5. Στους λογαριασμούς 45.14-45.15 παρακολουθούνται οι μακροπρόθεσμες υποχρεώσεις της οικονομικής μονάδας προς συνδεμένες επιχειρήσεις, ενώ στους λογαριασμούς 45.16-45.17 παρακολουθούνται οι μακροπρόθεσμες υποχρεώσεις της προς λοιπές επιχειρήσεις στις οποίες έχει συμμετοχικό ενδιαφέρον, επειδή διαθέτει συμμετοχές του λογαριασμού 18.01.

Σχετικά με τις συνδεμένες επιχειρήσεις, καθώς και με τις συμμετοχικού ενδιαφέροντος λοιπές επιχειρήσεις, ισχύουν όσα καθορίζονται στην περίπτ. 10 της παρ. 2.2.112, σε συνδυασμό με όσα καθορίζονται και στην περίπτ. 1 της αυτής παραγράφου.

6. Στους λογαριασμούς 45.18-45.99 παρακολουθούνται οι λοιπές μακροπρόθεσμες υποχρεώσεις της οικονομικής μονάδας. Ειδικά για τους λογαριασμούς 45.24 «μη δουλευμένοι τόκοι γραμματίων πληρωτέων σε Δρχ.», 45.25 «μη δουλευμένοι τόκοι γραμματίων πληρωτέων σε Ξ.Ν.» και 45.26 «μη δουλευμένοι τόκοι γραμματίων πληρωτέων εκδόσεως Ν.Π.Δ.Δ. και Δημοσίων Επιχειρήσεων» ισχύουν όσα καθορίζονται στην περιπτ. 3 της παρ. 2.2.502, με τη διαφορά ότι αντί για το λογαριασμό 65.06 χρησιμοποιούνται οι οικείοι υπολογαριασμοί του 65.01, δηλαδή οι υπολογαριασμοί 65.01.07 «τόκοι και έξοδα μακροπρόθεσμων γραμματίων πληρωτέων σε Δρχ.» και 65.01.08 «τόκοι και έξοδα μακροπρόθεσμων γραμματίων πληρωτέων σε Ξ.Ν.».

 48   ΛΟΓΑΡΙΑΣΜΟΙ ΣΥΝΔΕΣΜΟΥ ΜΕ ΤΑ ΥΠΟΚΑΤΑΣΤΗΜΑΤΑ

       Ανάπτυξη σύμφωνα με τις ανάγκες κάθε μονάδας σε περίπτωση
       παρακολουθήσεως όλων ή μερικών υποκαταστημάτων με αυτοτελή
       λογιστική. (Σχετικές Γνωμ. ΕΣΥΛ: 64/1414/ 1990, 108/1811/1992,
       131/1877/1993, 200/2125/1994)

2.2.409 Λογαριασμός 48 «Λογαριασμοί συνδέσμου με τα υποκαταστήματα»

1. Στο λογαριασμό 48 παρακολουθούνται οι δοσοληψίες μεταξύ κεντρικού και υποκαταστημάτων της οικονομικής μονάδας, στις περιπτώσεις εκείνες που τα υποκαταστήματα έχουν λογιστική αυτοτέλεια.

2. Η ανάπτυξη του λογαριασμού 48 σε υπολογαριασμούς γίνεται σύμφωνα με τις ανάγκες που κάθε οικονομική μονάδα επιθυμεί να ικανοποιεί, έτσι ώστε να επιτυγχάνεται η απαιτούμενη προσαρμογή, π.χ. στο είδος της λογιστικής αυτοτέλειας κάθε υποκαταστήματος ή στον αριθμό των υποκαταστημάτων.

3. Κατά την κατάρτιση του ισολογισμού τέλους χρήσεως εφαρμόζονται υποχρεωτικά οι ακόλουθοι κανόνες:  

α. Ο λογαριασμός 48 δεν εμφανίζεται στον ισολογισμό τέλους χρήσεως. Το υπόλοιπο του λογαριασμού αυτού συμψηφίζεται κατά την ενσωμάτωση των ισολογισμών (ατελών) των υποκαταστημάτων στον ισολογισμό (γενικό) της οικονομικής μονάδας, κατά την οποία ενσωμάτωση συμψηφίζονται αμοιβαία οι λογαριασμοί που αφορούν δοσοληψίες μεταξύ των υποκαταστημάτων, καθώς και μεταξύ τούτων και του κεντρικού. Κατά την ενσωμάτωση αυτή συμψηφίζονται και τα υπολογιστικά ποσά εσόδων ή εξόδων, τα οποία είχαν ενδεχόμενα λογιστεί κατά τις μεταξύ υποκαταστημάτων ή υποκαταστημάτων και κεντρικού δοσοληψίες.

β. Οι οικονομικές μονάδες, που έχουν στο εξωτερικό υποκαταστήματα, τα οποία τηρούν αυτοτελή λογιστική, ενσωματώνουν στους ισολογισμούς της έδρας τους, τους ισολογισμούς των υποκαταστημάτων τους, όπως ορίζεται στην προηγούμενη υποπερίπτωση α. Η ενσωμάτωση αυτή γίνεται έπειτα από προηγούμενη μετατροπή σε δραχμές των διάφορων στοιχείων που είναι εκφρασμένα σε ξένο νόμισμα, όπως ορίζεται στην υποπαρ. 2.3.302.

 49   ΠΡΟΒΛΕΨΕΙΣ - ΜΑΚΡΟΠΡΟΘΕΣΜΕΣ ΥΠΟΧΡΕΩΣΕΙΣ
        ΥΠΟΚΑΤΑΣΤΗΜΑΤΩΝ ή ΑΛΛΩΝ ΚΕΝΤΡΩΝ
        (Όμιλος λογαριασμών προαιρετικής χρήσεως)

       494   ΠΡΟΒΛΕΨΕΙΣ
                Ανάπτυξη αντίστοιχη του λογ. 44

       495   ΜΑΚΡΟΠΡΟΘΕΣΜΕΣ ΥΠΟΧΡΕΩΣΕΙΣ
                Ανάπτυξη αντίστοιχη του λογ. 45

       496   ..................................

       497   ..................................

       498   ΛΟΓΑΡΙΑΣΜΟΙ ΣΥΝΔΕΣΜΟΥ ΜΕ ΤΑ ΛΟΙΠΑ
                ΥΠΟΚΑΤΑΣΤΗΜΑΤΑ
                (Αυτοτελούς λογιστικής)
                Ανάπτυξη αντίστοιχη του λογ. 48

2.2.410 Όμιλος λογαριασμών 49 «Προβλέψεις - Μακροπρόθεσμες υποχρεώσεις
υποκαταστημάτων ή άλλων κέντρων» (όμιλος λογαριασμών προαιρετικής χρήσεως)

1. Σχετικά με τον τρόπο αναπτύξεως κάθε πρωτοβάθμιου λογαριασμού (494-498) ισχύουν όσα καθορίζονται στην περίπτ. 1 της παρ. 2.2.113.

2. Σχετικά με τον τρόπο λειτουργίας των πρωτοβάθμιων λογαριασμών 494-498 ισχύουν, αντίστοιχα, όσα ορίζονται παραπάνω στις παρ. 2.2.405 έως και 2.2.409 για τους πρωτοβάθμιους λογαριασμούς 44-48.

3. Σε περίπτωση που η οικονομική μονάδα κάνει χρήση του ομίλου λογαριασμών 49, τα κονδύλια των λογαριασμών του ομίλου αυτού, στον ισολογισμό χρήσεως, συναθροίζονται και εμφανίζονται μαζί με τα αντίστοιχα κονδύλια των λογαριασμών 44-48.


\chapter{ΒΡΑΧΥΠΡΟΘΕΣΜΕΣ ΥΠΟΧΡΕΩΣΕΙΣ}

\section{Λογαριασμοί}

\begin{tabularx}{\linewidth}{lX}

\end{tabularx}

50 Προμηθευτές

51 Γραμμάτια πληρωτέα

52 Τράπεζες - Λογαριασμοί βραχυπρόθεσμων υποχρεώσεων

53 Πιστωτές διάφοροι

54 Υποχρεώσεις από φόρους - τέλη

55 Ασφαλιστικοί οργανισμοί

56 Μεταβατικοί λογαριασμοί παθητικού

57 ...............................................................

58 Λογαριασμοί περιοδικής κατανομής

59 Βραχυπρόθεσμες υποχρεώσεις υποκαταστημάτων ή άλλων κέντρων 

2.2.5 ΟΜΑΔΑ 5η: ΒΡΑΧΥΠΡΟΘΕΣΜΕΣ ΥΠΟΧΡΕΩΣΕΙΣ

2.2.500 Περιεχόμενο και εννοιολογικοί προσδιορισμοί

1. Στην ομάδα 5 παρακολουθούνται οι βραχυπρόθεσμες υποχρεώσεις της οικονομικής μονάδας. Βραχυπρόθεσμες υποχρεώσεις είναι εκείνες για τις ποίες η προθεσμία εξοφλήσεώς τους λήγει μέχρι το τέλος της επόμενης χρήσεως.

2. Σχετικά με τις μακροπρόθεσμες υποχρεώσεις, που με την πάροδο του χρόνου μετατρέπονται σε βραχυπρόθεσμες, ισχύουν όσα καθορίζονται στην περίπτ. 2 της παρ. 2.2.406.

 50   ΠΡΟΜΗΘΕΥΤΕΣ

        50.00   Προμηθευτές εσωτερικού

        50.01   Προμηθευτές εξωτερικού

        50.02   Ελληνικό Δημόσιο

        50.03   Ν.Π.Δ.Δ. και Δημόσιες Επιχειρήσεις

        50.04   Προμηθευτές - Εγγυήσεις ειδών συσκευασίας

        50.05   Προκαταβολές σε προμηθευτές

        50.06   Προμηθευτές - Παρακρατημένες εγγυήσεις

        50.07   Προμηθευτές αντίθετος λογ/σμός ειδών συσκευασίας

        50.08   Προμηθευτές εσωτερικού λογ/σμός πάγιων στοιχείων

        .........

        50.90   Τρίτοι-λογαριασμοί πωλήσεων εμπορευμάτων για λογαριασμό τους
                    (Γνωμ. 165/2045/1993)

        .........

        50.99


 

2.2.501 Λογαριασμός 50 «Προμηθευτές»

1. Στους υπολογαριασμούς του 50 παρακολουθούνται οι κάθε φύσεως δοσοληψίες της οικονομικής μονάδας με τους προμηθευτές της, από τους οποίους αγοράζει περιουσιακά στοιχεία ή υπηρεσίες.

2. Στους λογαριασμούς 50.00 «προμηθευτές εσωτερικού» και 50.01 «προμηθευτές εξωτερικού» παρακολουθούνται οι υποχρεώσεις της οικονομικής μονάδας από τις «επί πιστώσει» αγορές της από προμηθευτές εσωτερικού και εξωτερικού, αντίστοιχα.

3. Στους λογαριασμούς 50.02 «Ελληνικό Δημόσιο» και 50.03 «Ν.Π.Δ.Δ. και Δημόσιες Επιχειρήσεις» παρακολουθούνται οι υποχρεώσεις της οικονομικής μονάδας από τις «επί πιστώσει» αγορές της από το Ελληνικό Δημόσιο ή από τα Ν.Π.Δ.Δ. και τις Δημόσιες Επιχειρήσεις, όταν έχουν την ιδιότητα του προμηθευτή.

4. Στο λογαριασμό 50.04 «προμηθευτές - εγγυήσεις ειδών συσκευασίας» παρακολουθούνται τα ποσά που καταβάλλει η οικονομική μονάδα στους προμηθευτές της ως εγγύηση για την κανονική επιστροφή των ειδών συσκευασίας που παραλαμβάνει με την υποχρέωση να τα επιστρέψει.

5. Στο λογαριασμό 50.05 «προκαταβολές σε προμηθευτές» είναι δυνατό να παρακολουθούνται τα ποσά που καταβάλλονται σε προμηθευτές προκαταβολικά για την εκτέλεση παραγγελιών, εκτός από όσα αφορούν πάγια στοιχεία, τα οποία παρακολουθούνται, ή στο λογαριασμό 15.09 «προκαταβολές κτήσεως πάγιων στοιχείων» ή στο λογαριασμό 50.08 «προμηθευτές εσωτερικού λογ. πάγιων στοιχείων» ή στο λογαριασμό 32.00 «παραγγελίες πάγιων στοιχείων», σύμφωνα με όσα καθορίζονται στην περίπτ. 5 της παρ. 2.2.109.

Ο λογαριασμός 50.05 κινείται στις περιπτώσεις που, κατά την κρίση της οικονομικής μονάδας, οι προκαταβολές οι οποίες δίνονται σε προμηθευτές αφορούν σημαντικές παραγγελίες που η εκτέλεσή τους απαιτεί πολύ χρόνο. Στις άλλες περιπτώσεις οι προκαταβολές σε προμηθευτές χρεώνονται απευθείας στους οικείους προσωπικούς λογαριασμούς τους. Με την ολική ή μερική εκτέλεση της παραγγελίας το υπόλοιπο - ολικό ή μερικό - του λογαριασμού 50.05 μεταφέρεται στη χρέωση του οικείου προσωπικού λογαριασμού του προμηθευτή.

Σε περίπτωση που, από υπαιτιότητα της οικονομικής μονάδας, δεν εκτελείται η παραγγελία και για το λόγο αυτό η προκαταβολή κρατείται από τον προμηθευτή π.χ.  σαν ποινική ρήτρα, ο λογαριασμός 50.05 πιστώνεται ισόποσα με χρέωση του λογαριασμού 81.00.02 «καταπτώσεις εγγυήσεων - ποινικών ρητρών».

6. Στο λογαριασμό 50.06 «προμηθευτές - παρακρατημένες εγγυήσεις» είναι δυνατό να παρακολουθούνται τα ποσά που η οικονομική μονάδα παρακρατεί για εγγύηση, σύμφωνα με σχετικούς συμβατικούς όρους συμφωνίας με τον προμηθευτή της. Όταν δεν γίνεται χρήση του λογαριασμού 50.06, οι παρακρατημένες εγγυήσεις παρακολουθούνται στους λογαριασμούς 50.00, 50.01, 50.02 και 50.03 κατά περίπτωση.

7. Στο λογαριασμό 50.07 «προμηθευτές αντίθετος λογαριασμός ειδών συσκευασίας» είναι δυνατό να παρακολουθούνται τα επιστρεπτέα σε προμηθευτές είδη συσκευασίας, ως εξής:

α. Ο λογαριασμός 50.07 χρεώνεται, με πίστωση του οικείου λογαριασμού του προμηθευτή, με την αξία που αναγράφεται στο σχετικό τιμολόγιο ή άλλο παραστατικό έγγραφο των επιστρεπτέων ειδών συσκευασίας.

β. Κατά την επιστροφή των ειδών συσκευασίας γίνεται αντίστροφη από την παραπάνω εγγραφή, δηλαδή πιστώνεται ο λογαριασμός 50.07 με την αξία που χρεώθηκε και χρεώνεται ο οικείος λογαριασμός του προμηθευτή.

γ. Σε περίπτωση που, κατά την επιστροφή των ειδών συσκευασίας σε όχι καλή κατάσταση, ο προμηθευτής αποτιμάει τα επιστρεφόμενα σε μέρος της αρχικής αξίας τους, ο λογαριασμός 50.07 πιστώνεται με την ολική αρχική αξία και χρεώνεται ο οικείος λογαριασμός του προμηθευτή, με το ποσό που αυτός αναγνωρίζει, και ο λογαριασμός 61.98.01 «αποζημιώσεις για φθορά ειδών συσκευασίας προμηθευτών», με τη διαφορά.

δ. Σε περίπτωση καταστροφής των επιστρεπτέων ειδών συσκευασίας πριν από την επιστροφή τους, χρεώνεται με την αρχική αξία τους ο παραπάνω λογαριασμός 61.98.01 και πιστώνεται ο λογαριασμός 50.07.

ε. Σε περίπτωση που η οικονομική μονάδα αποφασίζει να κρατήσει τα είδη συσκευασίας των προμηθευτών της για να τα χρησιμοποιήσει σαν δικά της, η αρχική αξία των ειδών αυτών μεταφέρεται στο λογαριασμό 28 «είδη συσκευασίας», με πίστωση του λογαριασμού 50.07.

Όταν δεν γίνεται χρήση του λογαριασμού 50.07 για την παρακολούθηση των επιστρεπτέων ειδών συσκευασίας, η παρακολούθηση αυτή γίνεται στους λογαριασμούς 50.00, 50.01, 50.02 και 50.03, κατά περίπτωση.

8. Οι συναλλαγματικές διαφορές που προκύπτουν όταν πληρώνονται υποχρεώσεις σε ξένο νόμισμα καταχωρούνται στο λογαριασμό 81.00.04, όταν είναι χρεωστικές, ή στο λογαριασμό 81.01.04, όταν είναι πιστωτικές, εφόσον δεν αφορούν κτήση πάγιων στοιχείων, ή στο λογαριασμό 16.15 όταν αφορούν πάγια στοιχεία, με πίστωση ή χρέωση των οικείων λογαριασμών προμηθευτών σε ξένο νόμισμα, σύμφωνα με όσα ορίζονται στην παρ. 2.3.2. Οι συναλλαγματικές διαφορές που προκύπτουν κατά την αποτίμηση, στο τέλος της χρήσεως, των υποχρεώσεων σε ξένο νόμισμα, αντιμετωπίζονται σύμφωνα με όσα ορίζονται στην παρ. 2.3.2.

 51   ΓΡΑΜΜΑΤΙΑ ΠΛΗΡΩΤΕΑ

        51.00   Γραμμάτια πληρωτέα σε Δρχ.

        51.01   Γραμμάτια πληρωτέα σε Ξ.Ν.

        51.02   Γραμμάτια πληρωτέα εκδόσεως Ν.Π.Δ.Δ. και Δημόσιων
                    Επιχειρήσεων

        51.03   Μη δουλευμένοι τόκοι γραμματίων πληρωτέων σε Δρχ.
                    (αντίθετος λογαριασμός)

        51.04   Μη δουλευμένοι τόκοι γραμματίων πληρωτέων σε Ξ.Ν.
                    (αντίθετος λογ/σμός)

        51.05   Μη δουλευμένοι τόκοι γραμματίων πληρωτέων εκδόσεως Ν.Π.Δ.Δ.
                    και Δημόσιων Επιχειρήσεων (αντίθετος λογαριασμός)

       .........

        51.90   Υποσχετικές επιστολές πληρωτέες σε δρχ. (Γνωμ. 79/1623/1991)

        51.91   Υποσχετικές επιστολές πληρωτέες σε Ξ.Ν. (Γνωμ. 79/1623/1991)

        51.92   Μη δουλευμένοι τόκοι υποσχετικών επιστολών πληρωτέων σε δρχ.
                    (λογ. αντίθετος) (Γνωμ. 79/1623/1991)

        51.93   Μη δουλευμένοι τόκοι υποσχετικών επιστολών πληρωτέων σε Ξ.Ν.
                    (λογ. αντίθετος) (Γνωμ. 79/1623/1991)

        .........

        51.99

 

2.2.502 Λογαριασμός 51 «Γραμμάτια πληρωτέα»

1. Στο λογαριασμό 51 παρακολουθούνται οι υποχρεώσεις - σε δραχμές και σε ξένο νόμισμα - της οικονομικής μονάδας, οι οποίες είναι ενσωματωμένες σε τίτλους συναλλαγματικών ή «γραμματίων εις διαταγήν». Ο λογαριασμός αυτός πιστώνεται με την αποδοχή των συναλλαγματικών ή την έκδοση των γραμματίων και χρεώνεται με την πληρωμή τους.

2. Σε περίπτωση δημιουργίας συναλλαγματικών διαφορών κατά την εξόφληση των συναλλαγματικών ή γραμματίων που εκφράζονται σε ξένο νόμισμα ή την αποτίμησή τους στο τέλος της χρήσεως, ισχύουν όσα καθορίζονται στην περίπτ. 8 της παρ.  2.2.501.

3. Οι τόκοι που περιλαμβάνονται στα άληκτα γραμμάτια πληρωτέα κατά το τέλος της χρήσεως καταχωρούνται στους αντίθετους λογαριασμούς 51.03 «μη δουλευμένοι τόκοι γραμματίων πληρωτέων σε δρχ.», 51.04 «μη δουλευμένοι τόκοι γραμματίων πληρωτέων σε Ξ.Ν.» και 51.05 «μη δουλευμένοι τόκοι γραμματίων πληρωτέων εκδόσεως Ν.Π.Δ.Δ. και Δημόσιων Επιχειρήσεων», κατά περίπτωση, και στον ισολογισμό εμφανίζονται αφαιρετικά από το συνολικό ποσό των γραμματίων πληρωτέων. Ο χειρισμός αυτός δεν είναι υποχρεωτικός για τις οικονομικές μονάδες, αν όμως γίνει σε κάποια χρήση θα γίνεται υποχρεωτικά και στις άλλες κατά τρόπο πάγιο.

Οι περιπτώσεις όπου στα άληκτα γραμμάτια πληρωτέα περιλαμβάνονται τόκοι, αντιμετωπίζονται με τους παρακάτω τρόπους:

α. Στην περίπτωση που οι τόκοι διαχωρίζονται από το κόστος προμήθειας πάγιων ή κυκλοφοριακών στοιχείων, ισχύουν τα εξής:

- Οι τόκοι των γραμματίων που γίνονται αποδεκτά και λήγουν μέσα στη χρήση καταχωρούνται απευθείας στους οικείους υπολογαριασμούς του 65.06 «τόκοι και έξοδα λοιπών βραχυπρόθεσμων υποχρεώσεων».

- Από τους τόκους των γραμματίων που γίνονται αποδεκτά μέσα στη χρήση και λήγουν μετά το τέλος της, εκείνοι που αναλογούν στη χρονική περίοδο μέχρι τη λήξη της χρήσεως αυτής καταχωρούνται απευθείας στους οικείους υπολογαριασμούς του 65.06 και εκείνοι που αναλογούν στη χρονική περίοδο μετά τη λήξη της χρήσεως αυτής καταχωρούνται στους αντίθετους λογαριασμούς 51.03, 51.04 και 51.05, κατά περίπτωση.

- Στο τέλος κάθε χρήσεως, οι δουλευμένοι τόκοι των γραμματίων που έληξαν μέσα στη χρήση αυτή (γραμμάτια που έγιναν αποδεκτά σε προηγούμενες χρήσεις), καθώς και οι τόκοι των λοιπών γραμματίων (γραμμάτια που έγιναν αποδεκτά σε προηγούμενες χρήσεις και λήγουν μετά το τέλος της χρήσεως) που αναλογούν στη χρονική περίοδο μέχρι τη λήξη της χρήσεως αυτής, μεταφέρονται από τους λογαριασμούς 51.03, 51.04 και 51.05, κατά περίπτωση, στους οικείους υπολογαριασμούς του 65.06.

β. Στην περίπτωση που οι τόκοι περιλαμβάνονται στο κόστος προμήθειας πάγιων ή κυκλοφοριακών στοιχείων, ο διαχωρισμός και η εμφάνισή τους στους αντίθετους λογαριασμούς 51.03, 51.04 και 51.05 είναι δυνητικός. Ο λογιστικός χειρισμός του διαχωρισμού αφήνεται στην κρίση κάθε οικονομικής μονάδας που επιθυμεί να κάνει χρήση της δυνητικής ευχέρειας της περιπτώσεως αυτής, υπό τον όρο όμως ότι δε θα μεταφέρονται σε αποτελεσματικούς λογαριασμούς εξόδων και εσόδων των ομάδων 6 και 7 οι μη δουλευμένοι τόκοι των γραμματίων πληρωτέων της κατηγορίας αυτής.

 52   ΤΡΑΠΕΖΕΣ - ΛΟΓΑΡΙΑΣΜΟΙ ΒΡΑΧΥΠΡΟΘΕΣΜΩΝ ΥΠΟΧΡΕΩΣΕΩΝ

        52.00   Τράπεζα Α'

        52.01   Τράπεζα Β'

        52.03

        .........

        52.99   Λοιπές Τράπεζες

 

2.2.503 Λογαριασμός 52 «Τράπεζες λογαριασμοί βραχυπρόθεσμων υποχρεώσεων»

1. Στο λογαριασμό 52 παρακολουθούνται οι υποχρεώσεις της οικονομικής μονάδας από βραχυπρόθεσμες, κάθε φύσεως, τραπεζικές χρηματοδοτήσεις προς αυτή.

2. Σε περίπτωση δημιουργίας συναλλαγματικών διαφορών κατά την εξόφληση υποχρεώσεων προς Τράπεζες σε ξένο νόμισμα ή την αποτίμησή τους στο τέλος της χρήσεως, ισχύουν όσα καθορίζονται στην περίπτ. 8 της παρ. 2.2.501.

Σημείωση: Ο 52 αναπτύσσεται σε δευτεροβάθμιους κατά υποκαταστήματα Τράπεζας και σε τριτοβάθμιους κατά τραπεζικό λογαριασμό, σύμφωνα με τις ανάγκες κάθε επιχειρήσεως (δηλ. χρησιμοποιούνται ελεύθερα και οι 100 δευτεροβάθμιοι). (Α.Υ.Ο. 1116200/885/0015/ΠΟΛ. 1282/30-10-1996). 

 53   ΠΙΣΤΩΤΕΣ ΔΙΑΦΟΡΟΙ

        53.00   Αποδοχές προσωπικού πληρωτέες

        53.01   Μερίσματα πληρωτέα

        53.02   Προμερίσματα πληρωτέα

        53.03   Οφειλόμενες αμοιβές προσωπικού

        53.04   Ομολογίες πληρωτέες

        53.05   Τοκομερίδια πληρωτέα

        53.06   Οφειλόμενες δόσεις συμμετοχών

        53.07   Οφειλόμενες δόσεις ομολογιών και λοιπών χρεογράφων

        53.08   Δικαιούχοι αμοιβών

        53.09   Δικαιούχοι χρηματικών εγγυήσεων

        53.10   Βραχυπρόθεσμες υποχρεώσεις προς συνδεμένες επιχειρήσεις σε Δρχ.

        53.11   Βραχυπρόθεσμες υποχρεώσεις προς συνδεμένες επιχειρήσεις σε Ξ.Ν.

        53.12   Βραχυπρόθεσμες υποχρεώσεις προς λοιπές συμμετοχικού
                     ενδιαφέροντος επιχειρήσεις σε Δρχ.

        53.13   Βραχυπρόθεσμες υποχρεώσεις προς λοιπές συμμετοχικού
                     ενδιαφέροντος επιχειρήσεις σε Ξ.Ν.

        53.14   Βραχυπρόθεσμες υποχρεώσεις προς εταίρους

        53.15   Δικαιούχοι ομολογιούχοι παροχών επί πλέον τόκου

        53.16   Μέτοχοι-αξία μετοχών τους προς απόδοση λόγω αποσβέσεως
                     ή μειώσεως του κεφαλαίου

        53.17   Μακροπρόθεσμες υποχρεώσεις πληρωτέες στην επόμενη χρήση
                     σε Δρχ.

        53.18   Μακροπρόθεσμες υποχρεώσεις πληρωτέες στην επόμενη χρήση
                     σε Ξ.Ν.

        ........

        53.90   Επιταγές πληρωτέες (μεταχρονολογημένες)
                     (Γνωμ. 38/1047/1988)

        ........

        53.98   Λοιπές βραχυπρόθεσμες υποχρεώσεις σε Δρχ.

        53.99   Λοιπές βραχυπρόθεσμες υποχρεώσεις σε Ξ.Ν.

 

2.2.504 Λογαριασμός 53 «Πιστωτές διάφοροι»

1. Στους υπολογαριασμούς του 53 παρακολουθούνται οι υποχρεώσεις της οικονομικής μονάδας οι οποίες δεν υπάγονται σε οποιαδήποτε κατηγορία υποχρεώσεων από εκείνες που παρακολουθούνται στους λοιπούς πρωτοβάθμιους λογαριασμούς της ομάδας 5.

2. Ο λογαριασμός 53.00 «αποδοχές προσωπικού πληρωτέες» χρησιμοποιείται στην περίπτωση που η λογιστικοποίηση των μισθοδοτικών καταστάσεων πληρωμής του προσωπικού γίνεται συμψηφιστικά. Στην πίστωση του λογαριασμού αυτού, με χρέωση των οικείων υπολογαριασμών του 60 «αμοιβές και έξοδα προσωπικού», καταχωρούνται οι καθαρές πληρωτέες αποδοχές του προσωπικού, ενώ στη χρέωσή του καταχωρούνται οι καταβολές προς τους δικαιούχους. Οι αποδοχές οι οποίες, μέσα σε εύλογο χρόνο, δεν ζητούνται από τους δικαιούχους, μεταφέρονται στην πίστωση του λογαριασμού 53.03 «οφειλόμενες αμοιβές προσωπικού».

3. Ο λογαριασμός 53.01 «μερίσματα πληρωτέα» λειτουργεί ως εξής:

α. Πιστώνεται, με χρέωση του λογαριασμού 88.99 «κέρδη προς διάθεση», με τα διανεμητέα μερίσματα χρήσεως.

β. Πιστώνεται, με χρέωση του λογαριασμού 53.02 «προμερίσματα πληρωτέα», με τα προμερίσματα που μένουν απλήρωτα στο τέλος της χρήσεως.

γ. Χρεώνεται, με πίστωση του λογαριασμού 33.06 «προμερίσματα» για τη μεταφορά των προμερισμάτων.

δ. Χρεώνεται, με πίστωση του λογαριασμού 54.09.00 «φόρος μερισμάτων», για τη μεταφορά του παρακρατημένου φόρου μερισμάτων.

ε. Χρεώνεται με το μέρος των μερισμάτων που προτείνεται να διατεθούν για αύξηση του μετοχικού κεφαλαίου, με πίστωση του λογαριασμού 43.02 «διαθέσιμα μερίσματα χρήσεως για αύξηση του μετοχικού κεφαλαίου».

στ. Χρεώνεται με τα μερίσματα που καταβάλλονται στους μετόχους από την ημερομηνία που αρχίζει η πληρωμή του μερίσματος μέχρι το τέλος της χρήσεως.

4. Ο λογαριασμός 53.02 «προμερίσματα πληρωτέα» πιστώνεται, με χρέωση του λογαριασμού 33.06 «προμερίσματα», με το συνολικό ποσό το οποίο αποφασίζεται νόμιμα να καταβληθεί στους μετόχους ως προμέρισμα. Ο λογαριασμός αυτός χρεώνεται με τα προμερίσματα που καταβάλλονται μέχρι το τέλος της χρήσεως στους μετόχους και με τον παρακρατημένο φόρο που αναλογεί σ' αυτά.

5. Στο λογαριασμό 53.04 «ομολογίες πληρωτέες» καταχωρείται το μέσα στην επόμενη χρήση πληρωτέο ποσό που αντιστοιχεί στις εξοφλητέες ομολογίες, με χρέωση του οικείου υπολογαριασμού των 45.00 έως και 45.05.

Σε περίπτωση που δημιουργούνται συναλλαγματικές διαφορές κατά την εξόφληση ομολογιών σε ξένο νόμισμα ή την αποτίμησή τους στο τέλος της χρήσεως, ισχύουν όσα καθορίζονται στην περίπτ. 8 της παρ. 2.2.501.

6. Στο λογαριασμό 53.05 «τοκομερίδια πληρωτέα» καταχωρείται η αξία των τοκομεριδίων ομολογιακών δανείων, κατά τη λήξη τους, με χρέωση των οικείων υπολογαριασμών του 65.00 «τόκοι και έξοδα ομολογιακών δανείων».

Σε περίπτωση που δημιουργούνται συναλλαγματικές διαφορές κατά την εξόφληση τοκομεριδίων σε ξένο νόμισμα ή την αποτίμησή τους στο τέλος της χρήσεως, ισχύουν όσα καθορίζονται στην περίπτ. 8 της παρ. 2.2.501.

7. Στο λογαριασμό 53.06 «οφειλόμενες δόσεις συμμετοχών» καταχωρούνται οι δόσεις που οφείλονται από συμμετοχές της περιπτ. 3 της παρ. 2.2.112.

8. Στο λογαριασμό 53.07 «οφειλόμενες δόσεις ομολογιών και λοιπών χρεογράφων» καταχωρούνται οι δόσεις που οφείλονται από χρεόγραφα της παρ. 2.2.305.

9. Στο λογαριασμό 53.08 «δικαιούχοι αμοιβών» καταχωρούνται οι αμοιβές που οφείλονται σε ελεύθερους επαγγελματίες, σε μέλη του διοικητικού συμβουλίου της οικονομικής μονάδας και σε τρίτους.

10. Στο λογαριασμό 53.09 «δικαιούχοι χρηματικών εγγυήσεων» καταχωρούνται οι χρηματικές εγγυήσεις που καταθέτονται στην οικονομική μονάδα, για διάφορους λόγους, από τρίτους εκτός από προμηθευτές, για τους οποίους οι παρακρατημένες εγγυήσεις καταχωρούνται στο λογαριασμό 50.06.

11. Στους λογαριασμούς 53.10 και 53.11 παρακολουθούνται οι βραχυπρόθεσμες υποχρεώσεις της οικονομικής μονάδας προς συνδεμένες επιχειρήσεις, ενώ στους λογαριασμούς 53.12 και 53.13 παρακολουθούνται οι βραχυπρόθεσμες υποχρεώσεις αυτής προς λοιπές επιχειρήσεις στις οποίες έχει συμμετοχικό ενδιαφέρον, επειδή διαθέτει συμμετοχές του λογαριασμού 18.01.

Σχετικά με τις συνδεμένες επιχειρήσεις καθώς και με τις συμμετοχικού ενδιαφέροντος λοιπές επιχειρήσεις, ισχύουν όσα καθορίζονται στην περίπτ. 10 της παρ. 2.2.112, σε συνδυασμό και με όσα καθορίζονται και στην περίπτωση 1 της ίδιας παραγράφου.

12. Στο λογαριασμό 53.14 «βραχυπρόθεσμες υποχρεώσεις προς εταίρους» παρακολουθούνται οι βραχυπρόθεσμες υποχρεώσεις της οικονομικής μονάδας προς εταίρους της.

13. Στο λογαριασμό 53.15 «δικαιούχοι ομολογιούχοι παροχών επί πλέον τόκου» καταχωρούνται οι τυχόν πρόσθετες παροχές που δίνονται σε ομολογιούχους της οικονομικής μονάδας, πέρα από τον τόκο των τοκομεριδίων. Οι παροχές αυτές, αν είναι μέρισμα, καταχωρούνται στο λογαριασμό 53.15 με χρέωση του λογαριασμού 88.99 «κέρδη προς διάθεση», αν όμως είναι πρόσθετος τόκος καταχωρούνται στο λογαριασμό αυτό με χρέωση του λογαριασμού 65.09 «παροχές σε ομολογιούχους επί πλέον τόκου».

14. Στο λογαριασμό 53.16 «μέτοχοι - αξία μετοχών τους προς απόδοση λόγω αποσβέσεως ή μειώσεως του κεφαλαίου» καταχωρούνται η ονομαστική αξία των μετοχών καθώς και τα ποσά τα οποία καταβάλλονται στους μετόχους ανώνυμης εταιρείας λόγω αποσβέσεως ή μειώσεως του κεφαλαίου αυτής, σύμφωνα με όσα καθορίζονται στις περιπτώσεις 9 και 10 της παρ. 2.2.401.

15. Στους λογαριασμούς 53.17 «μακροπρόθεσμες υποχρεώσεις πληρωτέες στην επόμενη χρήση σε Δρχ.» και 53.18 «μακροπρόθεσμες υποχρεώσεις πληρωτέες στην επόμενη χρήση σε Ξ.Ν.» παρακολουθούνται οι μακροπρόθεσμες υποχρεώσεις της οικονομικής μονάδας, οι οποίες μετατρέπονται σε βραχυπρόθεσμες, σύμφωνα με όσα καθορίζονται στην περίπτ. 2 της παρ. 2.2.406.

16. Στους λογαριασμούς 53.98 «λοιπές βραχυπρόθεσμες υποχρεώσεις σε Δρχ.» και 53.99 «λοιπές βραχυπρόθεσμες υποχρεώσεις σε Ξ.Ν.» παρακολουθούνται οι λοιπές βραχυπρόθεσμες υποχρεώσεις της οικονομικής μονάδας οι οποίες δεν εντάσσονται σε μία από τις προηγούμενες κατηγορίες λογαριασμών της ομάδας 5.

 54   ΥΠΟΧΡΕΩΣΕΙΣ ΑΠΟ ΦΟΡΟΥΣ - ΤΕΛΗ

        54.00   Φόρος προστιθέμενης αξίας (Γνωμ. 243/2162/1995 \& 136/1905/1993)
                      Οι υπογραμμισμένοι τριτοβάθμιοι λογαριασμοί της γνωματεύσεως
                      του ΕΣΥΛ 243/2162/1995 τηρούνται υποχρεωτικά. Για την κάλυψη των
                      απαιτήσεων της κείμενης νομοθεσίας για το ΦΠΑ και των αναγκών του
                      φορολογικού ελέγχου οι τριτοβάθμιοι αυτοί αναπτύσσονται σε
                      τεταρτοβάθμιους κατά συντελεστή Φ.Π.Α.

        54.01   Φόρος καταναλώσεως (Ν. 2127/1993)

        54.02   Χαρτόσημο τιμολογίων πωλήσεως
                     (Μετά την εφαρμογή του Ν. 1642/1986 ο λογ/σμός παραμένει κενός)

        54.03   Φόροι - Τέλη αμοιβών προσωπικού

                      54.03.00   Φόρος μισθωτών υπηρεσιών

                                01   .........................................

                                02   Χαρτόσημο και ΟΓΑ μισθωτών υπηρεσιών

                                03   .........................................

                                04   Φόρος αποζημιώσεων απολυομένων

                                05   .........................................

                                06   Χαρτόσημο και ΟΓΑ αποζημιώσεων απολυομένων

                      ..............

                      54.03.99

        54.04   Φόροι - Τέλη αμοιβών τρίτων

                      54.04.00   Φόρος αμοιβών ελεύθερων επαγγελματιών

                                01    Χαρτόσημο και ΟΓΑ αμοιβών ελεύθερων επαγγελματιών

                                02    Χαρτόσημο και ΟΓΑ λοιπών αμοιβών τρίτων

                      ...............

                      54.04.99

        54.05   Φόροι - Τέλη κυκλοφορίας μεταφορικών μέσων

        54.06   Φόροι - Τέλη τιμολογίων αγοράς

                      54.06.00    Φόρος τιμολογίων αγοράς αγροτικών προϊόντων

                                01   ...........................................

                                02   ...........................................

                      ...............

                      54.06.99

        54.07   Φόρος εισοδήματος φορολογητέων κερδών

        54.08   Λογαριασμός εκκαθαρίσεως φόρων - τελών ετήσιας δηλώσεως φόρου
                     εισοδήματος

        54.09   Λοιποί φόροι και τέλη

                      54.09.00   Φόρος μερισμάτων

                                01   Φόρος αμοιβών μελών διοικητικού συμβουλίου

                                02   Χαρτόσημο και ΟΓΑ αμοιβών μελών διοικητικού συμβουλίου

                                03   Φόρος τόκων

                                04   Χαρτόσημο και ΟΓΑ τόκων

                                05   Χαρτόσημο και ΟΓΑ εισοδημάτων από οικοδομές

                                06   Τέλη υδρεύσεως εισοδημάτων από οικοδομές
                                        (Τα τέλη υδρεύσεως καταργήθηκαν με το άρθρο 43 παρ. 2
                                        Ν. 2065/1992)

                                07   Φόροι ακίνητης περιουσίας (Ο φόρος ακίνητης περιουσίας
                                        καταργήθηκε με το άρθρο 37 Ν. 2065/1992 και με το άρθρο 24
                                        Ν. 2130/1993 θεσπίστηκαν τέλη υπέρ Δήμων και Κοινοτήτων.
                                        Για το λόγο αυτό τροποποιείται ανάλογα και ο τίτλος του
                                        λογαριασμού αυτού).

                                08   Φόροι - Τέλη ανεγειρόμενων οικοδομών

                                09   Τέλη καθαριότητας και φωτισμού

                                10   Χαρτόσημο και ΟΓΑ δανείων

                                11   Χαρτόσημο και ΟΓΑ κερδών προσωπικών εταιριών

                                12   Φόρος αμοιβών εργολάβων

                                13   Φόρος αμοιβών διαχειριστών εταίρων ΕΠΕ (άρθρο 55 παρ. 1-δ'
                                        Ν. 2238/1994)

                                14   Φόρος προμηθευτών (άρθρο 55 παρ. 1-στ' Ν. 2238/1994)

                      ..............

                      54.09.99

        54.10

        .........

        54.90   Αγγελιόσημο υπέρ ΤΣΠΕΑΘ (Γνωμ. 232/2209/1994)

        .........

        54.99   Φόροι - Τέλη προηγούμενων χρήσεων

2.2.505 Λογαριασμός 54 «Υποχρεώσεις από φόρους - τέλη»

1. Στους υπολογαριασμούς του 54 παρακολουθούνται οι υποχρεώσεις της οικονομικής μονάδας από φόρους και τέλη προς το Ελληνικό Δημόσιο, τους δήμους, τις κοινότητες και λοιπούς οργανισμούς δημοσίου δικαίου.

2. Στους λογαριασμούς 54.00 «φόρος κύκλου εργασιών», 54.01 «φόρος καταναλώσεως ειδών πολυτελείας» και 54.02 «χαρτόσημο τιμολογίων πωλήσεως» παρακολουθούνται οι φόροι και τα τέλη χαρτοσήμου που αναλογούν στις πωλήσεις αγαθών και υπηρεσιών. Οι λογαριασμοί αυτοί πιστώνονται κατά τη διάρκεια της χρήσεως με τα ποσά φόρων και τελών που πρόκειται να αποδοθούν και χρεώνονται, ως ακολούθως:

- με τα ποσά που καταβάλλονται για την απόδοση των παραπάνω φόρων και τελών,

- με τα ποσά φόρων και τελών που αναλογούν στις επιστροφές και ακυρώσεις πωλήσεων και

- με τα ποσά του φόρου κύκλου εργασιών που περιλαμβάνεται στην αξία αγοράς των προς βιομηχανοποίηση πρώτων υλών, όταν ο φόρος αυτός, σύμφωνα με τις φορολογικές διατάξεις που ισχύουν κάθε φορά, είναι συμψηφιστέος στο φόρο κύκλου εργασιών που πρόκειται να αποδοθεί.

Σε περίπτωση που ο φόρος κύκλου εργασιών αντικατασταθεί από το φόρο προστιθέμενης αξίας, ο λογαριασμός 54.00 θα χρησιμοποιηθεί για την παρακολούθηση του νέου αυτού φόρου, σύμφωνα με τις διατάξεις της φορολογικής νομοθεσίας που θα ισχύουν κάθε φορά.

3. Στο λογαριασμό 54.03 «φόροι - τέλη αμοιβών προσωπικού» πιστώνονται οι φόροι και τα τέλη χαρτοσήμου και ΟΓΑ που παρακρατούνται από τις αποδοχές του προσωπικού και από τις αποζημιώσεις του λόγω απολύσεως ή εξόδου από την υπηρεσία, καθώς και τα τέλη χαρτοσήμου μισθωτών υπηρεσιών και ο φόρος Α.Ν.  843/48 που επιβαρύνουν την οικονομική μονάδα. Η πίστωση του λογαριασμού 54.03 γίνεται με χρέωση των οικείων υπολογαριασμών του 60 «αμοιβές και έξοδα προσωπικού». Ο λογαριασμός 54.03 χρεώνεται με τα ποσά που καταβάλλονται για την εξόφληση των σχετικών υποχρεώσεων.

4. Στο λογαριασμό 54.04 «φόροι - τέλη αμοιβών τρίτων» πιστώνονται οι φόροι και τα τέλη χαρτοσήμου που η οικονομική μονάδα παρακρατεί για τις αμοιβές τρίτων.  Στον ίδιο λογαριασμό, με χρέωση των οικείων υπολογαριασμών του 61 «αμοιβές και έξοδα τρίτων», πιστώνονται οι φόροι και τα τέλη χαρτοσήμου που αναλογούν στις αμοιβές τρίτων, όταν δεν γίνεται παρακράτηση ή όταν γίνεται μερική παρακράτηση.

Το αυτό ισχύει και για όλες τις άλλες περιπτώσεις παρακρατημένων φόρων και τελών.

5. Στο λογαριασμό 54.05 «φόροι - τέλη κυκλοφορίας μεταφορικών μέσων», με χρέωση των οικείων υπολογαριασμών του 63.03, πιστώνονται οι φόροι και τα τέλη των μεταφορικών μέσων.

6. Στο λογαριασμό 54.06 «φόροι- τέλη τιμολογίων αγοράς» πιστώνονται οι φόροι, τα τέλη χαρτοσήμου και η εισφορά υπέρ του ΟΓΑ που αναλογούν στα τιμολόγια αγοράς. Οι υπολογαριασμοί του 54.06 χρεώνονται με τις καταβολές για την απόδοση των φόρων και των τελών της κατηγορίας αυτής, καθώς και με τα ποσά που αντιστοιχούν σε επιστροφές αγορών ή ακυρώσεις τιμολογίων αγοράς.

7. Στο λογαριασμό 54.07 «φόρος εισοδήματος φορολογητέων κερδών» καταχωρούνται, κατά το κλείσιμο του ισολογισμού, με χρέωση του λογαριασμού 88.08 «φόρος εισοδήματος και εισφορά ΟΓΑ», ο φόρος εισοδήματος και η υπέρ ΟΓΑ εισφορά, που αναλογούν στα φορολογητέα κέρδη της χρήσεως. Ο λογαριασμός 54.07 χρεώνεται με τη μεταφορά του υπολοίπου του στο λογαριασμό 54.08, σύμφωνα με όσα καθορίζονται στην παρακάτω περίπτ. 8.

8. Στο λογαριασμό 54.08 «λογαριασμός εκκαθαρίσεως φόρων-τελών ετήσιας δηλώσεως φόρου εισοδήματος» συγκεντρώνονται, κατά το κλείσιμο του ισολογισμού, όλοι οι λογαριασμοί που απεικονίζουν ποσά φόρων τα οποία περιλαμβάνονται στη δήλωση του φόρου εισοδήματος της κλειόμενης χρήσεως. Ο λογαριασμός 54.08 λειτουργεί ως εξής:

α. Στη χρέωσή του μεταφέρονται τα υπόλοιπα των λογαριασμών 33.13.00 «προκαταβολή φόρου εισοδήματος» και 33.13.01-07, σύμφωνα με όσα καθορίζονται στην περίπτ. 10 της παρ. 2.2.304.

β. Στην πίστωσή του μεταφέρονται τα υπόλοιπα των λογαριασμών 54.07, σύμφωνα με όσα καθορίζονται στην παραπάνω περίπτ. 7, 54.09.05 «χαρτόσημο και ΟΓΑ εισοδημάτων από οικοδομές» και 54.09.06 «τέλη υδρεύσεως εισοδημάτων από οικοδομές».

γ. Πιστώνεται, με χρέωση του λογαριασμού 33.13.00, με τον προκαταβλητέο φόρο για την επόμενη χρήση, ο οποίος, σύμφωνα με τα παραπάνω (α), μεταφέρεται στη χρέωσή του (του 54.08) κατά το κλείσιμο του ισολογισμού της επόμενης αυτής χρήσεως. Μετά από τις παραπάνω εγγραφές το υπόλοιπο του λογαριασμού 54.08 είναι ίσο με το ποσό του καταβλητέου φόρου που προκύπτει από την οικεία δήλωση φόρου εισοδήματος. Οι καταβολές, στην επόμενη χρήση, για την εξόφληση του φόρου αυτού καταχωρούνται στη χρέωση του 54.08.

Σε περίπτωση που το υπόλοιπο του λογαριασμού 54.08, μετά από τις παραπάνω εγγραφές, είναι χρεωστικό, το υπόλοιπο αυτό, που είναι ίσο με τον αντίστοιχο υπόλοιπο της οικείας δηλώσεως φόρου εισοδήματος, αντιστοιχεί στον επιστρεπτέο στην οικονομική μονάδα φόρο.

9. Οι υπολογαριασμοί του 54.09 «λοιποί φόροι και τέλη» λειτουργούν ως εξής:

α. Στον 54.09.00 «φόρος μερισμάτων», με χρέωση του 53.01 «μερίσματα πληρωτέα», καταχωρείται, αμέσως μετά την έγκριση του ισολογισμού από τη γενική συνέλευση των μετόχων, ο φόρος που αναλογεί στα διανεμόμενα μερίσματα. Στη χρέωση του λογαριασμού αυτού καταχωρούνται οι καταβολές για την απόδοση του φόρου μερισμάτων.

β. Στους 54.09.01 «φόρος αμοιβών μελών διοικητικού συμβουλίου» και 54.09.02 «χαρτόσημο και ΟΓΑ αμοιβών μελών διοικητικού συμβουλίου» καταχωρούνται τα ποσά των φόρων και τελών χαρτοσήμου που παρακρατούνται από την οικονομική μονάδα για τις αμοιβές μελών του διοικητικού της συμβουλίου.

γ. Στους 54.09.03 «φόρος τόκων» και 54.09.04 «χαρτόσημο και ΟΓΑ τόκων» καταχωρούνται τα ποσά των φόρων και τελών χαρτοσήμου που παρακρατούνται από την οικονομική μονάδα για τους τόκους που αυτή καταβάλλει σε τρίτους.

δ. Στους 54.09.05 «χαρτόσημο και ΟΓΑ εισοδημάτων από οικοδομές» και 54.09.06 «τέλη υδρεύσεως εισοδημάτων από οικοδομές» καταχωρούνται τα τέλη χαρτοσήμου και υδρεύσεως, εφόσον βαρύνουν την οικονομική μονάδα, με χρέωση των λογαριασμών 63.98.00 «χαρτόσημο μισθωμάτων» και 63.98.01 «τέλη υδρεύσεως», αντίστοιχα. Στο λογαριασμό 54.09.05 καταχωρούνται και τα ποσά χαρτοσήμου που εισπράττονται από μισθωτές ακινήτων της οικονομικής μονάδας. Στο τέλος της χρήσεως, τα υπόλοιπα των λογαριασμών 54.09.05 και 54.09.06 μεταφέρονται στο λογαριασμό 54.08, σύμφωνα με όσα αναφέρονται παραπάνω στην περίπτ. 8.

ε. Στους 54.09.07 «φόροι ακίνητης περιουσίας» και 54.09.08 «φόροι - τέλη ανεγειρόμενων οικοδομών» καταχωρούνται οι φόροι που αναλογούν στην ακίνητη περιουσία της οικονομικής μονάδας, με χρέωση του λογαριασμού 63.98.02 «φόρος ακίνητης περιουσίας» καθώς και οι φόροι και τα τέλη που καταβάλλονται για την έκδοση αδειών ανεγειρόμενων οικοδομών, με χρέωση των οικείων λογαριασμών της ομάδας 1 ή, κατά περίπτωση, του λογαριασμού 63.04.01 «φόροι και τέλη ανεγειρόμενων ακινήτων» (π.χ. τέλη πεζοδρομίων, χρήσεως δρόμων, τέλη οικοδομικών αδειών ή συντηρήσεως εγκαταστάσεων).

στ. Στον 54.09.09 «τέλη καθαριότητας και φωτισμού» καταχωρούνται, με χρέωση του λογαριασμού 63.04.00 «τέλη καθαριότητας και φωτισμού», τα ποσά που βεβαιώνονται σε βάρος της οικονομικής μονάδας για τέλη καθαριότητας και δημοτικού φωτισμού.

ζ. Στον 54.09.10 «χαρτόσημο και ΟΓΑ δανείων» καταχωρούνται τα τέλη χαρτοσήμου και ΟΓΑ που επιβαρύνουν την οικονομική μονάδα για δάνειά της, με χρέωση των οικείων υπολογαριασμών του 63.02 «τέλη συναλλαγματικών, δανείων και λοιπών πράξεων». Στο λογαριασμό αυτό (54.09.10) καταχωρούνται και τα τέλη χαρτοσήμου και ΟΓΑ που τυχόν εισπράττονται από οφειλέτες της οικονομικής μονάδας.

η. Στον 54.09.11 «χαρτόσημο και ΟΓΑ κερδών προσωπικών εταιρειών» καταχωρούνται τα τέλη χαρτοσήμου και ΟΓΑ που αναλογούν στα κέρδη της χρήσεως, με χρέωση του λογαριασμού 63.98.03 «χαρτόσημο κερδών».

θ. Στον 54.09.12 «φόρος αμοιβών εργολάβων» καταχωρούνται οι φόροι και τα τέλη χαρτοσήμου που παρακρατούνται από την οικονομική μονάδα για τις αμοιβές εργολάβων της.

10. Στο λογαριασμό 54.99 «φόροι - τέλη προηγούμενων χρήσεων» παρακολουθούνται οι φόροι και τα τέλη προηγούμενων χρήσεων που, για διάφορους λόγους, δεν προηγήθηκε η καταχώρισή τους στους οικείους υπολογαριασμούς του 54, ούτε ασκήθηκε προσφυγή στα αρμόδια δικαστήρια, οπότε ισχύουν όσα καθορίζονται στις περιπτ. 17 και 18 της παρ. 2.2.304 για το λογαριασμό 33.98. Ο λογαριασμός 54.99 πιστώνεται με χρέωση των οικείων υπολογαριασμών του 82.00 «έξοδα προηγούμενων χρήσεων».

 55   ΑΣΦΑΛΙΣΤΙΚΟΙ ΟΡΓΑΝΙΣΜΟΙ

        55.00   Ίδρυμα Κοινωνικών Ασφαλίσεων (ΙΚΑ)

                    55.00.00   Λογαριασμός τρέχουσας κινήσεως

                               01   Λογαριασμός δόσεων καθυστερούμενων κρατήσεων
                                      και εισφορών

                               02   Λογαριασμός τρέχουσας κινήσεως εισφορών ανεγειρόμενων
                                      οικοδομών

                               03   Λογαριασμός δωρόσημου ημερομισθίων οικοδομικών
                                      εργασιών

                    ..............

                    55.00.99

        55.01   Λοιπά Ταμεία κύριας ασφαλίσεως

        55.02   Επικουρικά Ταμεία

        55.03   ..............................

        .........

        55.99   Κρατήσεις και εισφορές καθυστερούμενες προηγούμενων χρήσεων

 

2.2.506 Λογαριασμός 55 «Ασφαλιστικοί οργανισμοί»

1. Στους υπολογαριασμούς του 55 παρακολουθούνται οι υποχρεώσεις της οικονομικής μονάδας προς τους διάφορους ασφαλιστικούς οργανισμούς από εισφορές εργοδότου και κρατήσεις εργαζομένων.

2. Στο λογαριασμό 55.00 «Ίδρυμα Κοινωνικών Ασφαλίσεων (ΙΚΑ)» παρακολουθούνται οι υποχρεώσεις της οικονομικής μονάδας προς το ΙΚΑ, ως εξής:

α. Στο τέλος κάθε μισθοδοτικής περιόδου, με το συνολικό ύψος των εισφορών εργοδότου και εργαζομένων που αναλογούν στις αποδοχές της περιόδου αυτής, πιστώνονται οι υπολογαριασμοί 55.00.00 «λογαριασμός τρέχουσας κινήσεως», 55.00.02 «λογαριασμός τρέχουσας κινήσεως εισφορών ανεγειρόμενων οικοδομών» και 55.00.03 «λογαριασμός δωρόσημου ημερομισθίων οικοδομικών εργασιών», κατά περίπτωση. Οι υπολογαριασμοί αυτοί χρεώνονται με τις καταβολές που γίνονται προς τον δικαιούχο ασφαλιστικό οργανισμό.

β. Σε περίπτωση διακανονισμού καθυστερημένων προς το ΙΚΑ υποχρεώσεων με δόσεις, οι σχετικές οφειλές μεταφέρονται στον υπολογαριασμό 55.00.01 «λογαριασμός δόσεων καθυστερούμενων κρατήσεων και εισφορών», με τον οποίο παρακολουθείται η εξόφλησή τους.

γ. Σε περίπτωση βεβαιώσεως οφειλών προς το ΙΚΑ, πέρα από εκείνες που εμφανίζονται στους οικείους υπολογαριασμούς του 55.00, αν οι οφειλές αυτές αφορούν τη χρήση μέσα στην οποία βεβαιώνονται, καταχωρούνται στους οικείους υπολογαριασμούς του 55.00, με χρέωση των οικείων υπολογαριασμών του 60, αν όμως αφορούν προηγούμενες χρήσεις, καταχωρούνται στο λογαριασμό 55.99, με χρέωση των οικείων υπολογαριασμών του 82.00 «έξοδα προηγούμενων χρήσεων».

3. Στους λογαριασμούς 55.01 «λοιπά ταμεία κύριας ασφαλίσεως», 55.02 «επικουρικά ταμεία» και 55.03 «εργατική εστία» παρακολουθούνται, κατά τρόπο ανάλογο με όσα καθορίζονται στην παραπάνω περίπτ. 2, οι υποχρεώσεις της οικονομικής μονάδας προς τους λοιπούς ασφαλιστικούς οργανισμούς.

4. Στο λογαριασμό 55.99 «κρατήσεις και εισφορές καθυστερούμενες προηγούμενων χρήσεων» παρακολουθούνται οι υποχρεώσεις της οικονομικής μονάδας προς ασφαλιστικούς οργανισμούς, όταν οι υποχρεώσεις αυτές αφορούν την προηγούμενη ή τις προηγούμενες χρήσεις και καθυστερεί η πληρωμή τους πέρα από την ημερομηνία κατά την οποία γίνονται ληξιπρόθεσμες.

 56   ΜΕΤΑΒΑΤΙΚΟΙ ΛΟΓΑΡΙΑΣΜΟΙ ΠΑΘΗΤΙΚΟΥ

        56.00   Έσοδα επόμενων χρήσεων
                    (Ανάπτυξη αντίστοιχη των λογαριασμών εσόδων)

        56.01   Έξοδα χρήσεως δουλευμένα (πληρωτέα)
                    (Ανάπτυξη αντίστοιχη των λογαριασμών εξόδων και κατά δικαιούχο)

        56.02   Αγορές υπό τακτοποίηση

        56.03   Εκπτώσεις επί πωλήσεων χρήσεως υπό διακανονισμό

        .........

        56.90   Πωλήσεις ανεγειρόμενων οικοδομών υπό τακτοποίηση
                    (Γνωμ.257/2257/96) .....

        56.99

2.2.507 Λογαριασμός 56 «Μεταβατικοί λογαριασμοί παθητικού»

1. Όπως οι μεταβατικοί λογαριασμοί ενεργητικού, έτσι και οι μεταβατικοί λογαριασμοί παθητικού εξυπηρετούν το σκοπό της αναμορφώσεως των λογαριασμών του ισολογισμού στο πραγματικό μέγεθός τους κατά την ημερομηνία λήξεως της χρήσεως, σύμφωνα με όσα καθορίζονται στην περίπτ. 1 της παρ. 2.2.307.

Ειδικότερα, στους μεταβατικούς λογαριασμούς παθητικού καταχωρούνται τα έσοδα της επόμενης χρήσεως που προεισπράττονται και τα πληρωτέα έξοδα της κλειόμενης χρήσεως, που πραγματοποιούνται δηλαδή μέσα στη χρήση, δεν πληρώνονται όμως μέσα σ' αυτή, ούτε είναι δυνατή η πίστωσή τους σε προσωπικούς λογαριασμούς, επειδή δεν είναι απαιτητά κατά το τέλος της χρήσεως.

2. Στο λογαριασμό 56.00 «έσοδα επόμενων χρήσεων», σε περίπτωση που δεν καταχωρούνται απευθείας σ' αυτόν, μεταφέρονται από τους οικείους λογαριασμούς εσόδων της ομάδας 7 όσα από αυτά δεν αφορούν την κλειόμενη, αλλά την επόμενη ή τις επόμενες χρήσεις.

Η ανάλυση του λογαριασμού 56.00 σε τριτοβάθμιους υπολογαριασμούς είναι αντίστοιχη των αναλύσεων των λογαριασμών εσόδων, στους οποίους μεταφέρονται τα κονδύλια που αφορούν τη νέα (επόμενη) χρήση, αμέσως μετά την έναρξή της.

3. Στο λογαριασμό 56.01 «έξοδα χρήσεως δουλευμένα (πληρωτέα)» καταχωρούνται, με αντίστοιχη χρέωση των οικείων λογαριασμών εξόδων της ομάδας 6, τα έξοδα που ανήκουν στην κλειόμενη χρήση, αλλά δεν πληρώνονται μέσα σ' αυτή, και τα οποία, σύμφωνα π.χ. με τις σχετικές συμβάσεις, δεν είναι στο τέλος της χρήσεως απαιτητά από τους δικαιούχους και για το λόγο αυτό δεν κρίνεται ορθό ή σκόπιμο να φέρονται σε πίστωση των οικείων λογαριασμών υποχρεώσεων.

4. Στο λογαριασμό 56.02 «αγορές υπό τακτοποίηση» παρακολουθούνται οι υπό τακτοποίηση αγορές αγαθών σε περίπτωση που το τιμολόγιο ή τα λοιπά δικαιολογητικά αγοράς δεν περιέρχονται στην οικονομική μονάδα κατά την παραλαβή των αγαθών. Σχετικά με τον τρόπο λειτουργίας του λογαριασμού αυτού ισχύουν όσα καθορίζονται στην περίπτ. 7 της παρ. 2.2.203.

5. Στο λογαριασμό 56.03 «εκπτώσεις επί πωλήσεων χρήσεως υπό διακανονισμό» καταχωρούνται, με αντίστοιχη χρέωση των οικείων υπολογαριασμών των 70-73, οι εκπτώσεις επί πωλήσεων που δικαιούνται οι πελάτες της οικονομικής μονάδας, για τις οποίες, κατά το κλείσιμο του ισολογισμού, δεν είναι γνωστό το ακριβές τους ύψος και από το λόγο αυτό δεν είναι δυνατή ή δεν κρίνεται σκόπιμη η πίστωση των λογαριασμών των πελατών. Οι εκπτώσεις αυτές, όταν κατά την επόμενη χρήση οριστικοποιούνται, μεταφέρονται από το λογαριασμό 56.03 στους οικείους προσωπικούς λογαριασμούς των δικαιούχων πελατών.

 58   ΛΟΓΑΡΙΑΣΜΟΙ ΠΕΡΙΟΔΙΚΗΣ ΚΑΤΑΝΟΜΗΣ

       Σημείωση: Στην παραπάνω ανάλυση έγινε τροποποίηση των προβλεπόμενων
       από το Ε.Γ.Λ.Σ. κωδικών αριθμών των δευτεροβάθμιων λογαριασμών
       του 58, ώστε αυτοί να ταυτίζονται με τους πρωτοβάθμιους λογαριασμούς των
       μάδων 2, 6, 7, και 8 με τους οποίους συλλειτουργούν.

        58.00   ...............................................................

        58.20    Προϋπολογισμένες αγορές εμπορευμάτων (Λ/20.99)

        58.21    ...............................................................

        .........

        58.24   Προϋπολογισμένες αγορές πρώτων και βοηθητικών υλών - υλικών
                     συσκευασίας (Λ/24.99)

        58.25   Προϋπολογισμένες αγορές αναλώσιμων υλικών (Λ/25.99)

        58.26   Προϋπολογισμένες αγορές ανταλλακτικών πάγιων στοιχείων
                     (Λ/26.99)

        58.27   .......................................................

        58.28   Προϋπολογισμένες αγορές ειδών συσκευασίας (Λ/28.99)

        .........

        58.60   Προϋπολογισμένες - Προπληρωμένες αμοιβές, έξοδα και παροχές
                     προσωπικού (Λ/69.60)

        58.61   Προϋπολογισμένες - Προπληρωμένες αμοιβές και έξοδα τρίτων
                     (Λ/61.99)

        58.62   Προϋπολογισμένες - Προπληρωμένες παροχές τρίτων (Λ/62.99)

        58.63   Προϋπολογισμένοι - Προπληρωμένοι φόροι - τέλη (Λ/63.99)

        58.64   Προϋπολογισμένα - Προπληρωμένα διάφορα έξοδα (Λ/64.99)

        58.65   Προϋπολογισμένοι - Προπληρωμένοι τόκοι και συναφή έξοδα
                     (Λ/65.99)

        58.66   Προϋπολογισμένες αποσβέσεις εκμεταλλεύσεως (Λ/66.99)

        58.67   ....................................................

        58.68   Προϋπολογισμένες προβλέψεις εκμεταλλεύσεως (Λ/68.99)

        58.69   ....................................................

        58.70   Προϋπολογισμένες πωλήσεις εμπορευμάτων (Λ/70.99)

        58.71   Προϋπολογισμένες πωλήσεις έτοιμων και ημιτελών προϊόντων
                     (Λ/71.99)

        58.72   Προϋπολογισμένες πωλήσεις λοιπών αποθεμάτων και άχρηστου
                     υλικού (Λ/72.99)

        58.73   Προϋπολογισμένες πωλήσεις υπηρεσιών (Λ/73.99)

        58.74   Προϋπολογισμένες - Προεισπραγμένες επιχορηγήσεις και διάφορα
                     έσοδα πωλήσεων (Λ/74.99)

        58.75   Προϋπολογισμένα - Προεισπραγμένα έσοδα παρεπόμενων ασχολιών
                     (Λ/75.99)

        58.76   Προϋπολογισμένα - Προεισπραγμένα έσοδα κεφαλαίων (Λ/76.99)

        58.77   ....................................................

        58.78   Προϋπολογισμένα τεκμαρτά έσοδα (Λ/78.99)

        58.79   ....................................................

        58.80   ....................................................

        58.81    Προϋπολογισμένα - Προπληρωμένα έκτακτα και ανόργανα
                     αποτελέσματα (Λ/81.99)

        58.82   Προϋπολογισμένα - Προπληρωμένα έξοδα και έσοδα προηγούμενων
                     χρήσεων (Λ/82.99)

        58.83   Προϋπολογισμένες προβλέψεις για έκτακτους κινδύνους (Λ/83.99)

        58.84   Προϋπολογισμένα έσοδα από προβλέψεις προηγούμενων χρήσεων
                     (Λ/84.99)

        58.85   Προϋπολογισμένες μη ενσωματωμένες στο λειτουργικό κόστος
                     αποσβέσεις (Λ/85.99)

2.2.509 Λογαριασμός 58 «Λογαριασμοί περιοδικής κατανομής»

1. Η καθιέρωση του λογαριασμού 58 αποβλέπει στη διευκόλυνση του προσδιορισμού των βραχύχρονων αποτελεσμάτων, τόσο στην περίπτωση που δε λειτουργεί η αναλυτική λογιστική των λογαριασμών της ομάδας 9, οπότε η λειτουργία του λογαριασμού 58 συμβάλλει αποφασιστικά στον προσδιορισμό ορθών βραχύχρονων αποτελεσμάτων, όσο και στην περίπτωση κατά την οποία λειτουργεί το σύστημα της αναλυτικής λογιστικής των λογαριασμών της ομάδας 9, οπότε με τη λειτουργία του λογαριασμού 58 επιτυγχάνεται αριθμητική συμφωνία των βραχύχρονων συνολικών αποτελεσμάτων της γενικής λογιστικής με τα βραχύχρονα αναλυτικά αποτελέσματα της αναλυτικής λογιστικής.

2. Με το λογαριασμό 58 δίνεται η ευχέρεια να καταχωρούνται στους οικείους λογαριασμούς εξόδων, εσόδων και αποθεμάτων της γενικής λογιστικής τα ποσά εκείνα τα οποία, μολονότι δεν έχουν πραγματοποιηθεί, είναι γνωστά (π.χ. δώρα Χριστουγέννων, ασφάλιστρα, ενοίκια) ή είναι δυνατό να προσδιορίζονται προϋπολογιστικά με ικανή προσέγγιση (π.χ. αποζημιώσεις απολυόμενου προσωπικού, φόροι - τέλη, τόκοι δανειακών λογαριασμών Τραπεζών).

3. Ο λογαριασμός 58 λειτουργεί κατά τους εξής δύο τρόπους:

α. Με παρεμβολή ενδιάμεσων λογαριασμών εξόδων, εσόδων, αποτελεσμάτων και αποθεμάτων.

Ο λογαριασμός 58 αναπτύσσεται στους δευτεροβάθμιους λογαριασμούς, οι οποίοι απεικονίζονται στο Σχέδιο Λογαριασμών (58.00, 58.01, 58.02, 58.03, 58.04, 58.05, 58.06, 58.07, 58.08, 58.09, 58.10, 58.11, 58.12, 58.13, 58.14, 58.15, 58.16, 58.17, 58.18, 58.19, 58.20, 58.21, 58.22, 58.23, 58.24, 58.25, 58.26, 58.27, και 58.28). Οι δευτεροβάθμιοι αυτοί λογαριασμοί είναι αντίστοιχοι με τους ενδιάμεσους λογαριασμούς παρακολουθήσεως των προϋπολογισμένων εξόδων της ομάδας 6 (60.99, 61.99, 62.99, 63.99, 64.99, 65.99, 66.99, 67.99 και 68.99), των προϋπολογισμένων έκτακτων και ανόργανων αποτελεσμάτων της ομάδας 8 (81.99, 82.99, 83.99, 84.99 και 85.99), των προϋπολογισμένων αγορών της ομάδας 2 (20.99, 24.99, 25.99, 26.99 και 28.99) και των προϋπολογισμένων εσόδων της ομάδας 7 (70.99, 71.99, 72.99, 73.99, 74.99, 75.99, 76.99 και 78.99).

Οι δευτεροβάθμιοι λογαριασμοί του 58, οι οποίοι αναπτύσσονται σύμφωνα με τις ανάγκες κάθε μονάδας, στο τέλος της περιόδου λογισμού (π.χ. στο τέλος του μήνα ή τριμήνου):

- Πιστώνονται με τα προϋπολογισμένα έξοδα, με χρέωση των αντίστοιχων ενδιάμεσων λογαριασμών των ομάδων 6 και 8.

- Πιστώνονται με τις προϋπολογισμένες αγορές, με χρέωση των αντίστοιχων ενδιάμεσων λογαριασμών της ομάδας 2.

- Χρεώνονται με τα προϋπολογισμένα έσοδα, με πίστωση των αντίστοιχων ενδιάμεσων λογαριασμών των ομάδων 7 και 8.

Στο τέλος της επόμενης περιόδου λογισμού (π.χ. στο τέλος του επόμενου, ή τριμήνου) ακυρώνονται οι εγγραφές προϋπολογισμένων εξόδων, εσόδων και αγορών, οι οποίες έγιναν στο τέλος της προηγούμενης περιόδου λογισμού, και διενεργούνται νέες εγγραφές προϋπολογισμένων εξόδων, εσόδων και αγορών, σύμφωνα με τα στοιχεία της νέας (επόμενης) περιόδου λογισμού, κατά τον τρόπο που αναφέρεται παραπάνω.

Οι δευτεροβάθμιοι λογαριασμοί του 58 στο τέλος της περιόδου λογισμού:

- Χρεώνονται με τα προπληρωμένα έξοδα που εμφανίζονται στους οικείους λογαριασμούς εξόδων των ομάδων 6 και 8, με πίστωση των αντίστοιχων ενδιάμεσων λογαριασμών των ομάδων αυτών (6 και 8).

- Πιστώνονται με τα προεισπραγμένα έσοδα που εμφανίζονται στους οικείους λογαριασμούς εσόδων των ομάδων 7 και 8, με χρέωση των αντίστοιχων ενδιάμεσων λογαριασμών των ομάδων αυτών (7 και 8).

Στο τέλος της επόμενης περιόδου λογισμού ακυρώνονται οι εγγραφές των προπληρωμένων εξόδων και των προεισπραγμένων εσόδων, οι οποίες έγιναν στο τέλος της προηγούμενης περιόδου λογισμού, και διενεργούνται νέες εγγραφές προπληρωμένων εξόδων και προεισπραγμένων εσόδων, σύμφωνα με τα στοιχεία της νέας (επόμενης) περιόδου λογισμού, όπως ακριβώς γίνεται και στην περίπτωση των προϋπολογισμένων εξόδων.

Τόσο στις περιπτώσεις των προϋπολογισμένων, π.χ. εξόδων, όσο και στις περιπτώσεις των προπληρωμένων εξόδων και προεισπραγμένων εσόδων, παρέχεται η δυνατότητα, αντί να ακυρώνονται οι εγγραφές της προηγούμενης περιόδου λογισμού και να γίνονται νέες στο τέλος της νέας (επόμενης) περιόδου λογισμού, να γίνονται συμπληρωματικές εγγραφές, έτσι ώστε οι παραπάνω λογαριασμοί, οι οποίοι λειτουργούν κατά το σύστημα των αντικριζόμενων χρεωπιστώσεων, να απεικονίζουν τα πράγματι προϋπολογισμένα, προπληρωμένα και προεισπραγμένα ποσά στο τέλος κάθε περιόδου λογισμού.

Ο λογαριασμός 58, στο τέλος της χρήσεως, εξισώνεται, για τους λόγους που αναφέρονται παρακάτω στην περιγραφή του δεύτερου τρόπου λειτουργίας του.

β. Με απευθείας χρεωπίστωση των λογαριασμών εξόδων, εσόδων και αποθεμάτων.

Ο λογαριασμός 58 αναπτύσσεται σε δευτεροβάθμιους και αναλυτικότερους λογαριασμούς σύμφωνα με τις ανάγκες κάθε οικονομικής μονάδας. Οι υπολογαριασμοί του 58, στο τέλος της περιόδου λογισμού (π.χ. στο τέλος του μήνα ή τριμήνου):

- Πιστώνονται με τα προϋπολογισμένα έξοδα, με χρέωση των οικείων λογαριασμών των ομάδων 6 και 8.

- Πιστώνονται με τις προϋπολογισμένες αγορές (π.χ. αγορές που δεν συνοδεύονται από τιμολόγιο ή άλλο παραστατικό αξίας), με χρέωση των οικείων λογαριασμών της ομάδας 2.

- Χρεώνονται με τα προϋπολογισμένα έσοδα (π.χ. μη πραγματοποιημένες επιχορηγήσεις ή πριμοδοτήσεις), με πίστωση των οικείων λογαριασμών των ομάδων 7 και 8.

Η χρέωση ή πίστωση των αναλυτικών λογαριασμών του 58, που ακολουθεί το λογισμό των προϋπολογισμένων εξόδων, εσόδων και αποθεμάτων, γίνεται κατά το διακανονισμό τους, με πίστωση ή χρέωση των λογαριασμών των χρηματικών διαθεσίμων ή των προσωπικών λογαριασμών των τρίτων, ή κατά το λογισμό των αποσβέσεων και των προβλέψεων, με πίστωση των αντίστοιχων λογαριασμών του ισολογισμού. Στο τέλος της χρήσεως ο λογαριασμός 58 εξισώνεται, επειδή, σε περίπτωση που έχουν πραγματοποιηθεί έξοδα ή έσοδα που αφορούν την επόμενη χρήση ή, προκειμένου για προϋπολογισμένες αγορές, δεν έχει παραληφθεί το οικείο παραστατικό αξίας, τα αντίστοιχα ποσά μεταφέρονται στους οικείους μεταβατικούς λογαριασμούς ενεργητικού (36) ή παθητικού (56).

Αν στη διάρκεια της χρήσεως διαπιστωθεί ότι τα προϋπολογισμένα ποσά, που λογίστηκαν, χρειάζεται να τροποποιηθούν για την ισομερή κατανομή τους (π.χ.  αλλαγή προγράμματος διαφημίσεων), οι τροποποιήσεις γίνονται έτσι ώστε στο τέλος της χρήσεως τα ποσά που λογίζονται να είναι ίσα με τα ποσά που πραγματοποιούνται.

Ο λογαριασμός 58 είναι δυνατό να λειτουργεί και στις περιπτώσεις προπληρωμένων εξόδων και προεισπραγμένων εσόδων, ώστε τα έξοδα και τα έσοδα που αφορούν επόμενες περιόδους να εμφανίζονται στους οικείους υπολογαριασμούς του. Στο τέλος κάθε περιόδου λογισμού, τα έξοδα και έσοδα που αναλογούν στην περίοδο αυτή μεταφέρονται από τους οικείους υπολογαριασμούς του 58 στους οικείους λογαριασμούς των ομάδων 6 και 7.

 59   ΒΡΑΧΥΠΡΟΘΕΣΜΕΣ ΥΠΟΧΡΕΩΣΕΙΣ ΥΠΟΚΑΤΑΣΤΗΜΑΤΩΝ
        ή ΑΛΛΩΝ ΚΕΝΤΡΩΝ
        (Όμιλος λογαριασμών προαιρετικής χρήσεως)

        590   ΠΡΟΜΗΘΕΥΤΕΣ
                  Ανάπτυξη αντίστοιχη του λ/σμού 50

        591   ΓΡΑΜΜΑΤΙΑ ΠΛΗΡΩΤΕΑ
                  Ανάπτυξη αντίστοιχη του λ/σμού 51

        592   ΤΡΑΠΕΖΕΣ ΛΟΓΑΡΙΑΣΜΟΙ ΒΡΑΧΥΠΡΟΘΕΣΜΩΝ ΥΠΟΧΡΕΩΣΕΩΝ
                  Ανάπτυξη αντίστοιχη του λ/σμού 52

        593   ΠΙΣΤΩΤΕΣ ΔΙΑΦΟΡΟΙ
                  Ανάπτυξη αντίστοιχη του λ/σμού 53

        594   ΥΠΟΧΡΕΩΣΕΙΣ ΑΠΟ ΦΟΡΟΥΣ - ΤΕΛΗ
                  Ανάπτυξη αντίστοιχη του λ/σμού 54

        595   ΑΣΦΑΛΙΣΤΙΚΟΙ ΟΡΓΑΝΙΣΜΟΙ
                  Ανάπτυξη αντίστοιχη του λ/σμού 55

        596   ΜΕΤΑΒΑΤΙΚΟΙ ΛΟΓΑΡΙΑΣΜΟΙ ΠΑΘΗΤΙΚΟΥ
                  Ανάπτυξη αντίστοιχη του λ/σμού 56

        597   .....................................

        598   ΛΟΓΑΡΙΑΣΜΟΙ ΠΕΡΙΟΔΙΚΗΣ ΚΑΤΑΝΟΜΗΣ
                  Ανάπτυξη αντίστοιχη του λ/σμού 58

2.2.510 Όμιλος λογαριασμών 59 «Βραχυπρόθεσμες υποχρεώσεις υποκαταστημάτων ή άλλων κέντρων» (όμιλος λογαριασμών προαιρετικής χρήσεως)

1. Σχετικά με τον τρόπο αναπτύξεως κάθε πρωτοβάθμιου λογαριασμού (590-598) ισχύουν όσα καθορίζονται στην περίπτ. 1 της παρ. 2.2.113.

2. Σχετικά με τον τρόπο λειτουργίας των πρωτοβάθμιων λογαριασμών 590-598 ισχύουν, αντίστοιχα, όσα ορίζονται παραπάνω στις παρ. 2.2.500 έως και 2.2.509 για τους πρωτοβάθμιους λογαριασμούς 50-58.

3. Σε περίπτωση που η οικονομική μονάδα κάνει χρήση του ομίλου λογαριασμών 59, τα κονδύλια των λογαριασμών του ομίλου αυτού, στον ισολογισμό τέλους χρήσεως, συναθροίζονται και εμφανίζονται μαζί με τα αντίστοιχα κονδύλια των λογαριασμών 50-58.


\chapter{ΟΡΓΑΝΙΚΑ ΕΞΟΔΑ ΚΑΤ' ΕΙΔΟΣ}

\section{Λογαριασμοί}

\begin{tabularx}{\linewidth}{lX}

\end{tabularx}

60 Αμοιβές και έξοδα προσωπικού

61 Αμοιβές και έξοδα τρίτων

62 Παροχές τρίτων

63 Φόροι - Τέλη

64 Διάφορα έξοδα

65 Τόκοι και συναφή έξοδα

66 Αποσβέσεις πάγιων στοιχείων ενσωματωμένες στο λειτουργικό κόστος

67 ............................................

68 Προβλέψεις εκμεταλλεύσεως

69 Οργανικά έξοδα κατ' είδος υποκαταστημάτων ή άλλων κέντρων

2.2.6 ΟΜΑΔΑ 6η: ΟΡΓΑΝΙΚΑ ΕΞΟΔΑ ΚΑΤ' ΕΙΔΟΣ

2.2.600 Περιεχόμενο και εννοιολογικοί προσδιορισμοί

1. Στην ομάδα 6 απεικονίζονται και παρακολουθούνται κατ' είδος τα έξοδα που αναφέρονται στην ομαλή εκμετάλλευση της χρήσεως (οργανικά), καθώς επίσης και οι ετήσιες επιβαρύνσεις για τη διενέργεια αποσβέσεων και προβλέψεων που ενσωματώνονται στο λειτουργικό κόστος.

2. Στους λογαριασμούς της ομάδας 6 δεν καταχωρούνται:

α. Ποσά που αφορούν επενδύσεις ή τοποθετήσεις. Τα ποσά αυτά καταχωρούνται στους οικείους λογαριασμούς των ομάδων 1 και 3, με εξαίρεση εκείνα που αφορούν τις ιδιοκατασκευές και λαμβάνονται υπόψη κατά την κοστολόγησή τους, οπότε με τα σχετικά ποσά χρεώνονται οι οικείοι λογαριασμοί της ομάδας 1, με πίστωση του λογαριασμού 78.00 «ιδιοπαραγωγή και βελτιώσεις παγίων».

β. Ποσά που αφορούν ζημίες και έξοδα εξαιρετικού χαρακτήρα, τα οποία καταχωρούνται στους οικείους υπολογαριασμούς του 81 «έκτακτα και ανόργανα αποτελέσματα».

γ. Ποσά που αφορούν ζημίες και έξοδα προηγούμενων χρήσεων, τα οποία καταχωρούνται στους οικείους υπολογαριασμούς του 82 «έξοδα και έσοδα προηγούμενων χρήσεων».

δ. Ποσά προβλέψεων που δεν αφορούν άμεσα την εκμετάλλευση, τα οποία καταχωρούνται στους οικείους υπολογαριασμούς του 83 «προβλέψεις για έκτακτους κινδύνους».

ε. Ποσά που αφορούν φόρο εισοδήματος επί των αδιανέμητων κερδών της χρήσεως, τα οποία, σαν αφαιρετικά στοιχεία των αποτελεσμάτων χρήσεως, καταχωρούνται στη χρέωση του λογαριασμού 88.08 «φόρος εισοδήματος και εισφορά ΟΓΑ».

στ. Τα υπολογιστικά ή τεκμαρτά έξοδα (π.χ. τόκοι ιδίων κεφαλαίων, αμοιβή επιχειρηματία στις προσωπικές εταιρείες και ατομικές επιχειρήσεις, αυτασφάλιστρα), τα οποία δε συνδέονται με εκταμίευση και δε λογιστικοποιούνται στο χρηματοοικονομικό κύκλωμα της γενικής λογιστικής.

3. Αν κατά το χρόνο που γίνονται οι εγγραφές καταχωρίσεως των εξόδων δεν είναι γνωστός ο χαρακτήρας ή ο προορισμός τους, τα ποσά των εξόδων αυτών είναι δυνατό να καταχωρούνται προσωρινά στους λογαριασμούς της ομάδας 6 και από αυτούς, είτε περιοδικά μέσα στη χρήση, είτε στο τέλος της κατά το κλείσιμο του ισολογισμού, να μεταφέρονται στους λογαριασμούς στους οποίους πραγματικά ανήκουν (δηλαδή στους λογαριασμούς του ενεργητικού ή στους λογαριασμούς της ομάδας 8).

4. Ο τρόπος διορθώσεως των λογαριασμών εξόδων της ομάδας 6, που περιγράφεται στην πιο πάνω περίπτ. 3, δεν εφαρμόζεται στις ακόλουθες δύο περιπτώσεις:

α. Στην περίπτωση που τα έξοδα αφορούν κατασκευές ή βελτιώσεις πάγιων στοιχείων, των οποίων το κόστος προσδιορίζεται από τους λογαριασμούς της αναλυτικής λογιστικής εκμεταλλεύσεως της ομάδας 9 ή, αν δε λειτουργεί η λογιστική αυτή, εξωλογιστικά με υπολογισμούς που βασίζονται σε λογιστικά στοιχεία. Στην περίπτωση αυτή οι διορθωτικές εγγραφές των εξόδων γίνονται με πίστωση του λογαριασμού 78.00 «ιδιοπαραγωγή και βελτιώσεις παγίων» και χρέωση των οικείων λογαριασμών της ομάδας 1.

β. Στην περίπτωση που, για τα έξοδα που πραγματοποιούνται μέσα στη χρήση και από τη φύση τους αφορούν λογαριασμούς της ομάδας 6, έχει προηγηθεί, σε προηγούμενες χρήσεις, ο σχηματισμός προβλέψεων. Στην περίπτωση αυτή, προκειμένου να εμφανίζεται στους λογαριασμούς της ομάδας 6 το πραγματικό ύψος των εξόδων που πραγματοποιούνται κατά τη χρήση, οι διορθωτικές εγγραφές των εξόδων γίνονται με πίστωση του λογαριασμού 78.05 «χρησιμοποιημένες προβλέψεις προς κάλυψη εξόδων εκμεταλλεύσεως» και χρέωση των οικείων υπολογαριασμών του 44 «προβλέψεις».

5. Σε περιπτώσεις που η οικονομική μονάδα καταλογίζει σε βάρος τρίτων έξοδα που πραγματοποιούνται για λογαριασμό τους, όπως π.χ. έξοδα εκτυπώσεως εντύπων ή γραφική ύλη, τα οποία για οποιοδήποτε λόγο έχουν καταχωρηθεί στους οικείους λογαριασμούς της ομάδας 6, τα αντίστοιχα ποσά μπορούν να μεταφέρονται με αντιλογισμό στη χρέωση των οικείων λογαριασμών των τρίτων.

2.2.601 Περιοδική κατανομή εξόδων μέσα στη χρήση

Σε περίπτωση που η οικονομική μονάδα υπολογίζει βραχύχρονα (π.χ. μηνιαία ή τριμηνιαία) αποτελέσματα ή καταρτίζει περιοδικές συγκρίσιμες καταστάσεις, η περιοδική κατανομή των εξόδων αντιμετωπίζεται, είτε με απευθείας χρεοπίστωση των οικείων λογαριασμών εξόδων, είτε με παρεμβολή ενδιάμεσων λογαριασμών εξόδων, (60.99, 61.99, 62.99, 63.99, 64.99, 65.99, 66.99 και 68.99), σύμφωνα με όσα καθορίζονται στην παρ. 2.2.509.

2.2.602 Τακτοποίηση λογαριασμών εξόδων στο τέλος της χρήσεως

1. Τα υπόλοιπα των λογαριασμών της ομάδας 6, στο τέλος της χρήσεως, μεταφέρονται στη χρέωση του λογαριασμού 80.00 «λογαριασμός γενικής εκμεταλλεύσεως». Σε περίπτωση που οι λογαριασμοί εξόδων περιλαμβάνουν και προπληρωμένα ποσά εξόδων που αφορούν επόμενες χρήσεις ή σε περίπτωση που οι λογαριασμοί αυτοί δεν περιλαμβάνουν ποσά δουλευμένων εξόδων, επειδή θα πληρωθούν κατά τις επόμενες χρήσεις, πριν από τη μεταφορά των υπολοίπων τους στο λογαριασμό 80.00 γίνονται εγγραφές τακτοποιήσεως, έτσι ώστε τα υπόλοιπα αυτά να απεικονίζουν το ακριβές ύψος όλων των δουλευμένων εξόδων εκμεταλλεύσεως της χρήσεως που κλείνει.

2. Οι εγγραφές τακτοποιήσεως της προηγούμενης περιπτώσεως γίνονται με τη βοήθεια μεταβατικών λογαριασμών ενεργητικού (λογαριασμός 36) και παθητικού (λογαριασμός 56), όπως καθορίζεται αντίστοιχα στις παρ. 2.2.307 και 2.2.507.

2.2.603 Δυνητική ευχέρεια αναπτύξεως λογαριασμών εξόδων

Η υποδεικνυόμενη ανάπτυξη των λογαριασμών τρίτου βαθμού στους οποίους αναλύονται οι δευτεροβάθμιοι των λογαριασμών 60-68 είναι ενδεικτική.

Κάθε οικονομική μονάδα έχει τη δυνατότητα, αντί ν' αναπτύξει κατ' είδος τους τριτοβάθμιους λογαριασμούς εξόδων, να τους αναπτύξει κατά προορισμό. Στην περίπτωση όμως αυτή οι υποχρεωτικοί τριτοβάθμιοι λογαριασμοί των εξόδων κατ' είδος εμφανίζονται υποχρεωτικά ως αναλυτικοί των περιληπτικών, κατά προορισμό, λογαριασμών, στους οποίους θα αναλύονται οι δευτεροβάθμιοι κατ' είδος λογαριασμοί των 60-68 πρωτοβάθμιων.

 60   ΑΜΟΙΒΕΣ ΚΑΙ ΕΞΟΔΑ ΠΡΟΣΩΠΙΚΟΥ

        60.00   Αμοιβές έμμισθου προσωπικού

                    60.00.00   Τακτικές αποδοχές (περιλαμβάνονται και προσαυξήσεις λόγω
                                       νυκτερινών, Κυριακών και εξαιρετέων)

                               01   Οικογενειακά επιδόματα

                               02   Αμοιβές υπερωριακής απασχολήσεως

                               03   Δώρα εορτών (Χριστουγέννων και Πάσχα)

                               04   .......................................

                               05   Αποδοχές ασθένειας

                               06   Αποδοχές κανονικής άδειας

                               07   Επιδόματα κανονικής άδειας

                               08   Αποζημιώσεις μη χορηγούμενων αδειών

                               09   Ποσοστά για πωλήσεις και αγορές

                               10   Έκτακτες αμοιβές (πριμ, βραβεία, επιδόματα, αποζημιώσεις για
                                       παροχές σε είδος κ.λπ.)

                               11   Αμοιβές εκτός έδρας (όταν δεν καλύπτουν έξοδα εκτός έδρας)

                               12   Αμοιβές μαθητευομένων (τακτικές, έκτακτες, αργιών,
                                       ασθένειας, άδειας κ.λπ.)

                    60.00.13

                    ..............

                    60.00.99

        60.01    Αμοιβές ημερομίσθιου προσωπικού

                    60.01.00   Τακτικές αποδοχές (περιλαμβάνονται και προσαυξήσεις λόγω
                                      νυκτερινών, Κυριακών και εξαιρετέων)

                               01   Οικογενειακά επιδόματα

                               02   Αμοιβές υπερωριακής απασχολήσεως

                               03   Δώρα εορτών (Χριστουγέννων και Πάσχα)

                               04   Αποδοχές επίσημων αργιών

                               05   Αποδοχές ασθένειας

                               06   Αποδοχές κανονικής άδειας

                               07   Επιδόματα κανονικής άδειας

                               08   Αποζημιώσεις μη χορηγούμενων αδειών

                               09   Ποσοστά για πωλήσεις και αγορές

                               10   Έκτακτες αμοιβές (πριμ, βραβεία, επιδόματα ή αποζημιώσεις
                                       για παροχές σε είδος κ.λπ.)

                               11   Αμοιβές εκτός έδρας (όταν δεν καλύπτουν έξοδα εκτός έδρας)

                               12   Αμοιβές μαθητευομένων (τακτικές, έκτακτες, αργιών,
                                       ασθένειας ή άδειας)

                    ..............

                    60.01.99

        60.02   Παρεπόμενες παροχές και έξοδα προσωπικού

                    60.02.00   Είδη ενδύσεως

                               01   Έξοδα στεγάσεως (π.χ. κατοικιών)

                               02   Επιχορηγήσεις και λοιπά έξοδα κυλικείου - εστιατορίου

                               03   Έξοδα ψυχαγωγίας προσωπικού (π.χ. κατασκηνώσεων,
                                       εκδρομών, χοροεσπερίδων ή εορταστικών εκδηλώσεων)

                               04   Έξοδα επιμορφώσεως προσωπικού (π.χ. δίδακτρα, έξοδα
                                       εκπαιδευτικών ταξιδιών ή έξοδα μετεκπαιδεύσεων)

                               05   Έξοδα ιατροφαρμακευτικής περιθάλψεως (π.χ. νοσήλια,
                                       φάρμακα, έξοδα εγχειρήσεων, έξοδα κηδειών)

                               06   Ασφάλιστρα προσωπικού (π.χ. ομαδικής ή ατομικής
                                       ασφαλίσεως)

                               07   Αξία χορηγούμενων αποθεμάτων (Γνωμ. 44/1129/1989)

                    ..............

                    60.02.99       Λοιπές παρεπόμενες παροχές και έξοδα προσωπικού

        60.03   Εργοδοτικές εισφορές και επιβαρύνσεις έμμισθου προσωπικού

                    60.03.00   Εργοδοτικές εισφορές ΙΚΑ

                               01   Εργοδοτικές εισφορές λοιπών ταμείων κύριας ασφαλίσεως

                               02   Εργοδοτικές εισφορές ταμείων επικουρικής ασφαλίσεως

                               03   .............................................

                               04   Χαρτόσημο μισθοδοσίας

                    .............

                    60.03.99

        60.04   Εργοδοτικές εισφορές και επιβαρύνσεις ημερομίσθιου προσωπικού

                    60.04.00   Εργοδοτικές εισφορές ΙΚΑ

                               01   Εργοδοτικές εισφορές λοιπών ταμείων κύριας ασφαλίσεως

                               02   Εργοδοτικές εισφορές ταμείων επικουρικής ασφαλίσεως

                               03   Δωρόσημο οικοδόμων (Γνωμ. 252/2244/1995)

                               04   Χαρτόσημο μισθοδοσίας

                    .............

                    60.04.99

        60.05   Αποζημιώσεις απολύσεως ή εξόδου από την υπηρεσία

                    60.05.00    Αποζημιώσεις απολύσεως ή εξόδου από την υπηρεσία
                                      έμμισθου προσωπικού

                               01   Αποζημιώσεις απολύσεως ή εξόδου από την υπηρεσία
                                       ημερομίσθιου προσωπικού

                    ..............

                    60.05.99

        60.06

        .........

        60.99   Προϋπολογισμένες - Προπληρωμένες αμοιβές, έξοδα και παροχές
                     προσωπικού (Λ/58.60)

 

 

2.2.604 Λογαριασμός 60 «Αμοιβές και έξοδα προσωπικού»

1. Στο λογαριασμό 60 καταχωρούνται όλα τα έξοδα της οικονομικής μονάδας που προκύπτουν από την απασχόληση προσωπικού της, το οποίο συνδέεται με αυτή με σύμβαση μισθώσεως εργασίας.

2. Στους λογαριασμούς 60.00 «αμοιβές έμμισθου προσωπικού» και 60.01 «αμοιβές ημερομίσθιου προσωπικού» καταχωρούνται οι κάθε είδους αμοιβές τους έμμισθου και ημερομίσθιου προσωπικού, αντίστοιχα. Οι λογαριασμοί αυτοί, στην περίπτωση που η μισθοδοτική κατάσταση λογιστικοποιείται με συμψηφιστική εγγραφή, χρεώνονται, με βάση μισθοδοτικές καταστάσεις ή ατομικές εκκαθαρίσεις, με τις ονομαστικές (μικτές) αποδοχές του προσωπικού, με πίστωση:

- των λογαριασμών 33.00, 33.01 και 33.02, με τα ποσά που ενδεχόμενα παρακρατούνται για την εξόφληση προκαταβολών, χρηματικών διευκολύνσεων και δανείων.

- των οικείων υπολογαριασμών των λογαριασμών 54 και 55, με τα ποσά που παρακρατούνται από τις αποδοχές του προσωπικού για φόρους, χαρτόσημο και εισφορές υπέρ των ασφαλιστικών οργανισμών.

- του λογαριασμού 53.00, με τα καθαρά ποσά που καταβάλλονται στο προσωπικό με χρέωση του λογαριασμού αυτού.

Στην περίπτωση που η μισθοδοτική κατάσταση λογιστικοποιείται ταμιακά, ο λογαριασμός 53.00 δε χρησιμοποιείται.

Στο λογαριασμό 60.00 καταχωρούνται και οι αποδοχές που καταβάλλονται σε διευθυντές, γενικούς διευθυντές και μέλη του διοικητικού συμβουλίου ανώνυμων εταιρειών, για υπηρεσίες που παρέχουν στην οικονομική μονάδα με βάση σύμβαση μισθώσεως εργασίας, όπως για το λοιπό έμμισθο προσωπικό.

Σε περίπτωση που η οικονομική μονάδα αδυνατεί ή δε θέλει να παρακολουθεί χωριστά τα οικογενειακά επιδόματα και τις αμοιβές μαθητευομένων στους προαιρετικούς τριτοβάθμιους λογαριασμούς 60.00.01, 60.00.12, 60.01.01 και 60.01.12, έχει τη δυνατότητα να παρακολουθεί τις κατηγορίες αυτές αμοιβών προσωπικού, μαζί με τις τακτικές αποδοχές, στους λογαριασμούς 60.00.00 και 60.01.00, κατά περίπτωση.

3. Στο λογαριασμό 60.02 «παρεπόμενες παροχές και έξοδα προσωπικού» καταχωρούνται τα ποσά που αντιπροσωπεύουν, εκτός από τις αμοιβές και εργοδοτικές εισφορές, λοιπές παροχές και έξοδα που πραγματοποιούνται για το προσωπικό της οικονομικής μονάδας. Στις περιπτώσεις εκείνες που η οικονομική μονάδα κρίνει σκόπιμο να παρακολουθεί τα διάφορα είδη που προορίζονται για το προσωπικό της (π.χ. είδη ενδύσεως ή φάρμακα) σε λογαριασμούς αποθεμάτων, κατά την αγορά των ειδών αυτών χρεώνονται οι οικείοι υπολογαριασμοί του λογαριασμού 25 «αναλώσιμα υλικά».

Τα ποσά που ενδεχόμενα εισπράττονται από το προσωπικό ή λογίζονται σε βάρος του για συμμετοχή στα παραπάνω έξοδα, φέρονται σε πίστωση του λογαριασμού 75.01 «έσοδα από παροχή υπηρεσιών στο προσωπικό».

4. Στους λογαριασμούς 60.03 «εργοδοτικές εισφορές και επιβαρύνσεις έμμισθου προσωπικού» και 60.04 «εργοδοτικές εισφορές και επιβαρύνσεις ημερομίσθιου προσωπικού» καταχωρούνται τα ποσά των εργοδοτικών εισφορών και λοιπών επιβαρύνσεων (χαρτόσημο, φόρος α.ν. 843/48) που αναλογούν στις αποδοχές που καταβάλλονται στο έμμισθο (60.03) και ημερομίσθιο (60.04) προσωπικό της οικονομικής μονάδας, με αντίστοιχη πίστωση των οικείων υπολογαριασμών των λογαριασμών 54 και 55.

Τα ποσά προστίμων και προσαυξήσεων που ενδεχόμενα επιβάλλονται στις εισφορές, π.χ. λόγω καθυστερημένης καταβολής τους, καταχωρούνται, αν αφορούν τη χρήση στο λογαριασμό 81.00 «έκτακτα και ανόργανα έξοδα», αν όμως αφορούν προηγούμενες χρήσεις, έστω και αν βεβαιώνονται μέσα στη χρήση, στο λογαριασμό 82.00 «έξοδα προηγούμενων χρήσεων».

5. Στο λογαριασμό 60.05 «αποζημιώσεις απολύσεως ή εξόδου από την υπηρεσία» καταχωρούνται οι αποζημιώσεις που καταβάλλονται από την οικονομική μονάδα στο προσωπικό που αποχωρεί από την υπηρεσία, είτε λόγω καταγγελίας της συμβάσεως εργασίας, είτε λόγω συμπληρώσεως του χρόνου συνταξιοδοτήσεως, ανεξάρτητα από το αν έχει προηγηθεί ο σχηματισμός σχετικής προβλέψεως (λογ. 44.00).

Στην περίπτωση που η οικονομική μονάδα κάνει χρήση της δυνητικής ευχέρειας της περίπτ. 5-β της παρ. 2.2.405, εφαρμόζονται οι διατάξεις του δεύτερου εδαφίου της περιπτώσεως αυτής (5-β).

 61   ΑΜΟΙΒΕΣ ΚΑΙ ΕΞΟΔΑ ΤΡΙΤΩΝ

        61.00   Αμοιβές και έξοδα ελεύθερων επαγγελματιών υποκείμενες σε
                     παρακράτηση φόρου εισοδήματος

                     61.00.00   Αμοιβές και έξοδα δικηγόρων

                                01   Αμοιβές και έξοδα συμβολαιογράφων (όταν υπόκεινται σε
                                        παρακράτηση φόρου εισοδήματος)

                                02   Αμοιβές και έξοδα τεχνικών

                                03   Αμοιβές και έξοδα οργανωτών - μελετητών - ερευνητών

                                04   Αμοιβές και έξοδα ελεγκτών

                                05   Αμοιβές και έξοδα ιατρών

                                06   Αμοιβές και έξοδα λογιστών

                                07   Αμοιβές φοροτεχνικών

                     ..............

                     61.00.99       Αμοιβές και έξοδα λοιπών ελεύθερων επαγγελματιών

        61.01   Αμοιβές και έξοδα μη ελεύθερων επαγγελματιών υποκείμενες σε
                     παρακράτηση φόρου εισοδήματος

                     61.01.00    Αμοιβές συνεδριάσεων μελών διοικητικού συμβουλίου

                                01   Αμοιβές και έξοδα διαφόρων τρίτων

                                02   Αμοιβές σε εταιρίες μελετών Τεχνικών Έργων εξωτερικού
                                        (Γνωμ. 252/2244/1995)

                     ..............

                     61.01.99

       61.02   Λοιπές προμήθειες τρίτων

                     61.02.00   Προμήθειες για αγορές

                                01   Προμήθειες για πωλήσεις

                                02   Προμήθειες εισπράξεως τιμολογίων και φορτωτικών εγγράφων

                                03   Μεσιτείες

                     ..............

                     61.02.99

        61.03   Επεξεργασίες από τρίτους

                     61.03.00    Επεξεργασίες (Facon)

                                01   Αμοιβές μηχανογραφικής επεξεργασίας (Service)

                     ..............

                     61.03.99

        61.04

        .........

        61.90   Αμοιβές τρίτων μη υποκείμενες σε παρακράτηση φόρου εισοδήματος
                    (Γνωμ. 252/2244/1995)

                     61.90.00   Αμοιβές σε εταιρίες μελετών τεχνικών έργων
                                      (Γνωμ. 252/2244/1995)

                                01   Αμοιβές για έρευνα αγοράς

                                02   ........................................

                                03   Αμοιβές Γραφείων επιλογής προσωπικού

                                04

        61.91   Πνευματικά και καλλιτεχνικά δικαιώματα τρίτων επί πωλήσεων
                    (Γνωμ. 95/1694/1992)

        61.92   Εισφορές υπέρ τρίτων για ελεύθερους επαγγελματίες

                     61.92.00   Εισφορές Ταμείου Νομικών έμμισθων δικηγόρων
                                      (άρθρο 37 Ν. 2145/1993)

                     61.92.01   Εισφορές Τ.Σ.Α.Υ. έμμισθων ιατρών

                                02   .................................

        61.93   Αμοιβές υπερβολάβων εκτελέσεως εργασιών τεχνικών έργων
                    (Γνωμ. 252/2244/1995)

        61.94   Εισφορές υπέρ Ασφαλιστικών Οργανισμών για κατασκευαζόμενα
                     Τεχνικά Έργα (Γνωμ. 252/2244/1995)

                     61.94.00   Εισφορά ΙΚΑ προσωπικού υπεργολάβων εκτελέσεως εργασιών
                                      Τεχνικών Έργων

                     61.94.01   Δωρόσημο οικοδόμων υπεργολάβων εκτελέσεως εργασιών
                                      Τεχνικών Έργων

                     61.94.02   ..............................................

                     61.94.05   Κρατήσεις - Εισφορές υπέρ Τ.Σ.Μ.Ε.Δ.Ε.

                     61.94.06   Κρατήσεις - Εισφορές υπέρ Τ.Π.Ε.Δ.Ε.

                     61.94.07   Κρατήσεις - Εισφορές υπέρ Μ.Τ.Π.Υ.

                     61.94.08   .....................................

                     κ.λπ.

                     ...............

        61.98   Λοιπές αμοιβές τρίτων

                     61.98.00    Χρήσεις δικαιωμάτων (Royalties)

                                01   Αποζημιώσεις για φθορά ειδών συσκευασίας προμηθευτών

                                ....

                     ..............

                     61.98.99

        61.99   Προϋπολογισμένες - Προπληρωμένες αμοιβές και έξοδα τρίτων
                     (Λ/58.61)

2.2.605 Λογαριασμός 61 «Αμοιβές και έξοδα τρίτων»

1. Στους υπολογαριασμούς του 61 καταχωρούνται οι αμοιβές που λογίζονται από την οικονομική μονάδα για εργασίες τρίτων, οι οποίοι Δε συνδέονται με αυτή με σχέση εξαρτημένης εργασίας.

2. Οι αμοιβές τρίτων που δεν περιλαμβάνονται στους υπολογαριασμούς 61.00-61.03 καταχωρούνται στο λογαριασμό 61.98 «λοιπές αμοιβές τρίτων». Στο λογαριασμό αυτό καταχωρούνται και τα ποσά που λογίζονται ή καταβάλλονται σε τρίτους κάθε χρόνο για την παραχώρηση της χρήσεως π.χ. σημάτων, μεθόδων παραγωγής ή διπλωμάτων ευρεσιτεχνίας. Αν τα ποσά αυτά προκαταβάλλονται για τη χρήση δικαιωμάτων και προνομίων για περισσότερα χρόνια, χρεώνονται οι οικείοι υπολογαριασμοί του λογαριασμού 16, σύμφωνα με όσα καθορίζονται στην παρ. 2.2.110.

Σημείωση: Εξυπακούεται ότι οι αμοιβές των τρίτων, των οποίων οι προσφερόμενες στην επιχείρηση υπηρεσίες αφορούν κατασκευή πάγιων στοιχείων (αρχιτεκτόνων, μηχανικών, μελετητών κ.λπ.), καταχωρούνται στους οικείους λογαριασμούς του παγίου και όχι στους υπολογαριασμούς του 61 (για διευκόλυνση καταρτίσεως της σχετικής φορολογικής δηλώσεως κ.λπ., δύνανται να καταχωρούνται στον 61 και αμέσως - με αντιλογιστική εγγραφή - να μεταφέρονται στο πάγιο).

Όταν έξοδα των τρίτων δεν υπόκεινται σε παρακράτηση φόρου, αποχωρίζονται των αμοιβών και καταχωρούνται στον 64.98.  (Α.Υ.Ο. 1116200/885/0015/ΠΟΛ. 1282/30-10-96)

 62   ΠΑΡΟΧΕΣ ΤΡΙΤΩΝ

        62.00   Ηλεκτρικό ρεύμα παραγωγής

        62.01   Φωταέριο παραγωγικής διαδικασίας

        62.02   Ύδρευση παραγωγικής διαδικασίας

        62.03   Τηλεπικοινωνίες

                    62.03.00   Τηλεφωνικά - Τηλεγραφικά

                                01    TELEX (Τηλέτυπο) - FAX

                                02   Ταχυδρομικά

                                03   Διάφορα έξοδα τηλεπικοινωνιών

                                ....

                    62.03.10   Αγορές τηλεκαρτών προς διάθεση (Γνωμ. 147/1968/1993)

                    ...............

                    62.03.99

        62.04   Ενοίκια

                    62.04.00    Ενοίκια εδαφικών εκτάσεων

                                01   Ενοίκια κτιρίων - τεχνικών έργων

                                02   Ενοίκια μηχανημάτων - τεχνικών εγκαταστάσεων - λοιπού
                                        μηχανολογικού εξοπλισμού

                                03   Ενοίκια μεταφορικών μέσων

                                04   Ενοίκια επίπλων

                                05   Ενοίκια μηχανογραφικών μέσων

                                06   Ενοίκια λοιπού εξοπλισμού

                                07   Ενοίκια φωτοαντιγραφικών μέσων

                                08   Ενοίκια φωτεινών επιγραφών

                                09   ..................................

                    62.04.10   Ενοίκια χρονομεριστικής μισθώσεως Ν. 1652/1986
                                      (Γνωμ. 85/1644/1991)

                    ................

                    62.04.99

        62.05   Ασφάλιστρα

                    62.05.00    Ασφάλιστρα πυρός

                                01   Ασφάλιστρα μεταφορικών μέσων

                                02   Ασφάλιστρα μεταφορών

                                03   Ασφάλιστρα πιστώσεων

                                ....

                                08

        62.06   Αποθήκευτρα

        62.07   Επισκευές και συντηρήσεις

                    62.07.00   Εδαφικών εκτάσεων

                                01   Κτιρίων - Εγκαταστάσεων κτιρίων - Τεχνικών έργων

                                02   Μηχανημάτων - Τεχνικών Εγκαταστάσεων - Λοιπού
                                        Μηχανολογικού εξοπλισμού

                                03   Μεταφορικών μέσων

                                04   Επίπλων και λοιπού εξοπλισμού

                                05   Εμπορευμάτων

                                06   Έτοιμων προϊόντων

                                07   Λοιπών υλικών αγαθών

                    ..............

                    62.07.99

        62.08

        .........

        62.91   Έξοδα μεταφορικού έργου (Γνωμ. 129/1875/1993)

        .........

        62.98   Λοιπές παροχές τρίτων

                    62.98.00    Φωτισμός (πλην ηλεκτρικής ενέργειας παραγωγής)

                                01   Φωταέριο (πλην φωταερίου παραγωγής)

                                02   Ύδρευση (πλην υδρεύσεως παραγωγής)

                                03   Έξοδα ξενοδοχείων για εξυπηρέτηση πελατών μας
                                        (Γνωμ. 219/2184/1994)

                    ..............

                    62.98.99

        62.99   Προϋπολογισμένες - Προπληρωμένες παροχές τρίτων (Λ/58.62)

 

2.2.606 Λογαριασμός 62 «Παροχές τρίτων»

Στους υπολογαριασμούς του 62 καταχωρούνται: (1) τα αντίτιμα των παροχών κοινής ωφελείας, (2) τα ενοίκια μισθώσεως πάγιων στοιχείων, εκτός από εκείνα που αφορούν στέγαση προσωπικού, τα οποία καταχωρούνται στο λογαριασμό 60.02.01 «έξοδα στεγάσεως», (3) τα κάθε μορφής ασφάλιστρα, εκτός από εκείνα που αφορούν ασφάλειες προσωπικού και καταχωρούνται στο λογαριασμό 60.02.06 «ασφάλιστρα προσωπικού», καθώς και εκείνα που αφορούν ασφάλειες μεταφοράς των αγοραζόμενων ειδών, τα οποία καταχωρούνται στους οικείους υπολογαριασμούς του 32 «παραγγελίες στο εξωτερικό» ή σε λογαριασμούς αποθεμάτων της ομάδας 2 ή πάγιων στοιχείων της ομάδας 1, (4) τα κάθε είδους αποθήκευτρα, (5) το κόστος επισκευής και συντηρήσεως πάγιων και λοιπών στοιχείων ενεργητικού, που γίνονται από τρίτους και (6) οι κάθε είδους παροχές τρίτων που δεν υπάγονται σε έναν από τους υπολογαριασμούς του 62.


Σημείωση: Για τα καταβαλλόμενα ενοίκια μισθώσεως πάγιων στοιχείων με σύμβαση leasing (βάσει του Ν. 1665/1986), με την 106/1804/1992 Γνωμ. του ΕΣΥΛ υποδεικνύονται να ανοίγονται ιδιαίτεροι τριτοβάθμιοι λογαριασμοί, ανάλογα με τις επιθυμητές πληροφορίες (όπως κατά ομοειδείς κατηγορίες κ.λπ.) Π.χ.:

62.04.20   Ενοίκια μισθώσεως leasing μηχανημάτων

62.04.21   Ενοίκια μισθώσεων leasing ...........

62.04.25   Ενοίκια μισθώσεων leasing μεταφορικών μέσων
                  (ή φορτηγών αυτοκινήτων) κ.λπ.

 (Α.Υ.Ο. 1116200/885/0015/ΠΟΛ. 1282/30-10-96)
 
63   ΦΟΡΟΙ - ΤΕΛΗ

        63.00    Φόρος εισοδήματος μη συμψηφιζόμενος

                    63.00.00   Φόρος εισοδήματος μη συμψηφιζόμενος εσωτερικού

                              01   Φόρος εισοδήματος μη συμψηφιζόμενος εξωτερικού

                    ...............

                    63.00.99

        63.01   Εισφορά ΟΓΑ

                    63.01.01   Εισφορά ΟΓΑ χαρτοσήμου

        63.02   Τέλη συναλλαγματικών, δανείων και λοιπών πράξεων

                    63.02.00   Χαρτόσημα συναλλαγματικών και αποδείξεων

                              01   Χαρτόσημα λοιπών πράξεων

                    ...............

                    63.02.99

        63.03   Φόροι - Τέλη κυκλοφορίας μεταφορικών μέσων

                    63.03.00    Αυτοκινήτων επιβατικών

                              01   Αυτοκινήτων φορτηγών

                              02   Σιδηροδρομικών οχημάτων

                              03   Πλωτών μέσων

                              04   Εναέριων μέσων

                    ..............

                    63.03.99

        63.04   Δημοτικοί φόροι - τέλη

                    63.04.00   Τέλη καθαριότητας και φωτισμού

                              01   Φόροι και τέλη ανεγειρόμενων ακινήτων

                              ....

                              03   Τέλη ακίνητης περιουσίας (άρθρο 24 Ν. 2130/1993)

                    ..............

                    63.04.99   Λοιποί δημοτικοί φόροι - τέλη

        63.05   Φόροι - Τέλη προβλεπόμενοι από διεθνείς οργανισμούς

        63.06   Λοιποί φόροι - τέλη εξωτερικού

        63.07

        .........

        63.90   Τέλη υπέρ τρίτων επί κατασκευαζόμενων Τεχνικών Έργων
                     (Γνωμ. 252/2244/1995)

        .........

        63.98   Διάφοροι φόροι - τέλη

                    63.98.00   Χαρτόσημο μισθωμάτων

                              01   Τέλη υδρεύσεως
                                      (Καταργήθηκαν με το άρθρο 43 παρ. 2 Ν. 2065/1992)

                              02   Φόρος ακίνητης περιουσίας
                                      (Καταργήθηκε με το άρθρο 37 Ν. 2065/1992)

                              03   Χαρτόσημο κερδών

                              04   Χαρτόσημο εσόδων από τόκους

                              05   ..........................................

                              06   Χαρτόσημο αμοιβών τρίτων

                              07   Κρατήσεις υπέρ Δημοσίου και τρίτων από πωλήσεις προς το
                                      Δημόσιο και τα Ν.Π.Δ.Δ.

                              08   ΦΠΑ εκπιπτόμενος στη φορολογίας εισοδήματος
                                      (Γνωμ. 243/2162/1995)

                              09   ΦΠΑ μη εκπιπτόμενος στη φορολογία εισοδήματος
                                      (Γνωμ. 243/2162/1995)

                    ..............

                    63.98.99       Λοιποί φόροι - τέλη

        63.99   Προϋπολογισμένοι - Προπληρωμένοι φόροι - τέλη (Λ/58.63)  

 

2.2.607 Λογαριασμός 63 «Φόροι - Τέλη»

1. Στους υπολογαριασμούς του 63 καταχωρούνται όλοι οι φόροι και τα τέλη που βαρύνουν την οικονομική μονάδα, εκτός από τους φόρους της επόμενης περίπτ. 2.

2. Στους υπολογαριασμούς του 63 δεν καταχωρούνται οι ακόλουθοι φόροι - τέλη:  - Ο φόρος εισοδήματος, ο οποίος, σαν αφαιρετικό στοιχείο των ετήσιων κερδών, καταχωρείται στο λογαριασμό 88.08 «φόρος εισοδήματος και εισφορά ΟΓΑ», εκτός αν πρόκειται για ποσά παρακρατημένου και μη συμψηφιζόμενου φόρου εισοδήματος, τα οποία καταχωρούνται στο λογαριασμό 63.00 «φόρος εισοδήματος μη συμψηφιζόμενος».

- Οι φόροι προηγούμενων χρήσεων, οι οποίοι καταχωρούνται στο λογαριασμό 82.00 «έξοδα προηγούμενων χρήσεων».

- Οι φορολογικές ποινές και τα πρόστιμα, που καταχωρούνται στο λογαριασμό 81.00.00 «φορολογικά πρόστιμα και προσαυξήσεις».

- Το χαρτόσημο μισθοδοσίας και ο φόρος α.ν. 843/48, που καταχωρούνται στους οικείους υπολογαριασμούς του 60.

- Το χαρτόσημο συμβάσεων, δανείων και χρηματοδοτήσεων, που καταχωρείται στο λογαριασμό 65.07 «χαρτόσημο συμβάσεων, δανείων και χρηματοδοτήσεων».

- Οι δασμοί και γενικά οι φόροι επί των αγορών, οι οποίοι καταχωρούνται στους λογαριασμούς αποθεμάτων της ομάδας 2, όταν αφορούν αγορές αποθεμάτων, και στους λογαριασμούς της ομάδας 1, όταν αφορούν αγορές πάγιων στοιχείων.

64   ΔΙΑΦΟΡΑ ΕΞΟΔΑ

        64.00   Έξοδα μεταφορών

                       64.00.00   Έξοδα κινήσεως (καύσιμα - λιπαντικά - διόδια) ιδιοκτητών
                                         μεταφορικών μέσων

                                 01     Έξοδα μεταφοράς προσωπικού με μεταφορικά μέσα τρίτων

                                 02   Έξοδα μεταφοράς υλικών - αγαθών αγορών με μεταφορικά
                                         μέσα τρίτων

                                 03   Έξοδα μεταφοράς υλικών - αγαθών πωλήσεων με μεταφορικά
                                 μέσα τρίτων

                                 04   Έξοδα διακινήσεων (εσωτερικών) υλικών - αγαθών με
                                 μεταφορικά μέσα τρίτων

                                 05

                       ..............

                       64.00.99

        64.01   Έξοδα ταξειδίων

                       64.01.00   Έξοδα ταξειδίων εσωτερικού

                       64.01.01   Έξοδα ταξειδίων εξωτερικού

                       ..............

                       64.01.99

        64.02   Έξοδα προβολής και διαφημίσεως

                       64.02.00   Διαφημίσεις από τον τύπο

                                 01   Διαφημίσεις από το ραδιόφωνο - τηλεόραση

                                 02   Διαφημίσεις από τον κινηματογράφο

                                 03   Διαφημίσεις από τα λοιπά μέσα ενημερώσεως

                                 04   Έξοδα λειτουργίας φωτεινών επιγραφών

                                 05   Έξοδα συνεδρίων - δεξιώσεων και άλλων παρεμφερών
                                         εκδηλώσεων

                                 06    Έξοδα υποδοχής και φιλοξενείας

                                 07   Έξοδα προβολής δια λοιπών μεθόδων (π.χ. χρηματοδότηση
                                         αθλητικών εκδηλώσεων ή αγώνων Rally)

                                 08   Έξοδα λόγω εγγυήσεως πωλήσεων (συμβατικές υποχρεώσεις)

                                 09   Έξοδα αποστολής δειγμάτων

                                 10   Αξία χορηγούμενων δειγμάτων (Γνωμ. 44/1129/1989)

                       ..............

                       64.02.99   Διάφορα έξοδα προβολής και διαφημίσεως

        64.03   Έξοδα εκθέσεων - επιδείξεων

                       64.03.00   Έξοδα εκθέσεων εσωτερικού

                                 01   Έξοδα εκθέσεων εξωτερικού

                                 02   Έξοδα επιδείξεων

                       ..............

                       64.03.99

        64.04   Ειδικά έξοδα προωθήσεως εξαγωγών

                       64.04.00    Ειδικά έξοδα εξαγωγών «δίχως δικαιολογητικά»
                                         (άρθρο 35 παρ. 3-5 Ν.Δ. 3323/1955, που εντάχθηκαν στο άρθρο
                                         31 παρ. 2-5 Ν. 2238/1994)

                                 01   ...................................................

                       ..............

                       64.04.99

        64.05   Συνδρομές - Εισφορές

                       64.05.00   Συνδρομές σε περιοδικά και εφημερίδες

                                 01   Συνδρομές - Εισφορές σε επαγγελματικές οργανώσεις

                                 02   Δικαιώματα Χρηματιστηρίου διαπραγματεύσεως τίτλων

                       ..............

                       64.05.99

        64.06   Δωρεές - Επιχορηγήσεις

                       64.06.00   Δωρεές για κοινωφελείς σκοπούς

                                 01    Επιχορηγήσεις για κοινωφελείς σκοπούς

                                 02   Αξία δωρεών αποθεμάτων για κοινωφελείς σκοπούς
                                         (Γνωμ. 44/1129/1989)

                       ..............

                       64.06.98   Λοιπές δωρεές

                       64.06.99   Λοιπές επιχορηγήσεις

        64.07   Έντυπα και γραφική ύλη

                       64.07.00   Έντυπα

                                 01   Υλικά πολλαπλών εκτυπώσεων

                                 02   Έξοδα πολλαπλών εκτυπώσεων

                                 03   Γραφική ύλη και λοιπά υλικά γραφείων

                       ..............

                       64.07.99

        64.08   Υλικά άμεσης αναλώσεως

                       64.08.00   Καύσιμα και λοιπά υλικά θερμάνσεως

                                 01   Υλικά καθαριότητας

                                 02   Υλικά φαρμακείου

                       ..............

                       64.08.99   Λοιπά υλικά άμεσης αναλώσεως

        64.09   Έξοδα δημοσιεύσεων

                       64.09.00   Έξοδα δημοσιεύσεως ισολογισμών και προσκλήσεων

                                 01   Έξοδα δημοσιεύσεως αγγελιών και ανακοινώσεων

                       ..............

                       64.09.99   Έξοδα λοιπών δημοσιεύσεων

        64.10   Έξοδα συμμετοχών και χρεογράφων

                       64.10.00   Προμήθειες και λοιπά έξοδα αγοράς συμμετοχών \&
                                         χρεογράφων

                       64.10.01   Προμήθειες και λοιπά έξοδα πωλήσεως συμμετοχών \&
                                         χρεογράφων

                       ..............

                       64.10.99   Λοιπά έξοδα συμμετοχών και χρεογράφων

        64.11    Διαφορές αποτιμήσεως συμμετοχών και χρεογράφων
                    (Από τη χρήση 1994 δεν χρησιμοποιείται ο 64.11, αλλά ο 68.01
                    «Προβλέψεις για υποτιμήσεις συμμετοχών και χρεογράφων».

        64.12    Διαφορές (ζημίες) από πώληση συμμετοχών και χρεογράφων

                       64.12.00    Διαφορές (ζημίες) από πώληση συμμετοχών

                                 01    Διαφορές (ζημίες) από πώληση σε συμμετοχών λοιπές πλην
                                         Α.Ε. επιχ/σεις

                                 02    Διαφορές (ζημίες) από πώληση χρεογράφων

                       ..............

                       64.12.99

        64.13

        .........

        64.91    Διαφορές (ζημίες) από πράξεις hedging (Γνωμ. 268/2272/1986)

        64.98    Διάφορα έξοδα

                       64.98.00    Κοινόχρηστες δαπάνες

                                 01    Έξοδα λειτουργίας Οργάνων Διοικήσεως (π.χ. έξοδα Γ.Σ.,
                                         Συμβουλίων ή Επιτροπών)

                                 02    Δικαστικά και έξοδα εξώδικων ενεργειών

                                 03    Έξοδα συμβολαιογράφων

                                 04    Έξοδα λοιπών ελεύθερων επαγγελματιών

                                 05    Έξοδα διαφόρων τρίτων

                       ..............

                       64.98.99

64.99 Προϋπολογισμένα - Προπληρωμένα διάφορα έξοδα (Λ/58.64)

2.2.608 Λογαριασμός 64 «Διάφορα έξοδα»

1. Στους υπολογαριασμούς του 64 καταχωρούνται όλα τα κατ' είδος οργανικά έξοδα που δεν καταχωρούνται σε οποιοδήποτε άλλο λογαριασμό της ομάδας 6.

2. Σχετικά με το περιεχόμενο των υπολογαριασμών του 64.00 «έξοδα μεταφορών» διευκρινίζονται τα ακόλουθα:

- Στο λογαριασμό 64.00.00 καταχωρούνται τα έξοδα κινήσεως των μεταφορικών μέσων της οικονομικής μονάδας, όταν τα μέσα αυτά ανήκουν κατά κυριότητα σ' αυτή.

- Στο λογαριασμό 64.00.01 καταχωρούνται τα έξοδα μεταφοράς του προσωπικού της οικονομικής μονάδας, όταν η μεταφορά γίνεται με μεταφορικά μέσα που ανήκουν σε τρίτους, οι οποίοι αναλαμβάνουν το έργο αυτό. Αν η μεταφορά γίνεται με μισθωμένα μεταφορικά μέσα, τα ενοίκια που καταβάλλονται ή λογίζονται καταχωρούνται στο λογαριασμό 62.04.03 «ενοίκια μεταφορικών μέσων».

- Στο λογαριασμό 64.00.02 καταχωρούνται τα έξοδα μεταφοράς των διάφορων υλικών - αγαθών που αγοράζονται από την οικονομική μονάδα, όταν η μεταφορά γίνεται με μεταφορικά μέσα που ανήκουν σε τρίτους, σύμφωνα με όσα ισχύουν στην παραπάνω δεύτερη υποπαράγραφο της περιπτώσεως 2.

Σημείωση: Πρόκειται για ειδικά έξοδα αγοράς, τα οποία είναι διαμορφωτικά της αξίας κτήσεως των αγοραζόμενων (βλ. και άρθρο 43 κωδ. Ν. 2190/1920) και καταχωρούνται στους οικείους υπολογαριασμούς του 32 «παραγγελίες εξωτερικού» ή στους λογαριασμούς της Ομάδας 2. Εδώ, στο λογαριασμό 64.00.02, καταχωρούνται, κατ' εξαίρεση, τα μεταφορικά εκείνα που (για διάφορους λόγους) λογιστικοποιούνται καθυστερημένα και τα αντίστοιχα αγαθά έχουν χρησιμοποιηθεί (οι πρώτες ύλες βιομηχανοποιήθηκαν \& έχει υπολογισθεί το μηνιαίο κόστος παραγωγής, τα εμπορεύματα πωλήθηκαν και υπολογίστηκαν τα βραχύχρονα αποτελέσματα κ.λπ.), οπότε, κατ' ανάγκη, βαρύνουν τον λογαριασμό 97. (ΑΥΟ 1116200/885/0015/ΠΟΛ. 1282/30-10-96)

- Στο λογαριασμό 64.00.03 καταχωρούνται τα έξοδα μεταφοράς των διάφορων υλικών - αγαθών που πωλούνται από την οικονομική μονάδα, όταν η μεταφορά γίνεται με μεταφορικά μέσα που ανήκουν σε τρίτους, σύμφωνα με όσα ισχύουν στην παραπάνω δεύτερη υποπαράγραφο της περιπτώσεως 2.

- Στο λογαριασμό 64.00.04 καταχωρούνται τα έξοδα εσωτερικής διακινήσεως των υλικών - αγαθών της οικονομικής μονάδας, όταν η διακίνηση αυτή, από τη μία εγκατάσταση στην άλλη, γίνεται με μεταφορικά μέσα που ανήκουν σε τρίτους, σύμφωνα με όσα ισχύουν στην παραπάνω δεύτερη υποπαράγραφο της περιπτώσεως 2.

3. Σχετικά με το περιεχόμενο των υπολογαριασμών του 64.01 «έξοδα ταξιδίων» διευκρινίζεται ότι, στις περιπτώσεις που οι λογαριασμοί εξόδων ταξιδίων περιλαμβάνουν και αμοιβές πέρα από εκείνες που καλύπτουν τα έξοδα, π.χ.  κινήσεως, διατροφής ή διανυκτερεύσεως, οι επί πλέον αυτές αμοιβές διαχωρίζονται και καταχωρούνται στο λογαριασμό 60.00.11 «αμοιβές εκτός έδρας», όταν πρόκειται για έμμισθο προσωπικό ή μέλη της διοικήσεως, ή στο λογαριασμό 60.01.11 «αμοιβές εκτός έδρας», όταν πρόκειται για ημερομίσθιο προσωπικό.

4. Σχετικά με το περιεχόμενο των υπολογαριασμών του 64.02 «έξοδα προβολής και διαφημίσεως» ορίζονται τα ακόλουθα:

- Στο λογαριασμό 64.02.04 «έξοδα λειτουργίας φωτεινών επιγραφών» καταχωρούνται έξοδα συντηρήσεως, επισκευών και άλλα των φωτεινών επιγραφών της οικονομικής μονάδας. Σε περίπτωση μισθώσεως δικαιωμάτων εγκαταστάσεως και λειτουργίας φωτεινών επιγραφών, τα ενοίκια καταχωρούνται στο λογαριασμό 62.04.08 «ενοίκια φωτεινών επιγραφών». Τα αρχικά έξοδα κατασκευής και εγκαταστάσεως των φωτεινών επιγραφών, όταν πρόκειται για περιπτώσεις αποσβέσεώς τους σε περισσότερες από μία χρήσεις, καταχωρούνται στον οικείο υπολογαριασμό του 14.09 «λοιπός εξοπλισμός».

- Στο λογαριασμό 64.02.08 «έξοδα λόγω εγγυήσεως πωλήσεων» καταχωρούνται τα έξοδα που καταβάλλονται από την οικονομική μονάδα σε πελάτης της με βάση τις εγγυήσεις που δίνονται σ' αυτούς για τα πωλημένα αγαθά. Στον ίδιο λογαριασμό καταχωρούνται οι διαφορές από την ενεργοποίηση των εγγυήσεων προμηθευτών, δηλαδή οι διαφορές μεταξύ κόστους αποκαταστάσεως, από την οικονομική μονάδα, ζημιών πελατών της και ποσών που καταβάλλουν οι προμηθευτές της για συμμετοχή στις ζημίες αυτές.

- Στο λογαριασμό 64.02.09 «έξοδα αποστολής δειγμάτων» καταχωρούνται τα έξοδα αποστολής, σε πελάτες ή υποψήφιους πελάτες, δειγμάτων από τα προς πώληση αποθέματα.

- Στο λογαριασμό 64.02.99 «διάφορα έξοδα προβολής και διαφημίσεως» καταχωρούνται όλα τα παρόμοιας φύσεως έξοδα που δεν εντάσσονται σε οποιαδήποτε κατηγορία των λογαριασμών 64.02.00-64.02.98.

5. Σχετικά με το περιεχόμενο των υπολογαριασμών του 64.03 «έξοδα εκθέσεων - επιδείξεων» ορίζεται ότι στους υπολογαριασμούς αυτούς καταχωρούνται τα κάθε είδους έξοδα συμμετοχής σε εκθέσεις εμπορικές και άλλες, όπου εκθέτονται τα αγαθά που προορίζονται για πώληση (λογ. 64.03.00 και 64.03.01), και τα κάθε είδους έξοδα που πραγματοποιούνται κατά την επίδειξη με οποιοδήποτε τρόπο των αγαθών που προορίζονται για πώληση (λογ. 64.03.02).

6. Σχετικά με το περιεχόμενο των υπολογαριασμών του 64.04 «ειδικά έξοδα προωθήσεως εξαγωγών» ορίζεται ότι στους υπολογαριασμούς αυτούς καταχωρούνται τα ειδικά εκείνα ποσά που καταβάλλονται από την οικονομική μονάδα, σύμφωνα με ειδικές διατάξεις της νομοθεσίας που ισχύει κάθε φορά (όπως το ν.δ. 4231/1962).

Σημείωση: Σχετικά με το περιεχόμενο του υπολογαριασμού 64.06.02 Η αξία των δωριζομένων κινητών αγαθών δεν εκπίπτει από τα ακαθάριστα έσοδα κατά τον υπολογισμό των φορολογητέων κερδών (άρθρο 31 παρ. 1 περ. α' υποπερ. γγ' ν. 2238/94).  (ΑΥΟ 1116200/885/0015/ΠΟΛ. 1282/30-10-96)

7. Σχετικά με το περιεχόμενο των υπολογαριασμών του 64.07 «έντυπα και γραφική ύλη» διευκρινίζεται ότι στους υπολογαριασμούς αυτούς καταχωρούνται τα έξοδα που πραγματοποιούνται από την οικονομική μονάδα για εκτύπωση και αγορά εντύπων (λογ. 64.07.00), για υλικά (π.χ. χαρτί) που προορίζονται για πολλαπλές εκτυπώσεις, είτε στις εγκαταστάσεις της οικονομικής μονάδας, είτε σε τρίτους (λογ. 64.07.01), για έξοδα πολλαπλών εκτυπώσεων όταν αυτές γίνονται από τρίτους με υλικά που προέρχονται από την οικονομική μονάδα (λογ. 64.07.02) και για γραφική ύλη και λοιπά υλικά γραφείων (λογ. 64.07.03).

8. Σχετικά με το περιεχόμενο των υπολογαριασμών του 64.08 «υλικά άμεσης αναλώσεως» ορίζεται ότι στους υπολογαριασμούς αυτούς καταχωρούνται τα διάφορα αναλώσιμα υλικά τα οποία κατά την αγορά τους δεν εισάγονται στις αποθήκες των λογαριασμών αποθεμάτων της ομάδας 2.

9. Σχετικά με το περιεχόμενο των υπολογαριασμών του 64.10 «έξοδα συμμετοχών και χρεογράφων» ορίζεται ότι στους υπολογαριασμούς αυτούς καταχωρούνται τα κάθε είδους και μορφής έξοδα που πραγματοποιούνται από την οικονομική μονάδα για την αγορά, πώληση και, γενικά, διαχείριση των συμμετοχών της παρ. 2.2.112 και των χρεογράφων της παρ. 2.2.305.

10. Σχετικά με το περιεχόμενο του λογαριασμού 64.11 «διαφορές αποτιμήσεως συμμετοχών και χρεογράφων» ορίζεται ότι στο λογαριασμό αυτό καταχωρούνται οι διαφορές ανάμεσα στη συνολική τιμή κτήσεως των συμμετοχών και χρεογράφων και στη συνολική τρέχουσα τιμή τους, σύμφωνα με όσα καθορίζονται στην περίπτ. 6-β της παρ. 2.2.112.

11. Σχετικά με το περιεχόμενο του λογαριασμού 64.12 «διαφορές από πώληση συμμετοχών και χρεογράφων» ορίζεται ότι στο λογαριασμό αυτό καταχωρούνται οι ζημίες που πραγματοποιούνται από την πώληση συμμετοχών και χρεογράφων, σύμφωνα με όσα καθορίζονται στην περίπτ. 5 της παρ. 2.2.112.

12. Σχετικά με το περιεχόμενο του λογαριασμού 64.98 «διάφορα έξοδα» ορίζονται τα ακόλουθα:

- Στο λογαριασμό αυτό καταχωρούνται τα έξοδα που δεν είναι δυνατό να ενταχθούν σε οποιοδήποτε άλλο δευτεροβάθμιο λογαριασμό του 64.

- Στους υπολογαριασμούς 64.98.03, 64.98.04 και 64.98.05 καταχωρούνται έξοδα τα οποία καταβάλλονται σε συμβολαιογράφους, σε λοιπούς ελεύθερους επαγγελματίες και σε διάφορους τρίτους, όταν για τα έξοδα αυτά δε γίνεται παρακράτηση φόρου εισοδήματος. Στις περιπτώσεις που γίνεται παρακράτηση φόρου εισοδήματος, τα έξοδα αυτά καταχωρούνται, μαζί με τις αμοιβές τρίτων, στους οικείους υπολογαριασμούς του 61

65   ΤΟΚΟΙ ΚΑΙ ΣΥΝΑΦΗ ΕΞΟΔΑ

        65.00   Τόκοι και έξοδα ομολογιακών δανείων

                    65.00.00   Τόκοι και έξοδα δανείων σε Δρχ. μη μετατρέψιμων σε μετοχές

                               01   Τόκοι και έξοδα δανείων σε Δρχ. μετατρέψιμων σε μετοχές

                               02   Τόκοι και έξοδα δανείων με ρήτρα Ξ.Ν. μη μετατρέψιμων
                                       σε μετοχές

                               03   Τόκοι και έξοδα δανείων με ρήτρα Ξ.Ν. μετατρέψιμων
                                       σε μετοχές

                               04   Τόκοι και έξοδα δανείων σε Ξ.Ν. μη μετατρέψιμων σε μετοχές

                               05   Τόκοι και έξοδα δανείων σε Ξ.Ν. μετατρέψιμων σε μετοχές

                     ..............

                     65.00.99

        65.01   Τόκοι και έξοδα λοιπών μακροπρόθεσμων υποχρεώσεων

                     65.01.00   Τόκοι και έξοδα Τραπεζικών μακροπρόθεσμων
                                       υποχρεώσεων σε Δρχ.

                               01   Τόκοι και έξοδα Τραπεζικών μακροπρόθεσμων
                                       υποχρεώσεων σε Δρχ. με ρήτρα Ξ.Ν.

                               02   Τόκοι και έξοδα Τραπεζικών μακροπρόθεσμων
                                       υποχρεώσεων σε Ξ.Ν.

                               03   Τόκοι και έξοδα μακροπρόθεσμων υποχρεώσεων
                                       προς Ταμιευτήρια

                               04   Τόκοι και έξοδα μακροπρόθεσμων υποχρεώσεων
                                       προς συνδεμένες επιχειρήσεις σε Δρχ.

                               05   Τόκοι και έξοδα μακροπρόθεσμων υποχρεώσεων προς
                                       συνδεμένες επιχειρήσεις σε Ξ.Ν.

                               06   Τόκοι και έξοδα μακροπρόθεσμων υποχρεώσεων προς
                                       εταίρους και διοικούντες

                               07   Τόκοι και έξοδα μακροπρόθεσμων γραμματίων
                                       πληρωτέων σε Δρχ.

                               08   Τόκοι και έξοδα μακροπρόθεσμων γραμματίων
                                       πληρωτέων σε Ξ.Ν.

                               09   Τόκοι και έξοδα μακροπρόθεσμων υποχρεώσεων
                                       προς το Δημόσιο από φόρους

                               10   Τόκοι και έξοδα μακροπρόθεσμων υποχρεώσεων
                                       προς ασφαλιστικά ταμεία

                     .............

                    65.01.98    Τόκοι και έξοδα λοιπών μακροπρόθεσμων
                                       υποχρεώσεων σε Δρχ.

                    65.01.99   Τόκοι και έξοδα λοιπών μακροπρόθεσμων
                                       υποχρεώσεων σε Ξ.Ν.

        65.02   Προεξοφλητικοί τόκοι και έξοδα Τραπεζών

        65.03   Τόκοι και έξοδα χρηματοδοτήσεων Τραπεζών
                     εγγυημένων με αξιόγραφα

        65.04   Τόκοι και έξοδα βραχυπρόθεσμων Τραπεζικών χορηγήσεων
                     για εξαγωγές

        65.05   Τόκοι και έξοδα λοιπών βραχυπρόθεσμων Τραπεζικών
                     χρηματοδοτήσεων

        65.06   Τόκοι και έξοδα λοιπών βραχυπρόθεσμων υποχρεώσεων

        65.07   Ειδικός φόρος τραπεζικών εργασιών και χαρτόσημο συμβάσεων
                     δανείων και χρηματοδοτήσεων

        65.08   Έξοδα ασφαλειών (π.χ. εμπράγματων) δανείων και χρηματοδοτήσεων

        65.09   Παροχές σε ομολογιούχους επί πλέον τόκου

        65.10   Προμήθειες εγγυητικών επιστολών

        .........

        65.90   Τόκοι και έξοδα εισπράξεως απαιτήσεων με σύμβαση Factoring
                     (Γνωμ. 216/2176/1994)

        .........

        65.98   Λοιπά συναφή με τις χρηματοδοτήσεις έξοδα

                     65.98.00   Εισπρακτικά γραμματίων εισπρακτέων

                     ...............

                    65.98.99   Διάφορα έξοδα τραπεζών

        65.99    Προϋπολογισμένοι - Προπληρωμένοι τόκοι και συναφή
                     έξοδα (Λ/58.65)

2.2.609 Λογαριασμός 65 «Τόκοι και συναφή έξοδα»

1. Στο λογαριασμό 65 καταχωρούνται οι τόκοι και τα συναφή με αυτούς έξοδα που αναφέρονται στο χρηματοοικονομικό κύκλωμα της οικονομικής μονάδας. Στους λογαριασμούς τόκων (65.00-65.06) καταχωρούνται, εκτός από τους τόκους, και οι προμήθειες που συνυπολογίζονται με αυτούς, καθώς και τα κάθε είδους παρεπόμενα με αυτούς έξοδα (π.χ. Φ.Κ.Ε., χαρτόσημο τόκων).

2. Στο λογαριασμό 65.09 «παροχές σε ομολογιούχους επί πλέον τόκου» καταχωρούνται οι τυχόν πρόσθετες παροχές που δίνονται σε ομολογιούχους της οικονομικής μονάδας επιπλέον του τόκου των τοκομεριδίων, σύμφωνα και με όσα καθορίζονται στην περίπτ. 13 της παρ. 2.2.504.

3. Στο λογαριασμό 65.98 «λοιπά συναφή με τις χρηματοδοτήσεις έξοδα» καταχωρούνται όλα τα έξοδα που αφορούν το χρηματοδοτικό κύκλωμα και δεν εντάσσονται σε οποιοδήποτε από τους λοιπούς δευτεροβάθμιους λογαριασμούς του 65. Εξαίρεση αποτελούν τα έξοδα που έχουν σχέση με τις συμμετοχές και τα χρεόγραφα, τα οποία καταχωρούνται στον υπολογαριασμό 64.10, σύμφωνα με όσα καθορίζονται στην περίπτ. 9 της παρ. 2.2.608.

66   ΑΠΟΣΒΕΣΕΙΣ ΠΑΓΙΩΝ ΣΤΟΙΧΕΙΩΝ ΕΝΣΩΜΑΤΩΜΕΝΕΣ
        ΣΤΟ ΛΕΙΤΟΥΡΓΙΚΟ ΚΟΣΤΟΣ

        66.00   Αποσβέσεις εδαφικών εκτάσεων

                     66.00.00

                               01   Αποσβέσεις Ορυχείων

                               02   Αποσβέσεις Μεταλλείων

                               03   Αποσβέσεις Λατομείων

                               04

                               05   Αποσβέσεις Φυτειών

                               06   Αποσβέσεις Δασών

                               07   .......................................

                               ....

                               11   Αποσβέσεις Ορυχείων εκτός εκμεταλλεύσεως

                               12   Αποσβέσεις Μεταλλείων εκτός εκμεταλλεύσεως

                               13   Αποσβέσεις Λατομείων εκτός εκμεταλλεύσεως

                               14

                               15   Αποσβέσεις Φυτειών εκτός εκμεταλλεύσεως

                               16   Αποσβέσεις Δασών εκτός εκμεταλλεύσεως

                               17

                     ..............

                     66.00.99

        66.01   Αποσβέσεις κτιρίων - εγκαταστάσεων κτιρίων - τεχνικών έργων

                     66.01.00   Αποσβέσεις κτιρίων - εγκαταστάσεων κτιρίων

                               01   Αποσβέσεις τεχνικών έργων εξυπηρετήσεως μεταφορών

                               02   Αποσβέσεις λοιπών τεχνικών έργων

                               03   Αποσβέσεις διαμορφώσεως γηπέδων

                               04   ..........................................

                               07   Αποσβέσεις κτιρίων - εγκαταστάσεων κτιρίων
                                       σε ακίνητα τρίτων

                               08   Αποσβέσεις τεχνικών έργων εξυπηρετήσεων
                                       μεταφορών σε ακίνητα τρίτων

                               09   Αποσβέσεις λοιπών τεχνικών έργων σε ακίνητα τρίτων

                               10   Αποσβέσεις διαμορφώσεως γηπέδων τρίτων

                               11   .............................................

                               14   Αποσβέσεις κτιρίων - εγκαταστάσεων κτιρίων
                                       εκτός εκμεταλλεύσεως

                               15   Αποσβέσεις τεχνικών έργων εξυπηρετήσεως μεταφορών εκτός
                                       εκμεταλλεύσεως

                               16   Αποσβέσεις λοιπών τεχνικών έργων εκτός εκμεταλλεύσεως

                               17   Αποσβέσεις διαμορφώσεων γηπέδων εκτός εκμεταλλεύσεως

                               18   ...............................................

                               21   Αποσβέσεις κτιρίων - εγκαταστάσεων κτιρίων σε ακίνητα
                                       τρίτων εκτός εκμεταλλεύσεως

                               22   Αποσβέσεις τεχνικών έργων εξυπηρετήσεως μεταφορών σε

                                       ακίνητα τρίτων εκτός εκμεταλλεύσεως

                               23   Αποσβέσεις λοιπών τεχνικών έργων σε ακίνητα τρίτων εκτός
                                       εκμεταλλεύσεως

                               24   Αποσβέσεις διαμορφώσεων γηπέδων τρίτων
                                       εκτός εκμεταλλεύσεως

                     .............

                     66.01.99

        66.02   Αποσβέσεις μηχανημάτων - τεχνικών εγκαταστάσεων - λοιπού
                     μηχανολογικού εξοπλισμού

                     66.02.00   Αποσβέσεις μηχανημάτων

                               01   Αποσβέσεις τεχνικών εγκαταστάσεων

                               02   Αποσβέσεις φορητών μηχανημάτων «χειρός»

                               03   Αποσβέσεις εργαλείων

                               04   Αποσβέσεις καλουπιών - ιδιοσυσκευών

                               05   Αποσβέσεις μηχανολογικών οργάνων

                               06   Αποσβέσεις λοιπού μηχανολογικού εξοπλισμού

                               07   Αποσβέσεις μηχανημάτων σε ακίνητα τρίτων

                               08   Αποσβέσεις τεχνικών εγκαταστάσεων σε ακίνητα τρίτων

                               09   Αποσβέσεις λοιπού μηχανολογικού εξοπλισμού
                                       σε ακίνητα τρίτων

                               10   Αποσβέσεις μηχανημάτων εκτός εκμεταλλεύσεως

                               11   Αποσβέσεις τεχνικών εγκαταστάσεων εκτός εκμεταλλεύσεως

                               12   Αποσβέσεις φορητών μηχανημάτων «χειρός» εκτός
                                       εκμεταλλεύσεως

                               13   Αποσβέσεις εργαλείων εκτός εκμεταλλεύσεως

                               14   Αποσβέσεις καλουπιών - ιδιοσυσκευών εκτός εκμεταλλεύσεως

                               15   Αποσβέσεις μηχανολογικών οργάνων εκτός εκμεταλλεύσεως

                               16   Αποσβέσεις λοιπού μηχανολογικού εξοπλισμού εκτός
                                       εκμεταλλεύσεως

                               17   Αποσβέσεις μηχανημάτων σε ακίνητα τρίτων
                                       εκτός εκμεταλλεύσεως

                               18   Αποσβέσεις τεχνικών εγκαταστάσεων σε ακίνητα τρίτων
                                       εκτός εκμεταλλεύσεως

                               19   Αποσβέσεις λοιπού μηχανολογικού εξοπλισμού σε ακίνητα
                                       τρίτων εκτός εκμεταλλεύσεως

                     ..............

                     66.02.99

        66.03   Αποσβέσεις μεταφορικών μέσων

                     66.03.00   Αποσβέσεις αυτοκινήτων λεωφορείων

                               01   Αποσβέσεις λοιπών επιβατικών αυτοκινήτων

                               02    Αποσβέσεις αυτοκινήτων φορτηγών - Ρυμουλκών - Ειδικής
                                       χρήσεως

                               03    Αποσβέσεις σιδηροδρομικών οχημάτων

                               04    Αποσβέσεις πλωτών μέσων

                               05    Αποσβέσεις εναέριων μέσων

                               06    Αποσβέσεις μέσων εσωτερικών μεταφορών

                               07    ........................................

                               09    Αποσβέσεις λοιπών μέσων μεταφοράς

                               10    Αποσβέσεις αυτοκινήτων λεωφορείων εκτός εκμεταλλεύσεως

                               11    Αποσβέσεις λοιπών επιβατικών αυτοκινήτων εκτός
                                       εκμεταλλεύσεως

                               12    Αποσβέσεις αυτοκινήτων φορτηγών - Ρυμουλκών - Ειδικής
                                       χρήσεως εκτός εκμεταλλεύσεως

                               13    Αποσβέσεις σιδηροδρομικών οχημάτων εκτός εκμεταλλεύσεως

                               14    Αποσβέσεις πλωτών μέσων εκτός εκμεταλλεύσεως

                               15    Αποσβέσεις εναέριων μέσων εκτός εκμεταλλεύσεως

                               16    Αποσβέσεις μέσων εσωτερικών μεταφορών
                                       εκτός εκμεταλλεύσεως

                               17    ............................................

                               19    Αποσβέσεις λοιπών μέσων μεταφοράς εκτός εκμεταλλεύσεως

                     .............

                     66.03.99

        66.04    Αποσβέσεις επίπλων και λοιπού εξοπλισμού

                     66.04.00    Αποσβέσεις επίπλων

                               01    Αποσβέσεις σκευών

                               02    Αποσβέσεις μηχανών γραφείων

                               03    Αποσβέσεις ηλεκτρονικών υπολογιστών και ηλεκτρονικών
                                       συγκροτημάτων

                               04    Αποσβέσεις μέσων αποθηκεύσεως και μεταφοράς

                               05    Αποσβέσεις επιστημονικών οργάνων

                               06    Αποσβέσεις ζώων για πάγια εκμετάλλευση

                               07    ........................................

                               08    Αποσβέσεις εξοπλισμού τηλεπικοινωνιών

                               09    Αποσβέσεις λοιπού εξοπλισμού

                               10    Αποσβέσεις επίπλων εκτός εκμεταλλεύσεως

                               11    Αποσβέσεις σκευών εκτός εκμεταλλεύσεως

                               12    Αποσβέσεις μηχανών γραφείων εκτός εκμεταλλεύσεως

                               13    Αποσβέσεις ηλεκτρονικών υπολογιστών και ηλεκτρονικών
                                       συγκροτημάτων εκτός εκμεταλλεύσεως

                               14    Αποσβέσεις μέσων αποθηκεύσεως και μεταφοράς
                                       εκτός εκμεταλλεύσεως

                               15    Αποσβέσεις επιστημονικών οργάνων εκτός εκμεταλλεύσεως

                               16    Αποσβέσεις ζώων για πάγια εκμετάλλευση εκτός
                                       εκμεταλλεύσεως

                               17    ........................................

                               18    Αποσβέσεις εξοπλισμού τηλεπικοινωνιών εκτός
                                       εκμεταλλεύσεως

                               19    Αποσβέσεις λοιπού εξοπλισμού εκτός εκμεταλλεύσεως

                     ..............

                     66.04.99

        66.05    Αποσβέσεις ασώματων ακινητοποιήσεων και εξόδων
                      πολυετούς αποσβέσεως

                     66.05.00    Αποσβέσεις υπεραξίας επιχειρήσεως

                               01    Αποσβέσεις δικαιωμάτων βιομηχανικής ιδιοκτησίας

                               02    Αποσβέσεις δικαιωμάτων εκμεταλλεύσεως ορυχείων -
                                       μεταλλείων - λατομείων

                               03    Αποσβέσεις λοιπών παραχωρήσεων

                               04    Αποσβέσεις δικαιωμάτων χρήσεως ενσώματων πάγιων
                                       στοιχείων

                               05    Αποσβέσεις λοιπών δικαιωμάτων

                     ..............

                     66.05.10    Αποσβέσεις εξόδων ιδρύσεως και πρώτης εγκαταστάσεως

                               11    Αποσβέσεις εξόδων ερευνών ορυχείων - μεταλλείων -
                                       λατομείων

                               12    Αποσβέσεις εξόδων λοιπών ερευνών

                               13    Αποσβέσεις εξόδων αυξήσεως κεφαλαίου και εκδόσεως
                                       ομολογιακών δανείων

                               14    Αποσβέσεις εξόδων κτήσεως ακινητοποιήσεων

                               15    ........................................

                               16    Αποσβέσεις διαφορών εκδόσεως και εξοφλήσεως ομολογιών

                     66.05.17    Αποσβέσεις εξόδων αναδιοργανώσεως

                     66.05.18    Αποσβέσεις τόκων δανείων κατασκευαστικής περιόδου

                               19    Αποσβέσεις λοιπών εξόδων πολυετούς αποσβέσεως

                     ..............

                     66.05.99

        66.06

        ........

        66.99 Προϋπολογισμένες αποσβέσεις εκμεταλλεύσεως (Λ/58.66)

 

2.2.610 Λογαριασμός 66 «Αποσβέσεις πάγιων στοιχείων ενσωματωμένες στο λειτουργικό κόστος»

1. Στο λογαριασμό 66 καταχωρούνται οι αποσβέσεις στοιχείων του πάγιου ενεργητικού που ενσωματώνονται στο λειτουργικό κόστος της οικονομικής μονάδας, δηλαδή στο λογαριασμό αυτό καταχωρούνται οι τακτικές αποσβέσεις που προβλέπονται από τη νομοθεσία που ισχύει κάθε φορά. Οι αποσβέσεις αυτές υπολογίζονται σύμφωνα με όσα καθορίζονται στην παρ. 2.2.102/ΙΙ.

2. Οι δευτεροβάθμιοι λογαριασμοί του 66 χρεώνονται με πίστωση των αντίστοιχων λογαριασμών της ομάδας 1 (10.99, 11.99, 12.99, 13.99, 14.99 και 16.99).

3. Οι αποσβέσεις που δεν ενσωματώνονται στο λειτουργικό κόστος της οικονομικής μονάδας (πρόσθετες) καταχωρούνται στους αντίστοιχους δευτεροβάθμιους λογαριασμούς του 85 «μη ενσωματωμένες στο λειτουργικό κόστος αποσβέσεις παγίων», με πίστωση των αυτών αντίστοιχων λογαριασμών της ομάδας 1 (10.99, 11.99, 12.99, 13.99, 14.99 και 16.99).

Α.Υ.Ο. 1116200/885/0015/ΠΟΛ. 1282/30-10-1996

Για τον κοστολογικό χειρισμό του υπολογαριασμού 66.05 «αποσβέσεις ασώματων ακινητοποιήσεων και εξόδων πολυετούς αποσβέσεως» σημειώνουμε τα ακόλουθα:

α) Για τα περισσότερα άϋλα στοιχεία και έξοδα του λογαριασμού 16 - οι αποσβέσεις των οποίων καταχωρούνται στον πιο πάνω λογ/σμό 66.05 - παρέχεται από το νόμο η ευχέρεια να αποσβένονται «είτε εφάπαξ είτε τμηματικά και ισόποσα μέσα σε μία πενταετία» (βλ. άρθρο 43 παρ. 3 και 4 κωδ. Ν. 2190/1920. Στις περιπτώσεις που επιλέγεται η εφάπαξ απόσβεσή τους δεν ενδείκνυται να επιβαρύνεται το λειτουργικό κόστος, αλλά πρέπει να καταχωρούνται στο λογ/σμό 85 «αποσβέσεις παγίων μη ενσωματωμένες στο λειτουργικό κόστος» (και όχι εδώ, στο 66.05).

β) Οι αποσβέσεις των λογαριασμών 66.05.01, 66.05.02 και 66.05.03 επιβαρύνουν το λογαριασμό 92.00 «έξοδα λειτουργίας παραγωγής».

γ) Οι αποσβέσεις των λογαριασμών 66.05.00, 66.05.10, 66.05.13 (των εξόδων αυξήσεως κεφαλαίου) και 66.05.17, επιβαρύνουν το λογαριασμό 92.01 «έξοδα διοικητικής λειτουργίας». Διευκρινίζεται ότι στο λογαριασμό 66.05.17 περιλαμβάνονται και οι αποσβέσεις των λογισμικών προγραμμάτων Η/Υ (που καταχωρούνται στο λογ/σμό 16.17 και αποσβένονται, σύμφωνα με το άρθρο 3 παρ. 3 του Π.Δ. 88/1973 όπως η παράγραφος αυτή συμπληρώθηκε με το άρθρο 5 παρ. 4 του Ν. 1947/1991, με συντελεστή 25%.

δ) Οι αποσβέσεις των λογαριασμών 66.05.11 και 66.05.12 βαρύνουν το λογαριασμό 92.02 «έξοδα λειτουργίας ερευνών και αναπτύξεως».

ε) Οι αποσβέσεις των λογαριασμών 66.05.13 (των εξόδων εκδόσεως ομολογιακών δανείων), 66.05.16 και 66.05.18 βαρύνουν το λογαριασμό 92.04 «έξοδα χρηματοοικονομικής λειτουργίας».

στ) Η επιβάρυνση των θέσεων κόστους του 92 με τις αποσβέσεις των λογαριασμών 66.05.04, 66.05.05, 66.05.14 και 66.05.19, γίνεται ανάλογα με τη χρησιμοποίηση των αντίστοιχων περιουσιακών στοιχείων (των λογαριασμών 16.04, 16.14 και 16.19).

68   ΠΡΟΒΛΕΨΕΙΣ ΕΚΜΕΤΑΛΛΕΥΣΕΩΣ

        68.00   Προβλέψεις για αποζημίωση προσωπικού λόγω εξόδου
                     από την υπηρεσία

        68.01   Προβλέψεις για υποτιμήσεις συμμετοχών και χρεογράφων

        ........

        68.09    Λοιπές προβλέψεις εκμεταλλεύσεως

        ........

        68.99    Προϋπολογισμένες προβλέψεις εκμεταλλεύσεως (Λ/58.68)

2.2.612 Λογαριασμός 68 «Προβλέψεις εκμεταλλεύσεως»

Στους υπολογαριασμούς του 68 καταχωρούνται οι προβλέψεις που γίνονται από την οικονομική μονάδα για κινδύνους εκμεταλλεύσεως, σύμφωνα με όσα καθορίζονται στην παρ. 2.2.405 για το λογαριασμό 44 «προβλέψεις».

 69   ΟΡΓΑΝΙΚΑ ΕΞΟΔΑ ΚΑΤ' ΕΙΔΟΣ ΥΠΟΚΑΤΑΣΤΗΜΑΤΩΝ
         Ή ΑΛΛΩΝ ΚΕΝΤΡΩΝ
        (Όμιλος λογαριασμών προαιρετικής χρήσεως)

        690    ΑΜΟΙΒΕΣ ΚΑΙ ΕΞΟΔΑ ΠΡΟΣΩΠΙΚΟΥ
                  Ανάπτυξη αντίστοιχη του λογ. 60

        691    ΑΜΟΙΒΕΣ ΚΑΙ ΕΞΟΔΑ ΤΡΙΤΩΝ
                  Ανάπτυξη αντίστοιχη του λογ. 61

        692    ΠΑΡΟΧΕΣ ΤΡΙΤΩΝ
                  Ανάπτυξη αντίστοιχη του λογ. 62

        693    ΦΟΡΟΙ - ΤΕΛΗ
                  Ανάπτυξη αντίστοιχη του λογ. 63

        694    ΔΙΑΦΟΡΑ ΕΞΟΔΑ
                  Ανάπτυξη αντίστοιχη του λογ. 64

        695    ΤΟΚΟΙ ΚΑΙ ΣΥΝΑΦΗ ΕΞΟΔΑ
                  Ανάπτυξη αντίστοιχη του λογ. 65

        696    ΑΠΟΣΒΕΣΕΙΣ ΠΑΓΙΩΝ ΣΤΟΙΧΕΙΩΝ ΕΝΣΩΜΑΤΩΜΕΝΕΣ
                  ΣΤΟ ΛΕΙΤΟΥΡΓΙΚΟ ΚΟΣΤΟΣ
                  Ανάπτυξη αντίστοιχη του λογ. 66

        697

        698    ΠΡΟΒΛΕΨΕΙΣ ΕΚΜΕΤΑΛΛΕΥΣΕΩΣ
                  Ανάπτυξη αντίστοιχη του λογ. 68

2.2.613 Όμιλος λογαριασμών 69 «Οργανικά έξοδα κατ' είδος υποκαταστημάτων ή άλλων κέντρων» (όμιλος λογαριασμών προαιρετικής χρήσεως)

1. Σχετικά με τον τρόπο αναπτύξεως κάθε πρωτοβάθμιου λογαριασμού (690-698) ισχύουν όσα καθορίζονται στην περίπτ. 1 της παρ. 2.2.113.

2. Σχετικά με τον τρόπο λειτουργίας των πρωτοβάθμιων λογαριασμών 690-698 ισχύουν, αντίστοιχα, όσα καθορίζονται παραπάνω στις παρ. 2.2.600 έως και 2.2.612 για τους πρωτοβάθμιους λογαριασμούς 60-68.

\chapter{ΟΡΓΑΝΙΚΑ ΕΣΟΔΑ ΚΑΤ' ΕΙΔΟΣ}

\section{Λογαριασμοί}

\begin{tabularx}{\linewidth}{lX}

\end{tabularx}

70 Πωλήσεις εμπορευμάτων

71 Πωλήσεις προϊόντων έτοιμων και ημιτελών

72 Πωλήσεις λοιπών αποθεμάτων και άχρηστου υλικού

73 Πωλήσεις υπηρεσιών (έσοδα από παροχή υπηρεσιών)

74 Επιχορηγήσεις και διάφορα έσοδα πωλήσεων

75 Έσοδα παρεπόμενων ασχολιών

76 Έσοδα κεφαλαίων

77 .....................................................

78 Ιδιοπαραγωγή παγίων-Τεκμαρτά έσοδα από αυτοπαραδόσεις ή
     καταστροφές αποθεμάτων

79 Οργανικά έσοδα κατ' είδος υποκαταστημάτων ή άλλων κέντρων

 

2.2.7 ΟΜΑΔΑ 7η: ΟΡΓΑΝΙΚΑ ΕΣΟΔΑ ΚΑΤ' ΕΙΔΟΣ

2.2.700 Περιεχόμενο και εννοιολογικοί προσδιορισμοί

1. Στην ομάδα 7 απεικονίζονται και παρακολουθούνται κατ' είδος τα έσοδα τα οποία αναφέρονται στην ομαλή εκμετάλλευση της χρήσεως (οργανικά).

2. Στους λογαριασμούς της ομάδας 7 δεν καταχωρούνται:

α. Κονδύλια που δε συνιστούν έσοδα, όπως η είσπραξη ποσών που η οικονομική μονάδα δανείζεται ή η επιστροφή σ' αυτή ποσών που η ίδια δανείζει σε τρίτους.

β. Έκτακτα και ανόργανα έσοδα, καθώς και έκτακτα κέρδη, τα οποία παρακολουθούνται στους οικείους λογαριασμούς της ομάδας 8.

3. Στους λογαριασμούς της ομάδας 7 απεικονίζονται και παρακολουθούνται οι εξής ειδικότερες κατηγορίες εσόδων:

α. Τα έσοδα από την πώληση υλικών αγαθών ή υπηρεσιών που συνιστούν το κύριο αντικείμενο της εκμεταλλεύσεως (κύκλος εργασιών ή τζίρος).

β. Τα έσοδα από επιχορηγήσεις και από διάφορες άλλες αιτίες που έχουν σχέση με τη δραστηριότητα των πωλήσεων.

γ. Τα έσοδα από παρεπόμενες ασχολίες.

δ. Τα έσοδα κεφαλαίων (συμμετοχών, χρεογράφων και τόκων).

ε. Η αξία κόστους των ιδιοπαραγόμενων πάγιων στοιχείων που χρησιμοποιούνται από την οικονομική μονάδα, καθώς και η αξία βελτιώσεως των στοιχείων αυτών.

στ. Οι χρησιμοποιημένες προβλέψεις για την κάλυψη εξόδων εκμεταλλεύσεως.

4. Οι πρωτοβάθμιοι λογαριασμοί της ομάδας 7 αναπτύσσονται σε δευτεροβάθμιους, τριτοβάθμιους και αναλυτικότερους υπολογαριασμούς σύμφωνα με τις ανάγκες κάθε μονάδας, εκτός από τους δευτεροβάθμιους λογαριασμούς που προβλέπονται ως υποχρεωτικοί από το Σχέδιο Λογαριασμών.

5. Κατά την ανάπτυξη των λογαριασμών πωλήσεων οι οικονομικές μονάδες είναι υποχρεωμένες να προβλέψουν λογαριασμούς για τη διάκρισή τους σε πωλήσεις εσωτερικού και πωλήσεις εξωτερικού.

2.2.701 Περιοδική κατανομή εσόδων μέσα στη χρήση

Σε περίπτωση που η οικονομική μονάδα προσδιορίζει βραχύχρονα (π.χ. μηνιαία ή τριμηνιαία) αποτελέσματα ή καταρτίζει περιοδικές συγκρίσιμες καταστάσεις, η χρονική εναρμόνιση των εσόδων γίνεται, είτε με απευθείας πιστωχρέωση των οικείων λογαριασμών εσόδων, είτε με παρεμβολή ενδιάμεσων λογαριασμών εσόδων (70.99, 71.99, 72.99, 73.99, 74.99, 75.99, 76.99 και 78.99), σύμφωνα με όσα καθορίζονται στην παρ. 2.2.509.

2.2.702 Τακτοποίηση λογαριασμών εσόδων στο τέλος της χρήσεως

1. Τα υπόλοιπα των λογαριασμών της ομάδας 7, στο τέλος της χρήσεως, μεταφέρονται στην πίστωση του λογαριασμού 80.00 «λογαριασμός γενικής εκμεταλλεύσεως». Σε περίπτωση που οι λογαριασμοί εσόδων περιλαμβάνουν και ποσά εσόδων που αφορούν επόμενες χρήσεις, επειδή έχουν προεισπραχτεί, ή σε περίπτωση που οι λογαριασμοί αυτοί δεν περιλαμβάνουν ποσά δουλευμένων εσόδων, επειδή η είσπραξή τους θα πραγματοποιηθεί στις επόμενες χρήσεις, πριν από τη μεταφορά των υπολοίπων τους στο λογαριασμό 80.00 γίνονται εγγραφές τακτοποιήσεως, έτσι ώστε τα υπόλοιπα αυτά να απεικονίζουν το ακριβές ύψος όλων των δουλευμένων εσόδων εκμεταλλεύσεως της χρήσεως που κλείνει.

2. Οι εγγραφές τακτοποιήσεως της προηγούμενης περιπτώσεως γίνονται με τη βοήθεια μεταβατικών λογαριασμών ενεργητικού (λογ. 36) και παθητικού (λογ. 56), όπως αντίστοιχα καθορίζεται στις παρ. 2.2.307 και 2.2.507.

2.2.703 Δυνητική ευχέρεια αναπτύξεως λογαριασμών εσόδων

Η υποδεικνυόμενη ανάπτυξη των λογαριασμών τρίτου βαθμού στους οποίους αναλύονται οι δευτεροβάθμιοι των λογαριασμών 70-78 είναι ενδεικτική.

Κάθε οικονομική μονάδα έχει τη δυνατότητα, αντί να αναπτύξει κατ' είδος τους τριτοβάθμιους λογαριασμούς εσόδων, να τους αναπτύξει κατά προορισμό. Στην περίπτωση όμως αυτή οι υποχρεωτικοί τριτοβάθμιοι λογαριασμοί των εσόδων κατ' είδος εμφανίζονται υποχρεωτικά σαν αναλυτικοί των περιληπτικών κατά προορισμό λογαριασμών, στους οποίους θα αναλύονται οι δευτεροβάθμιοι κατ' είδος λογαριασμοί των 70-78 πρωτοβαθμίων.

70   ΠΩΛΗΣΕΙΣ ΕΜΠΟΡΕΥΜΑΤΩΝ

        70.00

        70.01

                    Ανάπτυξη σύμφωνα με τις ανάγκες κάθε μονάδας με διάκριση σε
                    πωλήσεις εσωτερικού και εξωτερικού

        70.94

        70.95    Επιστροφές πωλήσεων

        70.96    Διάμεσος λογ. πωλήσεων

        70.97    Μη δουλευμένοι τόκοι γραμματίων εισπρακτέων

        70.98    Εκπτώσεις πωλήσεων

        70.99    Προϋπολογισμένες πωλήσεις εμπορευμάτων (Λ/58.70)

2.2.704 Λογαριασμός 70 «Πωλήσεις εμπορευμάτων»

1. Στο λογαριασμό 70 παρακολουθούνται οι πωλήσεις των εμπορευμάτων της οικονομικής μονάδας. Ο λογαριασμός αυτός αντιστοιχεί στο λογαριασμό 20 των αποθεμάτων και λειτουργεί σύμφωνα με όσα αναφέρονται στις παρακάτω περιπτώσεις.

2. Το αντίτιμο της πωλήσεως είναι έσοδο από τη στιγμή εκείνη που η πώληση θεωρείται πραγματοποιημένη. Η πώληση θεωρείται ότι πραγματοποιήθηκε αφότου το εμπόρευμα εξάγεται από την αποθήκη και παραδίδεται στον αγοραστή ή ταξιδεύει για λογαριασμό του ή, κατά περίπτωση, αφότου η υπηρεσία παρέχεται στον πελάτη. Το αντίτιμο από πωλήσεις που έχουν συνομολογηθεί χωρίς να θεωρούνται πραγματοποιημένες, σύμφωνα με τα παραπάνω, δε θεωρείται έσοδο. Το αντίτιμο από πωλήσεις που πραγματοποιούνται με τη συμφωνία το εμπόρευμα να παραμείνει στην αποθήκη του πωλητή προς φύλαξη για λογαριασμό του αγοραστή είναι έσοδο.

3. Η αξία των επιστροφών πωλήσεων καταχωρείται στη χρέωση των οικείων λογαριασμών πωλήσεων. Αν η οικονομική μονάδα επιθυμεί να παρακολουθεί ιδιαίτερα την αξία των επιστροφών, έχει τη δυνατότητα να χρησιμοποιεί το δευτεροβάθμιο λογαριασμό 70.95 «επιστροφές πωλήσεων» ή τριτοβάθμιους λογαριασμούς, τους οποίους αναπτύσσει σύμφωνα με τις ανάγκες της κάτω από κάθε δευτεροβάθμιο με τον οποίο παρακολουθούνται οι πωλήσεις κάθε κατηγορίας εμπορευμάτων. Σε περίπτωση χρησιμοποιήσεως του λογαριασμού 70.95, η ανάπτυξή του σε τριτοβάθμιους λογαριασμούς γίνεται κατά τρόπο που να προκύπτουν οι επιστροφές πωλήσεων για κάθε κατηγορία εμπορευμάτων.

4. Στο δευτεροβάθμιο λογαριασμό 70.96 «διάμεσος λογαριασμός πωλήσεων» είναι δυνατό να καταχωρούνται καθημερινά οι πωλήσεις εμπορευμάτων με τη συνολική τους αξία, με αντίστοιχη χρέωση των οικείων λογαριασμών της ομάδας 3. Στο τέλος κάθε μήνα ή ενδιάμεσα, ο λογαριασμός 70.96 χρεώνεται με πίστωση των οικείων υπολογαριασμών του 70, στους οποίους παρακολουθούνται οι πωλήσεις κατά κατηγορίες ή κατ' είδος εμπορευμάτων.

Η λειτουργία του λογαριασμού 70.96 υποδεικνύεται για τις περιπτώσεις εκείνες που τα εμπορεύματα που προορίζονται για πώληση κατατάσσονται και παρακολουθούνται σε πολλούς υπολογαριασμούς του 70, οπότε με τη χρησιμοποίηση του ενδιάμεσου αυτού λογαριασμού αποφεύγεται η καθημερινή ενημέρωση των αναλυτικών λογαριασμών πωλήσεων.

5. Στο δευτεροβάθμιο λογαριασμό 70.97 «μη δουλευμένοι τόκοι γραμματίων εισπρακτέων», ο οποίος έχει θέση αντίθετου λογαριασμού των λοιπών υπολογαριασμών του 70, στο τέλος κάθε χρήσεως ή, κατά περίπτωση, και κατά τη διάρκεια της χρήσεως, όταν γίνεται ανάλογος διαχωρισμός μη δουλευμένων τόκων, καταχωρούνται οι μη δουλευμένοι τόκοι γραμματίων εισπρακτέων που είναι ενσωματωμένοι στις πωλήσεις των υπολογαριασμών του 70, με πίστωση του λογαριασμού 31.06 ή, κατά περίπτωση, του 31.13.

Σχετικά με τον εκτοκισμό των άληκτων γραμματίων εισπρακτέων και με την κατανομή των μη δουλευμένων τόκων στις επιμέρους κατηγορίες εσόδων των λογαριασμών 70, 71, 72 και 73, ισχύουν όσα καθορίζονται στην περίπτ. 6 της παρ. 2.2.302.

6. Οι εκπτώσεις πωλήσεων είναι μειώσεις της τιμής πωλήσεως, οι οποίες δεν περιλαμβάνονται στα τιμολόγια πωλήσεως εμπορευμάτων για τα οποία χορηγείται η έκπτωση. Ανάλογα με την αιτία για την οποία χορηγούνται, οι εκπτώσεις διακρίνονται στις ακόλουθες κατηγορίες:

α. Εκπτώσεις για διαφορές στην ποιότητα των πωλημένων εμπορευμάτων σε σύγκριση με τη συμφωνημένη.

β. Εκπτώσεις τζίρου, οι οποίες χορηγούνται στο τέλος της συμφωνημένης περιόδου, επειδή έχει υπερκαλυφτεί το όριο ή τα όρια πωλήσεων που καθορίζονται ως στόχοι.

γ. Εκπτώσεις που αντικαθιστούν τις ποινικές ρήτρες, όταν παρατηρούνται καθυστερήσεις παραδόσεων ή δεν τηρούνται άλλοι όροι της συμφωνίας.

δ. Ταμιακές εκπτώσεις ή εκπτώσεις προεξοφλητικού διακανονισμού, οι οποίες χορηγούνται σε περιπτώσεις πωλήσεων «τοις μετρητοίς».

Οι εκπτώσεις πωλήσεων, σαν μειωτικό στοιχείο εσόδων, καταχωρούνται στη χρέωση των αντίστοιχων λογαριασμών των πωλήσεων. Αν τούτο δεν είναι εφικτό ή αν η οικονομική μονάδα επιθυμεί να παρακολουθεί τις εκπτώσεις χωριστά, έχει τη δυνατότητα να χρησιμοποιεί το δευτεροβάθμιο λογαριασμό 70.98 «εκπτώσεις πωλήσεων» ή τριτοβάθμιους λογαριασμούς, τους οποίους αναπτύσσει σύμφωνα με τις ανάγκες της κάτω από κάθε δευτεροβάθμιο με τον οποίο παρακολουθούνται οι πωλήσεις κάθε κατηγορίας εμπορευμάτων. Σε περίπτωση χρησιμοποιήσεως του λογαριασμού 70.98, η ανάπτυξή του σε τριτοβάθμιους λογαριασμούς γίνεται κατά τρόπο που να προκύπτουν οι εκπτώσεις πωλήσεων για κάθε κατηγορία εμπορευμάτων, έτσι ώστε οι εκπτώσεις αυτές να επιβαρύνουν τα μικτά αποτελέσματα της αντίστοιχης κατηγορίας. Αν η διάκριση των εκπτώσεων κατά κατηγορίες εμπορευμάτων είναι αδύνατη, οι εκπτώσεις αυτές επιβαρύνουν τα συνολικά μικτά αποτελέσματα εμπορευμάτων (λογ. 70).

7. Με την αξία των πραγματοποιούμενων πωλήσεων, που προκύπτει από τα τιμολόγια ή δελτία λιανικής πωλήσεως (Δ.Λ.Π.) που εκδίδονται, πιστώνονται οι οικείοι υπολογαριασμοί πωλήσεων του 70, με χρέωση των λογαριασμών πελατών (λογ. 30) ή των λογαριασμών χρηματικών διαθεσίμων (λογ. 38).

8. Ο λογαριασμός του πελάτη ή των χρηματικών διαθεσίμων χρεώνεται με το συνολικό ποσό του παραστατικού (τιμολογίου ή Δ.Λ.Π.), δηλαδή με την αξία των πωλημένων εμπορευμάτων, μειωμένη κατά την έκπτωση που αναγράφεται στο παραστατικό και αυξημένη κατά το φόρο κύκλου εργασιών ή άλλο φόρο καταναλώσεως (όταν επιβάλλεται τέτοιος φόρος), το χαρτόσημο και τα έξοδα αποστολής, όταν τα τελευταία βαρύνουν τον πελάτη και αναγράφονται στο παραστατικό.

9. Στην πίστωση των οικείων υπολογαριασμών πωλήσεων του 70 καταχωρείται η τιμολογιακή αξία των πωλημένων, μειωμένη κατά την έκπτωση που αναγράφεται στο παραστατικό, χωρίς τον τυχόν φόρο κύκλου εργασιών, που καταχωρείται στην πίστωση του λογαριασμού 54.00 «φόρος κύκλου εργασιών», χωρίς το τυχόν χαρτόσημο, που καταχωρείται στην πίστωση του λογαριασμού 54.02 «χαρτόσημο τιμολογίων πωλήσεως», χωρίς οποιοδήποτε φόρο, τέλος ή εισφορά που εισπράττονται από την οικονομική μονάδα για λογαριασμό του Δημοσίου ή τρίτου και που καταχωρούνται στην πίστωση των οικείων υπολογαριασμών του 54 «υποχρεώσεις από φόρους - τέλη» και χωρίς τα τυχόν τιμολογημένα έξοδα αποστολής, που καταχωρούνται στην πίστωση του λογαριασμού 75.10 «εισπραττόμενα έξοδα αποστολής αγαθών».

10. Για πωλήσεις εμπορευμάτων στο εξωτερικό, η σχετική εγγραφή χρεώσεως του λογαριασμού του πελάτη και πιστώσεως των λογαριασμών των πωλήσεων γίνεται την ημέρα κατά την οποία τα πωλημένα εξάγονται από την αποθήκη και εκδίδεται το παραστατικό πωλήσεως. Η αξία των πωλημένων, προκειμένου να γίνει σχετική εγγραφή, υπολογίζεται σε δραχμές με βάση της επίσημη τιμή συναλλάγματος (τιμή αγοράς της Τράπεζας της Ελλάδος) της ημέρας εκδόσεως του παραστατικού και εξαγωγής των πωλημένων από την αποθήκη.

Σε περιπτώσεις που προηγείται η εξαγωγή και επακολουθεί η έκδοση του παραστατικού, η αξία των πωλημένων υπολογίζεται με βάση την επίσημη τιμή συναλλάγματος της ημερομηνίας εκδόσεως του παραστατικού.

Τυχόν διαφορά ανάμεσα στην αξία των πωλημένων, όπως προσδιορίζεται σύμφωνα με τον παραπάνω τρόπο, και στην αξία που προκύπτει με βάση την τιμή συναλλάγματος της ημέρας διακανονισμού της αξίας, δηλαδή της ημέρας που εκδίδεται η εκκαθάριση της Τράπεζας η οποία μεσολαβεί στην Ελλάδα, αποτελεί ανόργανο έξοδο ή έσοδο και φέρεται αντίστοιχα στη χρέωση του λογαριασμού 81.00.04 ή στην πίστωση του λογαριασμού 81.01.04.

11. Σε περιπτώσεις που η οικονομική μονάδα, κατά την ημερομηνία κλεισίματος του ισολογισμού, έχει πραγματοποιήσει την εξαγωγή των πωλημένων εμπορευμάτων από τις αποθήκες της και τα έχει παραδόσει στον αγοραστή ή τα έχει φορτώσει και ταξιδεύουν για λογαριασμό του, χωρίς, για διάφορους λόγους, να έχει εκδόσει παραστατικό, χρεώνεται ο μεταβατικός λογαριασμός 36.01 «έσοδα χρήσεως εισπρακτέα», σύμφωνα με όσα καθορίζονται στην παρ. 2.2.307, με πίστωση των οικείων υπολογαριασμών πωλήσεων του 70. Κατά τη χρήση που ακολουθεί, όταν εκδίδεται το παραστατικό πωλήσεως, χρεώνεται ο λογαριασμός του πελάτη, με πίστωση του λογαριασμού 36.01.

12. Σε περιπτώσεις που εκδίδονται παραστατικά πωλήσεως πριν από την εξαγωγή των πωλημένων εμπορευμάτων από την αποθήκη και την παράδοσή τους στον αγοραστή, η αξία των παραστατικών αυτών δεν καταχωρείται στους οικείους υπολογαριασμούς πωλήσεων του 70. Τα παραστατικά αυτά ακυρώνονται στο τέλος της χρήσεως, αν μέχρι την ημέρα λήξεώς της τα εμπορεύματα δεν παραδοθούν στον αγοραστή ή δε φορτωθούν για λογαριασμό του.

71    ΠΩΛΗΣΕΙΣ ΠΡΟΪΟΝΤΩΝ ΕΤΟΙΜΩΝ ΚΑΙ ΗΜΙΤΕΛΩΝ

        71.00

        71.01

                     Ανάπτυξη σύμφωνα με τις ανάγκες κάθε μονάδας με διάκριση
                     σε πωλήσεις εσωτερικού και εξωτερικού

        71.94

        71.95    Επιστροφές πωλήσεων

        71.96    Διάμεσος λογ. πωλήσεων

        71.97    Μη δουλευμένοι τόκοι γραμματίων εισπρακτέων

        71.98    Εκπτώσεις πωλήσεων

        71.99    Προϋπολογισμένες πωλήσεις προϊόντων έτοιμων και
                     ημιτελών (Λ/58.71)

2.2.705 Λογαριασμός 71 «Πωλήσεις προϊόντων έτοιμων και ημιτελών»

Στο λογαριασμό 71 παρακολουθούνται οι πωλήσεις των έτοιμων και ημιτελών προϊόντων της οικονομικής μονάδας. Ο λογαριασμός αυτός, ο οποίος αντιστοιχεί στο λογαριασμό 21 των αποθεμάτων, λειτουργεί σύμφωνα με όσα καθορίζονται στην παρ. 2.2.704 για το λογαριασμό 70.

 72    ΠΩΛΗΣΕΙΣ ΛΟΙΠΩΝ ΑΠΟΘΕΜΑΤΩΝ ΚΑΙ ΑΧΡΗΣΤΟΥ ΥΛΙΚΟΥ

        72.00    Πωλήσεις άχρηστου υλικού 

        72.01    .......................................

        ........

        72.10    Ασφαλιστική αποζημίωση κλαπέντων ή απωλεσθέντων αποθεμάτων
                     (Γνωμ. 114/1839/1992 \& 217/2177/1994)

        72.11    Ασφαλιστική αποζημίωση καταστραφέντων αποθεμάτων
                     (Γνωμ. 107/1810/92)

        72.22    Πωλήσεις υποπροϊόντων και υπολειμμάτων 

        72.24    Πωλήσεις πρώτων και βοηθητικών υλών - υλικών συσκευασίας 

        72.25    Πωλήσεις αναλώσιμων υλικών 

        72.26    Πωλήσεις ανταλλακτικών πάγιων στοιχείων 

        72.28    Πωλήσεις ειδών συσκευασίας 

        72.94

        72.95    Επιστροφές πωλήσεων

        72.96    Διάμεσος λογ. πωλήσεων

        72.97    Μη δουλευμένοι τόκοι γραμματίων εισπρακτέων

        72.98    Εκπτώσεις πωλήσεων

        72.99    Προϋπολογισμένες πωλήσεις αποθεμάτων και άχρηστου υλικού
                     (Λ/58.72)

2.2.706 Λογαριασμός 72 «Πωλήσεις λοιπών αποθεμάτων και άχρηστου υλικού»

1. Στο λογαριασμό 72 παρακολουθούνται τα έσοδα της οικονομικής μονάδας από τις πωλήσεις: (1) των υποπροϊόντων και υπολειμμάτων, (2) των πρώτων και βοηθητικών υλών - υλικών συσκευασίας, (3) των αναλώσιμων υλικών, (4) των ανταλλακτικών πάγιων στοιχείων, (5) των ειδών συσκευασίας και (6) του άχρηστου υλικού.

2. Η ανάπτυξη του λογαριασμού 72 γίνεται σύμφωνα με τις ανάγκες κάθε μονάδας, κατά τρόπο όμως που να είναι δυνατή η ιδιαίτερη παρακολούθηση των παραπάνω κατηγοριών πωλήσεων.

3. Ο λογαριασμός 72, ο οποίος αντιστοιχεί στους λογαριασμούς αποθεμάτων 22, 24, 25, 26 και 28, λειτουργεί σύμφωνα με όσα καθορίζονται στην παρ. 2.2.704 για το λογαριασμό 70.

73    ΠΩΛΗΣΕΙΣ ΥΠΗΡΕΣΙΩΝ (Έσοδα από παροχή υπηρεσιών) 

        73.00

        73.01

                     Ανάπτυξη σύμφωνα με τις ανάγκες κάθε μονάδας με διάκριση
                     σε πωλήσεις εσωτερικού και εξωτερικού

        73.91    Διαφορές (κέρδη) από πράξεις Hedging (Γνωμ.268/2272/96) 

        73.94

        73.95

        73.96    Διάμεσος λογ. πωλήσεων

        73.97    Μη δουλευμένοι τόκοι γραμματίων εισπρακτέων

        73.98    Εκπτώσεις πωλήσεων

        73.99    Προϋπολογισμένες πωλήσεις εμπορευμάτων (Λ/58.73)

2.2.707 Λογαριασμός 73 «Πωλήσεις υπηρεσιών (Έσοδα παροχή υπηρεσιών)»

1. Στο λογαριασμό 73 παρακολουθούνται τα έσοδα της οικονομικής μονάδας από την πώληση υπηρεσιών σε τρίτους, εφόσον οι υπηρεσίες αυτές υπάγονται στις κύριες δραστηριότητές της. Στην αντίθετη περίπτωση, τα έσοδα αυτά καταχωρούνται στους οικείους υπολογαριασμούς του λογαριασμού 75 «έσοδα παρεπόμενων ασχολιών».

2. Ο λογαριασμός 73 λειτουργεί σύμφωνα με όσα καθορίζονται στην παρ. 2.2.704 για το λογαριασμό 70.

 74    ΕΠΙΧΟΡΗΓΗΣΕΙΣ ΚΑΙ ΔΙΑΦΟΡΑ ΕΣΟΔΑ ΠΩΛΗΣΕΩΝ

        74.00    Επιχορηγήσεις πωλήσεων 

        74.01    Επιστροφές δασμών και λοιπών επιβαρύνσεων 

        74.02    Επιστροφές τόκων λόγω εξαγωγών 

        74.03    Ειδικές επιχορηγήσεις - Επιδοτήσεις
                    (Γνωμ. 41/1063/1989, 47/1228/1989, 206/2138/1994)
                      Αναπτύσσεται σε τριτοβάθμιους κατά επιχορήγηση - επιδότηση Π.χ.:

                     74.03.00    Επιδοτήσεις ΟΑΕΔ

                               01    .........................................

                               02    .........................................

                               ....

                               99    .........................................

        74.04

        74.05    Επιδότηση επιτοκίου δανείων πάγιων επενδύσεων
                        (Γνωμ. 93/1687/1992)

                     74.05.00

                     ...............

        74.98    Διάφορα πρόσθετα έσοδα πωλήσεων

                     74.98.00    Αποζημιώσεις από πελάτες

                               01    Έσοδα από μερική χρησιμοποίηση ειδών συσκευασίας

                               02    Αποζημιώσεις από αβαρίες

                     .............

                     74.98.99

        74.99    Προϋπολογισμένες - Προεισπραγμένες επιχορηγήσεις και
                     διάφορα έσοδα πωλήσεων (Λ/58.74)

2.2.708 Λογαριασμός 74 «Επιχορηγήσεις και διάφορα έσοδα πωλήσεων»

1. Στο λογαριασμό 74 παρακολουθούνται τα έσοδα που πραγματοποιεί η οικονομική μονάδα από επιχορηγήσεις του Κράτους, από συμμετοχή του κρατικού προϋπολογισμού και των προϋπολογισμών διάφορων Οργανισμών στο κόστος της και από διάφορες άλλες αιτίες.

2. Επιχορηγήσεις (λογ. 74.00) είναι ποσά που χορηγούνται στην οικονομική μονάδα με οποιοδήποτε τρόπο από το Κράτος ή από Νομικά Πρόσωπα και Οργανισμούς που ελέγχονται από το Κράτος, για να πραγματοποιεί αυτή πωλήσεις ή άλλης μορφής εκμετάλλευση σε τιμές που για την ίδια θεωρούνται ασύμφορες. 

3. Επιστροφές δασμών και λοιπών επιβαρύνσεων (λογ. 74.01) είναι ποσά που επιστρέφονται στην οικονομική μονάδα εξαιτίας εξαγωγών της ή άλλης νόμιμης αιτίας, τα οποία είχαν καταβληθεί και συμπεριληφθεί στο κόστος πρώτων και βοηθητικών υλών που αγοράστηκαν από το εξωτερικό και αναλώθηκαν για την παραγωγή προϊόντων, που τελικά πωλήθηκαν στο εξωτερικό ή σε πελάτες του εσωτερικού στους οποίους έχει χορηγηθεί δασμολογική ατέλεια (όπως π.χ. ν.δ. 4171/1961). Σε περίπτωση που οι δασμοί, φόροι και τέλη, οι οποίοι καταβάλλονται κατά την εισαγωγή αγαθών από το εξωτερικό, καταχωρούνται στο λογαριασμό 33.14.01 «δασμοί και λοιποί φόροι εισαγωγής προς επιστροφή», κατά την επιστροφή τους καταχωρούνται στο λογαριασμό αυτό (33.14.01) και όχι στο λογαριασμό 74.01, σύμφωνα με όσα καθορίζονται στην περίπτ. 11-β της παρ. 2.2.304.

4. Επιστροφές τόκων λόγω εξαγωγών (λογ. 74.02) είναι ποσά τόκων χρηματοδοτήσεων που επιστρέφονται στην οικονομική μονάδα από τις Τράπεζες, λόγω εξαγωγών, σύμφωνα με τις αποφάσεις της Νομισματικής Επιτροπής που ισχύουν κάθε φορά.

5. Πρόσθετα έσοδα πωλήσεων (λογ. 74.98) είναι έσοδα που προκύπτουν άμεσα ή έμμεσα από τις πωλήσεις της οικονομικής μονάδας μετά την έκδοση των σχετικών παραστατικών πωλήσεως και τη διενέργεια των σχετικών εγγραφών. Ενδεικτικές περιπτώσεις τέτοιων εσόδων αποτελούν οι διάφορες αποζημιώσεις που καταβάλλουν οι πελάτες σε περίπτωση αθετήσεως όρων συμβάσεων, τα έσοδα από μερική χρησιμοποίηση ειδών συσκευασίας και αποζημιώσεις από αβαρίες, εφόσον δεν είναι εφικτή η μεταφορά τους σε μείωση της αξίας των αγαθών στα οποία έγινε η βλάβη.

6. Τα έσοδα από επιχορηγήσεις, επιστροφές δασμών και λοιπών επιβαρύνσεων και από επιστροφές τόκων καταχωρούνται στα βιβλία μόνο όταν είναι βέβαια και εκκαθαρισμένα, δηλαδή όταν δεν τελούν υπό αίρεση ή προθεσμία και αποδεικνύονται εγγράφως. Στη συγκεκριμένη περίπτωση, στους οικείους υπολογαριασμούς του 74 καταχωρούνται τα έσοδα για τα οποία η Τράπεζα της Ελλάδος ή οποιαδήποτε Αρχή έχει γνωρίσει εγγράφως στην οικονομική μονάδα ότι είναι δυνατή η είσπραξή τους ή όταν τα έσοδα αυτά προκύπτουν από απόλυτα δικαιολογημένους υπολογισμούς της οικονομικής μονάδας, που βασίζονται σε διατάξεις νόμων ή σε αποφάσεις αρμόδιων κρατικών ή εξουσιοδοτημένων από το κράτος οργάνων. Από τα βέβαια και εκκαθαρισμένα έσοδα, όσα αφορούν πωλήσεις της κλειόμενης χρήσεως καταχωρούνται στην πίστωση των οικείων υπολογαριασμών του 74, όσα όμως αφορούν πωλήσεις προηγούμενων χρήσεων καταχωρούνται στην πίστωση των οικείων υπολογαριασμών του 82 «έξοδα και έσοδα προηγούμενων χρήσεων».

7. Ο λογαριασμός 74 λειτουργεί σύμφωνα με όσα καθορίζονται στην παρ. 2.2.704 για το λογαριασμό 70, σε συνδυασμό και με όσα ειδικά ορίζονται για το λογαριασμό αυτό στις παραπάνω περιπτώσεις 1-6.

 75    ΕΣΟΔΑ ΠΑΡΕΠΟΜΕΝΩΝ ΑΣΧΟΛΙΩΝ

         75.00    Έσοδα από παροχή υπηρεσιών σε τρίτους 

                      75.00.00    Έσοδα από παροχή υπηρεσιών λογιστηρίου 

                                01    Έσοδα από μελέτες - έρευνες για λογαριασμό τρίτων 

                                02    Έσοδα από επεξεργασία (Facon) προϊόντων - υλικών τρίτων 

                                03    Έσοδα από επισκευές αγαθών τρίτων 

                                04    Έσοδα από παροχή υπηρεσιών σε πρωτοβάθμιους
                                       συνεταιρισμούς (Γνωμ. 55/1336/1990)

                      ..............

                      75.00.99    Λοιπά έσοδα από παροχή υπηρεσιών σε τρίτους

         75.01    Έσοδα από παροχή υπηρεσιών στο προσωπικό 

                      75.01.00    Έσοδα από παροχή κατοικιών 

                                01    Έσοδα εστιατορίου 

                                02    Έσοδα κυλικείου 

                      ..............

                      75.01.99    Λοιπά έσοδα από παροχή υπηρεσιών στο προσωπικό

         75.02    Προμήθειες - Μεσιτείες 

                      75.02.00    Προμήθειες από αγορές για λογαριασμό τρίτων 

                                01    Προμήθειες από πωλήσεις για λογαριασμό τρίτων 

                      ..............

                      75.02.99    Λοιπές προμήθειες και μεσιτείες

                      Σημείωση:    Οι προμήθειες - μεσιτείες από κύριες ασχολίες καταχωρούνται
                                             στον λογαριασμό 73

         75.03    Έσοδα από προνόμια και διοικητικές παραχωρήσεις 

                      Σημείωση:  Τα έσοδα από προνόμια και διοικητικές παραχωρήσεις, όταν
                                           πρόκειται για κύριες ασχολίες, καταχωρούνται στον λογαριασμό 73

         75.04    Ενοίκια εδαφικών εκτάσεων 

         75.05    Ενοίκια κτιρίων - τεχνικών έργων 

         75.06    Ενοίκια μηχανημάτων - τεχνικών εγκαταστάσεων - λοιπού
                      μηχανολογικού εξοπλισμού 

         75.07    Ενοίκια μεταφορικών μέσων 

         75.08    Ενοίκια επίπλων και λοιπού εξοπλισμού 

         75.09    Ενοίκια ασώματων ακινητοποιήσεων
                        (π.χ. μεταλλευτικών παραχωρήσεων)

         75.10    Εισπραττόμενα έξοδα αποστολής αγαθών 

         75.99    Προϋπολογισμένα - Προεισπραγμένα έσοδα παρεπόμενων ασχολιών
                        (Λ/58.75)

2.2.709 Λογαριασμός 75 «Έσοδα παρεπόμενων ασχολιών»

1. Στο λογαριασμό 75 παρακολουθούνται τα έσοδα που πραγματοποιεί η οικονομική μονάδα από παρεπόμενες ασχολίες, δηλαδή εκείνα που προέρχονται από παρεπόμενες δραστηριότητές της, σε σχέση με το κύριο αντικείμενό της. Αν μία από τις δραστηριότητες που προκαλούν άλλα έσοδα, εκτός από πωλήσεις ή έσοδα κεφαλαίων, συνιστά το κύριο αντικείμενο της οικονομικής μονάδας, τα έσοδα που προκύπτουν από τη δραστηριότητα αυτή καταχωρούνται στους οικείους υπολογαριασμούς του 73 «πωλήσεις υπηρεσιών» και όχι στο λογαριασμό 75.

2. Ο λογαριασμός 75 λειτουργεί σύμφωνα με όσα καθορίζονται στην παρ. 2.2.704 για το λογαριασμό 70, σε συνδυασμό και με τις ακόλουθες διευκρινίσεις:

α. Στο λογαριασμό 75.00 «έσοδα από παροχή υπηρεσιών σε τρίτους» καταχωρούνται τα έσοδα από υπηρεσίες που παρέχονται από την οικονομική μονάδα σε τρίτους, εφόσον οι υπηρεσίες αυτές δεν αποτελούν αντικείμενο της κύριας δραστηριότητάς της.

β. Στο λογαριασμό 75.01 «έσοδα από παροχή υπηρεσιών στο προσωπικό» καταχωρούνται τα έσοδα από τη συμμετοχή του προσωπικού στα έξοδα που πραγματοποιούνται για λογαριασμό του από την οικονομική μονάδα. Τα έξοδα αυτά, όταν πραγματοποιούνται, καταχωρούνται στη χρέωση του λογαριασμού 60.02 «παρεπόμενες παροχές και έξοδα προσωπικού», σύμφωνα με όσα καθορίζονται στην περίπτ. 3 της παρ. 2.2.604.

γ. Στο λογαριασμό 75.02 «προμήθειες - μεσιτείες» καταχωρούνται τα έσοδα από προμήθειες και μεσιτείες που η οικονομική μονάδα λαβαίνει από αγορές ή πωλήσεις που πραγματοποιεί για λογαριασμό τρίτων, εφόσον οι μεσολαβητικές αυτές ενέργειες δε συνιστούν το κύριο αντικείμενο της δραστηριότητάς της.

δ. Στο λογαριασμό 75.03 «έσοδα από προνόμια και διοικητικές παραχωρήσεις» καταχωρούνται τα έσοδα που προέρχονται από την παραχώρηση σε τρίτους του δικαιώματος εκμεταλλεύσεως άυλων περιουσιακών στοιχείων, όπως τεχνικών μεθόδων παραγωγής ή διπλωμάτων ευρεσιτεχνίας, εφόσον οι παραχωρήσεις αυτές δε συνιστούν το κύριο αντικείμενο της δραστηριότητας της οικονομικής μονάδας.

ε. Στους λογαριασμούς 75.04, 75.05, 75.06, 75.07, 75.08 και 75.09 καταχωρούνται, αντίστοιχα, τα έσοδα από την εκμίσθωση εδαφικών εκτάσεων, κτιρίων και τεχνικών έργων, μηχανημάτων - τεχνικών εγκαταστάσεων και λοιπού μηχανολογικού εξοπλισμού, μεταφορικών μέσων, επίπλων και λοιπού εξοπλισμού και ασώματων ακινητοποιήσεων, εφόσον οι εκμισθώσεις αυτές δε συνιστούν το κύριο αντικείμενο της δραστηριότητας της οικονομικής μονάδας (κτηματικές επιχειρήσεις). Τα έσοδα της κατηγορίας αυτής θεωρούνται ανόργανα και καταχωρούνται στους οικείους υπολογαριασμούς του 81.01 «έκτακτα και ανόργανα έσοδα», όταν τα πάγια περιουσιακά στοιχεία που εκμισθώνονται έχουν κτηθεί ευκαιριακά και δεν αφορούν το αντικείμενο της εκμεταλλεύσεως.

στ. Στο λογαριασμό 75.10 «εισπραττόμενα έξοδα αποστολής αγαθών» καταχωρούνται τα έξοδα αποστολής των πωλημένων αγαθών, που βαρύνουν τους πελάτες, οι οποίοι χρεώνονται είτε με το παραστατικό πωλήσεως, είτε με οποιοδήποτε άλλο παραστατικό.

76    ΕΣΟΔΑ ΚΕΦΑΛΑΙΩΝ 

        76.00    Έσοδα συμμετοχών 

                     76.00.00    Μερίσματα μετοχών εισαγμένων στο Χρηματιστήριο εταιριών
                                       εσωτερικού 

                               01    Μερίσματα μετοχών μη εισαγμένων στο Χρηματιστήριο
                                       εταιριών εσωτερικού 

                               02    Μερίσματα μετοχών εισαγμένων στο Χρηματιστήριο εταιριών
                                       εξωτερικού 

                               03    Μερίσματα μετοχών μη εισαγμένων στο Χρηματιστήριο
                                       εταιριών εξωτερικού 

                               04    Έσοδα από συμμετοχή σε προσωπικές εταιρίες εσωτερικού 

                               05    Έσοδα από συμμετοχή σε προσωπικές εταιρίες εξωτερικού 

                               06    Έσοδα από συμμετοχή σε κοινοπραξίες εσωτερικού 

                               07    Έσοδα από συμμετοχή σε κοινοπραξίες εξωτερικού 

                     ...............

                     76.00.99    Λοιπές έσοδα από συμμετοχές

        76.01    Έσοδα χρεογράφων 

                     76.01.00    Μερίσματα μετοχών εισαγμένων στο Χρηματιστήριο εταιριών
                                       εσωτερικού 

                               01    Μερίσματα μετοχών μη εισαγμένων στο Χρηματιστήριο
                                       εταιριών εσωτερικού 

                               02    Έσοδα ομολογιών ελληνικών δανείων 

                               03    Μερίσματα μεριδίων αμοιβαίων κεφαλαίων εσωτερικού 

                               04    Τόκοι έντοκων γραμματίων Ελληνικού Δημοσίου 

                               05    Μερίσματα μετοχών εισαγμένων στο Χρηματιστήριο εταιριών
                                       εξωτερικού 

                               06    Μερίσματα μετοχών μη εισαγμένων στο Χρηματιστήριο
                                       εταιριών εξωτερικού 

                               07    Έσοδα ομολογιών αλλοδαπών δανείων 

                               08    Μερίσματα μεριδίων αμοιβαίων κεφαλαίων εξωτερικού 

                     ...............

                     76.01.98    Έσοδα λοιπών χρεογράφων εσωτερικού

                     76.01.99    Έσοδα λοιπών χρεογράφων εξωτερικού

        76.02    Δουλευμένοι τόκοι γραμματίων εισπρακτέων 

        76.03    Λοιποί πιστωτικοί τόκοι 

                     76.03.00    Τόκοι καταθέσεων Τραπεζών εσωτερικού 

                               01    Τόκοι καταθέσεων Ταμιευτηρίων εσωτερικού 

                               02    Τόκοι καταθέσεων εξωτερικού 

                               03    Τόκοι χορηγημένων δανείων 

                               04    Τόκοι τρεχούμενων λογ/σμών πελατών 

                               05    Τόκοι λοιπών τρεχούμενων λογαριασμών 

                               06    Τόκοι καθυστερούμενων γραμματίων εισπρακτέων 

                     ...............

                     76.03.99    Λοιποί πιστωτικοί τόκοι

        76.04    Διαφορές (κέρδη) από πώληση συμμετοχών και χρεογράφων 

                     76.04.00    Διαφορές (κέρδη) από πώληση συμμετοχών 

                               01    Διαφορές (κέρδη) από πώληση συμμετοχών σε λοιπές
                                       πλήν Α.Ε. επιχειρήσεις 

                               02    Διαφορές (κέρδη) από πώληση χρεογράφων 

                     ...............

                     76.04.99

        ........

        76.98    Λοιπά έσοδα κεφαλαίων

                     76.98.00    Εκπτώσεις από εφάπαξ εξόφληση φόρων και τελών
                                       (Γνωμ. 31/1022/1988)

                               01

                     ..............

                     76.98.99

        76.99    Προϋπολογισμένα - Προεισπραγμένα έσοδα κεφαλαίων (Λ/58.76)

2.2.710 Λογαριασμός 76 «Έσοδα κεφαλαίων»

1. Στο λογαριασμό 76 παρακολουθούνται τα έσοδα που πραγματοποιεί η οικονομική μονάδα από τοποθετήσεις κεφαλαίων της σε συμμετοχές και χρεόγραφα και από δανεισμούς προς τρίτους. Τα έσοδα αυτά, αν προέρχονται από δραστηριότητα ή δραστηριότητες που συνιστούν το κύριο αντικείμενο απασχολήσεως της οικονομικής μονάδας (π.χ. εταιρείες αμοιβαίων κεφαλαίων ή Τράπεζες), καταχωρούνται στο λογαριασμό 73 ή και σε άλλους λογαριασμούς της ομάδας 7 (70-72), των οποίων οι τίτλοι τροποποιούνται σύμφωνα με τις ανάγκες της μονάδας.

2. Ο λογαριασμός 76 λειτουργεί σύμφωνα με όσα καθορίζονται στην παρ. 2.2.704 για το λογαριασμό 70, σε συνδυασμό και με τις ακόλουθες διευκρινίσεις:

α. Στους λογαριασμούς 76.00 «έσοδα συμμετοχών» και 76.01 «έσοδα χρεογράφων» καταχωρούνται τα έσοδα από μερίσματα συμμετοχών και χρεογράφων, καθώς και οι τόκοι από χρεόγραφα (π.χ. ομολογίες). Τα έσοδα αυτά καταχωρούνται στα ονομαστικά τους ποσά, ενώ ο φόρος που παρακρατείται καταχωρείται σε λογαριασμούς απαιτήσεων της ομάδας 3, ως εξής:

- Για μερίσματα μετοχών εταιρειών ημεδαπής εισαγμένων στο Χρηματιστήριο, στη χρέωση του λογαριασμού 33.13.01.

- Για μερίσματα μετοχών εταιρειών ημεδαπής μη εισαγμένων στο Χρηματιστήριο, στη χρέωση του λογαριασμού 33.13.02.

- Για μερίσματα μετοχών αλλοδαπής προελεύσεως, στη χρέωση του λογαριασμού 33.13.03.

- Για κέρδη από συμμετοχή σε αλλοδαπές ΕΠΕ, στη χρέωση του λογαριασμού 33.13.04.

- Για προσόδους από μερίδια αμοιβαίων κεφαλαίων, στη χρέωση του λογαριασμού 33.13.05.

- Για κέρδη από συμμετοχή σε ημεδαπές ΕΠΕ, ΟΕ, ΕΕ και κοινοπραξίες εκτελέσεως τεχνικών έργων ημεδαπής, στη χρέωση του λογαριασμού 33.13.07.

Η τύχη και ο λογιστικός χειρισμός των φόρων που παρακρατούνται ρυθμίζονται στην περίπτ. 10-β της παρ. 2.2.304.

β. Στο λογαριασμό 76.02 «δουλευμένοι τόκοι γραμματίων εισπρακτέων» καταχωρούνται οι δουλευμένοι τόκοι των γραμματίων εισπρακτέων. Η ανάπτυξη του λογαριασμού αυτού γίνεται σύμφωνα με τις ανάγκες κάθε οικονομικής μονάδας και η λειτουργία του σύμφωνα με όσα καθορίζονται στην περίπτ. 6 της παρ. 2.2.302.

γ. Στο λογαριασμό 76.03 «λοιποί πιστωτικοί τόκοι» καταχωρούνται τα ονομαστικά έσοδα από τόκους. Τυχόν ποσά φόρου εισοδήματος που παρακρατούνται κατά την είσπραξη ή το λογισμό των τόκων αυτών καταχωρούνται στη χρέωση του λογαριασμού 33.13.06, όπως καθορίζεται στην περίπτ. 10-β της παρ. 2.2.304.

δ. Στο λογαριασμό 76.04 «διαφορές (κέρδη) από πώληση συμμετοχών και χρεογράφων» καταχωρούνται τα κέρδη που πραγματοποιούνται από πωλήσεις συμμετοχών και χρεογράφων, σύμφωνα με όσα καθορίζονται στην περίπτ. 5 της παρ. 2.2.112 για τις συμμετοχές και στην περίπτ. 11 της παρ. 2.2.305 για τα χρεόγραφα.

ε. Στο λογαριασμό 76.98 «λοιπά έσοδα κεφαλαίων» καταχωρούνται τα έσοδα κεφαλαίων τα οποία δεν εντάσσονται σε οποιαδήποτε κατηγορία από αυτές των λοιπών υπολογαριασμών του 76.

78    ΙΔΙΟΠΑΡΑΓΩΓΗ ΠΑΓΙΩΝ - ΤΕΚΜΑΡΤΑ ΕΣΟΔΑ ΑΠΟ
        ΑΥΤΟΠΑΡΑΔΟΣΕΙΣ 'Η ΚΑΤΑΣΤΡΟΦΕΣ ΑΠΟΘΕΜΑΤΩΝ 

        78.00    Ιδιοπαραγωγή και βελτιώσεις παγίων 

                     78.00.10    Εδαφικών εκτάσεων 

                               11    Κτιρίων - Εγκαταστάσεων κτιρίων - Τεχνικών έργων 

                               12    Μηχανημάτων - Τεχνικών εγκαταστάσεων - Λοιπού
                                       μηχανολογικού εξοπλισμού 

                               13    Μεταφορικών μέσων 

                               14    Επίπλων και λοιπού εξοπλισμού 

                     78.00.14.06    Ιδιοπαραγωγή ζώων για πάγια εκμετάλλευση
                                           (Γνωμ. 183/2099/1993) 

                               15    Ακινητοποιήσεων υπό εκτέλεση 

                               16    Ασώματων ακινητοποιήσεων και εξόδων πολυετούς
                                       αποσβέσεως 

                     ..............

                     78.00.99

        78.01

         .........

         78.05    Χρησιμοποιημένες προβλέψεις προς κάλυψη εξόδων εκμεταλλεύσεως
                      (Ο λογαριασμός 78.05 έγινε προαιρετικός με τη Γνωμ. του ΕΣΥΛ 91/1683/1992)

                     78.05.00    Προβλέψεις για αποζημίωση προσωπικού λόγω εξόδου από την
                                       υπηρεσία

                     .............

                     78.05.09    Λοιπές προβλέψεις εκμεταλλεύσεως

                     ...........

                     78.05.99

        78.06

        ........

        78.10    Έσοδα από ιδιόχρηση αποθεμάτων (Γνωμ. 44/1129/1989) 

                     78.10.00    Αξία χορηγούμενων αποθεμάτων στο προσωπικό

                               01    Αξία χορηγούμενων δειγμάτων (δωρεάν) 

                               02    Αξία δωρεών αποθεμάτων για κοινωφελείς σκοπούς

                               03    Αξία σημαντικών δωρεών αποθεμάτων για κοινωφελείς
                                       σκοπούς

                               04    Ζημίες από καταστροφή ανασφάλιστων αποθεμάτων
                                       (Γνωμ 51/1282/1990)

                               05    Ζημίες από απώλεια ή κλοπή ανασφάλιστων αποθεμάτων
                                       (Γνωμ. 217/2177/1994)

                               06    ................................................

                               08    Αξία ιδιοχρησιμοποιούμενων αποθεμάτων ως παγίων
                                       (Γνωμ. 217/2177/1994, 251/2242/1995)

        78.11    Αξία καταστραφέντων ακατάλληλων αποθεμάτων
                        (Γνωμ. 51/1282/1990)

        ........

        78.99    Προϋπολογισμένη παραγωγή ιδιοχρησιμοποιουμένων πάγιων
                     στοιχείων και προϋπολογισμένη χρησιμοποίηση προβλέψεων Λ/58.78)

2.2.712 Λογαριασμός 78 «Ιδιοπαραγωγή παγίων και χρησιμοποιημένες προβλέψεις εκμεταλλεύσεως»

1. Ο λογαριασμός 78.00 «ιδιοπαραγωγή και βελτιώσεις παγίων» πιστώνεται, με χρέωση των οικείων λογαριασμών της ομάδας 1, με το κόστος παραγωγής των πάγιων στοιχείων που κατασκευάζονται ή δημιουργούνται από την οικονομική μονάδα με δικά της μέσα και για δική της χρήση, καθώς και με το κόστος βελτιώσεως των πάγιων στοιχείων. Το κόστος αυτό προσδιορίζεται από τους λογαριασμούς της αναλυτικής λογιστικής εκμεταλλεύσεως της ομάδας 9, ή αν δε λειτουργεί η λογιστική αυτή, εξωλογιστικά με υπολογισμούς που βασίζονται σε λογιστικά στοιχεία.

2. Για το λογαριασμό 78.05 «χρησιμοποιημένες προβλέψεις προς κάλυψη εξόδων εκμεταλλεύσεως» ισχύον όσα καθορίζονται στην περίπτ. 5-β της παρ. 2.2.405 για το λογαριασμό 44 «προβλέψεις».

3. Ο λογαριασμός 78 λειτουργεί σύμφωνα με όσα αναφέρονται στην παρ. 2.2.704 για το λογαριασμό 70, σε συνδυασμό και με όσα καθορίζονται στις παραπάνω περιπτώσεις 1-2 και στην παρ. 2.2.109.

Σημείωση: (1) Όταν η αξία των δωρούμενων αποθεμάτων βρίσκεται μέσα στα συνηθισμένα όρια κοινωνικής παραστάσεως της επιχειρήσεως καταχωρείται στη χρέωση του ομώνυμου λογ/σμού 64.06.02, ενώ όταν πρόκειται για μία έκτακτη, ασυνήθιστη και σημαντικής αξίας δωρεά χρεώνεται ο λογαριασμός 81.00.05 και έτσι δεν επιβαρύνεται η εκμετάλλευση ούτε το λειτουργικό κόστος. Υπόψη ότι η αξία των δωριζομένων κινητών αγαθών δεν εκπίπτει από τα ακαθάριστα έσοδα από τον υπολογισμό των φορολογητέων κερδών (άρθρο 31 παρ. 1 περίπτ. Α υποπ. γγ' Ν. 2238/1994.

(2) Εξυπακούεται ότι δεν χρησιμοποιείται ο λογ/σμός 78.10 (ούτε βέβαια χρεώνεται ο 81.02) όταν τα αποθέματα που καταστρέφονται ή κλέβονται είναι ασφαλισμένα, γιατί εισπράττεται η ασφαλιστική αποζημίωση (που καταχωρείται στον 72 κλπ., βλ. ανωτ. το λογ/σμό 72). (Α.Υ.Ο. 1116202/885/0015/ΠΟΛ. 1282/30-10-1996).

79    ΟΡΓΑΝΙΚΑ ΕΣΟΔΑ ΚΑΤ' ΕΙΔΟΣ ή ΑΛΛΩΝ ΚΕΝΤΡΩΝ
        (Όμιλος λογαριασμών προαιρετικής χρήσεως)

        790    ΠΩΛΗΣΕΙΣ ΕΜΠΟΡΕΥΜΑΤΩΝ
                  Ανάπτυξη αντίστοιχη του λογ. 70

        791    ΠΩΛΗΣΕΙΣ ΠΡΟΪΟΝΤΩΝ ΕΤΟΙΜΩΝ ΚΑΙ ΗΜΙΤΕΛΩΝ
                  Ανάπτυξη αντίστοιχη του λογ. 71

        792    ΠΩΛΗΣΕΙΣ ΛΟΙΠΩΝ ΑΠΟΘΕΜΑΤΩΝ ΚΑΙ ΑΧΡΗΣΤΟΥ ΥΛΙΚΟΥ
                  Ανάπτυξη αντίστοιχη του λογ. 72

        793    ΠΩΛΗΣΕΙΣ ΥΠΗΡΕΣΙΩΝ (έσοδα από παροχή υπηρεσιών)
                  Ανάπτυξη αντίστοιχη του λογ. 73

        794    ΕΠΙΧΟΡΗΓΗΣΕΙΣ ΚΑΙ ΔΙΑΦΟΡΑ ΕΣΟΔΑ ΠΩΛΗΣΕΩΝ
                  Ανάπτυξη αντίστοιχη του λογ. 74

        795    ΕΣΟΔΑ ΠΑΡΕΠΟΜΕΝΩΝ ΑΣΧΟΛΙΩΝ
                 Ανάπτυξη αντίστοιχη του λογ. 75

        796    ΕΣΟΔΑ ΚΕΦΑΛΑΙΩΝ
                 Ανάπτυξη αντίστοιχη του λογ. 76

        797    ................................................

        798    ΙΔΙΟΠΑΡΑΓΩΓΗ ΠΑΓΙΩΝ ΚΑΙ ΧΡΗΣΙΜΟΠΟΙΗΜΕΝΕΣ
                  ΠΡΟΒΛΕΨΕΙΣ ΕΚΜΕΤΑΛΛΕΥΣΕΩΣ
                 Ανάπτυξη αντίστοιχη του λογ. 78

2.2.713 Όμιλος λογαριασμών 79 «Οργανικά έσοδα κατ' είδος υποκαταστημάτων
ή άλλων κέντρων» (όμιλος λογαριασμών προαιρετικής χρήσεως)

1. Σχετικά με τον τρόπο αναπτύξεως κάθε πρωτοβάθμιου λογαριασμού (790-798) ισχύουν όσα καθορίζονται στην περίπτ. 1 της παρ. 2.2.113.

2. Σχετικά με τον τρόπο λειτουργίας των πρωτοβάθμιων λογαριασμών 790-798 ισχύουν, αντίστοιχα, όσα καθορίζονται παραπάνω στις παρ. 2.2.700 έως και 2.2.712 για τους πρωτοβάθμιους λογαριασμούς 70-78.

\chapter{ΛΟΓΑΡΙΑΣΜΟΙ ΑΠΟΤΕΛΕΣΜΑΤΩΝ}

\section{Λογαριασμοί}

\begin{tabularx}{\linewidth}{lX}

\end{tabularx}

80 Γενική εκμετάλλευση

81 Έκτακτα και ανόργανα αποτελέσματα

82 Έξοδα και έσοδα προηγούμενων χρήσεων

83 Προβλέψεις για έκτακτους κινδύνους

84 Έσοδα από προβλέψεις προηγούμενων χρήσεων

85 Αποσβέσεις παγίων μη ενσωματωμένες στο λειτουργικό κόστος

86 Αποτελέσματα χρήσεως

87 ..........................................

88 Αποτελέσματα προς διάθεση

89 Ισολογισμός  

2.2.8 ΟΜΑΔΑ 8η: ΛΟΓΑΡΙΑΣΜΟΙ ΑΠΟΤΕΛΕΣΜΑΤΩΝ

2.2.800 Περιεχόμενο - Ανάπτυξη λογαριασμών

1. Στην ομάδα 8 περιλαμβάνονται οι λογαριασμοί προσδιορισμού των αποτελεσμάτων εκμεταλλεύσεως, μικτών και καθαρών, καθώς και οι λογαριασμοί συγκεντρώσεως των μη προσδιοριστικών των μικτών κερδών εξόδων και εσόδων εκμεταλλεύσεως. Στην ίδια ομάδα περιλαμβάνονται οι λογαριασμοί συγκεντρώσεως των έκτακτων και ανόργανων αποτελεσμάτων, των εξόδων και εσόδων προηγούμενων χρήσεων, των προβλέψεων για έκτακτους κινδύνους, των εσόδων από προβλέψεις προηγούμενων χρήσεων και των μη ενσωματωμένων στο λειτουργικό κόστος αποσβέσεων πάγιων στοιχείων, καθώς και οι λογαριασμοί προσδιορισμού και διαθέσεως των αποτελεσμάτων χρήσεως.

2. Οι πρωτοβάθμιοι λογαριασμοί της ομάδας 8 αναπτύσσονται σε δευτεροβάθμιους υποχρεωτικούς λογαριασμούς και αυτοί αναπτύσσονται σε τριτοβάθμιους και αναλυτικότερους υπολογαριασμούς, σύμφωνα με τις ανάγκες κάθε οικονομικής μονάδας, με τον περιορισμό να τηρούνται οι υποχρεωτικοί τριτοβάθμιοι λογαριασμοί που προβλέπονται από το Σχέδιο Λογαριασμών.

2.2.801 Δυνητική ευχέρεια αναπτύξεως λογαριασμών τρίτου βαθμού

Η υποδεικνυόμενη από το Σχέδιο Λογαριασμών ανάπτυξη των λογαριασμών τρίτου βαθμού, στους οποίους αναλύονται οι δευτεροβάθμιοι των λογαριασμών 81-85, είναι ενδεικτική.

Κάθε οικονομική μονάδα έχει τη δυνατότητα, αντί να αναπτύξει κατ' είδος τους τριτοβάθμιους λογαριασμούς των 81 έως και 85, να τους αναπτύξει κατά προορισμό.

Στην περίπτωση όμως αυτή οι υποχρεωτικοί τριτοβάθμιοι λογαριασμοί των εξόδων ή εσόδων κατ' είδος εμφανίζονται υποχρεωτικά σαν αναλυτικοί των περιληπτικών κατά προορισμό λογαριασμών, στους οποίους θα αναλύονται οι δευτεροβάθμιοι κατ' είδος λογαριασμοί των 81-85 πρωτοβαθμίων.

2.2.802 Περιοδική κατανομή έκτακτων και ανόργανων αποτελεσμάτων μέσα στη χρήση

Σε περίπτωση που η οικονομική μονάδα προσδιορίζει βραχύχρονα (π.χ. μηνιαία ή τριμηνιαία) αποτελέσματα ή καταρτίζει περιοδικές συγκρίσιμες καταστάσεις, η χρονική εναρμόνιση των έκτακτων και ανόργανων αποτελεσμάτων γίνεται, είτε με απευθείας χρεοπίστωση των οικείων αποτελεσματικών λογαριασμών, είτε με παρεμβολή ενδιάμεσων λογαριασμών (81.99, 82.99, 83.99, 84.99 και 85.99), σύμφωνα με όσα καθορίζονται στην παρ. 2.2.509.

80    ΓΕΝΙΚΗ ΕΚΜΕΤΑΛΛΕΥΣΗ

        80.00    Λογαριασμός Γενικής Εκμεταλλεύσεως

        80.01    Μικτά αποτελέσματα (κέρδη ή ζημίες) εκμεταλλεύσεως

        80.02    Έξοδα μη προσδιοριστικά των μικτών αποτελεσμάτων

                     80.02.00    Έξοδα διοικητικής λειτουργίας

                               01    Έξοδα λειτουργίας ερευνών - αναπτύξεως

                               02    Έξοδα λειτουργίας διαθέσεως

                               03    Έξοδα λειτουργίας παραγωγής μη κοστολογηθέντα
                                       (κόστος υποαπασχολήσεως - αδράνειας) (Γνωμ. 46/1189/1989)

                               04    Προβλέψεις υποτιμήσεως συμμετοχών και χρεογράφων

                               05    Έξοδα και ζημίες συμμετοχών και χρεογράφων

                               06    Χρεωστικοί τόκοι και συναφή έξοδα

                     ..............

                     80.02.99

        80.03    Έσοδα μη προσδιοριστικά των μικτών αποτελεσμάτων

                     80.03.00    Άλλα έσοδα εκμεταλλεύσεως

                               01    Έσοδα συμμετοχών

                               02    Έσοδα χρεογράφων

                               03    Κέρδη πωλήσεως συμμετοχών και χρεογράφων

                               04    Πιστωτικοί τόκοι και συναφή έξοδα

                     ..............

                     80.03.99

         .........

         80.99

2.2.803 Λογαριασμός 80 «Γενική Εκμετάλλευση»

1. Ο λογαριασμός 80 χρησιμοποιείται μόνο στο τέλος της χρήσεως, οπότε καταρτίζεται υποχρεωτικά η κατάσταση του λογαριασμού γενικής εκμεταλλεύσεως σύμφωνα με το υπόδειγμα της παρ. 4.1.402. Ο λογαριασμός αυτός, ο οποίος, μαζί με το λογαριασμό 86 «αποτελέσματα χρήσεως», αποτελεί το αναγκαίο και αναπόσπαστο συμπλήρωμα του ισολογισμού, καταχωρείται στο βιβλίο απογραφών και ισολογισμών αμέσως μετά την καταχώριση του ισολογισμού και του λογαριασμού αποτελεσμάτων χρήσεως.

2. Σε αντίθεση με τον ισολογισμό και το λογαριασμό αποτελεσμάτων χρήσεως, που δημοσιεύονται σύμφωνα με τις διατάξεις της νομοθεσίας που ισχύει κάθε φορά, ο λογαριασμός της γενικής εκμεταλλεύσεως δε δημοσιεύεται υποχρεωτικά.

3. Ο λογαριασμός 80.00 «λογαριασμός γενικής εκμεταλλεύσεως» χρησιμεύει για τον προσδιορισμό των καθαρών τακτικών και οργανικών αποτελεσμάτων τα οποία πραγματοποιούνται, μέσα στη χρήση που κλείνει, από την εκμετάλλευση των διάφορων δραστηριοτήτων της οικονομικής μονάδας (κύριας, παρεπόμενων και δευτερεύουσας σημασίας).

4. Στο λογαριασμό 80.00, στο τέλος της χρήσεως, μεταφέρονται τα αρχικά αποθέματα τα τελικά αποθέματα και οι αγορές των λογαριασμών της ομάδας 2, τα έξοδα των λογαριασμών της ομάδας 6, εκτός από τους μη ενσωματωμένους στο λειτουργικό κόστος φόρους (π.χ. λογ. 63.98.02 «φόρος ακίνητης περιουσίας»), και τα έσοδα των λογαριασμών της ομάδας 7, αφού προηγουμένως οι λογαριασμοί των ομάδων αυτών υποστούν τις αναγκαίες τακτοποιήσεις, σύμφωνα με όσα καθορίζονται στις παρ. 2.2.202, 2.2.602 και 2.2.702, έτσι ώστε τα τελικά υπόλοιπά τους να αντιπροσωπεύουν τα δουλευμένα, τακτικά και οργανικά έξοδα και έσοδα της χρήσεως, δηλαδή εκείνα που αφορούν την ομαλή εκμετάλλευση της χρήσεως που κλείνει.

Ειδικότερα ο λογαριασμός 80.00 λειτουργεί ως εξής:

Ι. Χρεώνεται:

- με την αξία των αρχικών αποθεμάτων, δηλαδή των αποθεμάτων που υπήρχαν στην αρχή της χρήσεως που κλείνει, με πίστωση των οικείων υπολογαριασμών των πρωτοβάθμιων 20-28 ή των 290-298, κατά περίπτωση.

- με την αξία των αγορών εμπορευμάτων, πρώτων και βοηθητικών υλών - υλικών συσκευασίας, αναλώσιμων υλικών, ανταλλακτικών πάγιων στοιχείων και ειδών συσκευασίας, που έγιναν μέσα στη χρήση που κλείνει, με πίστωση των οικείων υπολογαριασμών των πρωτοβάθμιων 20, 24, 25, 26 και 28 ή των 290, 294, 295, 296 και 298, κατά περίπτωση.

- με την αξία των δουλευμένων εξόδων κατ' είδος, με πίστωση των οικείων λογαριασμών της ομάδας 6, δηλαδή των 60-68 ή των 690-698, κατά περίπτωση, οι οποίοι εξισώνονται.

- κατά περίπτωση, με τα καθαρά κέρδη εκμεταλλεύσεως της χρήσεως που κλείνει, με πίστωση του λογαριασμού 80.01.

ΙΙ. Πιστώνεται

- με την αξία των δουλευμένων εσόδων κατ' είδος, με χρέωση των οικείων λογαριασμών της ομάδας 7, δηλαδή των 70-78 ή των 790-798, κατά περίπτωση, οι οποίοι εξισώνονται.

- με την αξία των τελικών αποθεμάτων, δηλαδή των αποθεμάτων που προσδιορίζονται έπειτα από απογραφή στο τέλος της χρήσεως που κλείνει, όπως η αξία αυτή προσδιορίζεται με την αποτίμηση της ποσοτικής απογραφής, με χρέωση των οικείων υπολογαριασμών των πρωτοβάθμιων 20-28 ή των 290-298, κατά περίπτωση.

- κατά περίπτωση, με την καθαρή ζημία εκμεταλλεύσεως της χρήσεως που κλείνει, με χρέωση του λογαριασμού 80.01.

5. Ο λογαριασμός 80.01 «μικτά αποτελέσματα (κέρδη ή ζημίες) εκμεταλλεύσεως» χρησιμεύει για τον προσδιορισμό των μικτών αποτελεσμάτων (μικτών κερδών ή μικτών ζημιών), τα οποία πραγματοποιούνται, μέσα στη χρήση που κλείνει, από την εκμετάλλευση των διάφορων δραστηριοτήτων της οικονομικής μονάδας.

Στο λογαριασμό 80.01, στο τέλος της χρήσεως, μεταφέρονται τα καθαρά αποτελέσματα εκμεταλλεύσεως (καθαρά κέρδη ή καθαρές ζημίες), σύμφωνα με όσα καθορίζονται στην προηγούμενη περίπτ. 4. Έπειτα από τη μεταφορά αυτή, από το λογαριασμό 80.01 μεταφέρονται στο λογαριασμό 80.02 τα μη προσδιοριστικά των μικτών αποτελεσμάτων έξοδα, δηλαδή τα έξοδα διοικητικής λειτουργίας, τα έξοδα ερευνών και αναπτύξεως, τα έξοδα λειτουργίας διαθέσεως, οι διαφορές αποτιμήσεως συμμετοχών και χρεογράφων, τα έξοδα και οι ζημίες συμμετοχών και χρεογράφων και οι χρεωστικοί τόκοι και τα συναφή με αυτούς έξοδα. Από τον ίδιο λογαριασμό (80.01) μεταφέρονται στο λογαριασμό 80.03 τα μη προσδιοριστικά των μικτών αποτελεσμάτων έσοδα, δηλαδή τα διάφορα άλλα έσοδα, τα έσοδα συμμετοχών, τα έσοδα χρεογράφων, τα κέρδη από πωλήσεις συμμετοχών και χρεογράφων και οι πιστωτικοί τόκοι και τα συναφή με αυτούς έσοδα.

Έπειτα από τις παραπάνω μεταφορές και τις αντίστοιχες χρεοπιστώσεις του ο λογαριασμός 80.01 με το υπόλοιπό του (χρεωστικό ή πιστωτικό) απεικονίζει το οριστικό ύψος των μικτών αποτελεσμάτων εκμεταλλεύσεως (μικτών κερδών ή μικτών ζημιών).

Έξοδα μη προσδιοριστικά των μικτών αποτελεσμάτων είναι εκείνα τα οποία σύμφωνα με τις αρχές και τους κανόνες λειτουργίας της αναλυτικής λογιστικής εκμεταλλεύσεως, που περιγράφονται στο πέμπτο μέρος, τελικά δε βαρύνουν τα αποθέματα (τελικά αποθέματα, κόστος πωλημένων), αλλά τα αποτελέσματα χρήσεως.

Τα έξοδα αυτά προκύπτουν από τους οικείους λογαριασμούς της ομάδας 9, και ειδικότερα από τους λογαριασμούς 92.01 «έξοδα διοικητικής λειτουργίας», 92.02 «έξοδα λειτουργίας ερευνών και αναπτύξεως» και 92.03 «έξοδα λειτουργίας διαθέσεως» και από τους οικείους λογαριασμούς εξόδων κατ' είδος της ομάδας 6, και ειδικότερα από τους λογαριασμούς 64.10 «έξοδα συμμετοχών και χρεογράφων», 64.11 «διαφορές αποτιμήσεως συμμετοχών και χρεογράφων», 64.12 «διαφορές από πώληση συμμετοχών και χρεογράφων» και 65 «χρεωστικοί τόκοι και συναφή έξοδα».

Στις περιπτώσεις εκείνες που δε λειτουργεί λογιστικό σύστημα αναλυτικής λογιστικής εκμεταλλεύσεως, τα «έξοδα διοικητικής λειτουργίας», τα «έξοδα λειτουργίας ερευνών και αναπτύξεως» και τα «έξοδα λειτουργίας διαθέσεως» προσδιορίζονται εξωλογιστικά με βάση τα στοιχεία που προκύπτουν από τους λογαριασμούς της γενικής λογιστικής.

Έσοδα μη προσδιοριστικά των μικτών αποτελεσμάτων είναι εκείνα τα οποία δε συνυπολογίζονται στα έσοδα που συσχετίζονται με το κόστος πωλημένων, προκειμένου να προσδιοριστούν τα μικτά κέρδη ή οι μικτές ζημίες. Τα έσοδα αυτά προκύπτουν από τους οικείους λογαριασμούς εσόδων κατ' είδος της ομάδας 7, και ειδικότερα από τους λογαριασμούς 74, 75 και 78.05 τα άλλα έσοδα εκμεταλλεύσεως, από το λογαριασμό 76.00 τα έσοδα συμμετοχών, από το λογαριασμό 76.01 τα έσοδα χρεογράφων, από το λογαριασμό 76.04 τα κέρδη πωλήσεως συμμετοχών και χρεογράφων και από τους λογαριασμούς 76.02-76.98, πλην 76.04, οι πιστωτικοί τόκοι και τα συναφή έσοδα.

Ειδικότερα, ο λογαριασμός 80.01 λειτουργεί ως εξής:

Ι. Χρεώνεται:

- κατά περίπτωση, με την καθαρή ζημία εκμεταλλεύσεως της χρήσεως που κλείνει, με πίστωση του λογαριασμού 80.00, ο οποίος εξισώνεται.

- με τα μη προσδιοριστικά των μικτών αποτελεσμάτων έσοδα, όπως προσδιορίζονται παραπάνω, με πίστωση των οικείων υπολογαριασμών του 80.03.

- κατά περίπτωση, με τα μικτά κέρδη εκμεταλλεύσεως της χρήσεως που κλείνει, με πίστωση του λογαριασμού 86.00.00 «μικτά αποτελέσματα (κέρδη ή ζημίες)».

ΙΙ. Πιστώνεται:

- κατά περίπτωση, με τα καθαρά κέρδη εκμεταλλεύσεως της χρήσεως που κλείνει, με χρέωση του λογαριασμού 80.00, ο οποίος εξισώνεται,

- με τα μη προσδιοριστικά των μικτών αποτελεσμάτων έξοδα, όπως προσδιορίζονται παραπάνω, με χρέωση των οικείων υπολογαριασμών του 80.02,

- κατά περίπτωση, με τις μικτές ζημίες εκμεταλλεύσεως της χρήσεως που κλείνει, με χρέωση του λογαριασμού 86.00.00.

6. Ο λογαριασμός 80.02 «έξοδα μη προσδιοριστικά των μικτών αποτελεσμάτων» χρησιμεύει για τη συγκέντρωση των μη προσδιοριστικών των μικτών αποτελεσμάτων εξόδων, σύμφωνα με όσα παραπάνω καθορίζονται.

Τελικά, ο λογαριασμός 80.02 εξισώνεται με τη μεταφορά του υπολοίπου του στους οικείους υπολογαριασμούς του 86 «αποτελέσματα χρήσεως» και ειδικότερα στους υπολογαριασμούς 86.00.02 «έξοδα διοικητικής λειτουργίας», 86.00.03 «έξοδα λειτουργίας ερευνών και αναπτύξεως», 86.00.04 «έξοδα λειτουργίας διαθέσεως», 86.01.07 «διαφορές αποτιμήσεως συμμετοχών και χρεογράφων», 86.01.08 «έξοδα και ζημίες συμμετοχών και χρεογράφων» και 86.01.09 «χρεωστικοί τόκοι και συναφή έξοδα», οι οποίοι είναι αντίστοιχοι των υπολογαριασμών του 80.02.

7. Ο λογαριασμός 80.03 «έσοδα μη προσδιοριστικά των μικτών αποτελεσμάτων» χρησιμεύει για τη συγκέντρωση των μη προσδιοριστικών των μικτών αποτελεσμάτων εσόδων, σύμφωνα με όσα παραπάνω καθορίζονται.

Τελικά, ο λογαριασμός 80.03 εξισώνεται με τη μεταφορά του υπολοίπου του στους οικείους υπολογαριασμούς του 86 και ειδικότερα στους υπολογαριασμούς 86.00.01 «άλλα έσοδα εκμεταλλεύσεως», 86.01.00, «έσοδα συμμετοχών», 86.01.01 «έσοδα χρεογράφων», 86.01.02 «κέρδη πωλήσεως συμμετοχών και χρεογράφων» και 86.01.03 «πιστωτικοί τόκοι και συναφή έσοδα», οι οποίοι είναι αντίστοιχοι των υπολογαριασμών του 80.03.

81    ΕΚΤΑΚΤΑ ΚΑΙ ΑΝΟΡΓΑΝΑ ΑΠΟΤΕΛΕΣΜΑΤΑ

        81.00    Έκτακτα και ανόργανα έξοδα

                     81.00.00    Φορολογικά πρόστιμα και προσαυξήσεις

                               01    Προσαυξήσεις εισφορών ασφαλιστικών ταμείων

                               02    Καταπτώσεις εγγυήσεων - ποινικών ρήτρων

                               03    Κλοπές - Υπεξαιρέσεις

                               04    Συναλλαγματικές διαφορές

                               05    Αξία σημαντικών δωρεών αποθεμάτων για κοινωφελείς σκοπούς
                                       (Γνωμ. 44/1129/1989)

                     .............

                     81.00.10    Έκτακτη εφάπαξ εισφορά επί ακινήτων Ν. 2019/1992

                     81.00.11    Φόρος επί υπεραξίας αναπροσαρμογής ακινήτων
                                       (άρθρο 24 Ν. 2065/1992)

                     ...............

                     81.00.99    Λοιπά έκτακτα και ανόργανα έξοδα

        81.01    Έκτακτα και ανόργανα έσοδα

                     81.01.00

                               01

                               02    Καταπτώσεις εγγυήσεων - ποινικών ρήτρων

                               03

                               04    Συναλλαγματικές διαφορές

                               05    Αναλογούσες στη χρήση επιχορηγήσεις πάγιων επενδύσεων

                     ..............

                     81.01.99    Λοιπά έκτακτα και ανόργανα έσοδα

        81.02    Έκτακτες ζημίες

                     81.02.00    Ζημίες από εκποίηση ακινήτων

                               01    Ζημίες από εκποίηση τεχνικών έργων

                               02    Ζημίες από εκποίηση μηχανημάτων - τεχνικών εγκαταστάσεων
                                       - λοιπού μηχανολογικού εξοπλισμού

                               03    Ζημίες από εκποίηση μεταφορικών μέσων

                               04    Ζημίες από εκποίηση επίπλων και λοιπού εξοπλισμού

                               05    Ζημίες από μεταβίβαση δικαιωμάτων και λοιπών ασώματων
                                       ακινητοποιήσεων

                               06    Ζημίες από ανεπίδεκτες εισπράξεως απαιτήσεις

                               07    Ζημίες από καταστροφή ανασφάλιστων αποθεμάτων
                                       (Γνωμ. 51/1282/1990)

                               08    Ζημίες από απώλεια ή κλοπή ανασφάλιστων αποθεμάτων
                                      (Γνωμ. 217/2177/1994)

                     .............

                     81.02.10    Ζημίες από καταστροφή ακατάλληλων αποθεμάτων
                                      (Γνωμ. 51/1282/1990)

                     ..............

                     81.02.99    Λοιπές έκτακτες ζημίες

        81.03    Έκτακτα κέρδη

                    81.03.00    Κέρδη από εκποίηση ακινήτων

                               01    Κέρδη από εκποίηση τεχνικών έργων

                               02    Κέρδη από εκποίηση μηχανημάτων - τεχνικών εγκαταστάσεων
                                       - λοιπού μηχανολογικού εξοπλισμού

                               03    Κέρδη από εκποίηση μεταφορικών μέσων

                               04    Κέρδη από εκποίηση επίπλων και λοιπού εξοπλισμού

                               05    Κέρδη από μεταβίβαση δικαιωμάτων και λοιπών ασώματων
                                       ακινητοποιήσεων

                               06

                               07    Κέρδη από λαχνούς ομολογιακών δανείων

                     ...............

                     81.03.99    Λοιπά έκτακτα κέρδη

        81.04

        .........

        81.99    Προϋπολογισμένα - Προπληρωμένα έκτακτα και ανόργανα
                     αποτελέσματα (Λ/58.81)

2.2.804 Λογαριασμός 81 «Έκτακτα και ανόργανα αποτελέσματα»

1. Στο λογαριασμό 81 καταχωρούνται κατ' είδος τα έκτακτα και ανόργανα έξοδα και έσοδα της χρήσεως, καθώς και τα αποτελέσματα που πραγματοποιούνται από εξαιρετικές και έκτακτες πράξεις και εργασίες.

Η ανάλυση του λογαριασμού 81 σε δευτεροβάθμιους και τριτοβάθμιους λογιαριασμούς, κυρίως υποχρεωτικούς, περιλαμβάνει τις κυριότερες γνωστές κατηγορίες έκτακτων και ανόργανων εξόδων και αποτελεσμάτων. Η οικονομική μονάδα έχει τη δυνατότητα να δημιουργεί και άλλους τριτοβάθμιους λογαριασμούς για την ιδιαίτερη παρακολούθηση των περιπτώσεων που παρουσιάζονται, οπότε περιορίζεται το περιεχόμενο των προαιρετικών τριτοβάθμιων λογαριασμών 81.00.99, 81.01.99, 81.02.99 και 81.03.99.

2. Στο λογαριασμό 81.00 «έκτακτα και ανόργανα έξοδα» καταχωρούνται κατ' είδος τα έκτακτα και ανόργανα έξοδα που αφορούν τη χρήση. Στο λογαριασμό αυτό δεν καταχωρούνται έξοδα που αφορούν προηγούμενες χρήσεις. Τα τελευταία αυτά έξοδα καταχωρούνται στο λογαριασμό 82.00.

Ειδικά για τα φορολογικά πρόστιμα και τις προσαυξήσεις τους διευκρινίζεται ότι, αν πρόκειται για περιπτώσεις που δεν έχει προηγηθεί η άσκηση προσφυγής στα αρμόδια δικαστήρια, καταχωρούνται στο λογαριασμό 81.00, αλλιώς εφαρμόζονται όσα καθορίζονται για το λογαριασμό 33.98 στις περιπτ. 17 και 18 της παρ.  2.2.304.

Με την επιφύλαξη των σχετικών διατάξεων των περιπτ. 17 και 23 της παρ. 2.2.110, για τις συναλλαγματικές διαφορές που προκύπτουν κατά την αποτίμηση των απαιτήσεων, των λοιπών υποχρεώσεων και των διαθεσίμων της παρ. 2.3.2., καθώς και για εκείνες που προκύπτουν κατά την είσπραξη ή πληρωμή τους, ισχύουν όσα ορίζονται στην περίπτ. 4 της παρ. 2.3.2. Για τις συναλλαγματικές διαφορές που προκύπτουν κατά τη μετατροπή σε δραχμές των στοιχείων της περιπτ. 2 της υποπαρ.  2.3.302, ισχύουν όσα ορίζονται στην περίπτ. 3 της υποπαρ. 2.3.302.

3. Στο λογαριασμό 81.01 «έκτακτα και ανόργανα έσοδα» καταχωρούνται, κατ' είδος, τα έκτακτα και ανόργανα έσοδα που αφορούν τη χρήση. Στο λογαριασμό αυτό δεν καταχωρούνται έσοδα που αφορούν προηγούμενες χρήσεις. Τα τελευταία αυτά έσοδα καταχωρούνται στο λογαριασμό 82.01.

Ειδικά για τις πιστωτικές συναλλαγματικές διαφορές, που καταχωρούνται στον υπολογαριασμό 81.01.04, ισχύουν ανάλογα όσα αναφέρονται στην αμέσως προηγούμενη περίπτωση 2.

4. Στους λογαριασμούς 81.02 «έκτακτες ζημίες» και 81.03 «έκτακτα κέρδη» καταχωρούνται τα αποτελέσματα - ζημίες ή κέρδη - που προκύπτουν από εξαιρετικές και έκτακτες πράξεις και εργασίες, όπως π.χ. από εκποίηση πάγιων στοιχείων, από μεταβίβαση δικαιωμάτων και λοιπών ασώματων ακινητοποιήσεων, από ανεπίδεκτες εισπράξεως απαιτήσεις ή από λαχνούς ομολογιακών δανείων.

5. Κατά το κλείσιμο του ισολογισμού τα υπόλοιπα των υπολογαριασμών του 81 μεταφέρονται στους αντίστοιχους υπολογαριασμούς του 86.02 «έκτακτα και ανόργανα αποτελέσματα», έτσι ώστε ο λογαριασμός 81 να εξισώνεται.

82    ΕΞΟΔΑ ΚΑΙ ΕΣΟΔΑ ΠΡΟΗΓΟΥΜΕΝΩΝ ΧΡΗΣΕΩΝ

        82.00    Έξοδα προηγούμενων χρήσεων

                     82.00.00    Φορολογικά πρόστιμα και προσαυξήσεις

                               01    Προσαυξήσεις εισφορών ασφαλιστικών ταμείων

                               02    Καταπτώσεις εγγυήσεων - ποινικών ρήτρων

                               03    Κλοπές - Υπεξαιρέσεις

                               04    Φόροι και τέλη προηγούμενων χρήσεων (πλην φόρου
                                       εισοδήματος)

                               05    Οριστικοποιημένοι επίδικοι φόροι Δημοσίου (πλην φόρου
                                       εισοδήματος)

                               06    Εισφορές ασφαλιστικών ταμείων προηγούμενων χρήσεων

                               07    Χρεωστικές διαφορές μεταβατικού λογ/σμού 36.01
                                       (Γνωμ. 176/2087/1993)

                               08    Χρεωστικές διαφορές μεταβατικού λογ/σμού 36.03
                                       (Γνωμ. 176/2087/1993)

                               09    Χρεωστικές διαφορές μεταβατικού λογ/σμού 56.01
                                       (Γνωμ. 176/2087/1993)

                               10    Χρεωστικές διαφορές μεταβατικού λογ/σμού 56.03
                                       (Γνωμ. 176/2087/1993)

                     ............

                     82.00.60    Αποζημιώσεις απολύσεως ή εξόδου από την υπηρεσία
                                       (Γνωμ. 91/1683/1992)

                               61    ........
                               62    ........ (Τριτοβάθμιοι κατ'είδος εξόδων για τα οποία είχαν
                               63    ........ σχηματισθεί προβλέψεις σε προηγούμενες χρήσεις)
                               64    ........

                     ..............

                     82.00.99    Λοιπά έξοδα προηγούμενων χρήσεων

        82.01    Έσοδα προηγούμενων χρήσεων

                     82.01.00    Επιχορηγήσεις πωλήσεων

                               01    Επιστροφές δασμών και λοιπών επιβαρύνσεων

                               02    Επιστροφές τόκων λόγω εξαγωγών

                               03    Εισπράξεις αποσβεσμένων απαιτήσεων

                               04    Επιστροφές αχρεωστήτως καταβλημένων φόρων και τελών
                                       (πλην φόρου εισοδήματος)

                               07    Πιστωτικές διαφορές μεταβατικού λογ/σμού 36.01
                                       (Γνωμ. 176/2087/1993)

                               08    Πιστωτικές διαφορές μεταβατικού λογ/σμού 36.03
                                       (Γνωμ. 176/2087/1993)

                               09    Πιστωτικές διαφορές μεταβατικού λογ/σμού 56.01
                                       (Γνωμ. 176/2087/1993)

                               10    Πιστωτικές διαφορές μεταβατικού λογ/σμού 56.03
                                       (Γνωμ. 176/2087/1993)

                     ..............

                     82.01.99    Λοιπά έσοδα προηγούμενων χρήσεων

        .........

        82.02

        .........

        82.99    Προϋπολογισμένα - Προπληρωμένα έξοδα και έσοδα προηγούμενων
                        χρήσεων (Λ/58.82)

2.2.805 Λογαριασμός 82 «Έξοδα και έσοδα προηγούμενων χρήσεων»

1. Στο λογαριασμό 82 καταχωρούνται κατ' είδος τα έξοδα και έσοδα που πραγματοποιούνται μεν μέσα στη χρήση, ο χρόνος όμως και τα αίτια δημιουργίας τους ανάγονται σε δραστηριότητες προηγούμενων χρήσεων.

Η ανάλυση του λογαριασμού 82 σε δευτεροβάθμιους και τριτοβάθμιους λογαριασμούς, κυρίως υποχρεωτικούς, περιλαμβάνει τις κυριότερες γνωστές κατηγορίες εξόδων και εσόδων προηγούμενων χρήσεων. Η οικονομική μονάδα έχει τη δυνατότητα να δημιουργεί και άλλους τριτοβάθμιους λογαριασμούς για την ιδιαίτερη παρακολούθηση των περιπτώσεων που παρουσιάζονται, οπότε περιορίζεται το περιεχόμενο των προαιρετικών τριτοβάθμιων λογαριασμών 82.00.99 και 82.01.99.

2. Στο λογαριασμό 82.00 «έξοδα προηγούμενων χρήσεων» καταχωρούνται κατ' είδος τα έξοδα προηγούμενων χρήσεων, όπως οι φόροι και τα τέλη που επιβάλλονται για φορολογικές υποχρεώσεις που δημιουργούνται από πράξεις ή παραλείψεις προηγούμενων χρήσεων, αλλά η αποδοχή της υποχρεώσεως για πληρωμή τους γίνεται μέσα στη χρήση που τρέχει, χωρίς να προηγηθεί άσκηση προσφυγής στα αρμόδια δικαστήρια. Στον ίδιο λογαριασμό καταχωρούνται φόροι και τέλη που βεβαιώνονται μετά από οριστικοποίηση αποφάσεων των αρμόδιων δικαστηρίων.

Στους παραπάνω φόρους δεν περιλαμβάνεται ο φόρος εισοδήματος που αφορά προηγούμενες χρήσεις. Ο φόρος αυτός καταχωρείται στο λογαριασμό 42.04, σύμφωνα με όσα καθορίζονται στην περίπτ. 4 της παρ. 2.2.403.

3. Στο λογαριασμό 82.01 «έσοδα προηγούμενων χρήσεων» καταχωρούνται κατ' είδος τα έσοδα προηγούμενων χρήσεων, όπως οι εισπράξεις από αποσβεσμένες απαιτήσεις, οι επιστροφές αχρεωστήτως καταβλημένων σε προηγούμενες χρήσεις φόρων και τελών, εκτός από τις επιστροφές φόρου εισοδήματος που καταχωρούνται στο λογαριασμό 42.04 σύμφωνα με όσα καθορίζονται στην περίπτ. 4 της παρ. 2.2.403, οι επιχορηγήσεις, οι επιστροφές δασμών και λοιπών επιβαρύνσεων και οι επιστροφές τόκων λόγω εξαγωγών που αφορούν προηγούμενες χρήσεις.

4. Κατά το κλείσιμο του ισολογισμού τα υπόλοιπα των υπολογαριασμών του 82 μεταφέρονται στους αντίστοιχους υπολογαριασμούς του 86.02 «έκτακτα και ανόργανα αποτελέσματα», έτσι ώστε ο λογαριασμός 82 να εξισώνεται.

 83    ΠΡΟΒΛΕΨΕΙΣ ΓΙΑ ΕΚΤΑΚΤΟΥΣ ΚΙΝΔΥΝΟΥΣ

        83.00

        .........

        83.10    Προβλέψεις απαξιώσεων και υποτιμήσεων πάγιων στοιχείων

                     83.10.18    Προβλέψεις για κάλυψη ζημίας από συμμετοχή σε κοινοπραξία
                                       ή ο.ε. ή ε.ε. (Γνωμ. 118/1845/1993)

       83.11    Προβλέψεις για επισφαλείς απαιτήσεις

        83.12    Προβλέψεις για εξαιρετικούς κινδύνους και έκτακτα έξοδα

        83.13    Προβλέψεις για έξοδα προηγούμενων χρήσεων

        .........

        83.98    Λοιπές έκτακτες προβλέψεις

        83.99    Προϋπολογισμένες προβλέψεις για έκτακτους κινδύνους (Λ/58.83)

2.2.806 Λογαριασμός 83 «Προβλέψεις για έκτακτους κινδύνους»

1. Ο λογαριασμός 83 λειτουργεί σύμφωνα με όσα καθορίζονται στην περίπτ. 5 της παρ. 2.2.405 για το λογαριασμό 44 «προβλέψεις».

2. Κατά το κλείσιμο του ισολογισμού τα υπόλοιπα των υπολογαριασμών του 83 μεταφέρονται στον υπολογαριασμό 86.02.10 «προβλέψεις για έκτακτους κινδύνους», έτσι ώστε ο λογαριασμός 83 να εξισώνεται.

84    ΕΣΟΔΑ ΑΠΟ ΠΡΟΒΛΕΨΕΙΣ ΠΡΟΗΓΟΥΜΕΝΩΝ ΧΡΗΣΕΩΝ

        84.00    Έσοδα από αχρησιμοποίητες προβλέψεις προηγούμενων χρήσεων

                     84.00.00    Από προβλέψεις για αποζημίωση προσωπικού λόγω εξόδου από
                                       την υπηρεσία

                               01    Από προβλέψεις για υποτιμήσεις συμμετοχών και χρεογράφων

                               02

                               ....

                               09    Από λοιπές προβλέψεις εκμεταλλεύσεως

                               10    Από προβλέψεις απαξιώσεων και υποτιμήσεων πάγιων
                                       στοιχείων

                               11    Από προβλέψεις για επισφαλείς απαιτήσεις

                               12    Από προβλέψεις για εξαιρετικούς κινδύνους και έκτακτα έξοδα

                               13    Από προβλέψεις για έξοδα προηγούμενων χρήσεων

                     ..............

                     84.00.99    Από λοιπές έκτακτες προβλέψεις

        84.01    Έσοδα από χρησιμοποιημένες προβλέψεις προηγούμενων χρήσεων
                     για έκτακτους κινδύνους

                     84.01.00

                               01

                     ..............

                               12    Από προβλέψεις για εξαιρετικούς κινδύνους και έκτακτα έξοδα

                               13    Από προβλέψεις για έξοδα προηγούμενων χρήσεων

                     ...............

                     84.01.99    Από λοιπές έκτακτες προβλέψεις

        84.02

        .........

        84.91    Έσοδα από χρησιμοποιημένες προβλέψεις προηγούμενων χρήσεων
                     προς κάλυψη εξόδων εκμεταλλεύσεως (Γνωμ. 91/1683/1992)

                     84.91.00    Από προβλέψεις για αποζημίωση προσωπικού λόγω εξόδου από
                                       την υπηρεσία

                     ..............

                     84.91.09    Από λοιπές προβλέψεις εκμεταλλεύσεως

                     ...............

                     84.91.99

        84.99    Προϋπολογισμένα έσοδα από προβλέψεις προηγούμενων χρήσεων
                    (Λ/58.84)

2.2.807 Λογαριασμός 84 «Έσοδα από προβλέψεις προηγούμενων χρήσεων»

1. Ο λογαριασμός 84 λειτουργεί σύμφωνα με όσα καθορίζονται στην περίπτ. 5 της παρ. 2.2.405 για το λογαριασμό 44 «προβλέψεις».

2. Κατά το κλείσιμο του ισολογισμού τα υπόλοιπα των υπολογαριασμών του 84 μεταφέρονται στον υπολογαριασμό 86.02.03 «έσοδα από προβλέψεις προηγούμενων χρήσεων», έτσι ώστε ο λογαριασμός 84 να εξισώνεται.

85    ΑΠΟΣΒΕΣΕΙΣ ΠΑΓΙΩΝ ΜΗ ΕΝΣΩΜΑΤΩΜΕΝΕΣ
        ΣΤΟ ΛΕΙΤΟΥΡΓΙΚΟ ΚΟΣΤΟΣ

        85.00    Αποσβέσεις εδαφικών εκτάσεων

                    85.00.00

                               01    Αποσβέσεις Ορυχείων

                               02    Αποσβέσεις Μεταλλείων

                               03    Αποσβέσεις Λατομείων

                               04    ....................................

                               05    Αποσβέσεις Φυτειών

                               06    Αποσβέσεις Δασών

                               07    ....................................

                               11    Αποσβέσεις Ορυχείων εκτός εκμεταλλεύσεως

                               12    Αποσβέσεις Μεταλλείων εκτός εκμεταλλεύσεως

                               13    Αποσβέσεις Λατομείων εκτός εκμεταλλεύσεως

                               14    .....................................

                               15    Αποσβέσεις Φυτειών εκτός εκμεταλλεύσεως

                               16    Αποσβέσεις Δασών εκτός εκμεταλλεύσεως

                               17    ......................................

                    ..............

                    85.00.99

        85.01    Αποσβέσεις κτιρίων - εγκαταστάσεις κτιρίων - τεχνικών έργων

                    85.01.00    Αποσβέσεις κτιρίων - εγκαταστάσεων κτιρίων

                    85.01.01    Αποσβέσεις τεχνικών έργων εξυπηρετήσεως μεταφορών

                               02    Αποσβέσεις λοιπών τεχνικών έργων

                               03    Αποσβέσεις διαμορφώσεως γηπέδων

                               04    ....................................

                               07    Αποσβέσεις κτιρίων - εγκαταστάσεων κτιρίων σε ακίνητα
                                       τρίτων

                               08    Αποσβέσεις τεχνικών έργων εξυπηρετήσεως μεταφορών σε
                                       ακίνητα τρίτων

                               09    Αποσβέσεις λοιπών τεχνικών έργων σε ακίνητα τρίτων

                               10    Αποσβέσεις διαμορφώσεως γηπέδων τρίτων

                               11    ........................................

                               14    Αποσβέσεις κτιρίων - εγκαταστάσεων κτιρίων εκτός
                                       εκμεταλλεύσεως

                               15    Αποσβέσεις τεχνικών έργων εξυπηρετήσεως μεταφορών εκτός
                                       εκμεταλλεύσεως

                               16    Αποσβέσεις λοιπών τεχνικών έργων εκτός εκμεταλλεύσεως

                               17    Αποσβέσεις διαμορφώσεων γηπέδων εκτός εκμεταλλεύσεως

                               18    ...........................................

                               21    Αποσβέσεις κτιρίων - εγκαταστάσεων κτιρίων σε ακίνητα
                                       τρίτων εκτός εκμεταλλεύσεως

                               22    Αποσβέσεις τεχνικών έργων εξυπηρετήσεως μεταφορών σε
                                       ακίνητα τρίτων εκτός εκμεταλλεύσεως

                               23    Αποσβέσεις λοιπών τεχνικών έργων σε ακίνητα τρίτων εκτός
                                       εκμεταλλεύσεως

                               24    Αποσβέσεις διαμορφώσεων γηπέδων τρίτων εκτός
                                       εκμεταλλεύσεως

                    ..............

                    85.01.99

        85.02    Αποσβέσεις μηχανημάτων - τεχνικών εγκαταστάσεων - λοιπού
                     μηχανολογικού εξοπλισμού

                    85.02.00    Αποσβέσεις μηχανημάτων

                               01    Αποσβέσεις τεχνικών εγκαταστάσεων

                               02    Αποσβέσεις φορητών μηχανημάτων «χειρός»

                               03    Αποσβέσεις εργαλείων

                               04    Αποσβέσεις καλουπιών - ιδιοσυσκευών

                               05    Αποσβέσεις μηχανολογικών οργάνων

                               06    Αποσβέσεις λοιπού μηχανολογικού εξοπλισμού

                               07    Αποσβέσεις μηχανημάτων σε ακίνητα τρίτων

                               08    Αποσβέσεις τεχνικών εγκαταστάσεων σε ακίνητα τρίτων

                               09    Αποσβέσεις λοιπού μηχανολογικού εξοπλισμού σε ακίνητα
                                       τρίτων

                               10    Αποσβέσεις μηχανημάτων εκτός εκμεταλλεύσεως

                               11    Αποσβέσεις τεχνικών εγκαταστάσεων εκτός εκμεταλλεύσεως

                               12    Αποσβέσεις φορητών μηχανημάτων «χειρός» εκτός
                                       εκμεταλλεύσεως

                               13    Αποσβέσεις εργαλείων εκτός εκμεταλλεύσεως

                               14    Αποσβέσεις καλουπιών - ιδιοσυσκευών εκτός εκμεταλλεύσεως

                               15    Αποσβέσεις μηχανολογικών οργάνων εκτός εκμεταλλεύσεως

                               16    Αποσβέσεις λοιπού μηχανολογικού εξοπλισμού εκτός
                                       εκμεταλλεύσεως

                               17    Αποσβέσεις μηχανημάτων σε ακίνητα τρίτων εκτός
                                       εκμεταλλεύσεως

                               18    Αποσβέσεις τεχνικών εγκαταστάσεων σε ακίνητα τρίτων εκτός
                                       εκμεταλλεύσεως

                               19    Αποσβέσεις λοιπού μηχανολογικού εξοπλισμού σε ακίνητα
                                       τρίτων εκτός εκμεταλλεύσεως

                    ...............

                    85.02.99

        85.03    Αποσβέσεις μεταφορικών μέσων

                    85.03.00    Αποσβέσεις αυτοκινήτων λεωφορείων

                               01    Αποσβέσεις λοιπών επιβατικών αυτοκινήτων

                               02    Αποσβέσεις αυτοκινήτων φορτηγών - Ρυμουλκών - Ειδικής
                                       χρήσεως

                               03    Αποσβέσεις σιδηροδρομικών οχημάτων

                               04    Αποσβέσεις πλωτών μέσων

                               05    Αποσβέσεις εναέριων μέσων

                               06    Αποσβέσεις μέσων εσωτερικών μεταφορών

                               07    .......................................

                               09    Αποσβέσεις λοιπών μέσων μεταφοράς

                               10    Αποσβέσεις αυτοκινήτων λεωφορείων εκτός εκμεταλλεύσεως

                               11    Αποσβέσεις λοιπών επιβατικών αυτοκινήτων εκτός
                                       εκμεταλλεύσεως

                               12    Αποσβέσεις αυτοκινήτων φορτηγών - Ρυμουλκών - Ειδικής
                                       χρήσεως εκτός εκμεταλλεύσεως

                               13    Αποσβέσεις σιδηροδρομικών οχημάτων εκτός εκμεταλλεύσεως

                               14    Αποσβέσεις πλωτών μέσων εκτός εκμεταλλεύσεως

                               15    Αποσβέσεις εναέριων μέσων εκτός εκμεταλλεύσεως

                               16    Αποσβέσεις μέσων εσωτερικών μεταφορών εκτός
                                       εκμεταλλεύσεως

                               17    .............................................

                               19    Αποσβέσεις λοιπών μέσων μεταφοράς εκτός εκμεταλλεύσεως

                    ..............

                    85.03.99

        85.04    Αποσβέσεις επίπλων και λοιπού εξοπλισμού

                    85.04.00    Αποσβέσεις επίπλων

                               01    Αποσβέσεις σκευών

                               02    Αποσβέσεις μηχανών γραφείων

                               03    Αποσβέσεις ηλεκτρονικών υπολογιστών και ηλεκτρονικών
                                       συγκροτημάτων

                               04    Αποσβέσεις μέσων αποθηκεύσεως και μεταφοράς

                               05    Αποσβέσεις επιστημονικών οργάνων

                               06    Αποσβέσεις ζώων για πάγια εκμετάλλευση

                               07    ......................................

                               08    Αποσβέσεις εξοπλισμού τηλεπικοινωνιών

                               09    Αποσβέσεις λοιπού εξοπλισμού

                               10    Αποσβέσεις επίπλων εκτός εκμεταλλεύσεως

                               11    Αποσβέσεις σκευών εκτός εκμεταλλεύσεως

                               12    Αποσβέσεις μηχανών γραφείων εκτός εκμεταλλεύσεως

                               13    Αποσβέσεις ηλεκτρονικών υπολογιστών και ηλεκτρονικών
                                       συστημάτων εκτός εκμεταλλεύσεως

                               14    Αποσβέσεις μέσων αποθηκεύσεως και μεταφοράς εκτός
                                       εκμεταλλεύσεως

                    85.04.15    Αποσβέσεις επιστημονικών οργάνων εκτός εκμεταλλεύσεως

                               16    Αποσβέσεις ζώων για πάγια εκμετάλλευση εκτός
                                       εκμεταλλεύσεως

                               17    .........................................

                               18    Αποσβέσεις εξοπλισμού τηλεπικοινωνιών εκτός
                                       εκμεταλλεύσεως

                               19    Αποσβέσεις λοιπού εξοπλισμού εκτός εκμεταλλεύσεως

                    ..............

                    85.04.99

        85.05    Αποσβέσεις ασώματων ακινητοποιήσεων και εξόδων
                     πολυετούς αποσβέσεως

                    85.05.00    Αποσβέσεις υπεραξίας επιχειρήσεως

                               01    Αποσβέσεις δικαιωμάτων βιομηχανικής ιδιοκτησίας

                               02    Αποσβέσεις δικαιωμάτων εκμεταλλεύσεως ορυχείων -
                                       μεταλλείων - λατομείων

                               03    Αποσβέσεις λοιπών παραχωρήσεων

                               04    Αποσβέσεις δικαιωμάτων χρήσεως ενσώματων πάγιων
                                       στοιχείων

                               05    Αποσβέσεις λοιπών δικαιωμάτων

                               06

                     .............

                               10    Αποσβέσεις εξόδων ιδρύσεως και πρώτης εγκαταστάσεως

                               11    Αποσβέσεις εξόδων ερευνών ορυχείων - μεταλλείων -
                                       λατομείων

                               12    Αποσβέσεις εξόδων λοιπών ερευνών

                               13    Αποσβέσεις εξόδων αυξήσεων κεφαλαίου και εκδόσεως
                                       ομολογιακών δανείων

                               14    Αποσβέσεις εξόδων κτήσεως ακινητοποιήσεων

                               15    ..........................................

                               16    Αποσβέσεις διαφορών εκδόσεως και εξοφλήσεως ομολογιών

                               17    Αποσβέσεις εξόδων αναδιοργανώσεως

                               18    Αποσβέσεις τόκων δανείων κατασκευαστικής περιόδου

                               19    Αποσβέσεις λοιπών εξόδων πολυετούς αποσβέσεως

                    ..............

                    85.05.99

        85.06

         .........

         85.99    Προϋπολογισμένες μη ενσωματωμένες στο λειτουργικό κόστος
                      αποσβέσεις (Λ/58.85)

2.2.808 Λογαριασμός 85 «Αποσβέσεις παγίων μη ενσωματωμένες στο λειτουργικό κόστος»

1. Στο λογαριασμό 85 καταχωρούνται οι αποσβέσεις των πάγιων περιουσιακών στοιχείων που δεν ενσωματώνονται στο λειτουργικό κόστος, δηλαδή στο λογαριασμό αυτό καταχωρούνται οι πρόσθετες (επιταχυνόμενες) αποσβέσεις που προβλέπονται από τη νομοθεσία που ισχύει κάθε φορά.

2. Οι τακτικές αποσβέσεις, που ενσωματώνονται στο λειτουργικό κόστος, καταχωρούνται στο λογαριασμό 66, σύμφωνα με όσα καθορίζονται στην παρ. 2.2.610, και εμφανίζονται στην κατάσταση του λογαριασμού «Αποτελέσματα χρήσεως» αφαιρετικά από το σύνολο των αποσβέσεων (τακτικών και πρόσθετων), σύμφωνα με το υπόδειγμα της παρ. 4.1.202.

3. Ο λογιστικός χειρισμός των αποσβέσεων του λογαριασμού 85 γίνεται σύμφωνα με όσα καθορίζονται στην παρ. 2.2.610.

4. Κατά το κλείσιμο του ισολογισμού τα υπόλοιπα των υπολογαριασμών του 85 μεταφέρονται στους αντίστοιχους υπολογαριασμούς του 86.03 «μη ενσωματωμένες στο λειτουργικό κόστος αποσβέσεις παγίων», έτσι ώστε ο λογαριασμός 85 να εξισώνεται.

 86    ΑΠΟΤΕΛΕΣΜΑΤΑ ΧΡΗΣΕΩΣ

        86.00    Αποτελέσματα εκμεταλλεύσεως

                     86.00.00    Μικτά αποτελέσματα (κέρδη ή ζημίες) εκμεταλλεύσεως

                               01    Άλλα έσοδα εκμεταλλεύσεως

                               02    Έξοδα διοικητικής λειτουργίας

                               03    Έξοδα λειτουργίας ερευνών - αναπτύξεως

                               04    Έξοδα λειτουργίας διαθέσεως

                     86.00.05    Έξοδα λειτουργίας παραγωγής μη κοστολογηθέντα (κόστος
                                       υποαπασχολήσεως - αδράνειας (Γνωμ. 46/1189/1989)

                     ...............

                     86.00.99

        86.01    Χρηματοοικονομικά αποτελέσματα

                     86.01.00    Έσοδα συμμετοχών

                               01    Έσοδα χρεογράφων

                               02    Κέρδη πωλήσεως συμμετοχών και χρεογράφων

                               03    Πιστωτικοί τόκοι και συναφή έσοδα

                               04    ....................................

                               07    Διαφορές αποτιμήσεως συμμετοχών και χρεογράφων

                               08    Έξοδα και ζημίες συμμετοχών και χρεογράφων

                               09    Χρεωστικοί τόκοι και συναφή έξοδα

                     ..............

                     86.01.99

        86.02    Έκτακτα και ανόργανα αποτελέσματα

                     86.02.00    Έκτακτα και ανόργανα έσοδα

                               01    Έκτακτα κέρδη

                               02    Έσοδα προηγούμενων χρήσεων

                               03    Έσοδα από προβλέψεις προηγούμενων χρήσεων

                               04    .........................................

                               07    Έκτακτα και ανόργανα έξοδα

                               08    Έκτακτες ζημίες

                               09    Έξοδα προηγούμενων χρήσεων

                               10    Προβλέψεις για έκτακτους κινδύνους

                     ..............

                     86.02.99

        86.03    Μη ενσωματωμένες στο λειτουργικό κόστος αποσβέσεις παγίων

                     86.03.00    Εδαφικών εκτάσεων

                               01    Κτιρίων - εγκαταστάσεων κτιρίων - τεχνικών έργων

                               02    Μηχανημάτων - τεχνικών εγκαταστάσεων - λοιπού
                                       μηχανολογικού εξοπλισμού

                               03    Μεταφορικών μέσων

                               04    Επίπλων και λοιπού εξοπλισμού

                               05    Ασώματων ακινητοποιήσεων και εξόδων πολυετούς
                                       αποσβέσεως

                     ..............

                     86.03.99

        86.04

         .........

        86.99    Καθαρά αποτελέσματα χρήσεως

2.2.809 Λογαριασμός 86 «Αποτελέσματα χρήσεως»

1. Ο λογαριασμός 86 χρησιμοποιείται μόνο στο τέλος της χρήσεως, οπότε καταρτίζεται υποχρεωτικά η κατάσταση των αποτελεσμάτων χρήσεως, σύμφωνα με το υπόδειγμα της παρ. 4.1.202.

2. Η κατάσταση του λογαριασμού 86, στην οποία, πέρα από τα στοιχεία τα οποία προκύπτουν από τους αντίστοιχους υπολογαριασμούς του, περιλαμβάνονται και στοιχεία συνολικού κύκλου εργασιών (πωλήσεων) και κόστους πωλήσεων, δημοσιεύεται μαζί με τον ισολογισμό, σύμφωνα με τις διατάξεις της νομοθεσίας που ισχύει κάθε φορά.

3. Ο λογαριασμός 86 χρησιμεύει για τον προσδιορισμό των συνολικών καθαρών αποτελεσμάτων (κερδών ή ζημιών) που πραγματοποιούνται από το σύνολο των δραστηριοτήτων της οικονομικής μονάδας μέσα στη χρήση που κλείνει.

Στο λογαριασμό 86, στο τέλος της χρήσεως, μεταφέρονται τα μικτά αποτελέσματα εκμεταλλεύσεως και τα διάφορα άλλα έσοδα για να συσχετιστούν με τα έξοδα των λειτουργιών διοικητικής, ερευνών - αναπτύξεως και διαθέσεως. Στον ίδιο λογαριασμό μεταφέρονται επίσης τα χρηματοοικονομικά αποτελέσματα (έσοδα - έξοδα), τα έκτακτα και ανόργανα αποτελέσματα (έσοδα, κέρδη - έξοδα, ζημίες) και οι μη ενσωματωμένες στο λειτουργικό κόστος αποσβέσεις παγίων.

Από το συσχετισμό των παραπάνω στοιχείων, που γίνεται στο λογαριασμό 86 και ειδικότερα στον υπολογαριασμό 86.99, προκύπτουν τα συνολικά καθαρά αποτελέσματα της κλειόμενης χρήσεως πριν από την αφαίρεση των φόρων που βαρύνουν τα κέρδη (φόρων εισοδήματος και εισφοράς ΟΓΑ καθώς και λοιπών μη ενσωματωμένων στο λειτουργικό κόστος φόρων).

4. Ο υπολογαριασμός 86.99 «καθαρά αποτελέσματα χρήσεως» χρησιμεύει για τη συγκέντρωση των χρεωστικών και πιστωτικών υπολοίπων των λοιπών υπολογαριασμών του 86, από το συσχετισμό δε των υπολοίπων αυτών προκύπτουν τα συνολικά καθαρά αποτελέσματα χρήσεως (καθαρές ζημίες ή καθαρά κέρδη), τα οποία μεταφέρονται στο λογαριασμό 88.

5. Ειδικότερα ο λογαριασμός 86 λειτουργεί ως εξής:

Ι. Χρεώνεται:

- κατά περίπτωση, με τις μικτές ζημίες εκμεταλλεύσεως της κλειόμενης χρήσεως, με πίστωση του λογαριασμού 80.01,

- με τα έξοδα των λειτουργιών διοικητικής, ερευνών - αναπτύξεως και διαθέσεως, με πίστωση, αντίστοιχα, των λογαριασμών 80.02.00, 80.02.01 και 80.02.02,

- με τις διαφορές αποτιμήσεως συμμετοχών και χρεογράφων, με τα έξοδα και τις ζημίες συμμετοχών και χρεογράφων και με τους χρεωστικούς τόκους και τα συναφή με αυτούς έξοδα, με πίστωση, αντίστοιχα, των λογαριασμών 80.02.04, 80.02.05 και 80.02.06,

- με τα έκτακτα και ανόργανα έξοδα, με πίστωση του λογαριασμού 81.00,

- με τις έκτακτες ζημίες, με πίστωση του λογαριασμού 81.02,

- με τα έξοδα προηγούμενων χρήσεων, με πίστωση του λογαριασμού 82.00,

- με τις προβλέψεις για έκτακτους κινδύνους, με πίστωση του λογαριασμού 83,

- με τις μη ενσωματωμένες στο λειτουργικό κόστος αποσβέσεις παγίων, με πίστωση του λογαριασμού 85,

- με τα καθαρά κέρδη, με πίστωση του λογαριασμού 88.00 «καθαρά κέρδη χρήσεως».

ΙΙ. Πιστώνεται: 

- κατά περίπτωση, με τα μικτά κέρδη εκμεταλλεύσεως της χρήσεως που κλείνει, με χρέωση του λογαριασμού 80.01,

- με τα διάφορα άλλα έσοδα της εκμεταλλεύσεως, με χρέωση του λογαριασμού 80.03.00,

- με τα έσοδα από συμμετοχές, με χρέωση του λογαριασμού 80.03.01,

- με τα έσοδα χρεογράφων, με χρέωση του λογαριασμού 80.03.02,

- με τα κέρδη από πώληση συμμετοχών και χρεογράφων, με χρέωση του λογαριασμού 80.03.03,

- με τους πιστωτικούς τόκους και τα συναφή με αυτούς έσοδα, με χρέωση του λογαριασμού 80.03.04,

- με τα έκτακτα και ανόργανα έσοδα, με χρέωση το λογαριασμού 81.01,

- με τα έκτακτα κέρδη, με χρέωση του λογαριασμού 81.03,

- με τα έσοδα προηγούμενων χρήσεων, με χρέωση του λογαριασμού 82.01,

- με τα έσοδα από προβλέψεις προηγούμενων χρήσεων, με χρέωση του λογαριασμού 84,

- με τις ενδεχόμενες καθαρές ζημίες, με χρέωση του λογαριασμού 88.01.

 88    ΑΠΟΤΕΛΕΣΜΑΤΑ ΠΡΟΣ ΔΙΑΘΕΣΗ

        88.00    Καθαρά κέρδη χρήσεως

        88.01    Ζημίες χρήσεως

        88.02    Υπόλοιπο κερδών προηγούμενης χρήσεως

        88.03    Ζημίες προηγούμενης χρήσεως προς κάλυψη

        88.04    Ζημίες προηγούμενων χρήσεων προς κάλυψη

        88.05

        88.06    Διαφορές φορολογικού ελέγχου προηγούμενων χρήσεων

        88.07    Λ/σμός αποθεματικών προς διάθεση

        88.08    Φόρος εισοδήματος

        88.09    Λοιποί μη ενσωματωμένοι στο λειτουργικό κόστος φόροι

        88.10

        ..........

        88.98    Ζημίες εις νέο

        88.99    Κέρδη προς διάθεση

2.2.811 Λογαριασμός 88 «Αποτελέσματα προς διάθεση»

1. Ο λογαριασμός 88 χρησιμοποιείται μόνο στο τέλος της χρήσεως, όταν γίνεται διάθεση κερδών, οπότε καταρτίζεται υποχρεωτικά πίνακας διαθέσεως καθαρών κερδών, σύμφωνα με το υπόδειγμα της παρ. 4.1.302.

Ο πίνακας του λογαριασμού 88, στον οποίο εμφανίζονται τα στοιχεία των υπολογαριασμών του και ο τρόπος διαθέσεως των κερδών, δημοσιεύεται μαζί με τον ισολογισμό και τα αποτελέσματα χρήσεως, σύμφωνα με τις διατάξεις της νομοθεσίας που ισχύει κάθε φορά.

2. Ο λογαριασμός 88 χρησιμεύει για τη συγκέντρωση των καθαρών αποτελεσμάτων της χρήσεως, των κερδών της προηγούμενης ή προηγούμενων χρήσεων, των ζημιών της προηγούμενης ή προηγούμενων χρήσεων, όταν πρόκειται να συμψηφιστούν με κέρδη της κλειόμενης χρήσεως, των διαφορών φορολογικού ελέγχου προηγούμενων χρήσεων και, στην περίπτωση διανομής αποθεματικών, των προς διάθεση τέτοιων αποθεματικών.

3. Ειδικότερα ο λογαριασμός 88 λειτουργεί ως εξής:

Ι. Χρεώνεται:

- ο υπολογαριασμός 88.01 με τις καθαρές ζημίες χρήσεως, με πίστωση του λογαριασμού 86.99.

- ο υπολογαριασμός 88.03 με το υπόλοιπο ή μέρος των ζημιών προηγούμενης χρήσεως που πρόκειται να καλυφτεί από κέρδη της κλειόμενης χρήσεως, με πίστωση του λογαριασμού 42.01.

- ο υπολογαριασμός 88.04 με το υπόλοιπο ή μέρος των ζημιών προηγούμενων χρήσεων που πρόκειται να καλυφτεί από κέρδη της κλειόμενης χρήσεως, με πίστωση του λογαριασμού 42.02.

- ο υπολογαριασμός 88.06 με τις συμψηφιστικές χρεωστικές διαφορές που προκύπτουν από φορολογικό έλεγχο προηγούμενων χρήσεων, με πίστωση του λογαριασμού 42.04.

- ο υπολογαριασμός 88.08 με το φόρο εισοδήματος και την εισφορά υπέρ ΟΓΑ που αναλογούν στα συνολικά καθαρά αδιανέμητα φορολογητέα κέρδη της χρήσεως που κλείνει, με πίστωση του λογαριασμού 54.07, σύμφωνα με όσα καθορίζονται στην περίπτ. 7 της παρ. 2.2.505.

- ο υπολογαριασμός 88.09 με τους λοιπούς μη ενσωματωμένους στο λειτουργικό κόστος φόρους (π.χ. λογ. 63.98.02 «φόροι ακίνητης περιουσίας» για τους οποίους οι σχετικές διατάξεις της νομοθεσίας προβλέπουν ότι τελικά βαρύνουν τα κέρδη χρήσεως - ή τη ζημία χρήσεως - και όχι το λειτουργικό κόστος), με πίστωση των οικείων υπολογαριασμών του λογαριασμού 63 της ομάδας 6 στους οποίους παρακολουθούνται οι φόροι της κατηγορίας αυτής.

- οι υπολογαριασμοί 88.00, 88.02, 88.06 και 88.07 με τα υπόλοιπά τους, με πίστωση του υπολογαριασμού 88.98, όταν το τελικό υπόλοιπο του λογαριασμού 88 είναι χρεωστικό (ζημίες εις νέο), ή του υπολογαριασμού 88.99, όταν το τελικό υπόλοιπο του λογαριασμού 88 είναι πιστωτικό (κέρδη προς διάθεση).

- ο υπολογαριασμός 88.99 με τα προς διάθεση κέρδη, με πίστωση των οικείων υπολογαριασμών αποθεματικών του 41, του λογαριασμού 53.01 για τα καθαρά μερίσματα που διανέμονται, του λογαριασμού 43.02 με τα καθαρά μερίσματα για τα οποία αποφασίζεται η διάθεσή τους για αύξηση του μετοχικού κεφαλαίου, του λογαριασμού 54.09.00 για τους φόρους που παρακρατούνται από τα μερίσματα που διανέμονται και του λογαριασμού 42.00 για το υπόλοιπο κερδών που μεταφέρεται στην επόμενη χρήση (υπόλοιπο κερδών εις νέο).

ΙΙ. Πιστώνεται:

- ο υπολογαριασμός 88.00 με τα καθαρά κέρδη χρήσεως, με χρέωση του λογαριασμού 86.99.

- ο υπολογαριασμός 88.02 με το υπόλοιπο κερδών προηγούμενης χρήσεως, με χρέωση του λογαριασμού 42.00.

- ο υπολογαριασμός 88.06 με τις συμψηφιστικές πιστωτικές διαφορές που προκύπτουν από φορολογικό έλεγχο προηγούμενων χρήσεων, με χρέωση του λογαριασμού 42.04.

- ο υπολογαριασμός 88.07 με τα αποθεματικά για τα οποία αποφασίζεται η διάθεσή τους για την κάλυψη ζημιών ή τη διανομή μερισμάτων, με χρέωση των οικείων υπολογαριασμών αποθεματικών του 41 (σε περίπτωση διαθέσεως αποθεματικών, παράλληλα, γίνονται και ανάλογες εγγραφές πληρωμένων και οφειλόμενων φόρων εισοδήματος και άλλης φύσεως, όταν συντρέχει σχετική φορολογική υποχρέωση).

- οι υπολογαριασμοί 88.01, 88.03, 88.04, 88.06, 88.08 και 88.09 με τα υπόλοιπά τους, με χρέωση του υπολογαριασμού 88.98, όταν το τελικό υπόλοιπο του λογαριασμού 88 είναι χρεωστικό (ζημίες εις νέο), ή του υπολογαριασμού 88.99, όταν το τελικό υπόλοιπο του λογαριασμού 88 είναι πιστωτικό (κέρδη προς διάθεση).

- ο υπολογαριασμός 88.98 με τις ζημίες εις νέο, με χρέωση του λογαριασμού 42.01.

89    ΙΣΟΛΟΓΙΣΜΟΣ

        89.00    Ισολογισμός ανοίγματος χρήσεως

        89.01    Ισολογισμός κλεισίματος χρήσεως

2.2.812 Λογαριασμός 89 «Ισολογισμός»

1. Ο λογαριασμός 89 χρησιμοποιείται μόνο στο τέλος της χρήσεως, οπότε καταρτίζεται υποχρεωτικά η κατάσταση του ισολογισμού της χρήσεως, σύμφωνα με το υπόδειγμα της παρ. 4.1.103.

2. Η κατάσταση του ισολογισμού, στην οποία περιλαμβάνονται τουλάχιστο τα στοιχεία που προβλέπονται από το υπόδειγμα της παρ. 4.1.103, όταν συντρέχουν οι σχετικές προϋποθέσεις, δηλαδή, όταν από την οικονομική μονάδα τηρούνται οι σχετικοί λογαριασμοί, δημοσιεύεται μαζί με την κατάσταση του λογαριασμού αποτελεσμάτων, σύμφωνα με τις διατάξεις της νομοθεσίας που ισχύει κάθε φορά.

3. Ο λογαριασμός 89 χρησιμεύει για το κλείσιμο των λογαριασμών της χρήσεως που αναφέρεται ο ισολογισμός και για το άνοιγμα των λογαριασμών της νέας χρήσεως που ακολουθεί μετά την κατάρτιση του ισολογισμού.

4. Ειδικότερα ο λογαριασμός 89 λειτουργεί ως εξής:

α. Στο τέλος της χρήσεως στην οποία αναφέρεται ο ισολογισμός, μετά τη διενέργεια των εγγραφών κλεισίματός του, χρεώνεται (ο λογ. 89) με τα υπόλοιπα όλων των χρεωστικών λογαριασμών του, με πίστωση καθενός απ' αυτούς, που έτσι μηδενίζονται, και πιστώνεται με τα υπόλοιπα όλων των πιστωτικών λογαριασμών του, με χρέωση καθενός απ' αυτούς, που επίσης με τον τρόπο αυτό μηδενίζονται.

Οι εγγραφές χρεώσεως και πιστώσεως του λογαριασμού 89, στο τέλος κάθε χρήσεως, οι οποίες ονομάζονται «εγγραφές κλεισίματος των λογαριασμών του ισολογισμού», γίνονται, είτε αμέσως μετά τη διενέργεια όλων των εγγραφών κλεισίματος του ισολογισμού, είτε τμηματικά για κάθε λογαριασμό που το υπόλοιπό του οριστικοποιείται για την εμφάνισή του στον ισολογισμό.

β. Με την έναρξη της χρήσεως που ακολουθεί μετά την κατάρτιση του ισολογισμού, έπειτα από τη διαδικασία διενέργειας των εγγραφών κλεισίματος του ισολογισμού, που ολοκληρώνεται μέσα στη χρονική περίοδο (προθεσμία) η οποία προβλέπεται από τις διατάξεις της νομοθεσίας που ισχύει κάθε φορά, χρεώνεται (ο λογ. 89) με τα υπόλοιπα όλων των πιστωτικών λογαριασμών του, με πίστωση καθενός από αυτούς, και πιστώνεται με τα υπόλοιπα όλων των χρεωστικών λογαριασμών του, με χρέωση καθενός από αυτούς.

Οι εγγραφές χρεώσεως και πιστώσεως του λογαριασμού 89, κατά την έναρξη της χρήσεως που ακολουθεί την κατάρτιση του ισολογισμού (κατά τη διάρκεια της προθεσμίας κλεισίματος του ισολογισμού που τρέχει στην επόμενη χρήση), οι οποίες ονομάζονται «εγγραφές ανοίγματος των λογαριασμών του ισολογισμού», γίνονται, είτε αμέσως έπειτα από τη διαδικασία διενέργειας όλων των εγγραφών κλεισίματος του ισολογισμού, είτε τμηματικά για κάθε λογαριασμό που, μετά την οριστικοποίηση του υπολοίπου του για την εμφάνισή του στον ισολογισμό, κλείνει σύμφωνα με όσα καθορίζονται παραπάνω.

\end{document}